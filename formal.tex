\chapter{Formal Model of Component System Evolution}
\label{formal}
\epigraph{We are like sailors who on the open sea must reconstruct their ship but are never able to start afresh from the bottom.}
{\textit{Willard Van Orman Quine, Word and Object, 1960}}
In the previous chapter, component system evolution (CSE) was described as the continual alteration of a system of components by a composer.
The described component models each defined constraints on this change to ensure the resulting system is valid.

This chapter describes the model (\modelname) and language (CUDF*) used to describe the CSE process. 
They also enable the discussion of CSE as they formally define terms and concepts used in this research.

The model \modelname (\textbf{Co}mponent \textbf{Sy}stem \textbf{E}volution) aims for the qualities of a model described by \cite{Selic2003}:
\begin{itemize}
  \item \textbf{Abstractness}: A model is a reduction of reality, removing or hiding detail that is irrelevant.
  \item \textbf{Understandability}: A model directly appeals to intuition, to convey complex ideas with little intellectual effort.
  \item \textbf{Accuracy}: A model must provide a true to life representation of the modelled system.
  \item \textbf{Predictiveness}: A model must be able to be used to predict the modelled system's non-obvious yet interesting properties.
  \item \textbf{Inexpensive}: A model must be cheaper to construct and analyze than the modelled system.
\end{itemize}
\modelname describes CSE as a series of evolution steps, where each step is in response to a user request to alter the system. 

The Common Upgradeability Description Format (CUDF) \citep{treinen2009common} 
and the Mancoosi Optimisation Format (MOF) \citep{abate2011} have been used to describe a single change to a component system.
However, for the purpose of this thesis, these formats are insufficient due to the fact that CSE requires many changes.
To address this issue, the CUDF* language was developed as a combination and extension of the CUDF and MOF models.
A CUDF* document can be parsed to create an instance of \modelname, thus fully describe the evolution of a component system.

The relationships between the models presented in this chapter are described in figure \ref{formal.modeldiagram}.
\begin{figure}[htp]
\begin{center}
\digraph[scale=.5]{formalmodeldiagram}{
rankdir=BT;
\modelnamewx [label=<<TABLE  BORDER="0" CELLBORDER="1" CELLSPACING="5"><TR><TD WIDTH="150">\modelnamewx</TD></TR></TABLE>> shape=none];
FMI [label=<<TABLE BORDER="0" CELLBORDER="1" CELLSPACING="5"><TR><TD WIDTH="150">:\modelnamewx</TD></TR></TABLE>> shape=none];
CM[label=<<TABLE BORDER="0" CELLBORDER="1" CELLSPACING="5"><TR><TD WIDTH="150">CUDF</TD></TR></TABLE>> shape=none];
MOF[label=<<TABLE BORDER="0" CELLBORDER="1" CELLSPACING="5"><TR><TD WIDTH="150">MOF</TD></TR></TABLE>> shape=none];
CMS[label=<<TABLE BORDER="0" CELLBORDER="1" CELLSPACING="5"><TR><TD WIDTH="150">CUDF*</TD></TR></TABLE>> shape=none];
CMI[label=<<TABLE BORDER="0" CELLBORDER="1" CELLSPACING="5"><TR><TD WIDTH="150">:CUDF* Document</TD></TR></TABLE>> shape=none];
rank=max;
FMI -> \modelnamewx [ label = " instantiates"];
CMS -> CM [ label=" extends"];
CMS -> MOF [label=" extends"];
CMI -> CMS [ label=" instantiates"];
subgraph {
	rank=same;
	CMI -> FMI [ label="parsed to" ];
}
}
  \caption{The relationships between CUDF* and \modelname}
  \label{formal.modeldiagram}
\end{center}
\end{figure}

This chapter first 
defines the \modelname model then the CUDF* language.
The complete description of parsing a CUDF* document to an instance of \modelname is presented in Appendix \ref{apx.cudf}.
 
\section{\modelname Model of Component System Evolution}
\label{formal.step}
This section presents the \modelname model of component system evolution. 
\modelname describes the evolution of a component system as a series of evolution steps.
Each step tries to satisfy a user request for change along with the constraints that ensure the system is valid.
The selected component system is optimal with respect to a preference order.

\subsection{Evolution Problem}
The solutions to an evolution problem make up the set of valid component systems that can be evolved to.
To define this, first the basic elements of the model must be defined, such as components and constraints. 

%%%What a component is
Components are the atomic units of a component system's evolution.
\begin{defs}
\label{formal.componentdef}
Let $\mathcal{N}$ be the set of names, where a name is a finite string of characters; 
let $\mathcal{V}$ be the totally ordered set of versions\footnote{Given $\mathcal{V}$ is ordered under 
$\leq$, $\leq$ is antisymmetric ($v \leq w$ and $w \leq v$ then $v = w$), transitive ($v \leq w$ and $w \leq x$ then $v \leq x$) and total ($v \leq w$ or $w \leq v$)}; 
a \textbf{component}
is an element of $\mathbb{C}$, where $\mathbb{C} = \mathcal{N} \times \mathcal{V}$.
\end{defs}
A component $a$ is then a pair $\langle n,v \rangle$, where $n$ is the component's name, and $v$ is the component's version.
The characters  $a,b,c,\ldots$ are used to denote components, $m,n,o,\ldots$ used to denote names, and $v,w,x,\ldots$ used to denote versions.

\begin{defs}
\label{formal.componentsystemdef}
A \textbf{component system} is a finite subset of components, e.g. $\alpha$ is a component system where $\alpha \subseteq_{finite} \mathbb{C}$. 
\end{defs}
The characters $\alpha,\beta,\gamma$ are used to denote component systems\footnote{In hierarchical component models, such as Fractal \citep{Quma2006}, 
a component system would also be a component.
That is, a component system could have elements that are themselves component systems.
This definition is not included in this model, though it may be possible for it to be modified to include it in the future.}.

Constraints are used to describe valid systems.
\begin{defs}
\label{formal.constraintdefs}
A \textbf{constraint} $con$ is a set of component systems, i.e. $con \subseteq 2^{\mathbb{C}}$.
\end{defs}
Such constraints can be defined intentionally by specifying the necessary and sufficient conditions for a component system to be in a constraint.

In most component models described in chapter \ref{background}, only specific types of constraints are used, these types are:
\begin{enumerate}
  \label{formal.constrainttypes}
  \item \textbf{Exclusion}: $\neg a := \{\alpha \in 2^{\mathbb{C}} \mid a \not \in \alpha \}$ \label{formal.removeconstraint}.
  \item \textbf{Conflict}: $a \rightarrow \neg c := \{\alpha \in 2^{\mathbb{C}} \mid \mbox{if } a \in \alpha \mbox{ then } c \not \in \alpha\}$ \label{formal.conflictconstraint}.
  \item \textbf{Inclusive Disjunction}: $a_1 \vee \ldots \vee a_n := \{\alpha \in 2^{\mathbb{C}} \mid a_1 \in \alpha \mbox{ or }\ldots \mbox{ or } a_n \in \alpha\}$ \label{formal.keepconstraint}.
  \item \textbf{Dependence}: $a \rightarrow c_1 \vee \ldots \vee c_n := \{\alpha \in 2^{\mathbb{C}} \mid \mbox{if } a_1 \in \alpha \mbox{ then }  c_1 \in \alpha \mbox{ or } \ldots \mbox{ or } c_n \in \alpha\}$ \label{formal.dependencyconstraint}.
  \item \textbf{Exactly One}: $a_1 + \ldots + a_n = 1 := \{\alpha \in 2^{\mathbb{C}} \mid |\{a_1,\ldots,a_n\} \cap \alpha| = 1\}$ \label{formal.upgradeconstraint}.
\end{enumerate}
Note, for constraint types \ref{formal.keepconstraint}, \ref{formal.dependencyconstraint} and \ref{formal.upgradeconstraint} the value of $n$ can be equal to $1$.

The vocabulary used when discussing constraints is as follows:
\begin{defs}
\label{formal.constraintdefs}
{\ }
\begin{enumerate}
  \item A component system $\alpha$ is said to \textbf{satisfy} a constraint $con$ iff $\alpha \in con$.
  \item An \textbf{evolution problem} $EP$ is a finite set of constraints.
  \item A \textbf{solution} to an evolution problem $EP$ is a component system  $\alpha$  that satisfies all constraints in the evolution problem, i.e. $\alpha \in con_i$ for all $con_i \in EP$.
  \item An evolution problem is \textbf{unsatisfiable} if there exists no solution to it.  
\end{enumerate}
\end{defs}

Based on the definitions presented in this section, the following example illustrates how an evolution problem can be described and solved.
Assume the evolution problem $EP$ is defined as the set of two constraints $con_1$ and $con_2$, i.e. $EP = \{con_1,con_2\}$.
The constraint $con_1$ is $b$ (constraint type \ref{formal.keepconstraint}), which is defined as the set of all component systems with $b$ in them.
The constraint $con_2$ is $a \rightarrow c$ (constraint type \ref{formal.dependencyconstraint}), is defined as the set of all component systems where if $a$ is included, then $c$ is included.
The component systems that satisfy both $con_1$ and $con_2$ are solutions to $EP$.
Some of these solutions are $\{b\}$, $\{a,b,c\}$ and $\{a,b,c,d\}$.

\subsubsection{Complexity of an Evolution Problem}
When investigating the complexity of an evolution problem, it was observed that:
\begin{obs}
Finding a solution to an evolution problem is NP-complete \citep{cook1971}.
\end{obs}
This observation has been made before by \cite{Mancinelli2006} and \cite{abate2011}.
This can be shown by reducing a known NP-complete problem, called one-in-three satisfiability, to the evolution problem.
The one-in-three satisfiability problem is defined by \cite{Schaefer1978} as:
\begin{quote}
Given sets $S_1, \ldots, S_n$ each having at most 3 members, is there a subset $T$ of the members such that for each $i$, $|T \cap S_i|  =  1$.
\end{quote}

This can be reduced to an evolution problem where:
\begin{quote}
Given sets $S_1, \ldots, S_n$ each containing 3 components, and constraints $con_1,\ldots,con_n$
such that for each $i$, $S_i = \{a,b,c\}$ iff $con_i$ equals constraint $a + b + c = 1$ (constraint type \ref{formal.upgradeconstraint}).
A set of components $\alpha$ is such that for each $i$, $|\alpha \cap S_i|  =  1$ iff $\alpha$ is a solution to the evolution problem $EP = \{con_1,\ldots,con_n\}$.
\end{quote}

This reduction uses the ``exactly one'' constraint (type \ref{formal.upgradeconstraint}).
However, if this constraint type was not included into this formalism, the evolution problem would still be NP-Complete,
as one-in-three satisfiability can be reduced to a problem consisting of  constraint types \ref{formal.conflictconstraint} and \ref{formal.keepconstraint}.

\subsection{Constraints and Requests}
\label{formal.constraints}
An evolution problem consists of two sets of constraints; constraints created by the user request to change the system, 
and constraints required for the system to be valid.
To clarify the purpose of these constraints, they are separated into two sets.

\begin{defs}
The set of \textbf{user requests} $\Delta$ is the set of all constraints, where each constraint in $\Delta$ is either:
\begin{enumerate}
  \item an \textbf{installation} request of constraint type \ref{formal.keepconstraint}.
  \item a \textbf{remove} request of constraint type \ref{formal.removeconstraint}.
  \item a \textbf{*grade}\footnote{This referrers to either request an upgrade or a downgrade} request of constraint type \ref{formal.upgradeconstraint}.
\end{enumerate}
\end{defs}

Consider the examples of:
\begin{itemize}
  \item an installation request $a \vee b$ is a request by the user to install component $a$ or $b$ into the system.
  \item a remove request $\neg a$ is a request by the user to remove component $a$ from the system.
  \item a *grade request $a + b = 1$ is a request by the user to include either $a$ or $b$, but not both.
\end{itemize} 

The *grade request uses the ``exactly one'' constraint (type \ref{formal.upgradeconstraint}) to represent both downgrade and upgrade requests from the user.
The intuition is that after upgrading or downgrading a component, exactly one component should be included in the system \citep{treinen2009common}.

\begin{defs}
The set of \textbf{system constraints} $\Omega$ is a set of all constraints, where each constraint in $\Omega$ is either: 
\begin{enumerate}
  \item a \textbf{keep} constraint of type \ref{formal.keepconstraint}.
  \item a \textbf{dependency} constraint of type \ref{formal.dependencyconstraint}.
  \item a \textbf{conflict} constraint of type \ref{formal.conflictconstraint}.
\end{enumerate}
\end{defs}

Consider the examples of:
\begin{itemize}
  \item a keep constraint $a \vee b$ is a system constraint to keep component $a$ or $b$ in the system.
  \item a dependency constraint $a \rightarrow b \vee c$ is a system constraint where if $a$ is in the system, then $b$ or $c$ must be in the system.
  \item a conflict constraint $a \rightarrow \neg b$ is a system constraint where if $a$ is in the system then $b$ must not be in the system.
\end{itemize} 

The evolution of a component system happens over time.
This concept of ``time'' must be introduced to the model.
\begin{defs}
The set $T$ is the set of all times. $T$ is a strict total ordered under $<_{time}$, where times $t_i$ and $t_{i+1}$ in $T$ implies $t_i <_{time} t_{i+1}$
\end{defs}

At any time, there exists only a finite set of components. 
\begin{defs}
$\mathbb{C}_{t}$ is the finite set of components that exists at time $t$, $\mathbb{C}_{t} \subset_{finite} \mathbb{C}$
\end{defs}

At any point in time, the sets of system constraints and user request constraints are with respect to the set of components that exist at that time.
That is, a constraint can only describe systems that are composed of components that exist at a given time. 
\begin{defs}
Given time $t$, and the set of components $\mathbb{C}_{t}$, the set of \textbf{system constraints w.r.t. time} is $\Omega_{t}$, where $\Omega_{t} = \{con \cap 2^{\mathbb{C}_{t}} \mid con \in \Omega\}$,
and the set of \textbf{user requests w.r.t. time} is $\Delta_{t}$, where $\Delta_{t} = \{con \cap 2^{\mathbb{C}_{t}} \mid con \in \Delta\}$
\end{defs}

\begin{defs}
Given a time $t$, a \textbf{user request}, $\delta_{t}$, is a finite set of user requests at time $t$, i.e. $\delta_{t} \subseteq_{finite} \Delta_{t}$;
and a set of \textbf{system constraints}, $\omega_{t}$, is a finite set of system constraints at time $t$, i.e. $\omega_{t} \subseteq_{finite} \Omega_{t}$.  
\end{defs}

An evolution problem, $EP$, can be defined by combining a user request $\delta_{t}$ and a set of system constraints $\omega_{t}$, such that $EP = \delta_{t} \cup \omega_{t}$.

An example is presented to illustrate the relationship an evolution problem has to time.
Consider the system constraint $con_1$ that is defined as $a \rightarrow b \vee d$, and the user request constraint $con_2$ defined as $a$ (a singleton installation constraint).
The constraint $con_1$ can (but is not limited to) be satisfied by the component systems $\{\}$, $\{b\}$, $\{a,b\}$, $\{a,b,d\}$, and $\{a,d\}$,
and the constraint $con_2$ can (but is not limited to) be satisfied by component systems $\{a\}$, $\{a,b\}$ and $\{a,b,d\}$.

Assume that at time $t$, only the components $a$ and $b$ exist, i.e $\mathbb{C}_{t} = \{a,b\}$.
The set of all component systems at time $t$ is $2^{\mathbb{C}_{t}} = \{\{\},\{a\},\{b\},\{a,b\}\}$.
Therefore, at time $t$ component $d$ does not exist, it is not in $\mathbb{C}_{t}$, nor in any component system in $2^{\mathbb{C}_{t}}$.

The system constraint $con_1' \in \Omega_{t}$, is such that $con_1' = con_1 \cap 2^{\mathbb{C}_{t}} = \{\{\},\{b\},\{a,b\}\}$.
The user request constraint $con_2' \in \Delta_{t}$, is such that $con_2' = con_2 \cap 2^{\mathbb{C}_{t}} = \{\{a\},\{a,b\}\}$.

Assume a set of system constraints $\omega_{t} = \{con_1'\}$, and a user request constraints $\delta_{t} = \{con_2'\}$.
An evolution problem $\delta_{t} \cup \omega_{t}$, will have exactly one solution, the component system $\{a,b\}$.  

\subsection{Component System Evolution}
\label{formal.evo}
A component system is changed from one system to another to satisfy an evolution problem containing a user request and system constraints.
Many evolution problems have multiple valid solutions, as evolution problems are generally under-constrained \citep{Berre2008}.
Different solutions may have different properties,
and as a result, a user may prefer one system over another.
Introducing an ordering over the solutions addresses such user preferences.  

\begin{defs}
Given a component system $\alpha$, an \textbf{evolution preference order} w.r.t. $\alpha$ is a strict partial order $\prec_{\alpha} \subseteq 2^{\mathbb{C}} \times 2^{\mathbb{C}}$.
\end{defs}
A strict partial order has the properties of irreflexivity (not $\beta \prec_{\alpha} \beta$), asymmetry (if $\beta \prec_{\alpha} \beta'$ then not $\beta' \prec_{\alpha} \beta$), 
and transitivity (if $\beta \prec_{\alpha} \beta'$ and $\beta' \prec_{\alpha} \beta''$ then $\beta \prec_{\alpha} \beta''$).
The non-strict evolution preference order $\preceq_{\alpha}$ can be defined such that $\beta \prec_{\alpha} \beta'$ if and only if $(\beta \preceq_{\alpha} \beta'$ and $\beta \neq \beta')$. 

\begin{defs}
Given a component system $\alpha$, an evolution preference order w.r.t. $\alpha$ $\prec_{\alpha}$, and an evolution problem $EP$;
a component system $\beta$ is \textbf{optimal} if $\beta$ is a solution to $EP$, and there exists no other solution to $EP$ $\beta'$, such that $\beta \prec_{\alpha} \beta'$.   
\end{defs}

\begin{defs}
\label{formal.stepdef}
Given times $t_{i-1}$ and $t_i$, a component system $\alpha_{t_{i-1}}$, an evolution preference order $\prec_{\alpha_{t_{i-1}}}$, 
a user request $\delta_{t_i}$, and a set of system constraints $\omega_{t_i}$;
the \textbf{evolution step} function  $\epsilon(\alpha_{t_{i-1}},\delta_{t_i} \cup \omega_{t_i})$ returns the component system $\alpha_{t_i}$ such that:
\begin{itemize}
  \item iff the evolution problem $\delta_{t_i} \cup \omega_{t_i}$ is satisfiable, $\alpha_{t_i}$ is an optimal solution to $\delta_{t_i} \cup \omega_{t_i}$.
  \item iff the evolution problem $\delta_{t_i} \cup \omega_{t_i}$ is unsatisfiable, $\alpha_{t_i}$ equals $\alpha_{t_{i-1}}$.
\end{itemize}
\end{defs}
The evolution step function takes a previous system and changes it to satisfy a user's request and system constraints with a preferable system.
If the evolution problem is unsatisfiable, the evolution step function does not change the system and returns the previous system $\alpha_{t_{i-1}}$.
Such a situation occurs when the user requests a change that either conflicts with itself, or conflicts with the system constraints.
For example, consider the situation where a user requested to install components $a$ and $b$, but $a$ and $b$ conflict with each other.
If this situation happened in reality, a report will be generated and returned to the user explaining why the evolution problem is unsatisfiable.
This report may also include hints to the user as to how to make the request satisfiable.

If an evolution problem is unsatisfiable, it could be possible to return a component system that maximises the number of satisfied constraints in the user request.
This approach, however, is impossible to define as the intentions of a user request are unknown.
For example, consider the situation where a user requested to install components $a$ and $b$, but $a$ and $b$ conflict with each other.
Which component should be selected to be installed in the system?
Given the user request, this question cannot be answered.
The user may only require $a$ if $b$ is installed, or vice versa.
Making no assumptions about the intent of the user request is the prudent approach taken in this model.

\begin{defs}
Given a series of times $t_0,\ldots,t_n$, an initial component system $\alpha_{t_0}$, 
sets of components $C_{t_1},\ldots,C_{t_n}$, a series of user requests $\delta_{t_1},\ldots,\delta_{t_n}$, a series of system constraints $\omega_{t_1},\ldots,\omega_{t_n}$,
and a series of evolution preference orders $\prec_{\alpha_{t_0}},\ldots, \prec_{\alpha_{t_{n-1}}}$;
the \textbf{evolution} function $\epsilon^*$ is defined as\\
$\epsilon^*(\alpha_{t_0},\langle \delta_{t_1} \cup \omega_{t_1},\ldots,\delta_{t_n} \cup \omega_{t_n} \rangle)$ $=$ $\epsilon(\epsilon(\ldots \epsilon(\alpha_{t_0},\delta_{t_1} \cup \omega_{t_1}),\ldots),\delta_{t_n} \cup \omega_{t_n})$
$=$ $\alpha_{t_n}$.
\end{defs}

This evolution function is the core definition in \modelname as it describes the evolution process of repeatedly changing a component system.
That is, the evolution function $\epsilon^*$ is the ``closure'' of the evolution step function $\epsilon$.  
An instance of \modelname will contain: 
\begin{itemize}
  \item A series of times $t_0,\ldots,t_n$.
  \item Sets of components $\mathbb{C}_{t_0},\ldots,\mathbb{C}_{t_n}$.
  \item Sets of user requests $\delta_{t_1},\ldots,\delta_{t_n}$.
  \item Sets of system constraints $\omega_{t_1},\ldots,\omega_{t_n}$.
  \item Evolution preference orders $\prec_{\alpha_{t_0}},\ldots, \prec_{\alpha_{t_{n-1}}}$.
  \item Initial system $\alpha_{t_0}$.
\end{itemize}
This is enough information to calculate the system from $\alpha_{t_1}$ to $\alpha_{t_n}$, also known as \textbf{resolving} a \modelname instance.

The following observations can be made about the \modelname model.

\begin{obs}
It is not required for an evolution step to satisfy prior user requests.
\end{obs}
For example, if a user wants to install a text editor into the system, 
the user would expect that in future the text editor will remain in the system.
This property is not explicitly stated in the model, because (as described above) it is difficult to determine a user's intention with a request.

A similar problem is faced by the revision function in the domain of belief revision \citep{alchourron1985logic}.
Revision functions try to preserve consistency across a set of beliefs as new, and possibly inconsistent, information is introduced.
This is similar, as a set of previous user requests should try to remain satisfied with the component system, as the user requests to change the system.    
This problem with the evolution process is not further explored in this research, but belief revision may provide inspiration for future research.


\begin{obs}
Multiple evolution steps may produce a less optimal component system than an equivalent, single evolution step.
\end{obs}

This can be demonstrated by describing an instance of \modelname.
Let the evolution preference order select solutions with the minimum number of components altered (the symmetric difference between the solutions), 
i.e. $\beta \prec_{\alpha} \beta'$ iff $|\alpha \Delta \beta| > |\alpha \Delta \beta'|$.
Further, let:
\begin{itemize}
  \item The times $0$, $1$ and $2$.
  \item The set of components $\mathbb{C}_0 = \mathbb{C}_1 = \mathbb{C}_2 = \{a,b,c,d\}$.
  \item The system at time $0$ be $\alpha_0 = \{\}$.
  \item A series of user requests be $\delta_1 = \{a\}$ and $ \delta_2 = \{c\}$.
  \item The set of constraints $\omega_1 = \omega_2 = \{a \rightarrow b \vee c, c \rightarrow d\}$.
\end{itemize}

Consider the component evolution from $\alpha_0$ to $\alpha_2$:
\begin{itemize}
  \item $\alpha_0 = \{\}$.
  \item $\alpha_1 = \epsilon(\alpha_0,\delta_1 \cup \omega_1)$, given the preference order $\prec_{\alpha_0}$. $|\alpha_0 \Delta \{a,b\}| = 2$ is minimal, therefore $\alpha_1 = \{a,b\}$.
  \item $\alpha_2 = \epsilon(\alpha_1,\delta_2 \cup \omega_2)$, given the preference order $\prec_{\alpha_1}$. $|\alpha_1 \Delta \{a,b,c,d\}| = 4$ is minimal, therefore $\alpha_2 = \{a,b,c,d\}$.
\end{itemize}

Further, consider the component evolution directly from $\alpha_0$ to $\alpha_2$, where $\delta_2' = \delta_1 \cup \delta_2$:
\begin{itemize}
  \item $\alpha_0 = \{\}$.
  \item $\alpha_2' = \epsilon(\alpha_0,\delta_2' \cup \omega_2)$, given the preference order $\prec_{\alpha_0}$. $|\alpha_0 \Delta \{a,c,d\}| = 3$ is minimal, therefore $\alpha_2' = \{a,c,d\}$.
\end{itemize}

In this example, the system $\alpha_2$ is less optimal than the system $\alpha_2'$ w.r.t. the order $\prec_{\alpha_0}$.
This counter-intuitive result shows that the incomplete knowledge of future user requests may effect the optimality of the evolved component system.

\subsection{Multiple Criteria Preferences}
\label{formal.opt}
%%%The formal definition of how to evolve a component system is complicated because it is a multicriteria problem
Typically to define an evolution preference order, multiple criteria are considered. 
All systems presented in chapter \ref{background} (such as Eclipse P2, apt-get, aptitude, and MPM), define multiple criteria that express the preferences in their systems.
A multi-criteria framework is presented as part of \modelname, to enable the description of evolution preference orders. 

%%%A naive solution is to map all criteria to real numbers, then aggregate them through addition multiplication with weights.
A simple way to define the order $\prec_{\alpha}$, can be by defining a scoring function ($f: 2^{\mathbb{C}} \times 2^{\mathbb{C}} \rightarrow \mathbb{R}$) for each criterion. 
These functions can then be aggregated, e.g. using addition or multiplication (possibly with weights), and used to define $\prec_{\alpha}$.
For example, a function that returns the number of components that have changed between systems is defined $f_{change}(\alpha,\beta) = |\alpha \Delta \beta|$.
A function that returns the size of all components in a component system is defined $f_{size}(\alpha,\beta) = \sum \limits_{c \in \beta} size(c)$,
 where $size$ is a function that takes a component and returns its size in kilobytes.
The evolution preference order could then be defined as $\beta \prec_{\alpha} \beta'$ iff $f_{change}(\alpha,\beta) + f_{size}(\alpha,\beta) > f_{change}(\alpha,\beta') + f_{size}(\alpha,\beta')$.

%%%This has draw backs, some criteria are infinitly more important, it is not intuitive for users to weight.
Defining the evolution preference order in this manner has some drawbacks:
\begin{itemize}
  \item Information can be lost in the conversion to real numbers, e.g. the reason why a specific system was selected over another.
  \item The relationships between criteria are difficult to intuitively define, e.g. ``size $+$ change'' results in a confusing metric.
  \item Altering weights to express order over criteria does not convey meaningful information about decisions or their impact.
  For example, stating ``the size of a system is $.25$ multiplied by the change of a system'' does not convey what the implications of the weighting will be.
\end{itemize}

%%%Using composed orders
A possibly more intuitive approach to define the evolution preference order is to create a ranking function that measures a component system against an arbitrary range.
A criterion is then a ranking function and an order over its range. 
By not reducing the problem to only real numbers, information can be retained about the selection of a component system.

Using a lexicographic order, each criterion can then be composed together into a composite criterion.
The lexicographic composition lets an order over the preference of criteria be defined, without having to assign arbitrary weights.
This method has been used before in the MPM \citep{abate2011}.
It is the selected mechanism to define the preference order in this thesis.

\begin{defs}
A \textbf{ranking function}, $rank_{\alpha}$, is a function that ranks a component system to a range $A$ w.r.t. a component system $\alpha$, i.e. $rank_{\alpha} : 2^{\mathbb{C}} \rightarrow A$.
\end{defs}
The range $A$ could be real or natural numbers, integers, versions, components, \ldots

\begin{defs}
\label{formal.defcriteria}
A \textbf{criterion} is a tuple $\langle rank_{\alpha}, \leq \rangle$, 
where the ranking function is defined as $rank_{\alpha} : 2^{\mathbb{C}} \rightarrow A$, and $\leq$ (whose strict order is $<$) is a partial order over $A$.
\end{defs}
The order $\leq$ is a partial order; $\leq$ is reflexive ($a \leq a$), antisymmetry ($\text{if } a \leq b \text{ and } b \leq a \text{ then } a = b$), and transitive ($ \text{if } a \leq b  \text{ and } b \leq c  \text{ then } a \leq c$).

The evolution preference order is described with respect to a criterion.
\begin{defs}
\label{formal.defcrittooptimal}
The \textbf{evolution preference order} $\prec_{\alpha}$ can be defined with a criterion $\langle rank_{\alpha}, \leq \rangle$,
such that given component systems $\beta$ and $\beta'$, iff $rank_{\alpha}(\beta)$ is strictly less than ($<$) $rank_{\alpha}(\beta')$ then $\beta'$ is preferred, 
i.e. $\beta \prec_{\alpha} \beta'$ iff $rank_{\alpha}(\beta) < rank_{\alpha}(\beta')$.
\end{defs}


\begin{defs}
\label{formal.defcritlex}
The \textbf{lexicographic composition} of multiple criteria into a single criterion is defined with the operator $\oplus$.
Given the component system $\alpha$, and the two criteria $crit_1 = \langle rank^1_{\alpha}, \leq_{1} \rangle$ and $crit_2 = \langle rank^1_{\alpha}, \leq_{2} \rangle$,
where the range of $rank^1_{\alpha}$ is $A_1$, and the range of $rank^2_{\alpha}$ is $A_2$.
The lexicographical composition of $crit_1$ and $crit_2$, $crit_1 \oplus crit_2$, returns the tuple $\langle rank^L_{\alpha},\leq_L \rangle$.
The range of $rank^L_{\alpha}$ is $A_1 \times A_2$, and $rank^L_{\alpha}(\beta) = \langle rank^1_{\alpha}(\beta),rank^2_{\alpha}(\beta) \rangle$.
The order $\leq_L$ (whose strict order is $<_L$) is over $A_1 \times A_2$, such that $(a,b) \leq_L (a',b')$ iff $a <_1 a'$ or $(a = a'$ and $b \leq_2 b')$.
\end{defs}

The lexicographic composition of two criteria results in a criterion.
That is, by using the composition $\oplus$, two criteria are lexicographically composed into a tuple $\langle rank^L_{\alpha},\leq_L \rangle$.
The order $\leq_L$ is a partial order over the range of $rank^L_{\alpha}$.
Therefore, the tuple  $\langle rank^L_{\alpha},\leq_L \rangle$ is a criterion.
This means that any number of criteria can be composed together using $\oplus$,
and resulting criterion can then be used to define the evolution preference order $\prec_{\alpha}$


\section{CUDF* Language}
\label{formal.cudf}
For \modelname to be of use to study CSE, a language to describe CSE problems is required.
As mentioned already, this new language is called CUDF* which extends and combines the Common Upgradeability Description Format (CUDF)\footnote{This specification was previously discussed in section \ref{background.models}, with an example shown in figure \ref{CUDFmetadata}.
}
\citep{treinen2009common} and the Mancoosi Optimisation Format (MOF)\footnote{http://www.mancoosi.org/misc-2011/criteria/ accessed 6/3/2012}.
To describe CUDF*, this section first describes CUDF and MOF.
Then CUDF* is presented, and the description of how CUDF* is parsed to create a \modelname instance is given (for the full description refer to appendix \ref{apx.cudf}).

\subsection{CUDF}
\label{formal.cudfDesign}
CUDF was designed by Mancoosi for the Mancoosi International Solver Competition (MISC).
Mancoosi is a research project committed to the improvement of the process of evolving Free and Open Source Software (FOSS) distributions.
A contribution of Mancoosi was to hold the MISC that compared solvers of CUDF problems.
MISC was held to encourage researchers and practitioners into researching change made to component systems.

An issue faced by Mancoosi when organising MISC was that no common format existed to describe evolution problems.
To address this issue, the CUDF language \citep{treinen2009common} was defined.
This language allows the abstraction of evolution problems and the benchmarking of applications that find their solutions. 

\subsubsection{CUDF Language}
This section describes the CUDF format.
The goal of this description is to provide a general overview of CUDF rather than its complete specification.
Some aspects of CUDF (e.g. the formal type system) are not taken into account  as a complete redefinition of the original language is not the goal of this section.

A CUDF document is a single plain text file that represents all information necessary to define an evolution problem.
A CUDF document also contains the description of the set of components that make a component system.
An example of a CUDF document is presented in section \ref{background.models}.

As shown in figure \ref{formal:cudfstruct}, a CUDF document is separated into three sections:
\begin{itemize}
  \item \textbf{preamble} stanza: defines the additional types used in the CUDF document.
  \item \textbf{package description} stanzas: typically the bulk of a CUDF document, this defines the set of components (using the FOSS term package).
  \item \textbf{request} stanza: this defines the requested change that should be made to the system.
\end{itemize}

\begin{figure}[htp]
\centering
\begin{tabular}{|c|}
\hline preamble \\ \hline
\hline package description$_{1}$ \\ \hline
\hline package description$_{2}$ \\ \hline
 \multicolumn{1}{c}{\ldots}  \\ 
\hline package description$_{i}$\\ \hline
\hline request\\ \hline
\end{tabular}
\caption{Structure of the CUDF stanzas}
\label{formal:cudfstruct}
\end{figure}

Every stanza begins with a key of its type (\verb+preamble+, \verb+package+, \verb+request+), followed by several lines ending in an empty line.
Each line in a stanza defines a property using a key/value pair separated by the delimiter ``:''.
This structure is presented in figure \ref{formal.cudfstanza}

\begin{figure}[htp] 
\begin{center}
\begin{alltt}
\textit{stanza}: \textit{value}
\textit{key_1}: \textit{value_1}
\ldots
\textit{key_n}: \textit{value_n}
\end{alltt}
  \caption{Structure of a CUDF stanza}
  \label{formal.cudfstanza}
\end{center}
\end{figure}

\subsubsection{Package Description}
The package description stanza starts with the \verb+package+ key followed by the value of package name, a non-empty string.
The package description stanzas define the set of components $\mathbb{C}_{t_1}$, the set of system constraints $\omega_{t_1}$, and the component system $\alpha_{t_{0}}$.
The only mandatory property of this stanza is defined with the key \verb+version:+ whose value is a positive integer representing the version of the package.
The pair $\langle name, version \rangle$ of a package is unique, there can exist at most one package description for a given name and version.

Some relevant properties for a package are defined by the keys:
\begin{itemize}
  \item \verb+installed+: this Boolean property states whether the package is initially installed in the component system. 
  The CUDF document describes all packages, installed and not. 
  The set of packages with the property \verb+installed: true+ are included in the component system $\alpha_{t_{0}}$.
  \item \verb+keep+: the value of this property is mapped to a keep constraint, as defined in section \ref{formal.constraints}.
  Some possible values are:
	  	\begin{itemize}
	  		\item \verb+version+: keep this particular version of the package installed
	  		\item \verb+package+: keep at least one version of this package installed
		\end{itemize}
  \item \verb+provides+: this is a list of names of features or services that this package provides separated by the delimiter ``,''.
  e.g. \verb+provides: n, m+ means this package provides the features with names \verb+n+ and \verb+m+.
  Each feature name can be accompanied by a version, to state what particular version is provided, e.g. \verb+n=10+ means feature \verb+n+ version \verb+10+ is provided.
  When no version is specified for a feature, all versions of that feature are provided.
  \item \verb+depends+: the value of this property is mapped to a set of dependency constraints, as defined in section \ref{formal.constraints}.
  The value is a list of lists of \textbf{package formula}, first separated by the delimiter ``,'' then delimiter ``|''.
  It is a conjunction of disjunctions (conjunctive normal form), with each formula defining a set of packages.
  A package is defined using a name of either a package or feature $n$ and optionally an operator and version of the form  $n = v$, $n$ != $v$, $n > v$, $n < v$, $n >= v$, or $n <= v$.
  For example, \verb+depends: n , m | o < 10+ means the package depends on packages (or features) with ``name \verb+n+'',
   and  packages (or features) with ``(name \verb+m+ OR name \verb+o+ of version less than \verb+10+)''.  
  \item \verb+conflicts+: the value of this property is mapped to a set of conflict constraints, as defined in section \ref{formal.constraints}.
  The value is a list of \textbf{package formula} as defined above separated by the delimiter ``,''.
  For example, \verb+conflicts: n , m > 2+ means this package conflicts with all packages (or features) with ``name \verb+n+ AND name \verb+m+ with version greater than \verb+2+''.
  A caveat is that a component cannot conflict with itself.
\end{itemize}

An example of a package description stanza is given in figure \ref{formal.cudfpdstanza}.
\begin{figure}[htp] 
\begin{center}
\begin{alltt}
package: textEditor
version: 1
installed: true
depends: spellCheckerService
\end{alltt}
  \caption{Example of Package Description Stanza}
  \label{formal.cudfpdstanza}
\end{center}
\end{figure}

This package description stanza describes a package \texttt{textEditor} that is version \texttt{1}.
This package is installed in the initial system, and depends on a package or feature named \texttt{spellCheckerService}. 

\subsubsection{Preamble}
\label{cudf.preambledes}
The preamble stanza starts with the key \verb+preamble:+ (no value is necessary for this key).
The main function of the preamble is to provide the description of additional properties and their types, that can be used by package descriptions.
This is the mechanism which allows the extension the CUDF language.

Extension properties are described in the value to key \verb+property+.
This value is a list of properties that describe a name of the property, the type of the property and optionally the properties default value.
Each property is separated by the delimiter ``,'', the name is separated from the version by ``:'', and the default value is wrapped in ``['' and ``]'' and separated from the version by ``=''.

For example, \verb+property: size: int, bugs: int = [0]+ defines the integer property of \verb+size+, and the integer property of \verb+bugs+ with the default value \verb+0+.
An example where these extended properties are used in a package description stanza is shown in figure \ref{formal.cudfextensionexample}.

\begin{figure}[htp] 
\begin{center}
\begin{alltt}
preamble:
property: size: int, bugs: int = [0]

package: textEditor
version: 1
bugs: 10
size: 2
depends: spellCheckerService

\end{alltt}
  \caption{Example of preamble stanza demonstrating the extendable CUDF language}
  \label{formal.cudfextensionexample}
\end{center}
\end{figure}

\subsubsection{Request}
\label{formal.cudfdes.request}
The request stanza starts with the \verb+request:+ key (no value is required for this key).
This stanza is mapped to the user request $\delta_{t_1}$.
The request stanza contains three properties that define the user request to change the system.
Each value is a list of \textbf{package formula} separated by the delimiter ``,''.
The keys and values are:
\begin{itemize}
  \item \verb+install+ key, whose value is mapped to a set of installation requests, as described in section \ref{formal.constraints}.
  For example, \verb+install: n, m >= 2+ means install component with name \verb+n+, and to install component with name \verb+m+ and with version two or greater.
  \item \verb+remove+ key, whose value is mapped to a set of remove requests, as described in section \ref{formal.constraints}.
  For example, \verb+remove: n , m = 2+ means remove all components with name \verb+n+ and component \verb+m+ of version two.
  \item \verb+upgrade+ key, whose value is mapped to a set of upgrade requests, as described in section \ref{formal.constraints}.
  For example, \verb+upgrade: n+ means to remove all packages of name \verb+n+ and install one package of equal or greater versions to that of the maximum version of \verb+n+ currently installed. 
\end{itemize}


\begin{figure}[htp] 
\begin{center}
\begin{alltt}
request:
install: n > 2
remove: m
upgrade: o
\end{alltt}
  \caption{Example of CUDF request stanza}
  \label{formal.cudfrequeststanza}
\end{center}
\end{figure}

In figure \ref{formal.cudfrequeststanza}, an example of the request stanza is given.
In this example, a component of name \verb+n+ and with version greater than 2 is requested to be installed; all components of name \verb+m+ are requested to be removed;
and components of name \verb+o+ are requested to be upgraded.

%%%Upgrading with multiple packages
The CUDF request to upgrade a component is the most complex request due to the following requirements:
\begin{enumerate}
  \item exactly one component with  the selected name can be installed.
  \item exactly one component's version must be greater than, or equal to, the currently installed greatest version of a component with that name.
\end{enumerate}
For example, if a system has components $a_1 = \langle n,1 \rangle$ and $a_2 = \langle n,2 \rangle$ installed and it is requested to upgrade components with name $n$.
Component $a_1$ must be removed, and either $a_2$ can remain, or $a_2$ can be removed and a higher version of a component with name $n$ can be installed.

\subsection{CUDF Example}
\label{formal.example}
An example of a CUDF document is presented in figure \ref{formal.CUDFEXAMPLE},
it demonstrates how to define a set of components $\mathbb{C}_{t_1}$, a system $\alpha_{t_{0}}$, 
a set of system constraints $\omega_{t_1}$, 
and a user request $\delta_{t_1}$.

\begin{figure}[ht] 
\begin{center}
\begin{alltt}
preamble:
property: size: int = [0]

package: syslib
version: 1
installed: true

package: syslib
version: 2
conflicts: syslib

package: textEditor
version: 1
depends: spellChecker | spellCheckerService, syslib > 1

package spellChecker
version: 1
size: 1

package: tpspeller
version: 1
provides: spellCheckerService
size: 2

request:
install:textEditor
\end{alltt}
  \caption{Example of a CUDF document}
  \label{formal.CUDFEXAMPLE}
\end{center}
\end{figure} 



%%%Describe the preamble
The preamble from the example CUDF document presented defines the addition of one property \verb+size+ of type \verb+int+ whose default value is \verb+0+.
This extension allows any component in the document to define its \texttt{size}.

The set of components in the example CUDF document are defined from the package stanzas,
such that $\mathbb{C}_{t_1} = \{\langle \texttt{"syslib"},1 \rangle$, 
$\langle \texttt{"syslib"},2\rangle,$ 
$\langle \texttt{"textEditor"},1 \rangle,$ 
$\langle \texttt{"spellChecker"},1 \rangle,$ 
$\langle \texttt{"tpspeller"},1 \rangle\}$. 
For brevity's sake, these components are defined as:
$\langle \texttt{"syslib"},1 \rangle = syslib_1$, $\langle \texttt{"syslib"},2\rangle = syslib_2 $,  $\langle \texttt{"textEditor"},1 \rangle =  textEditor_1 $,  
$\langle \texttt{"spellChecker"},1 \rangle =  spellChecker_1 $,  $\langle \texttt{"tpspeller"},1 \rangle =  tpspeller_1 $.

The component system $\alpha_{t_{0}}$ includes all components whose \verb+installed+ property equals \verb+true+ (it defaults to \texttt{false}),
$\alpha_{t_{0}} = \{ syslib_1 \}$.

%%%Describe the depends constraints (including features and inequalities)
The system constraints $\omega_{t_1}$ can be extracted from the example CUDF document.
These are defined by the \texttt{depends} property of $textEditor_1$, and the \texttt{conflicts} property of $syslib_2$.

$textEditor_1$ depends on \texttt{spellChecker} or \texttt{spellCheckerService} AND \texttt{syslib} greater than version \texttt{1}.
The first dependency is on components that are named  \texttt{spellChecker} or \texttt{spellCheckerService},
or provide a feature named \texttt{spellChecker} or \texttt{spellCheckerService}.
Components that satisfy this criteria are $ spellChecker_1$ and $tpspeller_1$.
This results in the dependency constraint $textEditor_1 \rightarrow spellChecker_1 \vee tpspeller_1$.

The second dependency is on components that are named  \texttt{syslib}  greater than version \texttt{1},
or provide a feature named \texttt{syslib} with version greater than \texttt{1}.
The only component that satisfies this dependency is $syslib_2$,
resulting in the dependency constraint $textEditor_1 \rightarrow syslib_2$.

$syslib_2$ conflicts with components of name \texttt{syslib}, that is not itself.
The only component that satisfies this conflict is $syslib_1$,
resulting in the conflict constraint $syslib_2 \rightarrow \neg syslib_1$.

The set of system constraints $\omega_{t_1}$ in the example CUDF document is
$\{ textEditor_1 \rightarrow spellChecker_1 \vee tpspeller_1$,
$textEditor_1 \rightarrow syslib_2$, 
$syslib_2 \rightarrow \neg syslib_1\}$.


%%%Describe the request constraint
The user request $\delta_{t_1}$ is defined by the example CUDF documents request stanza.
The request is to install a package named \texttt{textEditor}.
The only component that satisfies this description is $textEditor_1$, which results in the install request $textEditor_1$.
Therefore, $\delta_{t_1} = \{textEditor_1\}$.

The evolution problem $\delta_{t_1} \cup \omega_{t_1}$ can be found by looking at the constraints defined by each package description and the request.
Firstly, the request constraints ensures that $textEditor_1$ must be installed.
If $textEditor_1$ is installed then the system library component $syslib_2$ must be installed, given $textEditor_1 \rightarrow syslib_2$.
As $syslib_2$ conflicts with $syslib_1$, $syslib_1$ must not be installed.
$spellChecker_1$ or $tpspeller_1$ must be installed, given $textEditor_1 \rightarrow spellChecker_1 \vee tpspeller_1$.

The three possible solutions to this evolution problem are;
\begin{itemize}
  \item $\alpha_{t_1}^1 = \{syslib_2, textEditor_1, spellChecker_1\}$,
  \item $\alpha_{t_1}^2 = \{syslib_2, textEditor_1, tpspeller_1\}$,
  \item $\alpha_{t_1}^3 = \{syslib_2, textEditor_1, spellChecker_1, tpspeller_1\}$
\end{itemize}

\subsection{Mancoosi Optimisation Format}
\label{formal.mancoosioptimisationformat}
%%%Describe the format by which mancoosi has required optimisation criteria formatted
MOF is a format defined by Mancoosi to express the lexicographic order of criteria.
The definition of what the criteria are is external to MOF.
A criterion is directly defined in \modelname and MOF is only used to define the lexicographic order. 

A MOF string is a list of lexicographically ordered criterion names separated by the delimiter ``\texttt{,}''.
Each criterion name is mapped to a criterion in \modelname.
A MOF string is therefore mapped to an evolution preference order by using the lexicographic composition of criteria defined in the string.

An example is presented to show how a MOF string can be used to create an evolution preference order.
This preference order is then applied to the example presented in section \ref{formal.example} to find the optimal solution.

A criterion is defined to minimise the change to a component system during evolution.
This criterion's name is \texttt{-changed} in MOF,
and is defined as $crit_{change} = \langle rank^{change}_{\alpha_{t_{0}}}, \leq \rangle$,
where $rank^{change}_{\alpha_{t_{0}}}(\beta) = |\alpha_{t_{0}} \Delta \beta|$.

Another criterion is defined to minimise the total size of the component system.
This criterion's name is \texttt{-size} in MOF,
and is defined as $crit_{size} = \langle rank^{size}_{\alpha_{t_{0}}}, \leq  \rangle$,
where $rank^{size}_{\alpha_{t_{0}}}(\beta) = \sum_{c \in \beta} c$.\verb+size+.
 
The MOF string \texttt{-change,-size} defines the lexicographically composed criteria $crit_{change} \oplus crit_{size}$ 
which equals $\langle rank^{L}_{\alpha_{t_{0}}}, \leq_L \rangle$ (as shown in definition \ref{formal.defcritlex}).
This criterion can be used to create an evolution preference order $\prec_{\alpha_{t_{0}}}$ (as described in definition \ref{formal.defcrittooptimal}).

In the previous example, the component system $\alpha_{t_{0}}$ was  $\{syslib_1\}$, and there were three possible solutions:
\begin{itemize}
  \item $\alpha_{t_1}^1 = \{syslib_2, textEditor_1, spellChecker_1\}$,
  \item $\alpha_{t_1}^2 = \{syslib_2, textEditor_1, tpspeller_1\}$,
  \item $\alpha_{t_1}^3 = \{syslib_2, textEditor_1, spellChecker_1, tpspeller_1\}$
\end{itemize}

This rank function $rank^{L}_{\alpha_{t_{0}}}$ applied to the possible solutions returns pairs of ($rank^{change}_{\alpha_{t_{0}}}$, $rank^{size}_{\alpha_{t_{0}}}$):
\begin{itemize}
  \item $rank^{L}_{\alpha_{t_{0}}}(\alpha_{t_1}^1 ) = r_1 = (4, 1)$, i.e. change is $4$ and size is $1$.
  \item $rank^{L}_{\alpha_{t_{0}}}(\alpha_{t_1}^2 ) = r_2 = (4, 2)$, i.e. change is $4$ and size is $2$.
  \item $rank^{L}_{\alpha_{t_{0}}}(\alpha_{t_1}^3 ) = r_3 = (5, 3)$, i.e. change is $5$ and size is $3$.
\end{itemize}
The order of these ranks over $\leq_L$ is then $r_3 \leq_L r_2 \leq_L r_1$.
The order of optimality ($\prec_{\alpha_{t_{0}}}$) over the possible solutions is $\alpha_{t_1}^3 \prec_{\alpha_{t_{0}}} \alpha_{t_1}^2$, 
$\alpha_{t_1}^3 \prec_{\alpha_{t_{0}}} \alpha_{t_1}^1$, and $\alpha_{t_1}^2 \prec_{\alpha_{t_{0}}} \alpha_{t_1}^1$.
This makes $\alpha_{t_1}^1$ an optimal solution, as no other solutions is greater w.r.t. $\prec_{\alpha_{t_{0}}}$. 
Therefore, given the CUDF document in figure \ref{formal.CUDFEXAMPLE} and MOF criteria \texttt{-change,-size} the resolved system is $\{syslib_2, textEditor_1, spellChecker_1\}$.
\subsection{CUDF*}
\label{formal.cudfstar}
CUDF* is an extension of both CUDF and MOF, created to define a single document that can be mapped to an instance of \modelname.
The differences between CUDF and CUDF* are:
\begin{itemize}
  \item In CUDF* each package description stanza must define the time that the component first existed.
  \item In CUDF* the preamble defines the time $t_0$. 
  \item In CUDF* there can be multiple requests stanza's, where each request stanza must define a time $t_i$, and MOF criteria string
  such that the request stanza maps to the request $\delta_{t_i}$ to $\prec_{\alpha_{t_{i-1}}}$.
  \item In CUDF* an \texttt{upgrade} request with value \texttt{*} attempts to upgrade all installed components.
\end{itemize}

By defining when each package first existed, the set of components $\mathbb{C}_{t_i}$ can be found for any arbitrary time $t_i$.
Additionally, the system constraints $\omega_{t_i}$ can also be extracted.

The overall structure of a CUDF* document where the preamble defines $t_0$, and there are multiple requests each defining a time is presented in figure \ref{formal.cudfstarstruct}.
\begin{figure}[htp]
\centering
\begin{tabular}{|c|}
\hline preamble$_{t_0}$ \\ \hline
\hline package description$_{1}$ \\ \hline
\hline package description$_{2}$ \\ \hline
 \multicolumn{1}{c}{\ldots}  \\ 
\hline package description$_{i}$\\ \hline
\hline request$_{t_1}$\\ \hline
\hline request$_{t_2}$\\ \hline
\multicolumn{1}{c}{\ldots}  \\
\hline request$_{t_j}$\\ \hline
\end{tabular}
\caption{Structure of the CUDF* stanzas}
\label{formal.cudfstarstruct}
\end{figure}

With these extensions, a CUDF* document is able to define the necessary elements in an instance of \modelname.
A full description of the process of parsing a CUDF* document to a \modelname instance can be found in appendix \ref{apx.cudf}.

\subsection{CUDF* Example}
\label{formal.cudfstarexample}
An example of a CUDF* document is presented in figure \ref{formal.CUDFstarEXAMPLE} that is similar to the example presented in \ref{formal.CUDFEXAMPLE}.
This example is designed to demonstrate the differences between CUDF, MOF and CUDF*.

\begin{figure}[htp] 
\begin{center}
\begin{alltt}
preamble: 100
property: size: int = [0]

package: syslib
version: 1
time: 50
installed: true

package: syslib
version: 2
time: 150
conflicts: syslib

package: textEditor
version: 1
time: 150
depends: spellChecker | spellCheckerService, syslib > 1

package spellChecker
version: 1
time: 150
size: 1

package: tpspeller
version: 1
time: 250
provides: spellCheckerService
size: 2

request: 200, -change,-size
install:textEditor

request: 300, -size,-change
install: textEditor, tpspeller

\end{alltt}
  \caption{Example of a CUDF* document}
  \label{formal.CUDFstarEXAMPLE}
\end{center}
\end{figure} 

Parsing this CUDF* document results in:
\begin{itemize}
  \item The initial component system $\alpha_{t_0}$ which is the set $\{syslib_1\}$ (the same from the previous example).
  \item The times are natural numbers, where $t_0$ is defined in the preamble as $t_0 = 100$, and $t_1$ and $t_2$ are defined in the requests as $t_1 = 200$ and $t_2 = 300$.
  \item The set of components $C_{t_i}$ is the set of components that existed at time $t_i$:
\begin{itemize}
  \item $C_{t_0} = \{syslib_1\}$
  \item $C_{t_1} = \{syslib_1, syslib_2, textEditor_1, spellChecker_1\}$
  \item $C_{t_2} = \{syslib_1, syslib_2, textEditor_1, spellChecker_1, tpspeller_1\}$
\end{itemize} 

\item The system constraints $\omega_{t_i}$ are constructed from the set of components $C_{t_i}$:
\begin{itemize}
  \item $\omega_{t_1}$ $=$ $\{ textEditor_1 \rightarrow spellChecker_1$, $textEditor_1 \rightarrow syslib_2$,  $syslib_2 \rightarrow \neg syslib_1\}$
  \item $\omega_{t_2}$ $=$ $\{textEditor_1 \rightarrow spellChecker_1 \vee tpspeller_1$,$textEditor_1 \rightarrow syslib_2$, $syslib_2 \rightarrow \neg syslib_1\}$.
\end{itemize}

\item The requests are installation requests:
\begin{itemize}
  \item $\delta_{t_1}$ $=$  $\{textEditor_1\}$
  \item $\delta_{t_2}$ $=$ $\{textEditor_1, tpspeller_1\}$.
\end{itemize}
\item The preference orders are extracted from the MOF criteria in each request.
These criteria are defined in section \ref{formal.mancoosioptimisationformat}..  
\begin{itemize}
  \item the MOF string \texttt{-change,-size} maps to  $crit_{change} \oplus crit_{size}$ which is used to define $\prec_{\alpha_{t_0}}$
  \item the MOF string \texttt{-size,-change} maps to  $crit_{size} \oplus crit_{change}$ which is used to define $\prec_{\alpha_{t_1}}$
\end{itemize}
\end{itemize}

Considering the evolution from $\alpha_{t_0}$ to $\alpha_{t_1}$,
the only component system that satisfies the evolution problem $\delta_{t_1} \cup \omega_{t_1}$
is $\{syslib_2, textEditor_1, spellChecker_1\}$.
Therefore, $\alpha_{t_1} = \{syslib_2, textEditor_1, spellChecker_1\}$.

Considering the evolution from $\alpha_{t_1}$ to $\alpha_{t_2}$,
there exist two component systems that satisfy the evolution problem $\delta_{t_2} \cup \omega_{t_2}$:
\begin{itemize}
  \item $s_1 = \{syslib_2, textEditor_1, tpspeller_1\}$,
  \item $s_2 = \{syslib_2, textEditor_1, spellChecker_1, tpspeller_1\}$
\end{itemize}

The evolution preference order $\prec_{\alpha_{t_1}}$ describes an order in which the minimum size of the system is the most significant attribute.
As the system $s_1$ has size $2$, and the system $s_2$ has size $3$, the optimal solution is $s_1$.
Therefore, $\alpha_{t_2} = \{syslib_2, textEditor_1, tpspeller_1\}$.

This CUDF* document therefore describes the evolution of a component system from 
$\{syslib_1\}$, to $\{syslib_2, textEditor_1, spellChecker_1\}$, then to $\{syslib_2, textEditor_1, tpspeller_1\}$.

This example demonstrates how a CUDF* document can be used to describe the evolution of a component system. 

\section{Summary}
This chapter described the \modelname model and the CUDF* language. 
These are used to describe the evolution of a component systems as a series of change requests made to the system by a user.
These changes are subject to constraints that must be satisfied in order to result in a valid system.
These describe the evolution of a system and not the user that requests changes to the system.
The following chapter defines the \usermodel model that describes the user, their system, and the changes they make.

