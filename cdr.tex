\chapter{Component Dependency Resolution}
\label{cdr}
%%%In this chapter we aim to first formally define CDR, discuss how its commonly implemented, finally giving examples of its use in different contexts

\section{Formal Definition}
%%%The formal definition of CDR is an expansion from our CBSE 2011 paper
First we denote the space of components as $\mathbb{C}$, each described by a name and version, such that no two components can have the same name and version.
We then define CDR as the function $C_R: 2^\mathbb{C} \rightarrow 2^\mathbb{C}$, 
such that it takes a set of user-requested components $X$ (an individual request is denoted $\Delta$),
and returns a set of components that satisfy all the constraints $R$.
For instance, given $X = \{a\}$ and a set of component constraints $R$,
$C_R(X)$ would try to resolve a set of components that contain $a$ while satisfying $R$.
The user-requested components must be in the system i.e. $X \subseteq C_R(X)$ as the user requirements are mandatory.
The returned set of components is seen as the resulting system of installed components, e.g. if $a \in C_R(X)$ then $a$ is installed.

The constraints in $R$ can be either: 
\begin{itemize}
  \item dependency requirements in the form $a \rightarrow c_1 \vee \ldots \vee c_n$ e.g. $a \rightarrow b \vee c$ means $a$ depends on $b$ or $c$
  \item conflict constraints in the form $a \rightarrow \neg c$, meaning $a$ conflicts with $c$
\end{itemize}
These constraints are defined semantically such that
a dependency constraint $a \rightarrow c_1 \vee \ldots \vee c_n$ means if $a \in C_R(X)$ then $c_1 \in C_R(X)$ or \ldots or $c_n \in C_R(X)$,
and a conflict constraint $a \rightarrow \neg c$ means if $a \in C_R(X)$ then $c \not \in C_R(X)$.
These types of constraints are expressive enough to describe many CDR problems.

%%%We extend this definition to include the possibility of optimisation
CDR is expressed w.r.t. a heuristic using a distance function $d(\mathbb{C}_1,\mathbb{C}_2)$ 
that returns a measure of distance between two sets of components as a non-negative real number.
Given a set of user-requests $X$, a system $C_R(X)$ and new user-request $\Delta$,
$C_R(X \cup \Delta) = \mathbb{C}_1$ iff there does not exist another system $\mathbb{C}_2$ such that $X \subset \mathbb{C}_2$, $\mathbb{C}_2$ satisfies $R$, 
and $d(C_R(X),\mathbb{C}_2) < d(C_R(X),\mathbb{C}_1)$.
That is, if two possible systems exist that $C_R$ could return, the system with the least distance from the original system is returned. 
By defining $d$ we can alter CDR w.r.t. our heuristics.

\section{Optimisation}
%%%Here we formalise notation of how we combine heuristics to create multi-criteria problem, this is a reduction (remove lattices, leave order theory) of our APSEC 2011 paper

\section{Common Upgradeability Description Format}
%%%Here we give an introduction to the CUDF format from Mancoosi, this format is used as a standard for Dependency Resolution problems

\subsection{CUDF Syntax}

\subsection{CUDF Semantics}




