\chapter{Component Dependency Resolution}
\label{cdr}
%%%This chapter stays abstracted over any implementation details, it aims to give motivation for CDR then define it formally.\\
%%%We start by giving a simplified 'real' example of a users requirements for a text editor, and the text editor requiring a spell checker.
Suppose a user wants to install a new text editor component ($c_t$) into their system.
The selected text editor has a dependency on a spell checker;
this means a spell checker component must be installed for the text editor to be functional.
There are three spell checker components available that satisfy this dependency ($c_1, c_2, c_3$),
this means that one of these spell checkers should be selected for installation as well.

%%%The problem becomes more complex when components, like the spell checkers, may require further components, and conflict with other components

%%%This problem is NP-Complete, as finding a solution is difficult but validating it is easy.

%%%CDR is more complex though, as there are typically many solutions that can be selected from, each with different properties, therefore it is an optimisation problem making it NP-Hard

In this chapter we give the formal defintion and notation of CDR while staying abstract. We finish by discussing the CUDF 
 
\section{Formal Definition}
%%%The formal definition of CDR is an expansion from our CBSE 2011 paper.
First we denote the space of components as $\mathbb{C}$, each described by a name and version, such that no two components can have the same name and version.
We then define CDR as the function $C_R: 2^\mathbb{C} \rightarrow 2^\mathbb{C}$, 
such that it takes a set of user-requested components $X$ (an individual request is denoted $\Delta$),
and returns a set of components that satisfy all the constraints $R$.
For instance, given $X = \{a\}$ and a set of component constraints $R$,
$C_R(X)$ would try to resolve a set of components that contain $a$ while satisfying $R$.
The user-requested components must be in the system i.e. $X \subseteq C_R(X)$ as the user requirements are mandatory.
The returned set of components is seen as the resulting system of installed components, e.g. if $a \in C_R(X)$ then $a$ is installed.

%%%The three constraints can be keep constraints, dependency requirements, or conflict constraints.//
The constraints in $R$ can be either: 
\begin{itemize}
  \item keep requirement in the form $a \vee \ldots \vee b$, e.g. $a \vee b$ means the $a$ or $b$ must be in the system
  \item dependency requirements in the form $a \rightarrow c_1 \vee \ldots \vee c_n$, e.g. $a \rightarrow b \vee c$ means $a$ depends on $b$ or $c$
  \item conflict constraints in the form $a \rightarrow \neg c$, meaning $a$ conflicts with $c$
  \item request constraints %TODO Explain the different request constraints that can be had
\end{itemize}
%%%Semantics of the constraints are then defined.
These constraints are defined semantically such that
a keep requirement  $a \vee \ldots \vee b$ means $a \in C_R(X)$ or $b \in C_R(X)$ (note: this can be unit i.e. $a$ is a valid keep constraint),
a dependency constraint $a \rightarrow c_1 \vee \ldots \vee c_n$ means if $a \in C_R(X)$ then $c_1 \in C_R(X)$ or \ldots or $c_n \in C_R(X)$,
and a conflict constraint $a \rightarrow \neg c$ means if $a \in C_R(X)$ then $c \not \in C_R(X)$.
These types of constraints are expressive enough to describe many CDR problems.

%%%We extend this definition of CDR to include optimisation through some measurement of improvement over the previous system.\\
CDR is expressed w.r.t. a heuristic using a distance function $d(\mathbb{C}_1,\mathbb{C}_2)$ 
that returns a measure of distance between two sets of components as a non-negative real number.
Given a set of user-requests $X$, a system $C_R(X)$ and new user-request $\Delta$,
$C_R(X \cup \Delta) = \mathbb{C}_1$ iff there does not exist another system $\mathbb{C}_2$ such that $X \subset \mathbb{C}_2$, $\mathbb{C}_2$ satisfies $R$, 
and $d(C_R(X),\mathbb{C}_2) < d(C_R(X),\mathbb{C}_1)$.
That is, if two possible systems exist that $C_R$ could return, the system with the least distance from the original system is returned. 
By defining $d$ we can alter CDR w.r.t. our heuristics.

%%%This optimisation function can be simplified to a function where the previous solution is not compared, but some static metric of the system is measured.

\subsection{NP-Complete Proof}
%%%Here we reduce the problem to SAT, which is simple as the constraints are very similar.

%%%We show that it is not a 'simple' SAT problem, due to conflict constraints, therefore it is NP-Complete.

%%%This proof has been published before; list of papers.

\section{Optimisation}
%%%This section is based on our APSEC 2011 paper, though we expand the ranking function to include an earlier system (dynamic).\\
%%%CDR can return potentially enormous amount of solutions, each has different properties that may make it more or less desirable than another, we need criteria to find an optimal solution.
Any given CDR problem can be resolved to a potentially enormous number of solutions \cite{Jenson2010},
and as each solution has different properties, suitable criteria should be included in the selection of an optimal solution.

%%%Information from differnet loactions and meta data can be used to define the criteria.
Information from component meta-data, component contracts \cite{Watkins1999}, repositories \cite{Guo2000}, 
composition testing \cite{XuejieZhang2008} and user preferences can be used to calculate these optimal solutions.
Additional criteria, such as size, reliability or licence, may change a solutions' desirability, 
therefore optimising for them will help return a better solution.

%%%This problem is multi-criteria as users will likely have more than one criteria to judge a solution by.

\subsection{Multi-Criteria CDR}
%%%Combining criteira is a challenge because criteria have different domains and orders, some have partial orders, arbitrary rankings, criterion orderings.
Optimising additional criteria proposes some challenges; firstly criteria may have different domains and orders, 
e.g., minimising the size in bytes while maximising percentages of reliability. 
Some criteria may have partial orders where two solutions can be incomparable,
e.g., one licence may not be better or worse than another.
Criteria may have arbitrary rankings, 
e.g., preferences for licences can differ significantly between users.
A criterion could also be more important than another, 
e.g., correct functionality of a system will always be more important than its size.

%%%A naive solution is to map all criteria to real numbers, then aggregate them through addition multiplication with weights.
A naive solution for this problem is to create a function to map each of the criteria to a real number.
Then by aggregating these numbers (through addition or multiplication) with user defined weightings 
(to express their preferences on properties), we create a single fitness function, 
and using this function we can search for optimal solutions.

%%%This has significant draw backs, some criteria are infinitly more important, it is not intuitive for users to weight.
This method has some significant drawbacks;
we lose some information in the conversion to the totally ordered real numbers,
making aspects like Pareto fronts \cite{pareto1964cours} (where multiple solutions are incomparable) impossible to represent.
The relationships between criteria are also difficult to define 
as complicated functions must be created to maintain proper ordering of solutions.
Asking users to assign preferences of criteria with arbitrary weights is not intuitive,
e.g., stating ``size is $.25$ more important than vendor'' does not convey meaningful information to the user about their weighting decision.

%%%We approach this problem formally using orders over partially ordered sets, where any sub set is bounded, this is a complete lattice.
The formal approach to solve this problem is to use orders over sets, 
where each criterion is a set ($A$) with an order ($\leq$).
Each criterion is then a partially ordered set (the order must be reflexive, antisymmetric and transitive).
Furthermore we require that any set of elements in a criterion must be bounded, 
i.e. have a least upper bound and a greatest lower bound.
A complete lattices satisfies these requirements, therefore is used as our representation. 

%%%We can then compose these complete lattice criteria using the product and lexicographic orders.

%%%This framework can then be used to describe and discuss the multicritera problem of CDR.
This framework defines criteria as a ranking system over a complete lattice.
The ranking system describes a mechanism in which to rank a solution to a complete lattice. 
The complete lattice gives an order to the solutions based on their ranking.
By combining these criteria (ranking systems and complete lattices) multi-criteria problems can be composed.
Then using a general algorithm that can search these structures, 
optimal solutions for any defined problem can be found.

\subsection{Complete Lattice}
%%%Complete Lattices have been used in other optimisation frameworks.
Complete lattices are an effective abstraction to represent rankings,
and have been used before when describing such criteria \cite{Bistarelli1997,Fernandez}.
They can represent problems that involve partial orders,
be combined into new lattices,
and be easily visualised and presented using Hasse diagrams \cite{davey1990introduction}.
They also have strict bounding, which enables general methods, such as branch and bound \cite{Land1960}, to find solutions.

%%%The formal definition for comlpete lattices is a partially ordered set where all subsets have a unique least upper bound and unique greatest lower bound
A complete lattice is a partially ordered set where all subsets have a unique least upper bound and unique greatest lower bound.
Formally we define:
\begin{defs}
A complete lattice is a pair $L = \langle A, \leq \rangle$ where $A$ is a non empty set 
and $\leq \subseteq A \times A$ is a relation such that:
\begin{itemize}
  \item $\leq$ is a partial order
  \item for all subsets of $A$ there exists a unique least upper bound (join or $\vee$) and a unique greatest lower bound (meet or $\wedge$)
\end{itemize}
\end{defs}

%%%The domiain of the complete lattice represents the value domain of the criteria.
The domain of the complete lattice ($A$) defines the value space (e.g., probability, cost, \ldots) 
that is used to rank solutions. 
The core attribute of a complete lattice is that all subsets of the lattice have both a unique least upper bound (join or $\vee$) 
and a unique greatest lower bound (meet or $\wedge$).
Using these operators we can define the greatest ($1$) and least ($0$) elements as
the join of the entire domain ($\bigvee A = 1$) and the meet of the domain ($\bigwedge A = 0$).

\subsection{Ranking and Solution}
%%%We define a ranking system and function, which returns a value for the complete lattice domain.
An optimal solution can now be defined through a ranking system with respect to complete lattices.

%%%A ranking system is a lattice a finite domain of variables, and their assignment domain
\begin{defs}
Consider a complete lattice $L = \langle A, \leq \rangle$; we define 
a ranking system over this lattice as a triple $RS = \langle L, D, V \rangle$, 
where $V$ is a finite set of variables and $D$ is their finite domain.
\end{defs}

%%%Given we specifically look at components, the variables are components and their domain is \{T,F\}, whether they are installed or not
Given we are focused on the component dependency resolution problem,
this abstract definition of a ranking system can be further defined.
Within dependency resolution, 
each variable in this problem is a component and their domain is whether they are included or excluded from the final solution.
Therefore, $D = \{F,T\}$ and $V = C$, where $C$ is the set of components in the problem.

%%%A solution (system) is a mapping of variables to the domain
\begin{defs}
A solution is a variable binding $\gamma: V \rightarrow D$, a mapping of each variable to an element in the domain,
and $\Gamma$ is the set of all solution functions.
\end{defs}

This definition of solution does not take order into account, where a solution could be the worst possible solution.

%%%The ranking function, takes two solutions, and returns the distance between them 
%TODO increase the ranking function to take two solutions and return a distance between them
\begin{defs}
Consider a complete lattice $L = \langle A, \leq \rangle$ and a ranking system $RS = \langle L, D, V \rangle$.
A ranking function ($Rank$) maps the distance from the current solution and a new solution to elements from $A$,
$Rank: \Gamma \times \Gamma \rightarrow A$.
\end{defs}

%%%The set of optimal solutions is then the set of solutions where there are no better solutions
\begin{defs}
Consider a complete lattice $L = \langle A, \leq \rangle$, a ranking system $RS = \langle L, D, V \rangle$, a previous solution $\gamma$ and a ranking function $Rank$.
The set of optimal solutions ($OSol \subseteq \Gamma$) is the set of solutions where there is no other solution whose rank is better,
such that $OSol = \{ s \in \Gamma : \nexists s' \mbox{ such that } Rank(\gamma,s) < Rank(\gamma,s')\}$
\end{defs}

%%%The set of optimial solutions may still be undesirable, as it is only solutions that are available
The set of optimal solutions only includes the highest ranked solutions,
as with the definition of solutions, these optimal solutions could be undesirable.
For instance given a problem which is ranked with either $Correct$ or $Incorrect$, 
the optimal solution could be $Incorrect$ if there exists no solutions that are $Correct$. 

\subsection{Composition of Complete Lattices}
%%%This section is mostly identicle to the section in our APSEC paper, it just proves that combining two complete lattices with lexicographic or product order returns a complete lattice
\label{sec:composition}
As a complete lattice with a ranking system is the representation of a criterion,
the composition of criteria is the way of expressing their relationships to one another. 
We first define two separate composition operators of complete lattices resulting in the product and lexicographic orders.

Here we prove that the lexicographical and product compositions of complete lattices result in a complete lattice,
this enables any combination of these orders to be used and remain within our framework.
This is proven by deriving the join (unique least upper bound) and meet (unique greatest lower bound) operations,
and showing there exists a least element ($0$) within the lattice \cite{davey1990introduction}. 

We then define how the ranking systems are composed with these orderings.

\begin{defs}
\label{def:product_order}
Consider two complete lattices $L_1 = \langle A_1, \leq_1 \rangle$ and $L_2 = \langle A_2, \leq_2 \rangle$;
$L_1 \otimes L_2$ creates a product composition $L_P = \langle A_1 \times A_2, \leq_P \rangle$
where $\leq_P$ is a product order; 
$(a,b) \leq_{P} (a',b') \mbox{ iff } a \leq_{1} a' \mbox{ and } b \leq_{2} b'$.
\end{defs}

The product order of a classical lattice $L = \langle \{T,F\}, \leq \rangle$ 
composed with its self is given in the diagram in Figure \ref{fig:product}.
%\begin{figure}[htp]
%\begin{center}
%  \includegraphics[height=.1\textheight]{productorder.png}
%  \caption{Hasse diagram of the product composition of classical lattices}
%  \label{fig:product}
%\end{center}
%\end{figure}

The product order is trivially reflexive, anti-symmetric and transitive, therefore it is a partial order.
\begin{prop}
The product composition of two complete lattices is a complete lattice.  
\end{prop}
\begin{prof}
For any two points $(a,b)$ and $(a',b')$;
their join is the join of each element $(a \vee a', b \vee b')$, 
inversely their meet is the meet of each element $(a \wedge a', b \wedge b')$.
This lattice is also complete as its least bound is the least bound of each set $(0_1,0_2)$.
$\Box$\end{prof}

\begin{defs}
\label{def:lex_order}
Consider two complete lattices $L_1 = \langle A_1, \leq_1 \rangle$ and $L_2 = \langle A_2, \leq_2 \rangle$;
$L_1 \odot L_2$ creates a lexicographical composition  $L_L = \langle A_1 \times A_2, \leq_L \rangle$
where $\leq_L$ is a lexicographic order;
$(a,b) \leq_{L} (a',b')$ iff $a <_{1} a'$ or $(a = a'$ and $b \leq_{2} b')$.
\end{defs}

The lexicographic order of a classical lattice $L = \langle \{T,F\}, \leq \rangle$ 
composed with its self is given in the diagram in Figure \ref{fig:lex}.
%\begin{figure}[htp]
%\begin{center}
%  \includegraphics[height=.1\textheight]{lexicograhicorder.png}
%  \caption{Hasse diagram of the lexicographical composition of classical lattices}
%  \label{fig:lex}
%\end{center}
%\end{figure}

The lexicographical order is trivially reflexive, anti-symmetric and transitive, therefore it is a partial order.
\begin{prop}
The lexicographical composition of two complete lattices is a complete lattice.
\end{prop}
\begin{prof}
Consider two points $(a,b)$ and $(a',b')$; 
the join of $(a,b)$ and $(a',b')$ is $(a',b')$ if $a <_1 a'$,
$(a,b)$ if $a' <_1 a$,
$(a, b \vee b')$ if $a = a'$,
or if $a$ and $a'$ are incomparable ($a||a'$) it is $(a \vee a', 0_2)$
as this is an upper bound, and any other upper bound must be bigger than this.
The meet is defined dually where the meet of $(a,b)$ and $(a',b')$ is $(a',b')$ if $a' <_1 a$,
$(a,b)$ if $a <_1 a'$,
$(a, b \wedge b')$ if $a = a'$,
or if $a$ and $a'$ are incomparable ($a||a'$) it is $(a \wedge a',1_2)$
as this is a lower bound, and any other lower bound must be smaller than this.

This lattice is also complete as its least bound is the least bound of each set $(0_1,0_2)$.
$\Box$\end{prof}

There are two things to note about the lexicographical composition; 
first it does not create a distributive lattice i.e. $\wedge_L$ does not distribute over $\vee_L$.
\begin{prop}
The lexicographic composition does not result in a distributive lattice.
\end{prop}
\begin{prof}
Consider two pairs $(a,b)$ and $(a', b') \in L_1 \odot L_2$ where $a || a'$ and $a = a''$ and $a' || a''$,
then $(a,b) \wedge_{L} ((a', b') \vee_{L} (a'', b'')) = (a,b)$,
where under the same conditions $((a,b) \wedge_{L} (a', b')) \vee_{L} ((a,b) \wedge_{L} (a'', b'')) = (a,b \wedge_2 b'')$.
$\Box$\end{prof}

Secondly, two lattices (not complete lattices) lexicographically combined together may not be a lattice. 
This is another reason for the choice of complete lattices as the structure of representation.
\begin{prop}
The lexicographic composition of two lattices may not result in a lattice.
\end{prop}
\begin{prof}
Suppose a lattice $K$ has a non linear order, and a lattice $L$ has no least bound ($0_L$). 
For two pairs $(a,b)$ and $(a',b)$ in $K \odot L$ have no join if $a$ and $a'$ are incomparable.
This would be defined as $(a,b) \vee (a',b) = (a \vee a',0_L)$, as $0_L$ does not exist this join does not exist.
$\Box$\end{prof}

In figure \ref{fig:stacking} is presented some examples of how our composition operators can be stacked
by using a classical lattice ($L = \langle \{T,F\}, \leq \rangle$) composed in different ways.
In these examples we have simplified the domain for brevity, 
e.g., $\langle T, \langle T,T\rangle\rangle$ is represented with $\langle T,T,T\rangle$
%\begin{figure}[htp]
%\begin{center}
%  \includegraphics[width=.45\textwidth]{all.png}
%  \caption{Hasse diagrams of the composition of a classical complete lattice}
%  \label{fig:stacking}
%\end{center}
%\end{figure}

The composition of the ranking systems is defined as:
\begin{defs}
Consider two complete lattices $L_1 = \langle A_1, \leq_1 \rangle$ and $L_2 = \langle A_2,\leq_2 \rangle$ and
the ranking systems over them $RS_1 = \langle L_1,D,V \rangle $ and $RS_2 = \langle L_2,D,V \rangle$,
with defined ranking functions as $Rank_1: \Gamma \rightarrow A_1 $ and $Rank_2: \Gamma \rightarrow A_2$.
The combination $RS_1 \odot RS_2$ (resp. $RS_1 \otimes RS_2$) results in the ranking system 
$RS = \langle L_1 \odot L_2, D, V \rangle$ (resp. $RS = \langle L_1 \otimes L_2, D, V \rangle$), 
and the combination of the ranking functions is 
$Rank: \Gamma \rightarrow A_1 \times A_2$ where $Rank(\gamma) = \langle Rank_1(\gamma), Rank_2(\gamma) \rangle$.
\end{defs}

The main point in this definition is that to combine ranking systems they must have the same set of variables 
and the same domain.

\subsection{Semi-ring constraint incompatibility}
%%%Discussion of the related yet incompatible work C-semirings; how they inspired this work but due to lexicographic order we are unable to use it.
Instances of different complete lattices used for constraint solving can be found in a similar framework based on c-semirings, 
by Bistarelli et al. \cite{Bistarelli1997}.
They prove that all instances of their framework are complete lattices, and their $+$ operator coincides with join
and if their $\times$ operator is idempotent, it coincides with meet.
Their examples include fuzzy, weighted, probabilistic and valued constraint problems which can be used in our framework.
In their framework $\times$ must distribute over $+$, this gives a distributive lattice,
since lexicographic orders may create non-distributive lattices, our overall frameworks are incompatible.
The c-semiring framework is still greatly useful as c-semirings are complete lattices all instances of this framework will work in ours.



\section{Common Upgradeability Description Format}
\label{form.cudf}

%%%Here we give an introduction to the CUDF format from Mancoosi, this format is used as a standard for Dependency Resolution problems
{}\cite{treinen2009common}

%%%It was made to standardise the problem, for two reasons, to make building solvers generalisable, and to compete solvers in the MISC competetions to compare solutions.

%%%In both regards it has succeded, there have been many competitions run, 
%%%and their creation of a Modular Package Manager(MPM) is currently moving through the Debian stages into their core repository

\subsection{CUDF Syntax}
%%%The CUDF syntax is made to resemble similar CDR meta-data through using key-value pairs (not XML, too verbose?)

%%%Package Formulas are the standard mechanism of dependence

%%%Versions as Integers

%%%The Syntax has been defined to be extensible within the language, so additional model specific information can be included without increasing the size of the standard

\subsection{CUDF Semantics}
%%%As CUDF was a format the was supposed to be open to allow people to build differnt solvers for it, and be used as a real package manager, it walked a fine line.

%%%It must be simple enough so that developers can quickly understand it to build or modify existing solvers capable of competing in MISC competitions

%%%It must be broad enough to define and solve real world problems



