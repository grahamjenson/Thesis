%=====================================================================================
%  User-defined Latex Commands
%=====================================================================================

% NATBIB Bibliography Style -- better names for harvard extended styles
%\newcommand {\citen}         [1]{\citet{#1}} % as a noun eg. Blogs (2000)
%\newcommand {\citeaffixed}   [1]{\citet{#1}}
%\newcommand {\possessivecite}[1]{\citet{#1}}
%\newcommand {\urlX}[1]{\textit{#1}}          % \url doesn't like to be movable

%%----------------------------------------------------------------------------------------

\newtheorem{defs}{Definition}
\newtheorem{prof}{Proof}
\newtheorem{prop}{Proposition}



% put a todo item in brackets, like   (todo: expand detail here )
\newcommand {\note}[1] {\todo[inline]{#1}}

% Put comments in a margin box
\newcommand {\thought}[1]{\marginpar{#1}}
\newcommand {\aside}[1]  {\marginpar{\tiny{#1}}}
\newcommand {\idea}[1]   {\marginpar{#1}}

% Hide some text
\newcommand {\hidden }[1]{}
\newcommand {\hide }  [1]{}

\newcommand {\product}[1]{\textit{#1}\texttrademark%
                          \index{#1@\textit{#1}}%
            }
\newcommand {\fullref}[1]{\ref{#1},~p\pageref{#1}}   % -->  4.5, p29
\newcommand {\fullrefi}[1]{\ref{#1}~(p\pageref{#1})} % inline -->  4.5 (p29)
\newcommand {\fullrefp}[1]{(sec. \fullref{#1})}      % add parentheses and sec. -> (sec. 4.5, p29)

\newcommand {\skipline} {\newline ~ \newline}
\newcommand {\gap}      {\newline ~ \newline}

\newcommand {\tab} {~~~}

\newcommand {\naive} {na\"{i}ve }
\newcommand {\degrees} {$^\circ$ }            % degree symbol

\newcommand {\LLLc} {Lifelong Learning }
\newcommand {\LLLs} {lifelong learning }
\newcommand {\LLLsn} {lifelong learning}

\newcommand {\ep} {ePortfolio}

% Registered (R) in a circle
\def\registered{{\ooalign{\hfil\raise.05ex\hbox{\tiny{R}}\hfil\crcr\mathhexbox20D}}}

% TEXT FORMATTING SHORTCUTS

\newcommand {\se}[1] {\small{\emph{#1}}}

% put a quotation in its own unindented paragraph
\newcommand {\inlinequote}[1] {\emph{#1}}

%%==========================================================================================
\newcommand {\shortquote}[1] {
   \noindent \begin{quote}
     \emph{#1}
   \end{quote}}

% Indented text, slightly smaller than usual, and italicised.
\newcommand {\smallquote}[1] {\begin{quote}\noindent\small{\textit{#1}} \end{quote}}

%%===========================================+=======================
\newcommand {\longquote}[1] {
   \noindent \begin{quotation}
     \emph{#1}
   \end{quotation}}
