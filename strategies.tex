\chapter{Strategies}
\label{strategies}
%%%An \textbf{Evolution Strategy} is the plan of action employed by a user through component resolution to evolve a system to attain the systems objectives

%%%The two core criteria a user must consider when evolving a component system (after the neccessity of having the systems constraints resolved)
%%%are that the system should be up-to date while changing as little as possible
The two core criteria considered during component system evolution are the maximisation of the versions of individual components, 
and the minimisation of the change caused by the evolution to the system.
Together these two criteria express that a system wants to be as up to date as possible, while changing as little as possible.
Although these criteria must be considered in a CDR algorithm,
they are not strictly defined, conflicting and for different users relate in different ways.
In this chapter we explore the nature and definitions of the two criteria,
and show that there is a significant gap between what is currently known and what we intend to explore. 

%%%These constraints are clearly conflicting, causing a pareto front of possible solutions
The first step toward finding optimal solutions is to define the criteria.
The first criterion of a valid component solution is whether it has all its dependencies resolved,
this can be tested by extracting information about dependencies from component meta-data.
Suppose the user has also defined their preference for free and open source components;
this is the second criterion where the licences of the components can be checked to match with this preference.
Further suppose the system is in a resource constrained environment where size matters, 
this motivates the final criterion to minimise solution size.

%%%They are also ill/un defined criteria and relationships. 
%%%This leaves a space open for exploration and experimentation of their defintion and relationships.


\section{Progressive vs. Conservative}
\begin{quotation}
Motion or change, and identity or rest, are the first and second secrets of nature: Motion and Rest. 
The whole code of her laws may be written on the thumbnail, or the signet of a ring.

Ralph Waldo Emerson in "Nature", Essays, Second Series (1844)
\end{quotation}

%%%The force and resistance to change is a natural problem, here we use the metaphor of politics to show this.
Changing to what is newer and better, but being wary of the change because of cost and risk is a conflicting problem in many domains.
For instance, in politics, conservitism is the philosophy that emphasises minimal and gradual change in society, %TODO cite
where progressivism promotes change and reform to governments.
These two conflicting ideaologies must be resolved, if a government is to function. 
As with component systems, 
the forces of change brought about by newer versions of components and the resistance to change brought about by the harm it may cause
are competing forces that must be resolved for a system to evolve.
In this section we discuss the nature of these criteria, why they are important to consider and how they conflict.

%%%Both forces are caused because of risk, risk of change introducing new problems, risk of not changing having old problems linger

\subsection{Updating vs. Minimal Change}
%%%These forces must also be resolved in software as well

%%%The reasons for keeping a system up-to-date; the fixing of bugs security, the adding of functionality, 

%%%The reasons for minimal change; through change unknown bugs could be introduced, the act of changing can create bugs, the effort of changing can consume resources  (e.g. network traffic).

%%%Problems with defining version based or change based criteria

%%%How Different users define and see these criteria (e.g. server admins like minimal change, developers like cutting edge)

They are also not strictly defined as the mechanisms to either compare two systems versions or the change from a system,
can be measured in different ways.

The measure to use when determining version of a system from the versions of installed components varies.
As a set of components can be versioned differently, 
we need to explore functions that can aggregate the versions of different components together in a meaningful way.

The Mancoosi organisation uses a metric that minimises the number of components in a system that have better versions, 
e.g. a system $a-1,b-1$ if $a-2$ exists is $1$ out of date. 
The Eclipse P2 implementation counts the amount of versions that are better for each component,
e.g. a system $a-1,b-1$ if $a-2$ and $a-3$ exist, is $2$ out of date.
Both of these take into account only the component and not the components that depend on it.

The measurement of change that a system goes through during evolution is also difficult to define.
The added, removed, updated, total changed, have all been considered as metrics before. 
As with the version metrics, none of these consider the dependencies when looking at the change that is performed on the system.

\subsubsection{Update Criteria}
{}In the previous chapter two version dependent criteria are presented and defined, one based on the Eclipse P2 concpet to minimise the ``uptodate distance'', 
{}and the second based on Mancoosi version to minimise the amount of packages that are notuptodate. 
{}The implications of using these criteria to update a system are discussed with relation to the order of the criteria, the average change in a system and the comparison of these values.

The order consequences of the order of the crtieria when updating are important given the semantics of the CUDF standard.
The key points when selecting to update a package in a CUDF standard is that there can only be one package and it must be of equal or greater version than the highest installed current version.
This means that if a package has only one installed package then it does not need to change to be valid in the returned solution.
Therefore, the returned system for any update command can be the initial system.
This means that if a criteria, such as changed, or hamming where minimised as the most important  

Dangers of Update Distance, as the distance for an uptodate components is 0, installing a new non upto date component costs nothing. If we set it to 1, then it will minimise solution size.
\subsection{The Relationship between Criteria}
%%%There are extremist views on both sides of these criteria; server admin will change a system reluctantly where developers will constantly stay up-to date.

%%%The actual relationships between these criteria though are unknown, thus the reasons for either view is based on intuition or assumption and not fact. 

%%%How fast will a system fall behind if a user never upgrades, or how much risk is introduced by continually upgrading?

%%%These questions can be answered by looking at the dependencies and relationships between criteria.

Each time a user decides to evolve a component system, the decision must be made about the risks of the evolution.
In an environment which is mission critical, all risk is eliminated and an unecessary change to the system is too risky.
In a development environment where the user may be trying to fix potential problem, 
or test different packages, then the risk is accepted as the system is essentially disposable, and a complete re-installation is not out of the question.
These two strategies are represented in by the current depenceny resolution implementations, however very little middle ground is available for the customisation of applications.
This was noted in the paper %TODO cite mancoosi CBSE 2011 paper


\subsection{Efficient Ordering of Criteria}
%%%Conflicting criteria should be lexicographically last, as they have the hardest time optimising if it times out it will still have a pretty good solution

%%%The last lex criteria should always be weighted such that it is availble for each component. The first half of the strategy guides the solution towards what is wanted.
%%%An exmaple is we want to maximise pagerank in our solutions, +sum(pagerank) will work, but the optimal solution would then be to install as many components as possible.
%%%So we minimise change first -change,+sum(pagerank), this will install as many components that have the same name. 
%%%-changed,-newc,+sum(pagerank) will minimise change from original system, minimise the amount of new components that can be added, and then maximise pagerank, 
%%%so only installing new components with the largest pagerank.

%%%Criteria that group components are more efficient as they have smaller constraints, criteria should be ordered lexicographically as efficiently as possible.

\section{Criteria}
%%%Here we specifically define different criteria/heuristics from academia (Mancoosi, Lehman), industry (JDepend, P2, aptitude, apt-get), and our own PageRank, predictive change,
%%%All defined formally, and ready for use for the simulation in the next chapter

\subsection{Simple}
%%%Hamming
The \textbf{Hamming} distance \cite{hamming1950error} is used here to define minimality of component systems.
This distance is defined as the size of the symmetric difference of two sets $C_1,C_2$, 
i.e. $d_H(C_1,C_2) = |C_1 \ominus C_2|$.
This is a greedy search for minimality, 
as selecting the minimal system based only on the component changes from the current system may result in more changes over the long term.


\subsection{Mancoosi Criteria}
%%%Here we define the criteria from Mancoosi

%%%not up to date

%%%removed

%%%new

%%%Reccommends

%%%Trendy which is lex(removed, not up to date, reccommends, new)

The \textbf{Paranoid} distance function, 
defined by the Mancoosi (\textit{MANaging the COmplexity of the Open Source Infrastructure}) project\footnote{http://www.mancoosi.org/}, 
is the lexicographic composition of two distance functions $removed$ and $changed$.
It is defined such that each component has a name $C \rightarrow Name$,
where the function $V: 2^C \times Name \rightarrow 2^C$ returns all components with a name in a set of components 
(these are the different versions of a component).
They are defined such that $removed(C_1,C_2)$ $=$ $|\{name \mid V(C_1,name) $ is nonempty and $V(C_2,name)$ is empty $\}|$,
and $changed(C_1,C_2)$ $=$ $|\{name \mid V(C_1,name)$ is different to $V(C_2,name)\}|$.
The removed function therefore only considers when all components of a name are removed. 

The distance function is then defined $d_P(C_1,C_2)$ to return a pair $(a,b)$ such that $a = removed(C_1,C_2)$ and $b = changed(C_1,C_2)$.
The distance order $\leq_{P}$ is lexicographically defined with respect to $C_R(X)$ as given two models $C_1,C_2$,
$(a,b)$ equals $d_P(C_R(X),C_1)$ and $(a',b')$ equals $d_P(C_R(X),C_2)$ then
$C_1 \leq_{P} C_2 $ iff $a < a' \vee (a = a' \wedge b \leq b')$.
This definition states that removing a component is infinitely worse than adding one, 
therefore we should expect this to generate larger solutions.

\subsection{Lehman}
%%%Lehman has outlined laws of evolution though empirical research, we can adapt these laws to see if they hold true and to what degree in component systems

\subsection{JDepend}
%%%Coupling 

%%%Instability

\subsection{P2}
%%%Looking at the criteria from the P2 solver
Eclipse \textbf{P2} \cite{le_berre_dependency_2009,leBerre2010} is the provisioning system for the Eclipse IDE platform.
Its optimisation function minimises the removed components, while maximising the version of installed components, 
we denote it's distance function as $d_{P2}$.
This function is fully described by Le Berre and Parrin in \cite{leBerre2010}.
The goal of combining these two criteria is to simultaneously update components while altering the system.
This strategy can be seen as a preemptive change, so that a user will not need to later update currently installed components,
therefore minimising future change.
As this is the only minimal criteria that includes the goal to maximise the currently installed version 
we expect more change during evolution.

\subsection{Propagation Based Criteria}
%%%We can create more informative criteria by analyising progation effects.\\
{}A components dependence on another component creates a tangible connection that should be considered when evolving a system.
{}This connection can be seen as a form of trust, given a developer selects the components to have their components depend on,
{}or as a measure of importance to a system, as a component that is highly depended on has more responsibility to the system.
{}This should be considered when making the choice to either remove or add a component to a system.
{}By analysing dependencies between components we can give a weight to a component-based on the structure of its relationships.

%%%Page rank as a metric, calculate page rank value for all 
The structure of component dependencies can be abstracted to a graph, 
and by analysing the importance of nodes in this graph using the \textbf{PageRank} with priors function \cite{White:2003:AER:956750.956782}, 
we attempt to quantify the value of a component in a system.
PageRank with priors estimates the probability of stopping on a node in a random walk over a directed graph given a set of starting nodes.
We see this as an approximation to the probability that a component is depended on in a system.
A directed graph $G = (V,E)$ is generated from the
dependency rules such that $V$ is the set of components, and $(a,b) \in E$ iff there is a dependency constraint in $R$ 
where $a \rightarrow c_1 \vee \ldots \vee c_n$ where $b = c_i$ for some $i$.
The function $PR(a)$ denotes the PageRank of $a$,
so given the set $C_R(X) \cup \{\Delta\}$ is the starting set of nodes, and the generated graph $G$,
the distance function $d_{PR}(C_1,C_2) = \sum_{a \in (C_1 \ominus C_2)} PR(a)$,
i.e. the distance is the sum of the PageRank values in the symmetric difference of the systems.
This distance is the only one to consider the components relationships when calculating the effect of removing or adding them.
Our hope is that by minimising the impact on the structure of the system, 
this will minimise the necessary changes over a long period of time.

%%%HITS, where we can identify hubs and authorities of packages and use this as criteria

%%%Propagation of positive and negative, changin a package that has very little depending on it should be better than changing a large package

\section{Optimisation}
%%%What can be optimised? Here we go over these criteria and discuss given the limitation that PBO imposes on optimisation, how we can optimise.

\section{Other Criteria}
{}There are many other types of criteria that can be considered when judging the optimality of a solution.
{}As we are trying to stay component model agnostic, the only information that can be used comes from the definition of components and component models laid out in Chapter \ref{background}.
{}However, there are other criteria that maybe component model dependent from information in component meta-data (e.g., size, vendor and licence),
{}component contracts and composition testing (e.g., reliability and speed),
{}and community sourced meta-data stored in repositories (e.g., popularity).

%%%Discuss information from component met-data

%%%Discuss information from component testing (e.g. treaty, unit tests)

%%%Discuss community stored information, e.g. app-install-data, reccommendation systems, community forums\ldots

