\chapter{Introduction}
\label{introduction}
\epigraph{E pluribus unum -- Latin for "Out of many, one"}
{\textit{de facto motto of the United States 1782 - 1956}}
%%%Interesting AND answer WHAT CSE, WHY prediciting behaviour in order to reccommend actions, HOW Simulation!
The Greek philosopher Plutarch proposed in his work \textit{Life of Theseus} a paradox that is in essence the question:
\begin{quote}
If an object has all its component parts replaced, is it the same object?
\end{quote}
The themes this question addresses involve objects made of components, the change of their component parts,
and how this change effects the object in question.

A software system, like the object in Plutarch's question, can be constructed from components and changed by altering its composition.
A \textbf{component system} can be constructed by composing together software components \citep{Szyperski2002},
and the process by which software systems change over time is \textbf{software evolution} \citep{lehman1980}.
The focus of this thesis is \textbf{component system evolution} (CSE), which is the process by which component systems change over time.
A component system is changed by altering its composition of components and \textbf{evolved} through repeated changes.
This research asks questions about CSE to find out how component systems change, and what the effect of that change is.

\section{Objective}
%%%What do I want to find out?
The primary objective of this research is to study the process of component system evolution to better understand its behavior.
The specific behaviours of CSE that are focused on are the amount of change made to systems, and how out-of-date the systems are during evolution.
Through understanding these behaviours of CSE this research can inform users of the consequences of their choices when changing their component systems.
Additionally, this understanding can be used to recommend adjustments to how component systems are currently changed in order to achieve better outcomes for users. 

\section{Motivation}
%%%Why research this?
An example of a popular component system is the Ubuntu GNU/Linux operating system, which has an estimated 20 million users\footnote{http://www.ubuntu.com/ accessed 16/5/2012}. 
Another example of a component system is the Eclipse IDE, which is used by thousands of companies and millions of users\footnote{http://www.eclipse.org/org/community\_survey/Eclipse\_Survey\_2011\_Report.pdf  accessed 29/5/2012}.
Ubuntu and Eclipse systems are constructed from components, called packages and bundles respectively, 
and can be changed by adding or removing components to and from their systems.
Repeatedly applying such changes allows Ubuntu and Eclipse systems to evolve, which in the context of this thesis is the CSE process.
By studying the behaviour of CSE and providing recommendations, this research has the potential to impact millions of users and their systems.

\section{Research Approach}
%%%How are we going to study CSE?
To study the evolution of a single component system, all changes to that system over a period of time must be examined.
Additionally, to gain a broad and useful perspective of CSE behaviour, the evolution of many systems must be studied.
To this end this research simulates the evolution of multiple Ubuntu component systems as they evolve.
The reason for selecting Ubuntu as the component system, and simulation as the method of study are described further in this section.

\subsection{Why Ubuntu?}
The Ubuntu operating system was selected as the component system to study CSE for the following reasons:
\begin{itemize}
  \item it has significant size and complexity which ensures that the results will not be trivial;
  \item it has open and accessible information that can be used for simulation;
  \item it has an active research community (notably the Mancoosi project) which provides a base for research;
  \item it has 20 million users, so outcomes of this research have potentially significant impact.
\end{itemize} 

\subsection{Why Simulate?}
To study the CSE process, the evolution of many component systems must be studied.
The method selected for this study was to model the relevant aspects of CSE, then simulate it using collected data and reasonable approximations.
Using a simulation provides the benefit of having control over variables that impact CSE.
This control allows for testing extreme as well as normal situations.
It also allows for the quick experimentation with novel or proposed techniques.
Therefore, the time to gather results is decreased, which allows a wide range of testing of new methods.

The core hurdle in creating a simulation is ensuring that the returned results are similar enough to reality to draw meaningful conclusions, i.e. the simulation is valid.
In this context:
\begin{quotation}
``\textit{Validation} is the process of determining whether a simulation is an accurate representation of the system, for the particular object of the study.'' \citep{Law2005}
\end{quotation}

The methodology that \cite{Law2005} outlines was selected to create such a valid simulation.
This methodology was selected because it gave practical guidance to creating and using a simulation.
The methodology was created after the observation that validation was often ``attempted after the simulation models had already been developed'' \citep{Law2005}.
Also observed was that simulations that are not validated can produce erroneous information that leads to bad, possibly costly decisions being made.

This methodology has a seven step approach to creating a valid simulation:
\begin{itemize}
  \item \textbf{Step 1: Formulate the problem}: %Intro, Background and Formal Models of the problem
	The problem should be described as clearly as possible.
	The core artifacts at this stage are the overall objectives of the study, the scope of the study.
\item \textbf{Step 2: Collect information/data to construct conceptual model}: %Simulation
	The conceptual model is a description of how the simulation and system work in relation to the studies objectives.
	It contains all variables used to configure the simulation. 
\item \textbf{Step 3: Validate conceptual model}: %Simulation
	The validation of the conceptual model is accomplished through interviews and discussions with the stakeholders of the study.
\item \textbf {Step 4: Implement the conceptual model}: % implementation and Ubuntu simulation --describing where the 
	The implementation of the conceptual model must be executed and documented in a way that allows others to replicate and repeat the process.
\item \textbf{Step 5: Validate simulation implementation}: %Through MISC
	The most definitive way to validate a simulation is to closely compare its results to that from an actual system \citep{Law2005}.
\item \textbf{Step 6: Design, conduct and analyse experiments} :
	Experiments use the simulation to measure behaviours and test hypothesises. 
	For each of the experiments, the configuration and number of independent runs must be defined.
\item \textbf{Step 7: Document and present results}:
	This presentation is required to promote the future re-use of the models, through describing the validation process.
\end{itemize}

The above described methodology was created for large scale industrial projects with a substantial amount of resources available.
It describes the employment and use of experts for validation and simulation analysts to ensure validity.
The available resources for this project is less than such projects, therefore some of the steps have been decreased in scope.
This may reduce the validity of the end model, but these restrictions have been made only when necessary,
and done so in a manner that attempts to minimise negative effects.

\section{Contributions}
As stated above, CSE is studied through simulating the evolution of Ubuntu systems guided by the methodology outlined by \cite{Law2005}.
To create this simulation, a variety of artifacts had to be developed.
Each of these artifacts, as well as the simulation and its results are contributions of this research. 
These contributions are:
\begin{enumerate}
  \item A formal model \textbf{\modelname} (\textbf{Co}mponent \textbf{Sy}stem \textbf{E}volution) that describes CSE. 
  \item The \textbf{CUDF* syntax} that defines documents that describe the evolution of a component system.
  \item The \textbf{conceptual model} of CSE.
  \item \textbf{GJSolver} is an implementation that is used to simulate the evolution of a component system.
  \item A \textbf{simulation of the evolution of Ubuntu operating systems}.
  \item The results, analysis and conclusions from simulations of the evolution Ubuntu systems.
\end{enumerate}

\section{Thesis Overview}
This thesis is organised and presented in the order that resembles the steps of the above described methodology. 

Chapter \ref{background} aims to put CSE in historical context by exploring the concepts of software evolution and component systems.
It also attempts to give a suitable definition of what a software component with respect to CSE.
This is explored by examining various component models that fit the definition, and comparing them against one another.

Chapter \ref{formal} presents the \modelname model and the CUDF* syntax used to exactly describe CSE.
A \modelname instance describes the evolution of a component system as a series of user requested changes made to the system,
and a CUDF* document is used to serialise a \modelname instance.
These define the process of CSE so that it can be simulated and studied.

Chapter \ref{simulation} describes the development of the simulations conceptual model.
By exploring the role users have in changing their component systems, through a conduced survey,
significant variables can be defined to be included in the conceptual model.
Some such variables that are ``the probability the user update the system per day'' and ``the probability a user installs a certain package''.
The conceptual model is then presented, and how it can be used to configure a simulation is described.
To accurately simulate the evolution of an Ubuntu system, data collection and conversion  are necessary.
These actions are described in detail, as they will have significant impact on the validity of the simulation.

Chapter \ref{implementation} describes the algorithms and implementations used to simulate CSE.
The main activity in simulation is taking a \modelname instance and calculating all the changes made to the component system.
This process is called resolving a \modelname instance.
Resolving a \modelname instance may require significant computational effort,
therefore the algorithms and implementation used are an important aspect of research.
The description of these algorithms is important to ensure the reproducibility of the results from the experiments.
This chapter also describes the validation of the simulation by using the presented implementations,
and creating simulation configurations and comparing them against collected data.

Chapter \ref{ubunutsimulation} describes the experiments conducted by simulating the evolution of Ubuntu systems.
These simulations are used to explore how CSE currently effects component systems.
After studying these behaviours, some novel ideas are proposed and tested to see if they can improve some aspects of CSE.
Finally, an experiment is conducted to compare the effects of changing a system with the current methods vs. the proposed ones.

This thesis concludes in chapter \ref{conclusion} with a discussion of the experimental results, the contributions of this research, related work, and possible future research.
