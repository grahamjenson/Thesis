\chapter{Introduction}
\label{introduction}
\epigraph{E pluribus unum -- Latin for "Out of many, one"}
{\textit{de facto motto of the United States 1782 - 1956}}
%%%Interesting AND answer WHAT CSE, WHY prediciting behaviour in order to reccommend actions, HOW Simulation!
The Greek philosopher Plutarch proposed in his work \textit{Life of Theseus} a paradox that is in essence the question:
\begin{quote}
If an object has all its component parts replaced, is it the same object?
\end{quote}
The themes of this question involve objects made of components, the change of their component parts,
and how this change effects the object in question.

A \textbf{component system} can be constructed by composing together software components \citep{Szyperski2002}
and the process by which software systems change over time is \textbf{software evolution} \citep{lehman1980}.
The focus of this thesis is \textbf{component system evolution} (CSE), which is the process by which component systems change over time.
Similar to Plutarch's question, CSE asks questions about how component systems change, and what the effect that hange has.

%%%What
A component system is changed by altering its composition by adding or removing components.
A component system evolves through repeated changes over time.
An example of a popular component system is the Ubuntu GNU/Linux operating system, which has an estimated 20 million users\footnote{http://www.ubuntu.com/ accessed 16/5/2012}. 
This system is constructed by components (called packages) that can be added and removed to the system by its users.
By studying how such systems evolve, this thesis aims to develop an understanding to predict CSE behaviour,
and provide information about consequences of a user's choices.



\section{Objective}
%%%What
The primary objective of this research is to study the process of component system evolution to better understand its behavior.
Through this understanding information can be provided to user's about the consequences of their choices when changing a component systems. 
From these objectives, recommendations to changing a component system to achieve desirable outcomes can be derived. 

%%%How 
CSE is studied by simulating the evolution of Ubuntu systems using developed models and algorithms.
The results from these simulations illustrate the effect users' choices have on the evolution of their component systems.
From such results recommendations can be derived in order to better inform users on the effect of their changes.
Furthermore, by simulating can be used to analyse experimental approaches to evolve component systems, and then judge their effectiveness.

\section{Contributions}
As stated above, CSE is studied using a developed simulation.
To create this simulation a variety of artifacts had to be developed.
Each of these artifacts, as well as the simulation and its results are seen as contributions from this research. 
The contributions from this research are:
\begin{enumerate}
  \item A formal model \textbf{\modelname} (\textbf{Co}mponent \textbf{Sy}stem \textbf{E}volution) that describes CSE. 
  \item The \textbf{CUDF* syntax} that defined documents that can be parsed to an instance of \modelname.
  \item The novel \textbf{lexicographic iterative strengthening algorithm} used to calculate the evolved component systems of a \modelname instance.
  \item \textbf{GJSolver} is an implementation of the parsing from a CUDF* document to \modelname instance, and then the resolving using the lexicographic iterative strengthening algorithm.
  \item A \textbf{conceptual model} of CSE based on the results of a user survey. 
  \item A \textbf{simulation of the evolution of Ubuntu operating systems}.
  \item The results from simulations of the evolution Ubuntu systems
\end{enumerate}

\section{Thesis Overview}
This thesis follows the above order of contributions

Chapter \ref{background} introduces the background of software evolution and component based systems.
This background is presented to illustrate the context in which this research occurs.
This chapter also attempts to give a suitable definition of what a software component is, and then examine different component models that fit such a definition.

Chapter \ref{formal} describes the \modelname model, and the CUDF* syntax.
The process of parsing a CUDF* document to create a \modelname instance is given, where the full description is in appendix \ref{apx.cudf}.

Chapter \ref{implementation} describes the resolving of a \modelname instance through mapping each evolution step to a \modelimpl problem.
First the \modelimpl problem is defined, then a mapping from an evolution step in a \modelname instance to a \modelimpl problem is described.
The lexicographic iterative strengthening algorithm to solve \modelimpl problems is then presented, and the use of this algorithm to resolve a \modelname instance is described.
Finally, this chapter describes the GJSolver implementation and its verification.

Chapter \ref{simulation} describes the development of the CSE simulation.
First the reason for using a simulation is discussed, then the methodology used to create the simulation is described.
The user survey is then discussed (for full results see appendix \ref{apx.survey}) and used to create a conceptual model.
This model is used to create CUDF* documents that describe evolutions of component systems. 

Chapter \ref{ubunutsimulation} discusses the simulation of the  the Debian component model \citep{Barth2005} with Ubuntu  components. 

This thesis concludes in chapter (\ref{conclusion}) with a discusion of the contributions of this research, related work, and potential future research.
