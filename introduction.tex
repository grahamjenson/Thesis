\chapter{Introduction}
\label{introduction}
\epigraph{E pluribus unum -- Latin for ``Out of many, one''}
{\textit{de facto motto of the United States 1782 - 1956}}
%%%Interesting AND answer WHAT CSE, WHY prediciting behaviour in order to reccommend actions, HOW Simulation!
The Greek philosopher Plutarch proposed in his work \textit{Life of Theseus} a paradox that is in essence the question:
\begin{quote}
If an object has all its component parts replaced, is it the same object?
\end{quote}
This question addresses many themes including how objects are made of components, how they change their component parts,
and how this change effects the object in question.

Like the object in Plutarch's question, a software system can be constructed from components and changed by altering its composition.
A \textbf{component system} can be constructed by composing together software components \citep{Szyperski2002},
and the process by which a software system changes over time is called \textbf{software evolution} \citep{lehman1980}.
The focus of this thesis is \textbf{component system evolution} (CSE), which is the process by which a component system changes over time.
A component system is changed by altering its composition of components and \textbf{evolves} through repeated changes.
This research asks questions about how component systems evolve and what the effects this evolution has.

\section{Objective}
%%%What do I want to find out?
The primary objective of this research is to study the process of component system evolution. 
The aspects of CSE that are focused on is the amount of change made to systems and how out-of-date the systems become over time.
Through understanding this change and out-of-dateness,
this research can inform developers and users of the consequences their choices have to the CSE process.
Additionally, this research can be used to alter how component systems are changed in order to reduce change and out-of-dateness during evolution.

%%%TODO Thesis statement
\section{Motivation}
%%%Why research this?
An example of a popular component system is the Ubuntu GNU/Linux operating system, which has an estimated 20 million users\footnote{http://www.ubuntu.com/ accessed 16/5/2012}. 
Another example of a component system is the Eclipse IDE, which is used by thousands of companies and millions of users\footnote{http://www.eclipse.org/org/community\_survey/Eclipse\_Survey\_2011\_Report.pdf  accessed 29/5/2012}.
Ubuntu and Eclipse systems are constructed from components, called packages and bundles respectively, 
and can be changed by adding or removing components to and from their systems.
Repeatedly applying such changes allows Ubuntu and Eclipse systems to evolve, which in the context of this thesis is the CSE process.
By studying the effects of CSE and providing recommendations, this research has the potential to impact millions of users and their systems.

\section{Research Approach}
%%%How are we going to study CSE?
To study the evolution of a single component system, all changes to that system over a period of time must be examined.
Additionally, to gain a broad and useful perspective of CSE effects, the evolution of many systems must be studied.
To this end, this research simulates the evolution of many Ubuntu component systems.
The reason for selecting Ubuntu as the component system, and simulation as the method of study are described further in this section.

\subsection{Why Ubuntu?}
The Ubuntu operating system was selected as the component system to study CSE for the following reasons:
\begin{itemize}
  \item It has significant size and complexity which ensures that the results will not be trivial.
  \item It has open and accessible information that can be used to create a simulation.
  \item It has an active research community which provides a base for research.
  \item It has an estimated 20 million users, potentially allowing the outcomes of this research to have a significant impact.
\end{itemize} 

\subsection{Why Simulate?}
To study the CSE process, the evolution of many component systems must be studied.
The method selected for this study is to model the relevant aspects of CSE, then simulate the evolution of many Ubuntu systems using collected data and reasonable approximations.
Using a simulation provides the benefit of having control over variables that impact CSE.
This control allows for testing normal as well as extreme situations.
It also allows for the quick experimentation with novel or proposed techniques.

The core hurdle in creating a simulation is ensuring that the returned results are similar enough to reality to draw meaningful conclusions, i.e. the simulation is valid.
In this context:
\begin{quotation}
``\textit{Validation} is the process of determining whether a simulation is an accurate representation of the system, for the particular object of the study.'' \citep{Law2005}
\end{quotation}

The methodology that \cite{Law2005} outlines was selected to create such a valid simulation.
This methodology was selected because it gave practical guidance to creating and using a simulation.
The methodology was created after the observation that validation was often ``attempted after the simulation models had already been developed'' \citep{Law2005}.
It was also observed was that simulations that are not validated can produce erroneous information that leads to bad, possibly costly decisions being made.

This methodology has a seven step approach to creating a valid simulation:
\begin{itemize}
  \item \textbf{Step 1: Formulate the problem}: %Intro, Background and Formal Models of the problem
	The problem should be described as clearly as possible.
	The core artifacts at this stage are the overall objectives and the the scope of the study.
\item \textbf{Step 2: Collect information/data to construct a conceptual model}: %Simulation
	The conceptual model is a description of how the simulation and system work in relation to the study's objectives.
	It contains all variables used to configure the simulation. 
\item \textbf{Step 3: Validate the conceptual model}: %Simulation
	The validation of the conceptual model is accomplished through interviews and discussions with the stakeholders of the study.
\item \textbf {Step 4: Implement the conceptual model}: % implementation and Ubuntu simulation --describing where the 
	The implementation of the conceptual model must be executed and documented in a way that allows others to replicate and repeat the process.
\item \textbf{Step 5: Validate the simulation implementation}: %Through MISC
	The most definitive way to validate a simulation is to compare its results to those from an actual system \citep{Law2005}.
\item \textbf{Step 6: Design, conduct and analyse experiments} :
	Experiments use the simulation to measure effects and test hypothesises. 
	For each of the experiments, the configuration and number of independent runs must be defined.
\item \textbf{Step 7: Document and present results}:
	This presentation is required to promote the future re-use of the models, through describing the validation process.
\end{itemize}

The above described methodology was created for large scale industrial projects with a substantial resources available.
It describes the employment and use of experts and analysts to ensure validity.
The available resources for this project are less than other such projects, therefore some of the steps have been decreased in scope.
This may reduce the validity of the final simulation, but these restrictions have been made only when necessary,
and done so in a manner that attempts to minimise negative effects.

\section{Contributions}
As stated above, CSE is studied through simulating the evolution of Ubuntu systems guided by the methodology outlined by \cite{Law2005}.
To create this simulation, a variety of artifacts had to be developed.
Each of these artifacts, as well as the simulation and its results are contributions of this research. 
These contributions are:
\begin{enumerate}
  \item A formal model \textbf{\modelname} (\textbf{Co}mponent \textbf{Sy}stem \textbf{E}volution) that describes CSE. 
  \item The \textbf{CUDF* syntax} that is used to define documents that describe the evolution of a component system.
  \item The \textbf{\usermodel} (\textbf{Sim}ulated \textbf{User}) model that approximates a user that changes their system.
  \item The \textbf{GJSolver} which is an implementation that calculates the changes made to a system as it evolves (called resolving).
  \item A \textbf{simulation of the evolution of Ubuntu operating systems} using \modelname, CUDF*, \usermodel and GJSolver.
  \item The results, analysis and conclusions from experiments using the simulation.
\end{enumerate}

Some of these contributions are Ubuntu specific, e.g. the \usermodel model and the results and analysis.
However, the \modelname model and CUDF* syntax are generalizable to other component systems.
Though this generalization is not explored in this research, a case for further research in this area is made in chapter \ref{conclusion}.

\section{Thesis Overview}
This thesis is organised and presented in the order that resembles the steps of the above described methodology
where first the problem is described (chapter \ref{background}), then the models are presented and validated (chapters \ref{formal} and \ref{simulation}),
the implementation and validation of the simulation is described (chapter \ref{implementation}), and finally the experiments and their results are discussed (chapter \ref{experiments}).  

Chapter \ref{background} explores the backgrounds of CSE and its related domains.
This aims to put CSE in historical context and to give a suitable definitions to the elements of CSE.

Chapter \ref{formal} presents the \modelname model and the CUDF* syntax that describe the formal aspects of the evolution of component systems.
\modelname can be used to describe the evolution of a component system and CUDF* is used as the syntax to serialise \modelname instances.

Chapter \ref{simulation} presents \usermodel that is used to describe users that request changes to their component systems.
This model includes assumptions and variables that are necessary to simulate the evolution of Ubuntu systems. 
It is defined using a conducted survey, whose results are also presented.
As this model relies on many assumptions that may impact the validity of the simulation, the validation of this model is also discussed. 

Chapter \ref{implementation} describes the algorithms used to create the GJSolver implementation.
The resolving of a component system can require significant computational effort,
therefore the algorithms and implementation used are an important aspect of research.
The verification of GJSolver, and validation of the simulation are also discussed in this chapter.

Chapter \ref{experiments} describes the experiments, results and analysis that are conducted using the developed simulation.
The effects examined are the changes that the systems go through, and how out-of-date the systems become during evolution.
Through these experiments some causes of change and out-of-dateness are found, and to reduce their effects some novel changes to CSE are proposed.
The effects of these changes are measured using the simulation and are shown to have benefits during CSE.

This thesis concludes in chapter \ref{conclusion} by describing the contributions of this research, related work, and possible future research.
