\chapter{Introduction}
\label{introduction}
\epigraph{E pluribus unum -- Latin for "Out of many, one"}
{\textit{de facto motto of the United States 1782 - 1956}}
Plutarch, a Greek philosopher, in his work \textit{Life of Theseus} 
proposed a paradox that in essence is the question:
\begin{quote}
If an object has all its component parts replaced, is it the same object?
\end{quote}
This question deals with the themes of objects made of components, the change of the component parts,
and how this change effects the object in question.
These themes are at the center of this thesis.

A \textbf{component system} can be constructed by composing together software components \citep{Szyperski2002}
and the process by which software systems change over time is the study of \textbf{software evolution} \citep{lehman1980}.
Therefore, \textbf{component system evolution} (CSE) is the study of how a component system changes over time.
In parallel to Plutarch's question, this research attempts to answer "how does a component system change as it evolves?".

\section{Objective}
The objective of this research is to study the process of component system evolution by using a developed simulation.
This simulation is used to model a variety of users and illustrate the effect their choices have on their component systems as it evolves.
The results of this can be used to answer questions like ``what is the average amount of components that are upgraded, if a user chooses to upgrade their system every day?''.
Furthermore, the simulation can be used to analyse experimental approaches to evolve component systems, and then judge their effectiveness.

\section{Contributions}
As stated above, CSE is studied using a developed simulation.
To create this simulation a variety of artifacts had to be developed.
Each of these artifacts, as well as the simulation and its results are seen as contributions from this research. 
The contributions from this research are:
\begin{enumerate}
  \item A formal model \textbf{\modelname} (\textbf{Co}mponent \textbf{Sy}stem \textbf{E}volution) that describes CSE. 
  \item The \textbf{CUDF* syntax} that defined documents that can be parsed to an instance of \modelname.
  \item The novel \textbf{lexicographic iterative strengthening algorithm} used to calculate the evolved component systems of a \modelname instance.
  \item \textbf{GJSolver} is an implementation of the parsing from a CUDF* document to \modelname instance, and then the resolving using the lexicographic iterative strengthening algorithm.
  \item A \textbf{conceptual model} of CSE based on the results of a user survey. 
  \item A \textbf{simulation of the evolution of Ubuntu operating systems}.
  \item The results from simulations of the evolution Ubuntu systems to answer the question "how does a component system change as it evolves?".
\end{enumerate}

\section{Thesis Overview}
This thesis follows the above order of contributions

Chapter \ref{background} introduces the background of software evolution and component based systems.
This background is presented to illustrate the context in which this research occurs.
This chapter also attempts to give a suitable definition of what a software component is, and then examine different component models that fit such a definition.

Chapter \ref{formal} describes the \modelname model, and the CUDF* syntax.
The process of parsing a CUDF* document to create a \modelname instance is given, where the full description is in appendix \ref{apx.cudf}.

Chapter \ref{implementation} describes the resolving of a \modelname instance through mapping each evolution step to a \modelimpl problem.
First the \modelimpl problem is defined, then a mapping from an evolution step in a \modelname instance to a \modelimpl problem is described.
The lexicographic iterative strengthening algorithm to solve \modelimpl problems is then presented, and the use of this algorithm to resolve a \modelname instance is described.
Finally, this chapter describes the GJSolver implementation and its verification.

Chapter \ref{simulation} describes the development of the CSE simulation.
First the reason for using a simulation is discussed, then the methodology used to create the simulation is described.
The user survey is then discussed (for full results see appendix \ref{apx.survey}) and used to create a conceptual model.
This model is used to create CUDF* documents that describe evolutions of component systems. 

Chapter \ref{ubunutsimulation} discusses the simulation of the  the Debian component model \citep{Barth2005} with Ubuntu  components. 

This thesis concludes in chapter (\ref{conclusion}) with a discusion of the contributions of this research, related work, and potential future research.