\chapter{Introduction}
\label{introduction}
\epigraph{E pluribus unum -- Latin for ``Out of many, one''}
{\textit{de facto motto of the United States 1782 - 1956}}
%%%Interesting AND answer WHAT CSE, WHY prediciting behaviour in order to reccommend actions, HOW Simulation!
The Greek philosopher Plutarch proposed in his work \textit{Life of Theseus} a paradox that is in essence the question\footnote{The actual question asked was whether Theseus's ship, which was restored by replacing all wooden parts, remained the same ship.}:
\begin{quote}
If an object has all its component parts replaced, is it the same object?
\end{quote}
This question addresses many themes including how objects are made of components, how they change their component parts,
and what effect does change have on the object in question.
Although ancient, 
these themes are still relevant today and this research explores each of them in the domain of software.

%how objects are made of components
\textit{How is a software system made from components?}
A \textbf{component system} can be created by composing together \textbf{software component}s \citep{Szyperski2002}.
A software component is an encapsulated unit that requires functionality supplied by other components, in order to provide its own functionality. 
A component system is valid only if every contained component has all its required functionality provided by another component.
For example, given a text-editor component requires spell-checker functionality, 
for a system to be valid that includes the text-editor must also include a component that provides spell-checker functionality.


%how/why they change their component parts %Change, change must be valid, many valid choices, technical aspects automated, must trust the technology
\textit{How are component systems changed?}
A component system can be changed by adding, removing or replacing components, however these actions must result in a valid component system.
This means that adding, removing, or replacing a single component may cause a propagation of change to the component system.
For example, when adding a text-editor to a component system 
a component that provides spell-checker functionality must also be added, otherwise the system will not be valid.

Additional complexity arises when changing a component system as there may be more than one valid component system to select from.
Furthermore, different component systems may have different properties and one system may be preferable to another.
Therefore, when changing a component system the resulting system should be valid and preferable.  
For example, given a text-editor component is being added to a component system and two components provide spell-checker functionality, 
a component system with either spell-checker component is valid, however the component system selected should be preferable to the other.

These complex and technical aspects of component system change can be automated. 
Such automation allows users to change their component systems without performing the tedious, error prone task of finding a preferable, valid component system.
However, when using such automated mechanisms a user must trust that the change will reflect their preferences.


%how this change effects the object in question
\textit{What effects does changing a component system have?}
Over time, a user will continually change their system to meet their requirements and keep it up-to-date.
This gradual change is the \textbf{component system evolution} (CSE) process.
The mechanisms by which the systems are changed will effect the system as it evolves,
particularly the criteria by which preferable component systems are selected.
However, what exactly are the effects these mechanisms have on an evolving system is yet to be answered. 
This research aims to provide answers to this questions by looking at component systems, the mechanisms that change them, and CSE.

\section{Motivation}
%%%Why research this?
Examples of popular component systems are the Ubuntu GNU/Linux operating systems, which have an estimated 20 million users\footnote{http://www.ubuntu.com/, accessed 16/5/2012},
ans Eclipse IDEs, which are used by thousands of companies and millions of users\footnote{http://www.eclipse.org/org/community\_survey/Eclipse\_Survey\_2011\_Report.pdf, \\accessed 29/5/2012}.
Ubuntu and Eclipse systems are constructed from components, called packages and bundles respectively. 
Ubuntu systems can be changed with the application \texttt{apt-get} \citep{Barth2005} and Eclipse systems can be changed with Eclipse P2 \citep{leBerre2010}.
These allow the user to change their system by adding, removing and upgrading components.
Repeatedly applying such changes allows the Ubuntu and Eclipse systems to evolve through the CSE process.
By studying the CSE process and understanding the effects it has on systems, this research has the potential to impact millions of users and their systems.

\section{Objective}
%%%What do I want to find out?
The objective of this research is to study the process of component system evolution and its effect on component systems.
The primary effects that are focused on are the amount of change made to systems and how out-of-date the systems become over time.
In section \ref{strat.usersurvey}, these two effects are argued to be a users' primary concerns during CSE.
Through understanding this change and out-of-dateness,
this research aims to inform developers and users of the consequences their choices have on evolving systems,
and reduce these negative effects.

The objective of this research has lead to the thesis:
\begin{quote}
\textit{It is possible to reduce the negative effects of component system evolution by altering the mechanisms by which systems are changed.} 
\end{quote}

The necessary steps to validate this thesis are:
\begin{itemize}
  \item To develop a reproducible and controllable environment in which to measure the effects of CSE.
  \item To use this environment to study how systems evolve.
  \item To alter the mechanisms by which systems are changed and study their impact on CSE.
  \item To demonstrate a reduction on change and out-of-dateness using such alterations.
\end{itemize}

Once these steps are completed, this thesis will be shown to be supported.  

\section{Research Method}
%%%How are we going to study CSE?
To study the evolution of a single component system, all changes to that system over a period of time must be examined.
Additionally, to gain a broad and useful perspective of CSE characteristics, the evolution of many systems must be studied.
To this end, this research simulates the evolution of many Ubuntu systems.
The reason for selecting Ubuntu and simulation as the method of study are detailed further in this section.

\subsection{Why Ubuntu?}
The Ubuntu operating system was selected as the component system to study CSE for the following reasons:
\begin{itemize}
  \item It has significant size and complexity which ensures that the results will not be trivial.
  \item It has an active research community whose results can be used and built upon.
  \item It has an estimated 20 million users, potentially allowing the outcomes of this research to have a significant impact.
  \item There are many open resources of information available to be collected and used for simulation.
\end{itemize} 

Another important question is ``Why only Ubuntu?'', 
as only focusing on the evolution of one component system may make the results from this research non-generalisable.
There are many similarities between component models (some are discussed in section \ref{background.models})
including the relationships between components and the mechanisms used to change component systems.
These similarities may make many of the contributions from this research component system independent.

Another factor in the decision to only simulate the evolution of Ubuntu systems was the significant effort invested in the collection of data.
This effort was reduced because of Ubuntu's open nature and active community that provided tools and resources to be used.
It may be more difficult to collect equally valid data for other component systems as
many systems, like Eclipse, do not have their history archived as precisely as Ubuntu.
Therefore, the simulation of other types of component system is not within the scope of this research,
however the simulation of further component systems is proposed as future research in section \ref{conclusion.futurework}.

\subsection{Why Simulate?}
The method selected to study CSE is to model its relevant aspects, then simulate the evolution of many Ubuntu systems.
This method was selected as a simulation provides the necessary control over the variables of CSE to validate the thesis.

Another method that was considered when approaching this thesis was to study the evolving systems of real users.
The difficulty of this method would be finding participants willing to allow experimental techniques change their (possibly mission critical) systems.
Finding enough participants to produce meaningful results would have required time and resources that were not available to this research.
Additionally, as evolution occurs over long periods of time, problems could take months to identify and correct, increasing the risk when using this method for this research. 
Therefore, the simulation approach to validating the thesis was preferred as it has a lower cost and less risk.

The core hurdle in creating a simulation is ensuring that the returned results are similar enough to reality to draw meaningful conclusions, i.e. the simulation is valid:
\begin{quotation}
``\textit{Validation} is the process of determining whether a simulation is an accurate representation of the system, for the particular object of the study.'' \citep{Law2005}
\end{quotation}

The methodology that \cite{Law2005} outlines was selected because it gave practical guidance to creating and using a valid simulation.
The methodology was created after the observation that validation was often ``attempted after the simulation models had already been developed'' \citep{Law2005}.
It was also observed was that non-validated simulations can produce erroneous information that leads to bad, possibly costly decisions being made.

This methodology has a seven step approach to creating a valid simulation:
\begin{itemize}
  \item \textbf{Step 1: Formulate the problem}: %Intro, Background and Formal Models of the problem
	The problem should be described as clearly as possible.
	The core artifacts at this stage are the overall objectives and the scope of the study.
\item \textbf{Step 2: Collect information/data to construct a conceptual model}: %Simulation
	The conceptual model is a description of how the simulation and system work in relation to the study's objectives.
	It contains all variables used to configure the simulation. 
\item \textbf{Step 3: Validate the conceptual model}: %Simulation
	The validation of the conceptual model is accomplished through interviews and discussions with the stakeholders of the study.
\item \textbf {Step 4: Implement the conceptual model}: % implementation and Ubuntu simulation --describing where the 
	The implementation of the conceptual model must be executed and documented in a way that allows others to replicate and repeat the process.
\item \textbf{Step 5: Validate the simulation implementation}: %Through MISC
	The most definitive way to validate a simulation is to compare its results to those from an actual system \citep{Law2005}.
\item \textbf{Step 6: Design, conduct and analyse experiments} :
	Experiments use the simulation to measure effects and test hypothesises. 
	For each of the experiments, the configuration and number of independent runs must be defined.
\item \textbf{Step 7: Document and present results}:
	This presentation is required to promote the future re-use of the models, through describing the validation process.
\end{itemize}

The above described methodology was created for large scale industrial projects with a substantial resources available.
It describes the employment and use of experts and analysts to ensure validity.
The available resources for this project is fewer than these large scale projects, therefore some of the steps have been decreased in scope.
This may reduce the validity of the final simulation, but these restrictions have been made only when necessary,
and done so in a manner that attempts to minimise negative effects.

\section{Contributions}
As stated above, CSE is studied through simulating the evolution of Ubuntu systems guided by the methodology outlined by \cite{Law2005}.
The simulation artifacts, the alterations defined to reduce negative effects during CSE, 
and results from the simulation are all contributions of this research. 
These contributions are:
\begin{enumerate}
  \item A formal model \textbf{\modelname} (\textbf{Co}mponent \textbf{Sy}stem \textbf{E}volution) that describes CSE. 
  \item The \textbf{CUDF* syntax} that is used to define documents that describe the evolution of a component system.
  \item \textbf{\usermodel} (\textbf{Sim}ulated \textbf{User}) that models a user that changes their system.
  \item The \textbf{GJSolver} which is an efficient implementation that calculates the changes made to a system as it evolves (called resolving).
	GJSolver was independently validated through the MISC competition hosted by the Mancoosi project\footnote{http://www.mancoosi.org/, accessed 8/8/2012}. 
  \item A \textbf{simulation of the evolution of Ubuntu operating systems} using \modelname, CUDF*, \usermodel and GJSolver.
  \item The defined methods to reduce the out-of-dateness and change during CSE. 
  \item The results, analysis and conclusions from experiments using the simulation.
\end{enumerate}

These contributions overlap with published papers from this research:
\begin{enumerate}
  \item An empirical study into the search space of resolving component systems \citep{Jenson2010} in the proceedings of Component Based Software Engineering (CBSE) 2010.
  \item A formal framework to describe a users preferences during CSE \citep{Jenson2010a} in the proceedings of Asia/Pacific Software Engineering Conference (APSEC) 2010.
  \item An empirical study into the evolution of component systems \citep{jenson2011} in the proceedings of CBSE 2011.
\end{enumerate}

\section{Thesis Overview}
This thesis is organised and presented in the order that resembles the steps of the above described methodology:
the problem is described (chapter \ref{background}), the models are presented and validated (chapters \ref{formal} and \ref{simulation}),
the implementation and validation of the simulation are described (chapter \ref{implementation}), and the experiments and their results are discussed (chapter \ref{experiments}).  

Chapter \ref{background} explores the backgrounds of CSE and its related domains.
This aims to put CSE in historical context and to give a suitable definitions to the elements of CSE.

Chapter \ref{formal} presents the \modelname model and the CUDF* syntax that describe the formal aspects of the evolution of component systems.
\modelname is used to describe the evolution of a component system and CUDF* is used as the syntax to serialise \modelname instances.

Chapter \ref{simulation} presents the \usermodel model which is used to describe users that request changes to their component systems.
This model includes assumptions and variables that are necessary to simulate the evolution of Ubuntu systems. 
It is developed from the results of a conducted survey.
As this model relies on many assumptions that may impact the validity of the simulation, the validation of this model is also discussed. 

Chapter \ref{implementation} describes the algorithms used to create the GJSolver implementation.
The resolving of a component system can require significant computational effort,
therefore the algorithms and implementation used are an important aspect of research.
The verification of GJSolver and validation of the simulation are also discussed in this chapter.

Chapter \ref{experiments} describes the experiments, results and analysis that are conducted using the developed simulation.
The effects examined are the changes that the systems go through, and how out-of-date the systems become during evolution.
Through these experiments, causes of additional change and out-of-dateness are identified.
These causes are addressed with some novel changes to CSE, and through using the simulation the impact of these changes is measured.
The effects of these changes are measured using the simulation and are shown to have benefits during CSE.

This thesis concludes in chapter \ref{conclusion} by describing the contributions of this research and possible future research.
