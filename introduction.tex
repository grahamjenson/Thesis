\chapter{Introduction}
\label{introduction}
\epigraph{E pluribus unum -- Latin for "Out of many, one"}
{\textit{de facto motto of the United States 1782 - 1956}}
%%%Software evolution is important
Software evolution is the repeated alteration to a software system in order to maintain the system or satisfy changing requirements.
Software evolution is regularly cited as over 50\% the total cost of a project \citep{Grubb2003},
and it is also a significant risk as any change to software may introduce new errors \citep{Brooks1975}.
Software evolution is also necessary, as any sufficiently complex system will require change to remain satisfactory \citep{lehman1980}. 
By studying the process of software evolution, it is hoped that risks can be mitigated and the costs can be lowered.

%%%How to make software less costly during evolution
To address the issues of risk and cost during software evolution, software can be created using various techniques.
For example, creating software with the hierarchical decomposition of object orientation \citep{goldberg1983smalltalk}, 
the decoupling of relationships through dependency inversion \citep{martin1996dependency}, 
separation of concerns through aspect-oriented programming \citep{Kiczales1997},
and the abstraction of knowledge through model driven engineering \citep{schmidt2006}.
Another method is to construct software systems by composing together independent units called \textbf{software components} \citep{Szyperski2002}.

By creating a system from components, the process of software evolution can be split into two different parts;
component evolution and component system evolution (CSE).
Component evolution is the change of a component by a developer at the code level,
e.g. adding functionality to a component then releasing a new version.  
CSE is the change of a system by a composer (user) at the component level,
i.e. removing an old version of a component from a system, then replacing it with a newer version.

\section{Objective}
The objective of this research is to study the process of component system evolution using a developed simulation.
This simulation can simulate a variety of users and illustrate the effect their choices have on their component systems.
Answering questions like what is the average amount of components that are upgraded, if a user chooses to upgrade their system every day?
Furthermore, the simulation can be used to study experimental approaches to evolve component systems, and then judge their effectiveness.

\section{Contributions}
As stated above, CSE is studied using a developed simulation.
To create this simulation a variety of artifacts had to be developed.
Each of these artifacts, as well as the simulation and the results from the simulation are seen as contributions from this research. 
The contributions from this research are:
\begin{enumerate}
  \item A formal model \modelname (\textbf{Co}mponent \textbf{Sy}stem \textbf{E}volution) was developed to describes CSE. 
  This model views component system evolution as a series of evolution steps, where each step tries to satisfy a user request and system constraints by changing the component system.
  \item A syntax CUDF* was developed such that a CUDF* document could be parsed to an instance of \modelname. 
  \item Although an instance of \modelname describes a component systems evolution, calculating the exact evolution may require some effort.
  This calculation is known as resolving.
  In order resolve a \modelname instance, each evolution step in the instance is mapped to a \modelimpl problem.
  The novel lexicographic iterative strengthening algorithm was developed to solve \modelimpl problems efficiently, thus allowing the efficient resolving of a \modelname instance.
  \item GJSolver was developed to implement the parsing from a CUDF* document to \modelname instance, and then the resolving using the lexicographic iterative strengthening algorithm.
  GJSolver was verified through a third party competition, MISC.
  \item A conceptual model of CSE was created based on the results of a user survey. 
  This model is used to describe a realistic evolution of a component system as a CUDF* document. 
  \item By creating instances of the conceptual model using data from collected user logs and the Ubuntu GNU/Linux operating system,
  the evolution of a component systems can be simulated.
  This simulation returns results that are used to answer questions about CSE, and explore experimental approaches to CSE.
\end{enumerate}


\section{Thesis Overview}
%%%A list and breif explination of each of the chapters and how they relate to our Research Goals
This thesis first describes the background of software evolution and component based systems in chapter \ref{background}.
This chapter also attempts to give a suitable definition of what a software component is, and then examines different component models that fit such a definition.
Other concepts such as users, software evolution, component evolution, and component system evolution are also explored in this chapter.

Chapter \ref{formal} describes the \modelname model, and the CUDF* syntax.
A CUDF* document can be parsed to an instance of \modelname which describes the evolution of a component system.

Chapter \ref{implementation} describes the resolving of a \modelname instance through mapping each evolution step to a \modelimpl problem.
The lexicographic iterative strengthening algorithm is described as the method to find a solution to a \modelimpl problem.
This chapter also describes the GJSolver implementation, which parses a CUDF* document to a \modelname instance, and resolves it using the lexicographic iterative strengthening algorithm.
Also described is the verification of GJSolver.

Chapter \ref{simulation} describes the process of creating the simulation using a method presented in \citep{Law2005}.
A core element of the simulation is the conceptual model, which models the reality of CSE.
Instances of this model are used to create realistic component system evolution problems, in order to be studied.

Chapter \ref{ubunutsimulation} discusses the simulation using the Debian component model \citep{Barth2005} with Ubuntu \citep{Hill2006} components. 

This thesis concludes in chapter (\ref{conclusion}) with a discusion of the contributions of this research, related work, and potential future research.