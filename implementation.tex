\chapter{Implementation}

\section{Common Upgradeability Description Format}

\section{Debian Model to CUDF mapping}
The mapping of the Debian dpkg component meta-data to CUDF is mostly a direct process as the meta data is very similar. 
However, there are a few instances where the Debian semantic causes complication with the CUDF format, and in this section we describe these in detail.
The first of these complications occuurs with versioning, as in Debian the version model is different to the CUDF model.
The second instance is because of the special relationships that virtual packages have in the Debian model, 
these must be intergrated into the CUDF mapping. 
Lastly the priority ranking, recommends, suggests and other extra information that is not standard in the CUDF format,
but with CUDF's extencibility can be represented and used in the resolution process.

The versions in Debian follow a model that first describes the epoch,
the upstream version, then the Debian revision (further described in the Debian policy manual). %TODO cite debian package manager
When comparing two versions these values are compared lexicographically such that
if the epoch is equal then the upstream versions are compared, 
and if these are equal then the Debian revision is compared.
As the CUDF version model is merely a single integer number, 
the Debian versions are not easily mapped.
To map these versions we order all Debian versions that are referenced in a repository into a single list, 
such that the least version has the index 0, and the last element in the list is the greatest version.
The index of the a Debian version in this list is then used as the CUDF version.

A virtual package in


\section{OSGi Component Model}
\subsection{Mapping to CUDF}

\section{Comparison}




