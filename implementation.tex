\chapter{Resolving \modelname}
\label{implementation}
\epigraph{What I cannot create, I do not understand.}
{\textit{Richard Feynman, 1988}}
The previous chapters presented the \modelname model that describes the evolution of a component system as a series of evolution steps.
Each evolution step alters the component system to satisfy a user's request to change it.
Calculating all the component systems that occur because of these requests is described as \textbf{resolving} a \modelname instance.

Resolving a \modelname instance can require significant computational effort due to the complexity of satisfying each request.
Therefore, an efficient problem representation and algorithm are required.
To resolve a \modelname instance each evolution step can be mapped to a Boolean Lexicographic Optimization (\modelimpl) problem \cite{marque2011blex}.
\modelimpl problems consists of finding an optimal assignment to Boolean variables,
where assignments must satisfy a set of clauses and linear inequalities,
and the optimisation criteria are lexicographic.
The clauses and linear inequalities are defined using a SAT formula extended with pseudo-Boolean (PB) constraints (SAT+PB) \citep{dixon2004automating},
and the criteria are defined to either maximise or minimise PB functions.

These relationships are presented in figure \ref{impl.modelsatdiagram}.

\begin{figure}[htp]
\begin{center}
\digraph[scale=0.5]{implsatgraph}{
rankdir=BT;
SAT [label=<<TABLE BORDER="0" CELLBORDER="1" CELLSPACING="5"><TR><TD WIDTH="150">SAT problem</TD></TR></TABLE>> shape=none];
PBCrit [label=<<TABLE BORDER="0" CELLBORDER="1" CELLSPACING="5"><TR><TD WIDTH="150">PB criteria</TD></TR></TABLE>> shape=none];
SATPB [label=<<TABLE BORDER="0" CELLBORDER="1" CELLSPACING="5"><TR><TD WIDTH="150">SAT+PB problem</TD></TR></TABLE>> shape=none];
SS [label=<<TABLE BORDER="0" CELLBORDER="1" CELLSPACING="5"><TR><TD WIDTH="150">\modelimplwx</TD></TR></TABLE>> shape=none];
SSI [label=<<TABLE BORDER="0" CELLBORDER="1" CELLSPACING="5"><TR><TD WIDTH="150">:\modelimplwx</TD></TR></TABLE>> shape=none];
\modelnamewx [label=<<TABLE  BORDER="0" CELLBORDER="1" CELLSPACING="5"><TR><TD WIDTH="150">\modelnamewx</TD></TR></TABLE>> shape=none];
FMI [label=<<TABLE BORDER="0" CELLBORDER="1" CELLSPACING="5"><TR><TD WIDTH="150">:\modelnamewx</TD></TR></TABLE>> shape=none ];
SS -> PBCrit [label=uses];
SS -> SATPB [label=uses];
SSI -> SS [label=instanciates];
subgraph {
rank=same
	FMI -> SSI [ label="maps to 1..*" ];
}
SATPB -> SAT [label=extends];
FMI -> \modelnamewx [ label = "instantiates"];
}
  \caption{A description of the relationships of \modelimpl and \modelname}
  \label{impl.modelsatdiagram}
\end{center}
\end{figure}

The algorithm employed to solve \modelimpl problems is the lexicographic-iterative-strengthening (LIS) algorithm.
This algorithm uses the Davis-Putnam-Logemann-Loveland (DPLL) \citep{Davis1960, davis1962machine} algorithm to find solutions to SAT+PB problems,
and the iterative strengthening \citep{calistri1994iterative, le2010sat4j} algorithm to optimise finding a solution.
This algorithm is similar to the ``iterative pseudo-Boolean solving'' algorithm presented by \cite{marque2011blex}.
However, LIS directly depends on the iterative-strengthening algorithm, where the ``iterative pseudo-Boolean solving'' does not specify such specific details. 

Some criteria that can be used by the LIS algorithm are also presented in this chapter. 
These criteria are mapped to MOF strings, and \modelname criteria as they are useful for CSE.
The criteria can be grouped into to general areas, minimising change and minimising how out-of-date a system is.
The presented criteria that minimise change where defined by Mancoosi and the criteria to minimise the out-of-dateness is defined by \cite{leBerre2010}.
 
This chapter also describes the implementation of GJSolver that takes a CUDF* document, parses it into a \modelname instance (as described in appendix \ref{apx.cudf}),
then resolves it with using the LIS algorithm.
It also describes the verification of GJSolver through through the Mancoosi International Solver Competition (MISC)\footnote{http://www.mancoosi.org/misc-2011/criteria/ accessed 6/3/2012}.

This chapter starts with section \ref{impl.mapping} that describes the \modelimpl problem and its mapping to an evolution step from a \modelname instance.
In section \ref{impl.algorithms}, the algorithms used to solve a \modelimpl problem, and resolve a \modelname instance are described.
Section \ref{impl.criteria} presents and defies the mapping of selected criteria.
Finally, in section \ref{impl.gjsolver} GJSolver is discussed, followed by the description of its verification through MISC.

\section{Boolean Lexicographic Optimization Problem}
\label{impl.mapping}
A \modelimpl problem consists of trying to assign a value to a set of Boolean variables such that is optimises some lexicographic criteria.
Additionally the possible assignments are restricted because they must satisfy various constraints, either Boolean disjunctions, of PB constraints.
In this thesis each criterion is defined using a PB function, and the constraints are defined using a SAT+PB formula.
This section defines both PB criteria and SAT+PB formula, and describes the mapping from a \modelname instance's evolution step to a \modelimpl problem.

\subsection{Boolean Satisfiability Problem (SAT)}
Boolean satisfiability (SAT)  is the problem of determining if the variables in a Boolean equation can be assigned in such a way that the equation returns true.
SAT was the first identified NP-Complete problem \citep{cook1971}, meaning there is no known algorithm that efficiently solves all instances of SAT problems.
The fundamental difficulty of SAT problems, 
combined with the ability to map many problems to SAT, has spawned a community\footnote{http://www.satcompetition.org/ accessed 6/3/2012} 
dedicated to creating, enhancing, and testing various SAT solver implementations. 
SAT solvers have been used in various domains to address problems such as electronic design automation \citep{Marques-Silva2000}, 
model verification \citep{dennis2006}, and component system evolution \citep{leBerre2010}.

A common representation of a SAT formula is in Conjunctive Normal Form (CNF).
CNF is defined as a conjunction of clauses, 
where each clause is a disjunction of literals, e.g. $(a \vee b) \wedge (\neg b \vee c)$.

A SAT problem in CNF is defined as:
\begin{defs}
\label{impl.defSAT}
{\ }
\begin{enumerate}
    \item Let $V$ be a set of variables
    \item A \textbf{literal} is a variable $v$ or its negation $\neg v$.
    \item Given a set of literals $P$, $\neg P := \{\neg v \mid v \in P\}$
    \item A literal $\neg \neg v := v$
    \item A \textbf{clause} is a set of literals 
    \item A \textbf{formula} is a set of clauses \label{impl.whatisaformula}
    \item A \textbf{SAT problem} is a set of variables $V$ and a formula $F$
    \item A set of literals is \textbf{consistent} if for any variable $v$, the set of literals does not contain both $v$ and its negation $\neg v$.
    \item A clause $\mathcal{C}$ is \textbf{satisfied} by a set of literals $P$ if there exists a literal in $\mathcal{C}$ that is also in $P$.
    \item A \textbf{solution} to a formula $F$ is a consistent set of literals $P$, such that for every clause $\mathcal{C}$ in $F$, $\mathcal{C}$ is satisfied by $P$.
    \item A \textbf{partial solution} to $F$ is a subset of any solution, i.e. given $P$ is a solution, $P'$ is a partial solution iff $P' \subseteq P$.
    \item A SAT problem is \textbf{satisfiable} if there exists a set of literals $P$ that is a solution to $F$, otherwise the instance of the SAT problem is \textbf{unsatisfiable}. 
\end{enumerate}
\end{defs}

For example, consider a SAT problem where $V = \{a,b,c\}$ and $F = \{\mathcal{C}_1,\mathcal{C}_2\}$, where clauses $\mathcal{C}_1 = \{a,b\}$ and $\mathcal{C}_2 = \{\neg b, c\}$.
A solution for this problem could be $\{a,\neg b,c\}$ as $a \in \mathcal{C}_1$ and $\neg b \in \mathcal{C}_2$.
However, $\{a, \neg b, b\}$ is not a solution because it is not consistent, and $\{a, b, \neg c\}$ is not a solution because it does not contain a literal in $\mathcal{C}_2$.  


\subsubsection{Pseudo-Boolean Extension of SAT to SAT+PB}
A typical extension of the SAT problem is the inclusion of pseudo-Boolean constraints \citep{dixon2004automating} into SAT formula.
This extends the SAT problem to a SAT+PB problem.

A pseudo-Boolean function takes a set of Boolean literals and returns an integer, e.g. given a set of literals $P$, $f(P) = 4$.
\begin{defs}
\label{impl.PBfunction}
Given a tuple of literals $\langle l_1,\ldots,l_n \rangle$ and a tuple of integers $\langle a_1,\ldots,a_n\rangle$,
a \textbf{pseudo-Boolean function} takes a set of literals $P$, and returns an integer such that:

$f(P) = \sum \limits_{i=0}^n f_i(P)$
where $f_i(P) = \begin{cases} a_i & l_i \in P \\ 0 & l_i \not \in P \end{cases}$
\end{defs}
For example, consider the pseudo-Boolean function $f$ defined with a tuple of literals $\langle x_1, \neg x_2\rangle$ and a tuple of integers $\langle 1,3\rangle$.
The value of $f(\{x_1\})$ will equal $1$, $f(\{\neg x_2\})$ will equal $3$, and $f(\{x_1,\neg x_2\})$ equals $4$.

A pseudo-Boolean constraint is a relation between a PB function and an integer, e.g. $f(P) \leq 3$.
\begin{defs}
\label{impl.PBConstraint}
A \textbf{pseudo-Boolean constraint} is a tuple of a pseudo-Boolean function $f$, a relationship  $R$ in $\{\geq,>,\leq,<,=\}$, and a positive integer $k$, 
a PB constraint is $\langle f,R,k\rangle$.
\end{defs}

\begin{defs}
\label{impl.PBSATis}
A pseudo-Boolean constraint $\langle f,R,k\rangle$ is \textbf{satisfied} by a set of literals $P$ iff $ f(P)$ $R$ $k$.
\end{defs}
For example, the pseudo-Boolean function $f$ described in the above example can be combined with a relation $>$ and integer $2$ to create the constraint $\langle f, >, 2 \rangle$.  
This constraint will be satisfied by the set of literals $P_1 = \{ x_1, \neg x_2 \}$ as $f(P_1) = 4$ and $4 > 2$, 
but not with the set of literals $P_2 = \{ x_1,  x_2 \}$ as $f(P_2) = 1$  and $ 1 \not > 2$.

A SAT problem can be extended to include pseudo-Boolean constraints by first including definitions \ref{impl.PBfunction}, \ref{impl.PBConstraint} and \ref{impl.PBSATis}. 
Secondly, by modifying the definition \ref{impl.defSAT}, specifically what a SAT formula can contain and what a solution to a SAT problem is:
\textit{
\begin{enumerate}
\setcounter{enumi}{5}
    \item A \textbf{formula} is a set of clauses and pseudo-Boolean constraints
\end{enumerate}
\begin{enumerate}
\setcounter{enumi}{9}
    \item A \textbf{solution} to a formula $F$ is a consistent set of literals $P$, such that for every clause $\mathcal{C}$ in $F$, $\mathcal{C}$ is satisfied by $P$,
    and for every pseudo-Boolean constraint $pb$ in $F$, $pb$ is satisfied by $P$. 
\end{enumerate}
}

To illustrate how the modified definition from SAT to SAT+PB problem works, 
consider a SAT+PB problem where $V = \{a,b,c\}$ and $F = \{\mathcal{C}_1,\mathcal{C}_2, \langle f, <, 2 \rangle\}$. 
Clauses $\mathcal{C}_1 = \{a,b\}$ and $\mathcal{C}_2 = \{\neg b, c\}$, 
and the function $f$ is defined with the tuple of literals $\langle a, b\rangle$ and a tuple of integers $\langle 1,3\rangle$.
A solution for this problem could be $\{a,\neg b,\neg c\}$ as $a \in \mathcal{C}_1$, $\neg b \in \mathcal{C}_2$, and $f(\{a,\neg b,\neg c\}) < 2$.
However, $\{\neg a, b, c\}$ is not a solution because $f(\{\neg a, b, c\}) \not < 2$. 

Pseudo-Boolean constraints can be translated into CNF clauses as described by \cite{dixon2004automating}.
However, a pseudo-Boolean constraint can be exponentially more concise, and more quickly solved \citep{dixon2004automating} than the translated clauses.
Additionally, satisfying PB constraints may only require minor amendments to existing algorithms (some such amendments are described by \cite{Sheini2006}).
For these reasons, the \modelimpl problem is defined to allow SAT formula with PB constraints (SAT+PB).

\subsection{\modelimpl Problem}
\label{impl.lexsatproblem}
In order to create a \modelimpl problem, some criteria used to define what properties an optimal solution have are required.
A pseudo-Boolean criterion is defined to try either maximise of minimise the value of a PB function.
By composing such criteria together into a list in order of lexicographic preference, an optimal solution to a \modelimpl problem can be defined.

\begin{defs}
A \textbf{pseudo-Boolean criterion} $\mathfrak{crit}$ is a tuple consisting of a pseudo-Boolean function $f$,
a relation over integers $R$ that is either $<$ or $>$, and a SAT+PB formula $I$,
i.e. $\mathfrak{crit} = \langle f, R , I \rangle$.
\end{defs}
The formula $I$ is used to define auxiliary variables.
Such variables are used to express aspects of the problem that are not directly defined in the problems formula. 
For example, if a user wanted to minimise the number different software licences in a component system (to potentially lower legal fees), 
then each licence would be represented by an auxiliary variable, and the PB function is defined using the licence auxiliary variables.
The formula $I$ should not restrict the solution space of the formula from the problem.
Auxiliary variables have been used before in similar implementations \citep{argelich2010solving}.
An example using auxiliary variables is presented in section \ref{impl.mappingexample}.

\modelimpl uses a list of PB criteria ordered by lexicographic preference to find an optimal solution.
That is, the most preferred criterion to optimise is at the start of the list, and the least at the end. 
The lexicographic order is defined : $(a,b)$  is lexicographically better than $(a',b')$ iff $a$ is better than $a'$ or $(a$ equals $a'$ and $b$ is better than $b')$.
These terms are translated to the \modelimpl domain.
\begin{defs}
Given a SAT+PB formula $F$, a pseudo-Boolean criteria $\langle f,R, I \rangle$, and two sets of literals $P$ and $P'$, 
$P$ and $P'$ are \textbf{equal} w.r.t. $\langle f,R, I \rangle$
iff $P$ and $P'$ are solutions to $F \cup I$ and $f(P) = f(P')$.
\end{defs}
Note that both $P$ and $P'$ must be solutions to the formula $F \cup I$, not just the formula $F$.
This is done to ensure that the auxiliary variables required by the criteria are available.

\begin{defs}
Given a SAT+PB formula $F$, a pseudo-Boolean criteria $\langle f,R, I \rangle$, and two sets of literals $P$ and $P'$, 
$P$ is \textbf{better than}  $P'$ w.r.t. $\langle f,R, I \rangle$
iff $P$ and $P'$ are solutions to $F \cup I$ and $f(P')$ $R$ $f(P)$.
\end{defs}
The relation $R$ that is either $<$ or $>$, is therefore used to either maximise or minimise the PB function.
It is the used as the primary definition of what solution is better than another.

These definitions are used to define the lexicographic order:
\begin{defs}
Given a formula $F$, a tuple of PB criteria $\langle \mathfrak{crit}_1,\ldots,\mathfrak{crit}_n \rangle$, and two sets of literals $P$ and $P'$,
$P$ is \textbf{lexicographically better than}  $P'$ w.r.t. to $\langle \mathfrak{crit}_1,\ldots,\mathfrak{crit}_n \rangle$
iff there exists an $i$ between $1$ and $n$ where for all $j < i$, $P$ is equal to $P'$ w.r.t. to $\mathfrak{crit}_j$ and $P$ is better than $P'$ w.r.t. $\mathfrak{crit}_i$.
\end{defs}
This means that for a solution to be lexicographically better than another, 
it must be better w.r.t. to a PB criterion and at least equal to all criteria that are before it in the tuple of criteria. 

This lexicographic order is then used to define an optimal solution of a \modelimpl problem:
\begin{defs}
Given a formula $F$ and lexicographically ordered PB criteria $\langle \mathfrak{crit}_1,\ldots,\mathfrak{crit}_n \rangle$
an \textbf{optimal solution} is a solution $P$ to $F$ 
where no other solution $P'$ to $F$ exists such that $P'$ is lexicographically greater than $P$ w.r.t. to $\langle \mathfrak{crit}_1,\ldots,\mathfrak{crit}_n \rangle$. 
\end{defs}

This means that given a \modelimpl problem that consists of:
\begin{itemize}
  \item a tuple of PB criteria $\langle \mathfrak{crit}_1,\ldots,\mathfrak{crit}_n \rangle$
  \item a SAT+PB formula $F$
\end{itemize}
A solution to \modelimpl problem is the optimal solution to $F$ w.r.t. to its PB criteria.

\subsection{Mapping \modelname Instance to \modelimpl Problem}
\label{impl.mapping}
Following from the definition of \modelname in section \ref{formal.step}, 
an instance of \modelname consists of a series of evolution steps at time $t_i$, where $i$ is from $1$ to $n$.
Each step consists of:
\begin{itemize}
  \item a time $t_i$
  \item the set of components $\mathbb{C}_{t_i}$ 
  \item an evolution problem $\delta_{t_i} \cup \omega_{t_i}$
  \item an evolution preference order $\prec_{\alpha_{t_{i-1}}}$
  \item a previous system $\alpha_{t_{i-1}}$.
\end{itemize}

To resolve a \modelname instance each step is mapped to a \modelimpl problem which consists of a SAT+PB formula $F$ 
and PB criteria $\langle \mathfrak{crit}_1,\ldots,\mathfrak{crit}_n \rangle$.

The set of components $\mathbb{C}_{t_i}$ are variables in the problem, i.e $\mathbb{C}_{t_i} \subseteq V_{t_i}$.
Not all variables in the problem are components, as auxiliary variables may be required.

\begin{defs}
	A component system $\alpha_{t_{i-1}}$ is mapped to a set of literals where
   $\alpha_{t_{i-1}} := \alpha_{t_{i-1}} \cup \{\neg c \mid c \in \mathbb{C}_{t_i}$ and $ c \not \in \alpha_{t_{i-1}}\}$.
\end{defs}
A component system $\alpha_{t_{i-1}}$ is mapped to a set of literals that are positive if they are in the system or negative if they are not.

The reverse mapping, from a set of literals to a component system, is:
\begin{defs}
A set of literals $P$ is mapped to a component system $\beta$ such that $\beta := \{c \mid c \in \mathbb{C}_t$ and $x \in P\}$
\end{defs}
A component system is the set of components that are not negative in the set of literals.  
Note: the mapping from literals to components is surjective, and the mapping from components to literals is injective.


The evolution problem $\delta_{t_i} \cup \omega_{t_i}$ is mapped to the formula $F$
by mapping each constraint to a SAT clause or PB constraint:
\begin{enumerate}
  \item \textbf{Exclusion}: $\neg a := \{\neg a\}$
  \item \textbf{Conflict}: $a \rightarrow \neg c := \{\neg a, \neg c\}$ 
  \item \textbf{Inclusive Disjunction}: $a_1 \vee \ldots \vee a_n := \{a_1, \ldots,  a_n\}$ 
  \item \textbf{Dependence}: $a \rightarrow c_1 \vee \ldots \vee c_n := \{\neg a, c_1, \ldots, c_n\}$
  \item \textbf{Exactly One}: $a_1 + \ldots + a_n = 1 := $ a pseudo-Boolean constraint $\langle f,=, 1 \rangle$, 
  where $f$ is defined with the tuple of literals $\langle a_1 ,\ldots , a_n\rangle$ and the tuple of natural numbers $\langle 1_1,\ldots,1_n \rangle$.
\end{enumerate}


\subsection{Evolution Preference Order Mapping}
\label{impl.mappingexample}
The evolution preference order $\prec_{\alpha_{t_{i-1}}}$ (as defined in chapter \ref{formal}) defined with a lexicographic composition of criteria $crit_{1} \oplus \ldots \oplus crit_{n}$ 
can be mapped to a tuple of PB criteria $\langle \mathfrak{crit}_1,\ldots,\mathfrak{crit}_n\rangle$
by mapping each criterion $crit_{i}$ to a PB criterion $\mathfrak{crit}_i$.

A criterion $\langle rank_{\alpha} ,\leq \rangle$ maps to a PB criteria $\langle f, R , I \rangle$ iff:
\begin{itemize}
  \item $R$ equals the strict order of $\leq$.
  \item given a solution $P$ to formula $I$, and $P$ maps to the component system $\beta$, $f(P) = rank_{\alpha}(\beta)$. 
\end{itemize} 

To map an evolution preference order to the PB criteria,
it must be defined with the lexicographically composed criteria.
To map a criterion to a PB criterion, it must satisfy the above constraints.
This means that:
\begin{itemize}
  \item not all evolution preference orders can be described with lexicographically ordered PB criteria
  \item not all criteria can be mapped to PB criteria
\end{itemize} 
Therefore, the mapping from evolution preference orders to PB criteria is a partial mapping.

The partial mapping from \modelname criteria to PB criteria is not presented here.
This is because many of the \modelname criteria that can be defined are not useful, 
e.g. a criteria that maximises the number of components whose name starts with the letter \texttt{a}.
Only specific criteria are mapped to PB criteria, these are presented in the next section. 

To further illustrate the mapping from a \modelname criteria to a PB criteria, an example is presented.
Consider two components $a$ and $b$, a criterion $\langle rank_{\alpha},< \rangle$ whose ranking function is defined as:

$rank_{\alpha}(\beta) = \begin{cases} 1 & a \in \beta \text{ or } b \in \beta\\ 0 & \text{otherwise} \end{cases}$

This criterion expresses the preference of having either components $a$ or $b$ (or both) in the system.

Further consider the pseudo-Boolean criterion $\langle f, < , I \rangle$.
The auxiliary variable $x$ is defined such that $x \Leftrightarrow a \in \beta \vee b \in \beta$.
This variable must be  converted to the set of CNF clauses, and included in $I$,
i.e. $I = \{\{\neg x, a,b\}, \{\neg a,x\}, \{\neg b, x \}\}$.
The PB function $f$ is defined with the tuple of literals $\langle x \rangle$ and natural numbers $\langle 1 \rangle$.

The criterion $\langle rank_{\alpha}, \leq \rangle$ maps to $\langle f, < , I \rangle$,
as the strict order pf $\leq$ from the criterion equals the order $<$ in the PB criterion.
Given a solution $P$ to $I$, where $P$ maps to the component system $\beta$, $f(P) = rank_{\alpha}(\beta)$.
This can is shown in table \ref{impl.critmapexmp}.
\begin{table}[h!]
\centering
\begin{tabular}{| c | c | c | c |}
\hline
$P$                                &    $\beta$            & $rank_{\alpha}(\beta)$     & $f(P)$\\ \hline    
$\{\neg a, \neg b, \neg x\}$     & $\{\}$                & 0                        & 0 \\
$\{\neg a,  b, x\}$             & $\{b\}$                & 1                        & 1 \\
$\{ a,  \neg b, x\}$             & $\{a\}$                & 1                        & 1 \\
$\{ a,  b, x\}$                 & $\{a,b\}$                & 1                        & 1 \\ \hline
\end{tabular}
\caption{Values to show $f$  maps to $rank_{\alpha}$}
\label{impl.critmapexmp}
\end{table}

The above example describes how a \modelname criterion can be mapped to a PB criterion.
In the following section, some specific criteria are presented and mapped.

\section{Criteria}
\label{impl.criteria}
This section presents a selection of criteria and their mapping to between \modelname criteria, PB criteria, and MOF.
They are presented in two general categories, minimising change to the system and minimising the out-of-dateness of the system.
The minimising change criteria were developed by Mancoosi for MISC\footnote{http://www.mancoosi.org/misc-2011/criteria/ accessed 6/3/2012} 
and the criterion used to minimise the out-of-dateness was developed for Eclipse P2 solver \citep{leBerre2010}.

\subsection{Change Criteria}
Simple definitions of criteria to minimise change can often have negative effects during CSE.
For example, the most direct measurement of change of a component system is the measurement of the total changed components.
This is defined as a \modelname criterion:
\begin{defs}
	The \textbf{changed components} criteria is defined as $crit_{changec} = \langle rank^{changec}_{\alpha}, \leq \rangle$,
	where $rank^{changec}_{\alpha}(\beta) = |\alpha \Delta \beta|$.
\end{defs}
That is, the number of components in the symmetric difference between component systems is the minimised with this criterion.
This criterion's measurements are an inaccurate representation of the risks of changing a system.
For example, a component being replaced by another version of itself is intuitively less risky than it being replaced with an entirely different component.  
Yet using the $crit_{changec}$ criterion, these would be seen as equivalent changes.

To define a change criterion that takes into account changing between components of the same name, is less risky, a components name can be considered.
For this purpose the function $V$ is defined:
\begin{defs}
The function $V: 2^{\mathbb{C}} \times \mathcal{N} \rightarrow 2^{\mathbb{C}}$ takes a set of components $\alpha$ and a component name $n$, and returns a set of components with name $n$ that are in $\alpha$,
i.e. $V(\alpha,n) = \{\langle n',v \rangle \mid \langle n',v \rangle \in \alpha $ and $ n' = n \}$
\end{defs}

This function can be used to define a criterion that considers the name of a component:
\begin{defs}
	The \textbf{change} criterion is defined as $crit_{change} = \langle rank^{change}_{\alpha}, \leq \rangle$,
	where $rank^{change}_{\alpha}(\beta) = |\{n \mid n \in \mathcal{N}$ and $V(\alpha,n) \neq V(\beta,n) \}|$.
\end{defs}

This criterion can be altered to also consider new and removed component names:
\begin{defs}
	The \textbf{new} criterion is defined as $crit_{new} = \langle rank^{new}_{\alpha}, \leq \rangle$,
	where $rank^{new}_{\alpha}(\beta) = |\{n \mid n \in \mathcal{N}$ and $V(\alpha,n) = \emptyset$ and $V(\beta,n) \neq \emptyset\}|$.
\end{defs}

\begin{defs}
	The \textbf{removed} criterion is defined as $crit_{removed} = \langle rank^{removed}_{\alpha}, \leq \rangle$,
	where $rank^{removed}_{\alpha}(\beta) = |\{n \mid n \in \mathcal{N}$ and $V(\alpha,n) \neq \emptyset$ and $V(\beta,n) =\emptyset\}|$.
\end{defs}

The change, new and remove criteria were taken directly from the MISC competitions definition of criteria.

The mapping between these criteria, PB criteria and MOF are presented in table \ref{impl.ccritmapping}.
A full description of the mapping is presented in appendix \ref{apx.critmapping}.
\begin{table}
\begin{tabular}{c | c | c}
\textbf{MOF name} 		& \textbf{\modelname criterion} & \textbf{PB criterion} \\
\texttt{-changed} 	& $crit_{change} = \langle rank^{change}_{\alpha}, \leq \rangle$ & $\langle f_{change}, <, I_{changed} \rangle$ \\
\texttt{-removed} 	& $crit_{removed} = \langle rank^{removed}_{\alpha}, \leq \rangle$ & $\langle f_{removed}, <, I_{removed} \rangle$ \\
\texttt{-new} 	& $crit_{new} = \langle rank^{new}_{\alpha}, \leq \rangle$ & $\langle f_{new}, <, I_{new} \rangle$ \\
\end{tabular}
\caption{Mapping between these elements}
\label{impl.ccritmapping}
\end{table}

\subsubsection{Out-of-date criteria}
When changing a component system it is a good idea to select more recent versions of components, as they keep the system from becoming out-of-date.
However, it is a difficult task to define an appropriate measurement of how out-of-date a component system is.
This difficulty comes from two properties of component versions must be considered:
\begin{itemize}
  \item A version of a component can only be compared to version of another component with the same name, 
  i.e. given two components $\langle n,v \rangle$ and  $\langle n',v' \rangle$ comparing $v$ and $v'$ is only useful if $n = n'$.
  \item The sum of versions is not a useful metric as two lesser versions will not be better than one greater version, 
  e.g. given components $\langle n,2 \rangle$,  $\langle n,3 \rangle$ and  $\langle n,4 \rangle$,
  a system with only $\langle n,2 \rangle$ and $\langle n,3 \rangle$ is not better than a system with just $\langle n,4 \rangle$. 
\end{itemize}
Not considering these properties when defining criteria could lead to significant problems.

A useful criterion has been defined in Eclipse P2 that minimises a measurement of the out-of-dateness of a component name.
This out-of-dateness measures the amount of components that exist that have greater versions, and have a component with the same name installed.
It is defined as:
\begin{defs}
The function $uptodatedistance$ takes a component $\langle n, v \rangle$ and a set of components $\mathbb{C}_t$ and returns the number of components with the same name and a greater version,
i.e. $uptodatedistance(\langle n, v \rangle,\mathbb{C}_t) = |\{\langle n, v' \rangle \mid \langle n, v' \rangle \in \mathbb{C}_t $ and $ v' > v \}|$
\end{defs}

A criterion using this measurement can then be defined:
\begin{defs}
	Given the set of components $\mathbb{C}_t$, the \textbf{uptodate distance} criterion is defined as $crit_{utdd} = \langle rank^{utdd}_{\alpha}, \geq \rangle$,
	where $rank^{utdd}_{\alpha}(\beta) = \sum_{c \in \beta} uptodatedistance(c,\mathbb{C}_t)$.
\end{defs}
That is, the measurement to be minimised is the number of components that have the same name and a greater version than a component that is currently installed. 

The mapping between these elements is presented in table \ref{impl.ccritmapping}.
A full description of the mapping is presented in appendix \ref{apx.critmapping}.
\begin{table}
\begin{tabular}{c | c | c}
\textbf{MOF} 		& \textbf{\modelname criterion} & \textbf{PB criterion} \\
\texttt{-uptodatedistance} 	& $crit_{utdd} = \langle rank^{utdd}_{\alpha}, \geq \rangle$ & $\langle f_{utdd}, <, I_{utdd} \rangle$ \\
\end{tabular}
\caption{Mapping between these elements}
\label{impl.ccritmapping}
\end{table}


\section{Solving a \modelimpl problem}
\label{impl.algorithms}
The efficient solving of \modelimpl problems will allow the efficient resolving of \modelname instances.
This section describes the lexicographic-iterative-strengthening-algorithm (LIS) to solve \modelimpl problems.
This algorithm uses the DPLL algorithm \citep{Davis1960, davis1962machine} to find a solution to a SAT+PB formula,
and the iterative strengthening algorithm \citep{calistri1994iterative, le2010sat4j} to find an optimal solution to a SAT+PB problem given a single PB criterion.
This section also describes how a \modelname instance is resolved by mapping to \modelimpl problems.
The algorithms presented in this section are described to give an overview of how \modelimpl problems were solved.

\subsection{Davis-Putnam-Logemann-Loveland algorithm for SAT Solvers}
%%%A successful algorithm for solving SAT problems is the DPLL algorithm, here we describe it in overview
The Davis-Putnam-Logemann-Loveland (DPLL) algorithm \citep{Davis1960, davis1962machine} is a complete (meaning it will find a solution if one exists), 
backtracking-based search algorithm for solving SAT and SAT+PB problems.


DPLL takes a formula $F$ and a set of literals $P$ (described as a partial assignment), and returns a solution to $F$ if $P$ is a partial solution, 
otherwise returning \verb+UNSATISIFABLE+.
When \texttt{DPLL} is called without a value $P$, $P$ is defaulted to equal the empty set .
By first calling \texttt{DPLL} with $P$ as the empty set, then adding literals to $P$ and recursively calling itself; 
the DPLL function searches for a solution to show the formula is satisfiable.
The DPLL algorithm in defined in figure \ref{impl.DPLL} (a slight modification of the algorithm presented in \citep{dixon2004automating}):
\begin{figure}[h]
\begin{center}
\begin{alltt}
function DPLL(\(F, P\)):
   \(P\) = unit-propagate(\(F, P\))
   if \(P\) is not consistent:
       then return UNSATISIFABLE;
   if \(P\) is a solution to \(F\):
       then return \(P\);
   \(l\) = decide\((P)\);
   \(answer\) = DPLL\((F, P \cup \{l\})\)
   if answer != UNSATISIFABLE 
       return answer
   else
       return DPLL\((F, P \cup \{\neg l\})\);
\end{alltt}
  \caption{Recursive DPLL algorithm}
  \label{impl.DPLL}
\end{center}
\end{figure}

\texttt{DPLL} first calls the \texttt{unit-propagation} function (further described later in this section) which derives literals that must be in $P$ if it is a solution.
Next DPLL checks whether $P$ is inconsistent, which means it is not a partial assignment.
Then DPLL checks if $P$ is a solution to $F$, if it is then $P$ is returned.
The \texttt{decide} function (further described later in this section) returns a literal $l$ that is not, nor whose negation is in $P$.
The literal $l$ is added to $P$, which is then checked to be a partial solution by recursively calling \texttt{DPLL}.
If $P$ with $l$ is a solution then the found solution is returned,
otherwise the search continues by adding $\neg l$ to $P$ and checking if it is a partial solution by calling \texttt{DPLL}.

\subsubsection{Unit Propagation}
\label{impl.unit}
The first line in the  DPLL algorithm calls the \texttt{unit-propagation} function.
This function uses the clauses in the formula, and the partial assignment to identify and add literals to $P$ that must be included if $P$ is to be a partial solution.

\begin{defs}
Given a partial assignment $P$, a clause $\mathcal{C}$ is called \textbf{unit} iff $\mathcal{C}$ is not satisfied by $P$, and $P$ contains all but one of the literals in $\neg \mathcal{C}$.
The literal whose negation is not in $P$ is called a \textbf{unit literal}. 
\end{defs}

For example, a clause $\{a,b,c\}$ is unit if the partial assignment contains $\neg b$ and $\neg c$ but neither $a$ or $\neg a$.
The literal $a$ is then a unit literal.

For a formula to be satisfiable given partial assignment, each unit literal must be included in the partial assignment,
because if their negation is included the clause is not satisfied by the partial assignment.
For example, given a formula $\{\mathcal{C}\}$, where $\mathcal{C} = \{a,b\}$;
given the assignment $\{\neg a\}$ the clause $\mathcal{C}$ is unit and unit literal is $b$.
If $\neg b$ were in the partial assignment, $\mathcal{C}$ would not be satisfied by $P$.
Therefore, $b$ must be in $P$ for $\mathcal{C}$ to be satisfied.

The process of unit propagation is defined in figure \ref{impl.propagation}.
\begin{figure}[htp]
\begin{center}
\begin{alltt}
unit-propagate(\(F, P\)):
  while \(P\) is consistent and there exists a \(\mathcal{C} \in F\) that given \(P\) is unit:
    \(l\) = unit literal in \(\mathcal{C}\)
    \(P\) = \(P \cup \{l\}\)
  return \(P\)
\end{alltt}
  \caption{Pseudo code of Unit Propagation}
  \label{impl.propagation}
\end{center}
\end{figure}

\subsubsection{Decide}
\label{impl.litorder}
The function \verb+decide+ takes a partial assignmnet $P$ and returns a literal $l$ such that $l \not \in P$ and $\neg l \not \in P$.
That is, if $l =$ \verb+decide+$(P)$, then $\{l\} \not \in P$ and $\{\neg l\} \not \in P$.
This literal is the point which the algorithm branches.
This \verb+decide+ function greatly impacts the efficiency of DPLL, as selecting literals that are in a solution (if one exists) would quickly find a result.

\subsubsection{DPLL Advancements}
Though the DPLL algorithm is the basis of many modern SAT solvers, the actual implementations have been altered to increase efficiency.
Some of these improvements are:
\begin{itemize}
  \item Conflict learning \citep{stallman1976,sorensson2009} is a technique to cache previously tried sets of assignments in order to stop re-solving the same sub-problems.
	\item Backjumping \citep{Gaschnig1979} is the technique which determines how far to up the search tree to backtrack when a conflict is found.
The higher up the tree the technique ``jumps'' to, the greater reduction of the search space.
\item  More efficient unit-propagation through watched literals \citep{Madigan2001,Moskewicz2001}.
\end{itemize}
These advancements make current SAT solvers very efficient.

\subsection{Iterative Strengthening}
The iterative strengthening algorithm  \citep{calistri1994iterative, le2010sat4j} 
is an anytime algorithm using constraint satisfaction to iteratively find better solutions to a SAT+PB formula w.r.t. a PB criterion.
This algorithm can be used to find optimal solutions to an evolution problem, given a pseudo-Boolean criterion.
This is done by first finding a solution, then iteratively adding constraints (created using the criterion) to ensure the next solution found will be better than the previous solution.
This is the strengthening process.
Strengthening continues until either the strengthened formula is found to be unsatisfiable, or the algorithm is interrupted, at which point the best solution currently found is returned. 
This algorithm is defined in figure \ref{impl.strength}.

\begin{figure}[htp]
\begin{center}
\begin{alltt}
iterative-strengthening(\(F\),\(\langle f, R, I \rangle\)):
    \(F\) = \(F \cup I\)
    \(answer\) = DPLL(\(F\))
    if \(answer\) = UNSATISIFABLE:
        return UNSATISIFABLE
    do:
        \(model\) = \(answer\)
        \(J\) = strengthen(\(model\),\(\langle f, R, I \rangle\))
        \(F\) = \(F \cup J\)
        \(answer\) = DPLL(\(F\))
    while not (interrupted() or  \(answer\) == UNSATISIFABLE)
    return \(model\) 
\end{alltt}
  \caption{Pseudo code of Iterative Strengthening Algorithm}
  \label{impl.strength}
\end{center}
\end{figure}

The first action in the \texttt{iterative-strengthening} is to include the formula $I$ that defines auxiliary variables used by the criterion.
This ensures that any solution returned by DPLL is also a solution to $I$.

The next action is to check if the formula is satisfiable.
This is accomplished by passing the formula to the \texttt{DPLL} function, and assigning its output to the variable $answer$.
If the output from \texttt{DPLL} is \texttt{UNSATISIFABLE}, then the algorithm stops and returns \texttt{UNSATISIFABLE}, as there are no solutions.

The main loop of this algorithm is then defined.
The first action in this loop is to assign the contents of the variable $answer$ to the variable $model$.
The variable $model$ is a set of literals, used as a store of the currently best found solution.

The function \texttt{strengthen} is then called to create a formula $J$.
The formula $J$ ensures that if \texttt{DPLL} is called with formula $F \cup J$, either:
\begin{itemize}
  \item a solution is returned that is better (w.r.t. the criterion) than the currently best found solution $model$. 
  \item \texttt{UNSATISIFABLE} is returned, showing that no better solution exists.
\end{itemize}

\begin{defs}
Given a set of literals $model$, and a criterion $\langle f, R, I \rangle$,
the function \texttt{strengthen} returns a formula $J$ consisting of a single pseudo-Boolean constraint, i.e. $ J = \{ \langle f,R,f(model) \rangle \}$.
\end{defs}
That is, \texttt{strengthen} returns a formula $J$,
that ensures any solution $P$, to the formula $F \cup J$, must have a better ( w.r.t. $R$) value of $f(P)$ than the previously best solutions value of $f(model)$.

The next steps are then to add the formula $J$ to $F$, then search for a new solution using \texttt{DPLL}.

The main loop will end given the condition that either the \texttt{interrupted} function returns $true$, or the \texttt{DPLL} function returns \texttt{UNSATISIFABLE}.
The \texttt{interrupted} method is typically defined to return false until some external input (like a user stopping the algorithm or a timer running out) is encountered.
The \texttt{interrupted} method has the additional responsibility of stopping the \texttt{DPLL} function, 
so that if at any point \texttt{interrupted} returns true the \texttt{DPLL} function immediately returns.
When the main loop ends, the currently best found solution $model$ is returned.

\subsection{Lexicographic Optimisation}
\label{impl.lexiterstre}
The iterative strengthening algorithm can be used find lexicographically optimal solutions to a tuple of criteria by iteratively strengthening each criteria in order.
This lexicographic-iterative-strengthening algorithm is presented in figure \ref{impl.lexstrength}.

\begin{figure}[htp]
\begin{center}
\begin{alltt}
lexicographic-iterative-strengthening(\(F\),\(\langle{}\mathfrak{crit}\sb{1}, \ldots ,\mathfrak{crit}\sb{n}\rangle\)):  
    \(answer\) = DPLL(\(F\))
    if \(answer\) = UNSATISIFABLE:
        return UNSATISIFABLE
    \(i\) = 0
    \(model\) = \(answer\)
    do:
        \(i = i + 1\)
        \(model\) = iterative-strengthening(\(F\),\(\mathfrak{crit}\sb{i}\))
        \(K\) = lock(\(model\),\(\mathfrak{crit}\sb{i}\))
        \(F\) = \(F \cup K\)
    while not (interrupted() or \(i\) == \(n\))
    return \(model\) 
\end{alltt}
  \caption{Pseudo code of the Lexicographic Iterative Strengthening Algorithm}
  \label{impl.lexstrength}
\end{center}
\end{figure}

The first action of the \texttt{lexicographic-iterative-strengthening} algorithm is to check if the formula is satisfiable using the \texttt{DPLL} algorithm.
If it is unsatisfiable, this algorithm returns \texttt{UNSATISIFABLE}.

The variable $i$ is then defined, this variable is a counter used to select the criterion to be strengthened.
Also, $model$ is assigned to be the set of literals $answer$, and is used to store the currently best found solution.

The main loop of this algorithm is then defined.
This loop first increments the counter $i$, to select the appropriate criterion to be strengthened.

The \texttt{iterative-strengthening} algorithm is called to find an optimal solution to $F$ given the criterion $\mathfrak{crit}_i$.
As the formula $F$ is known at this point in the algorithm to be satisfiable, the only possible returned value from \texttt{iterative-strengthening}
is a solution that is no worse than $model$.
This means that the returned value could be equivalent to the previously defined solution, if there exists no better solution than what has already been found.

The function \texttt{lock} is then called to return a formula $K$.
The formula $K$ ensures that if \texttt{DPLL} is called with $F \cup K$ a solution is returned that is not worse (w.r.t. the criterion $\mathfrak{crit}_i$) than the solution $model$. 
\begin{defs}
Given a set of literals $model$, and a criterion $\langle f, R, I \rangle$,
the function \texttt{lock} returns a formula consisting of a single pseudo-Boolean constraint, $\{ \langle f,=,f(model) \rangle \}$,
\end{defs}
That is, \texttt{lock} returns a formula $K$,
that ensures any solution $P$, to the formula $F \cup K$, must have the best value of $f$ found so far.
Note that $K$ does not effect the satisfiability of $F$, as $model$ is still a valid solution to $F \cup K$.
The formula $K$ is then added to the formula $F$, i.e. $F = F \cup K$,

The main loop will iterate until either all the criteria have been optimised, or the function is interrupted.
When the loop ends, the currently best found solution $model$ will be returned.

Some enhancements to the implementation of the lexicographic iterative strengthening algorithm can be made.
For example, when the \texttt{iterative-strengthening} is called, it is known that $F$ is satisfiable. 
Therefore, checking its satisfiability again within the \texttt{iterative-strengthening} function is not necessary.

The lexicographic iterative strengthening algorithm is an anytime algorithm.
It has been designed to return a solution to a \modelimpl problem, even if it is interrupted.
The reason for this anytime behaviour is that \modelimpl problems can take an impractical amount of time to solve.
It is therefore practically necessary to limit the time this algorithm searches for an optimal solution, 
and interrupt it when this time limit is reached. 

\subsection{Resolving a \modelname instance}
Resolving a \modelname instance involves finding the series of component systems $\alpha_{t_1},\ldots,\alpha_{t_n}$.
To do this, each evolution step starting at $t_1$ and ending at $t_n$ is mapped to an \modelimpl problem and solved using the lexicographic iterative strengthening algorithm.
This resolver algorithm is presented in figure \ref{impl.resolver}.
\begin{figure}[h!t!]
\begin{center}
\begin{alltt}
resolver:  
    for \(t\sb{i}\) in \(t\sb{1}\ldots,t\sb{n}\):
        \(F\) and \(\langle{}\mathfrak{crit}\sb{1}, \ldots ,\mathfrak{crit}\sb{n}\rangle\) mapped from evolution step at \(t\sb{i}\)
        \(answer\) = lexicographic-iterative-strengthening(\(F\),\(\langle{}\mathfrak{crit}\sb{1}, \ldots ,\mathfrak{crit}\sb{n}\rangle\))
        if \(answer\) equals UNSATISFIABLE:
            \(\alpha\sb{t\sb{i}}\) = \(\alpha\sb{t\sb{i-1}}\)
        else:
            \(\alpha\sb{t\sb{i}}\) mapped from \(answer\)
    return \(\alpha\sb{t\sb{1}},\ldots,\alpha\sb{t\sb{n}}\) 
\end{alltt}
  \caption{Pseudo code of the Lexicographic Iterative Strengthening Algorithm}
  \label{impl.resolver}
\end{center}
\end{figure}

In this algorithm, the systems are calculated starting at time $t_1$ and stopping at time $t_n$.
For each evolution step at time $t_i$ is mapped to a SAT + PB formula $F$ and PB criteria $\langle \mathfrak{crit}_{1}, \ldots ,\mathfrak{crit}_{n} \rangle$.
The algorithm \texttt{lexicographic-iterative-strengthening} is then used to find an optimal solution to $F$ with respect to the PB criteria, and return $answer$.
Either $answer$ equals \texttt{UNSATISFIABLE} at which point the system $\alpha_{t_i}$ is assigned as the previous system $\alpha_{t_{i-1}}$ (according to definition \ref{forml.stepdef}).
Otherwise, $answer$ is a set of literals that can be mapped back to the component system $\alpha_{t_i}$.
This algorithm, once completed, returns the set of component systems $\alpha_{t_1},\ldots,\alpha_{t_n}$ that resolve the \modelname instance.

\section{GJSolver}
\label{impl.gjsolver}
GJSolver is the implementation of the process from a CUDF* document to a resolved \modelname instance.
GJSolver grew through the course of this research to satisfy the need for an implementation to study CSE.
This implementation takes a CUDF* document, parses it to an instance of \modelname, which is then resolved by \modelimpl.

This process is described in figure \ref{impl.modelofgjsolver}.
\begin{figure}[htp]
\begin{center}
\digraph[scale=0.5]{implgjsolver}{
rankdir=LR;
subgraph cluster_0 {
label="GJSolver";
style=filled;
color=lightgrey;
FMI [label=<<TABLE  BORDER="0" CELLBORDER="1" CELLSPACING="5"><TR><TD WIDTH="150">:\modelnamewx</TD></TR></TABLE>> shape=none];
\modelimplwx [label=<<TABLE  BORDER="0" CELLBORDER="1" CELLSPACING="5"><TR><TD WIDTH="150">\modelimplwx</TD></TR></TABLE>> shape=none]
CMI[label=<<TABLE BORDER="0" CELLBORDER="1" CELLSPACING="5"><TR><TD WIDTH="150">:CUDF*</TD></TR></TABLE>> shape=none];
CMI -> FMI [ label="parsed to" ];
FMI -> \modelimplwx [ label="mapped to" ];
}
}
  \caption[labelInTOC]{GJSolver implementation description}
  \label{impl.modelofgjsolver}
\end{center}
\end{figure}

In this section the implementation and the validation of GJSolver are discussed.

\subsection{GJSolver Implementation}
\label{impl.gjsolverimpl}
The first decision made about the design of GJSolver was to base it on another similar implementation, Eclipse P2 \citep{le_berre_dependency_2009,leBerre2010}.
Basing the design on an existing implementation allowed the reuse of tools, and most importantly the reduction in risks during implementation.
The basis of GJSolver on Eclipse P2 lead to the following choices:
\begin{itemize}
  \item Java as the main implementation language.
  \item SAT4J as the core SAT+PB solver.
  \item Optimisation using PB criteria.
\end{itemize}

The main aspects where GJSolver does not reuse, or replicate Eclipse P2 is because P2 is designed especially for the OSGi and Eclipse component model, where GJSolver is designed for CUDF*.
Some of the differences between Eclipse P2 and GJSolver are:
\begin{itemize}
  \item No OSGi or Eclipse specific code in GJSolver.
  \item The internal representation of components is not based on OSGi.
\end{itemize}

Another important difference between Eclipse P2 and GJSolver is that Eclipse uses the iterative strengthening algorithm to optimise.
Eclipse P2 does this by compressing the tuple of PB criteria into a single criteria to be optimised (further described by \cite{leBerre2010}).  
How the optimisation approach used by GJSolver compares to the approach used by Eclipse P2 has not been explored in this research,
though the different approaches have been compared by \cite{marque2011blex}.

\subsubsection{SAT4J}
Given the use of SAT4J in GJSolver, a brief background of its development is presented here.

MiniSAT presented in \citep{een2003}, is a simple SAT solver implementation written in C, and designed for speed and extensibility.
It uses the DPLL based conflict driven algorithm as discussed in section \ref{impl.DPLL}.
This solver has become popular and is the basis of many other SAT solvers due to its open source distribution.
This has also lead to a track in the 2011 SAT competitions\footnote{http://www.satcompetition.org/2011/ accessed 6/3/2012} that deals with only altering MiniSAT to increase performance.
This means that MiniSAT has been repeatedly validated for performance by third parties across many different SAT problems. 

SAT4J \citep{le2010sat4j} is a re-implementation, and extension, of MiniSAT in the Java programming language.
SAT4J has been extended to efficiently solve a variety of related problems to SAT, including SAT+PB problems.
Due to the easily modifiable and transparent implementation of SAT4J, it has been able to be adapted to be used in various domains.

\subsection{Verification of GJSolver}
\label{impl.verif}
The verification of GJSolver is through the MISC competition held in 2011, and organised by Mancoosi.
This competition compares CUDF solvers by asking them to correctly parse and resolve hundreds of CUDF problems, and return optimal solutions with respect to various criteria.
MISC was created to promote interest in the component system evolution problems.

By viewing a CUDF problem as a single evolutionary step in a CUDF* problem, GJSolver has been modified to also solve CUDF problems.
Using such a modification, GJSolver could then be entered into the Mancoosi International Solver Competition (MISC) competition.

GJSolver was entered twice into MISC, firstly in a MISC Live event, which is an interim competition held during the year;
secondly at the MISC 2011 event\footnote{The results for MISC 2011 were announced at the Workshop on Logics for Component Configuration\footnote{http://www.pps.jussieu.fr/~treinen/lococo/2011/ accessed 6/3/2012}.}, 
which is the main competition. 

The main difference to GJSolver as described above, and the GJSolver entered into the MISC competition occurs when the evolution problem is unsatisfiable.
In the evolution step definition (\ref{formal.stepdef}) the component system $\alpha_{t_{i-1}}$ is returned if the evolution problem is unsatisfiable.
However, in this case, MISC requires that a file with only the text \texttt{FAIL} is written to state that no solution was found.
This is to correctly score a solver returning an incorrect solution to an evolution problem, 
and a solver finding a problem unsatisfiable.

\subsubsection{Mancoosi International Solver Competition}
\label{impl.MISCDEF}
Given a goal of the GJSolver implementation is to compete in the MISC, the interface and standards defined for this competition must be followed.
How the entered solvers are executed, what environment they are executed in, and the output required are all important aspects.

%%%They are executed on the command line
The way in which the entered solvers are executed is standardised to allow the automation of the competition.
This standard requires the entered solvers to be able to be executed on the command line with three arguments, \verb+cudfin+, \verb+cudfout+ and \verb+criteria+.
These arguments are defined as:
\begin{itemize}
  \item \verb+cudfin+: is a relative path to a CUDF document (as specified in section \ref{formal.cudf}) that describes the problem to be solver.
  \item \verb+cudfout+: is a relative path to a non-existent file, which is created by the solver to output the solution.
  \item \verb+criteria+: is a Mancoosi optimisation format (as described in section \ref{formal.mancoosioptimisationformat}) description of the criteria to select an optimal solution. 
\end{itemize}
The format of the output file, located at the path defined with \verb+cudfout+ argument, is a list of packages serialised as a list of stanzas with package and version properties.

%%%The environment POSIX, with 5minutes 1GB of memory
The environment in which the solver is executed is a virtual machine running a GNU/Linux system in a x86 architecture with 1GB of RAM.
It contains a Java runtime environment, allowing the use of Java as a primary language.
The time in which the solver is allowed to run is five minutes, after this time the solver will be forcibly stopped.
This time limit ensures that the competition can be run in an practical time frame.

\subsubsection{Tracks and Scoring}
Each MISC competition is broken down into three possible tracks, where each track is defined by the criteria used.
The first basic track, is 'paranoid', the second more advanced track is 'trendy', and third track is 'user'.
Both 'paranoid' and 'trendy' have pre-defined criteria, e.g. the 'paranoid' tracks criteria in MOF is \texttt{-removed,-changed}.
The 'user' track creates unknown criteria from a set of pre-defined criteria, so that the exact optimisation criteria is unknown before the competition.
This means that 'paranoid' and 'trendy' can have solvers tailored to their specific criteria, where the 'user' track cannot.
The exact criteria that is used for these tracks can be found on the MISC website\footnote{http://www.mancoosi.org/misc-2011/criteria/ accessed 12/5/2012}.

For each track, a set of solvers is entered.
Each track has a set of evolution problems defined in CUDF.
The solvers for each track are then called to return optimal solutions for all problems given the tracks criteria.

As MISC competitions were created to compare various solvers, a scoring system was developed. 
When a solver is given a CUDF problem and some criteria, the returned solution falls into one of three classes:
\begin{itemize}
  \item a \textbf{real solution} is any solution to the CUDF problem.
  \item \textbf{no solution} occurs when a solver finished without returning a solution. This can happen because of error, timeout, or there not being a satisfiable solution.
  \item an \textbf{incorrect solution} occurs when the solver returns an answer that is not a solution to the CUDF problem.
\end{itemize}

Given $m$ is the number of solvers that entered into the track, the scoring of a solvers solution to an individual CUDF problem is as follows:
\begin{itemize}
  \item a \textbf{real solution} is given $1$ point, with an additional $1$ point for every solver that found a better solution.
  \item for \textbf{no solution} $2\times m$ points are given
  \item for an \textbf{incorrect solution} $3 \times m$ points are given.
\end{itemize}
This means, if a solver returns an optimal solution to a CUDF problem, it will receive $1$ point.
However, if other better solutions are returned, then the solver could be given up to $m$ points.

For each track, all solvers in that track are assigned points based on their solutions to all problems in that track.
For a given track a solvers points are summed to give a final score.
If more than one solver has the same amount of points at the end of a track, then the time it took for them to find each solution is summed.
This total time value is used as the tie breaker.

\subsubsection{MISC Live}
%%%In the first competition we had only partially implemented much of the functionality, so we did not expect great results.
The MISC Live was entered when GJSolver was only partially implemented.
Therefore, the only track that was possible to enter was the 'paranoid' track.
The results for this track\footnote{http://mancoosi.org/misc-live/20101126/paranoid/ accessed 6/3/2012} where promising, though some improvements were necessary.
GJSolver's deficiencies were identified and corrected.

\subsubsection{MISC}
The main verification of GJSolver was through the MISC 2011 event.
In this event GJSolver was entered into all tracks of this event.
The 'paranoid' track had a total of 5 solvers, the 'trendy' track had a total of 6 solvers, 
and the 'user' track had a total of 4 solvers.
Each track was also entered by the solver which GJSolver is based on, Eclipse P2, and another efficient solver aspuncud.
These two solvers will form the basis of GJSolver's comparison.

The scores and the times for each of the track compared to that from Eclipse P2 and aspuncud are in table \ref{impl.misc2011}.
\begin{table}
\begin{tabular}{| l | c | c | c | c |}\hline
Track & \# of Problems & GJSolver & P2 & aspuncud\\ \hline
paranoid & 129 & (190 : 5,294) & (181 : 4,646) & (147 : 1,035) \\ \hline
trendy & 129 & (197 : 13,073) & (232 : 13,435) & (151 : 1,767) \\ \hline
user & 400 & (656 : 73,522) & (1392 : 87,956) & (1215 : 39,905) \\ \hline
\end{tabular}
\caption{Results from MISC 2011, results are (score:time in seconds)}
\label{impl.misc2011}
\end{table}

The winner for both 'paranoid' and 'trendy' tracks was the aspuncud solver.
The winner for the 'user' track was GJSolver that got nearly half the points of second place aspuncud, but took nearly twice the time.

\subsubsection{Analysis}
The results from the MISC competition show:
\begin{itemize}
  \item During the competition GJSolver had very consistent results.
These results allowed it to compete with the other solvers, and win the 'user' track.
 \item No CUDF problem in the competition was incorrectly solved by GJSolver. 
 This could be used to argue that the parsing of CUDF to \modelname, mapping to \modelimpl problems, and the implementations of the algorithms are correct.
 \item When compared to the similar implementation of Eclipse P2, GJSolver produces similar results, in a similar time. 
 As GJSolver was based on Eclipse P2, this is seen as a validation that the differences between the two are not detrimental, and possibly improvements. 
\end{itemize}

Given these reasons GJSolver is seen as a verified implementation and will be used in this research's simulation.


\section{Simulation Validation}
%%%This is validated to measure the difference to reality to find if the error is tolerable
The validation of the implementation of the simulation is an important step in the use of this simulation.
As no simulation exactly describes reality, this step is mostly an exploration of the differences between the simulation and the reality of component system evolution.
Therefore, this not only highlights the similarities to real component evolution, but also the differences.
The identification of differences is important, as the assumptions made will lead to errors, these can be measured to find if the simulation is within tolerable error. 

%%%This is done by ensuring the 
This simulation is validated though looking at the results outputted from the simulation and ensuring they conform to expectations, 
then compare them to the output of an actual system.

%%%Face Validation
Face validation is the process by which output of the simulation checked in order to find if it behaves as it should.
The core output from this simulation is the resulting systems created during execution of the simulation.
Looking at this output alone is difficult to state that it is correct, only that it looks correct.
This process was iteratively done throughout development, as the core stakeholder is also the simulation developer.

\subsection{Validation via Recorded System}
%%%Controlled Environment
To attempt to validate this simulation a controlled system was created that updates every day over a time frame of November 1st to 30th in 2011.
A computer had Ubuntu 11.10 (Oneiric Ocelot) installed on it, every day a script that used APT to update the system, and then saved the repository and the system state.

To validate the simulation, first we can look at the series of repositories, to find how different the assumption that the user uses all repositories is different from the default real user.
Secondly the evolution of the system can be looked at and compared to a similar evolution within the simulation.

\subsubsection{Comparing Repositories}
%%%Look at the repositories, amount of components \ldots amount of versions\ldots 
A core assumption with this simulation is that a user uses all components in the repository, 
this was caused by the fact that the usual meat-repository data files are not stored at the required resolution.
How significant the impact of this assumption can be measured through looking at the metrics and change of the repositories saved through this controlled system.

%%%TODO Complete this validation

\subsubsection{Comparing Systems}
%%%Compare this output to the output from the simulation
The output of this controlled system can be compared to the simulations output when setting the update criteria to mimic the APT criteria.
The simulated evolution of the system can metrics can be compared, like the rate at which the system goes out of date, and the amount of change that occurs.
These can be used as a broad measure of validity, as they are symptoms of the evolution they will be similar but not exact.

%%%TODO Complete this validation

\subsection{Validation via User Logs}
%%%Real User Log
To further validate this simulation, the submitted user logs were compared with the output of the simulation, over a month.
Some of the user logs fall within the dates of the simulation, this means that by generating custom user action sets these users actions can be directly simulated.
Their update probabilities and install distributions can also be extracted and used to create similar simulated users, to validate the creating of user action sets.

\subsubsection{Directly comparing User Logs}
%%%Creating custom user action sets to validate that the simulation returns about the right results
From these logs a configuration is generated; the update probability, the install distribution and the range of dates are extracted from the log.
The update and install criteria that used by the APT application, and the rest are the default contexts.  

%%%TODO Complete this validation

\subsubsection{Validating Generating User Actions Sets}
%%%Creating generated users, from the simulated versions to validate the generating of the users returns about the right results.
The entire simulation depends on being able to generate users that abstract from real users.
As discussed in this, and the previous, chapter this is a very difficult challenge.
Through comparing generated user action sets with the real users that are meant to abstract them, the difference can be determined.
Then by simulating these generated users, and comparing the results of the actions to the real system log the error that is caused by the generation can be measured.  

%%%Comparing the user action sets to the logs

%%%Comparing multiple generated users to the real thing

%%%TODO Actually complete this validation



\section{Summary}
This chapter described the \modelimpl problem and how it mapped to the evolution steps from a \modelname instance.
The algorithms used to solve a \modelimpl problem where also discussed.
The implementation GJSolver was presented.
GJSolver takes a CUDF* document, parses it to a \modelname instance, and resolves it by mapping it to \modelimpl problems and solving them.
GJSolver was verified through the MISC competition, and was shown to perform well when compared against other solvers.

The following chapter discusses the simulation of CSE and discusses the users motivations for change.