\chapter{Component Dependency Resolution}
\label{implementation}
\epigraph{What I cannot create, I do not understand.}
{\textit{Richard Feynman, 1988}}

%%%CDR Has been shown to be NP-Complete to find a solution and NP-Hard to find an optimial one.
In chapter \ref{formal} the evolutionary step function was defined to take an evolution problem, and an order over component systems, 
and return an optimal solution to the evolution problem.
This function can be automated into a process called component dependency resolution (CDR).

%%%Formally it resembles the SAT problem, which already has efficient implementations, and has been used to solve this problem
In this chapter, the CDR implementation named GJSolver is presented.
GJSolver is implemented to map evolution problems to Boolean Satisfiability Problems (SAT),
and map the criteria that define an optimal solution using pseudo-Boolean functions \citep{dixon2004automating}. 
Then, through using algorithms like the Davis-Putnam-Logemann-Loveland (DPLL) \citep{Davis1960, davis1962machine} (to find solutions to SAT problems),
and iterative strengthening \citep{calistri1994iterative, le2010sat4j} (to find optimal solutions given criteria),
an optimal solution to an evolution problem can be found.

%%%In this chapter\ldots
In this chapter GJSolver is described.
Firstly, in section \ref{impl.requirement}, the requirements and design decisions of GJSolver are discussed.
In section \ref{impl.mapping}, the SAT problem and pseudo-Boolean functions are defined then mapped to the evolution problem and criteria defined in the chapter \ref{formal}.
The algorithms used to find an optimal solution to an optimal solution to the SAT problem given the criteria are described in chapter \ref{impl.algorithms}
The CDR implementation, called GJSolver, created through this research is then described in section \ref{impl.gjsolver}.
The specific requirements that GJSolver was developed towards, the formal mapping, and the implementation decisions are described.

This chapter ends with a discussion of the process of verification of GJSolver through the Mancoosi International Solver Competition (MISC).  

\section{Requirements of GJSolver}
\label{impl.requirement}
Through the course of this research, the CDR implementation GJSolver grew out of the need to have a modifiable base to experiment with component dependency resolution.

Given the context under which GJSolver was developed, the set of requirements that it was designed according to are listed here:
\begin{enumerate}
  \item \textbf{Mancoosi International Competition Ready}: The MISC gives a set of standards to solve CUDF problems with criteria defined using the Mancoosi optimisation format.
  A goal of the GJSolver was to enter it in the MISC in order to be compared against other solvers.
  This comparison would be the final step in the validation and verification of the implementation.
  Following these standards ensures that the solver can be entered, and then competing can show the relative speed and correctness of the GJSolver implementation.
  This will ensure that GJSolver is a valid and efficient implementation. 
  \item \textbf{Anytime Algorithm}: Return a solution, even if it is not an optimal solution, within a predefined amount of time. 
  CDR problems can be large and complex, finding the ``best'' solution can be difficult. 
  This requirement ensures that the algorithm will return a solution, within a time frame.
  \item \textbf{Easily Creatable Criteria}: The ability to quickly implement and test criteria in a manner that enables experimentation, will increase the speed of research.
\end{enumerate}

\subsection{Mancoosi International Solver Competition}
Given a goal of the GJSolver implementation is to compete in the MISC, the interface and standards defined for this competition must be followed.
How the entered solvers are executed, what environment they are executed in, and the output required are all important aspects to the development of GJSolver.

%%%They are executed on the command line
The way in which the entered solvers are executed is standardised to allow the automation of the competition.
This standard requires the entered solvers to be able to be executed on the command line with three arguments, \verb+cudfin+, \verb+cudfout+ and \verb+criteria+.
These arguments are defined as:
\begin{itemize}
  \item \verb+cudfin+: is a relative path to a CUDF document (as specified in section \ref{formal.cudf}) that describes the problem to be solver.
  \item \verb+cudfout+: is a relative path to a non-existent file which is created by the solver to output the solution.
  \item \verb+criteria+: is a Mancoosi optimisation format (as described in section \ref{formal.mancoosioptimisationformat}) that describes the criteria to select a solution. 
\end{itemize}
The format of the output file, located at the path defined with \verb+cudfout+ argument, is a sub set of CUDF.
This output only requires package stanzas with the package name, version and installed properties.
This lowers the size of the output file required to be written by the solver.

%%%The environment POSIX, with 5minutes 1GB of memory
The environment in which the solver is executed is a virtual machine running a GNU/Linux system in a x86 architecture with 1GB of RAM.
It contains a Java runtime environment, allowing the use of Java as a primary language.
The time in which the solver is allowed to run is 5 minutes, after this time the solver will be forcibly executed.
This time limit ensures that the competition can be run in an acceptable time frame.

\subsection{Anytime Algorithm}
The time required to find an optimal solution to a component evolution problem can be longer than the allowed time within the MISC environment.
For this reason, GJSolver will be implemented with an anytime algorithm at its core.
An anytime algorithm can return a valid solution to a problem if it is interrupted before it ends.
This interruption could be from a user or some other stimuli.
In the case of GJSolver, the interruption will be caused by an internal timer dedicated to ensure that the algorithm does not run in time exceeding the 5 minute deadline.

Such anytime algorithms therefore create a trade-off between time and optimality, where the more time that is set aside for the algorithm to run the more optimal the solution. 
They can also create inconsistent results, as any input may, on different runs, return many different possible solutions.
This means that if it is necessary to interrupt the algorithm, it is possible to return non-optimal solutions.
However, this is only if interruption is necessary, as GJSolver will encounter a wide range of difficult problems, 
it is difficult, or impossible, to judge beforehand if a problem will require the algorithm to be interrupted. 

\subsection{Expendable Criteria}
The ability to test and experiment with a wide range of possible criteria to use in CDR is a goal of this research.
To satisfy this requirement, GJSolver was developed in a modular manner, in which the criteria can be extended, modified, and tested when solving problems quickly.

\subsection{Design}
%%%TODO this section
The first decision made about the design of GJSolver was to base it on another CDR implementation, Eclipse P2 \citep{le_berre_dependency_2009,leBerre2010}.
Basing the design on an existing implementation allowed the reuse of tools, and most importantly the reduction in risk that the implementation will not satisfy the requirements.
Eclipse P2 satisfies two of the three requirements: being MISC ready as it has a MISC implementation, and using an anytime algorithm to find solutions. 

The basis of GJSolver on Eclipse P2 lead to the following choices:
\begin{itemize}
  \item Java as the main implementation language
  \item SAT4J as the core SAT and PB solver
  \item Represents criteria as pseudo-Boolean function
  \item Finds an optimal solution through iterative strengthening
\end{itemize}

The main aspects where GJSolver does not reuse, or replicate Eclipse P2 is because P2 is designed especially for the OSGi and Eclipse component model, where GJSolver is designed for CUDF.
Som of the differences between Eclipse P2 and GJ Solver are:
\begin{itemize}
  \item No OSGi or Eclipse specific code in GJSolver
  \item The internal representation of components is based on CUDF and not on OSGi
  \item Not representing all criteria as a single pseudo-Boolean function
\end{itemize}

By starting with Eclipse P2 as a model, GJSolver was able to be efficiently implemented.

\subsection{SAT4J}
Given the use of SAT4J in GJSolver, a brief background of its development is presented here.

MiniSAT presented in \citep{een2003}, is a simple SAT solver implementation written in C, and designed for speed and extensibility.
It uses the DPLL based conflict driven algorithm as discussed above.
This solver has become popular and is the basis of many other SAT solvers due to its open source distribution.
This has also lead to a track in the 2011 SAT competitions\footnote{http://www.satcompetition.org/2011/ accessed 6/3/2012} that deals with only altering MiniSAT to increase performance.
This means that MiniSAT has been repeatedly validated for performance by third parties across many difference SAT problems. 

SAT4J \citep{le2010sat4j} is a Java re-implementation, and extension, of MiniSAT in the Java programming language.
The extensions SAT4J makes to MniSAT include the ability to find resolve pseudo-Boolean constraints.
SAT4J was developed in order to quickly test combinations of advancements in SAT solving technology.
This goal has created an easily modifiable and transparent implementation, able to be adapted to be used in various domains.
SAT4J in this right has been a success.

\section{Formal Mapping}
\label{impl.mapping}
GJSolver will represent an evolution problem as a SAT problem, and the set of criteria which define optimality as a list of pseudo-Boolean functions.
In this section, a SAT problem and pseudo-Boolean functions are defined and mapped to the the formal model presented in chapter \ref{formal}. 

\subsection{Boolean Satisfiability Problem}
Boolean satisfiability (SAT) is the problem of determining if the variables in a Boolean equation can be assigned in such a way that the equation returns true.
SAT was the first identified NP-Complete problem, meaning there is no known algorithm that efficiently solves all instances of SAT problems.
The fundamental difficulty of SAT problems, 
combined with the ability to map many problems to SAT has spawned a community\footnote{http://www.satcompetition.org/ accessed 6/3/2012} 
dedicated to creating, enhancing, and testing various SAT solver implementations. 
Such SAT solvers have been used in various domains to tackle problems such as electronic design automation \citep{Marques-Silva2000}, 
model verification \citep{dennis2006}, and, of course, component system evolution \citep{leBerre2010}.

A common representation of a SAT equation is in Conjunctive Normal Form (CNF).
CNF is defined as a conjunction of clauses, 
where each clause is a disjunction of literals, e.g. $(a \vee b) \wedge (\neg b \vee c)$.

This CNF representation of a SAT problem can be defined as such:
\begin{defs}
\label{impl.defSAT}
.
\begin{enumerate}
	\item Let $V$ be a set of variables, a \textbf{variable} is an element of $V$
	\item A \textbf{literal} is a variable $v$ or its negation $\neg v$.
	\item A literal $\neg \neg v \equiv v$
	\item Given a set of literals $P$, $\neg P \equiv \{\neg v \mid v \in P\}$
	\item A \textbf{clause} is a set of literals \label{impl.whatisaclause}
	\item A \textbf{formula} is a set of clauses
	\item An instance of a \textbf{SAT problem} is a set of variables $V$ and a formula $F$
	\item A set of literals is \textbf{consistent} if for any variable $v$, the set of literals does not contain both $v$ and its negation $\neg v$.
	\item A clause $c$ is \textbf{satisfied} by a set of literals $P$ if there exists a literal in $c$ that is also in $P$.
	\item A \textbf{solution} to a SAT formula $F$ is a consistent set of literals $P$, such that for every clause $c$ in $F$, $c$ is satisfied by $P$.
	\item A \textbf{partial solution} is a subset of any solution, e.g. given $P$ is a solution $P'$ is a partial solution iff $P' \subseteq P$.
	\item An instance of a SAT problem is \textbf{satisfiable} if there exists a set of literals $P$ that is a solution to $F$, otherwise the instance of the SAT problem is \textbf{unsatisfiable}. 
\end{enumerate}
\end{defs}

Assume a SAT problem where $V = \{a,b,c\}$ and $F = \{c_1,c_2\}$, where clauses $c_1 = {a,b}$ and $c_2 = {\neg b, c}$.
A solution for this problem could be $\{a,\neg b,c\}$ as $a \in c_1$ and $\neg b \in c_2$.
However, $\{a, \neg b, b\}$ is not a solution because it is not consistent, and $\{a, b, \neg c\}$ is not a solution because it does not contain a literal in $c_2$.  


\subsection{Pseudo-Boolean Extension of CNF}
%%%Optimisation of SAT solvers is typically done through extending their possible constraints to include Psuedo Boolean inequalities
A typical extension of the SAT problem is where a clause can be a pseudo-Boolean (PB) constraint \citep{dixon2004automating}.
PB constraints consist of a linear relation over Boolean variables.

A pseudo-Boolean function takes a set of Boolean literals and returns a natural number, e.g. given a set of literals $P$, $f(P) = 4$.
A pseudo-Boolean constraint is a relation between a the function and a number, e.g. given a set of literals $P$, $f(P) \leq 3$.
A pseudo-Boolean function is defined as:
\begin{defs}
\label{impl.PBfunction}
Given a tuple of literals $lits = \langle l_1,\ldots,l_n \rangle$ and a tuple of coefficients $cos = \langle a_1,\ldots,a_n\rangle$,
a pseudo-Boolean function takes a set of literals $x$, and is defined as

$f(x) = \sum \limits_{l_i \in (lits \cap x)} a_i$
\end{defs}
A pseudo-Boolean function can be fully described by defining the tuples $lits$ and $cos$.
For example, the pseudo-Boolean function $f$ where $lits = \langle x_1, \neg x_2\rangle$ and $cos \langle 1,3\rangle$,
for the input $\{x_1,\neg x_2,x_3\}$ will equal $4$.

Such a function can become a constraint:
\begin{defs}
\label{impl.PBConstraint}
Given an integer $k$, and a pseudo-Boolean function $f$, a pseudo-Boolean constraint is of the form

$f(x) \geq k$, $f(x) > k$, $f(x) \leq k$, $f(x) < k$ or $f(x) = k$


\end{defs}
A pseudo-Boolean constraint can be described with a pseudo-Boolean function $f$, an relation in $\{\geq,>,\leq,<,=\}$, and a number $k$.

Pseudo-Boolean constraints can be included in a SAT problem by modifying the definition \ref{impl.defSAT}.\ref{impl.whatisaclause} to:
\begin{defs}
.
\begin{enumerate}
\setcounter{enumi}{4}
	\item A \textbf{clause} is a set of literals or a pseudo-Boolean constraint.
\end{enumerate}

\end{defs}

Adding the definition of what a PB constraint is, and what satisfies a PB constraint:
\begin{defs}
.
\begin{enumerate}
\setcounter{enumi}{12}
  \item A \textbf{pseudo-Boolean constraint} is a pseudo-Boolean function $f$, a relation $R \in \{\geq,>,\leq,<,=\}$, and a number $k$
  \item A pseudo-Boolean constraint is \textbf{satisfied} by a set of literals $P$ iff $\langle f(P),k \rangle \in R$.
\end{enumerate}
\end{defs}

For example, the pseudo-Boolean function $f$ described above with can be used to create the constraint $f(x) > 2$.
This constraint will be satisfied by the set of literals $s_1 = \{ x_1, \neg x_2 \}$ as $f(s_1) = 4 > 2$, but not with the set of literals $s_2 = \{ x_1,  x_2 \}$ as $f(s_2) = 1 \not > 2$.

Pseudo-Boolean constraints, can be translated into CNF clauses, but the original pseudo-Boolean representation has been shown to be exponentially more concise \citep{dixon2004automating}.
Also given the proper amendments to unit propagation and other algorithms (some of which is described in \citep{Sheini2006}), 
it can be faster to find solutions to problems represented in pseudo-Boolean constraints rather than their translated SAT constraints \citep{dixon2004automating}.
Both of these reasons give ample justification to use the pseudo-Boolean extension to DPLL when mapping the component evolution problem to a set of constraints. 

\subsection{Component System}
A component system is a set of components which is mapped to in the SAT problem as a set of literals.
Firstly, the set of components is a subset of variables in the SAT problem, i.e. $\mathbb{C} \subseteq V$.
This means that all components are variables, though there can be variables that are not componnets.


\begin{itemize}
  \item A set of literals $P$ is mapped to a component system such that $P := \{x \mid x \in \mathbb{C} \wedge x \in P\}$
  \item A component system, $\alpha$ maps to a set of literals such that $\alpha := \alpha \cup \{\neg x \mid x \in \mathbb{C} \wedge x \not \in \alpha\}$
\end{itemize}
That is, a set of literals $P$ is mapped to a component system such that all components that are positive literals in $P$, are in the component system.
A component system $\alpha$ is mapped to a set of literals that is the union of $\alpha$ and the set of negative literals of components not in $\alpha$.
For example, a set of literals $\{a,b,c,\neg d,\neg e,f\}$ where $a,b,c,d,e$ are components, is mapped to the component system $\{a,b,c\}$.
To map this system back to a set of literals results in $\{a,b,c,\neg d,\neg e\}$.

As seen in the above example this mapping loses the information about the literal $f$, as it is not a component.
This literal, may represent some other information in the SAT problem.
This does not represent a problem, however, as such mappings only occur at points that do not effect CDR.

\subsubsection{Evolution Problem Mapping}
The evolution problem is defined as as a set of five types of constraints.
Each type of constraint in an evolution problem can be mapped to a SAT clause or PB constraint:
\begin{enumerate}
  \item $\neg a := \{\neg a\}$
  \item $a \rightarrow \neg c := \{\neg a, \neg c\}$ 
  \item $a_1 \vee \ldots \vee a_n := \{a_1, \ldots,  a_n\}$ 
  \item $a \rightarrow c_1 \vee \ldots \vee c_n := \{\neg a, c_1, \ldots, c_n\}$
  \item $a_1 + \ldots + a_n = 1 := $ a pseudo-Boolean constraint described with the function where $lits = \langle a_1 ,\ldots , a_n\rangle$ and $cos = \langle 1_1,\ldots,1_n \rangle$,
  whose relation is $=$ to the number $1$, as described in definition \ref{impl.PBConstraint}.  
\end{enumerate}

By mapping each constraint in an evolution problem $EP$ as defined above, will create a SAT formula $F$.

\subsubsection{Criteria Mapping}
Criteria in chapter \ref{formal} is defined as $\langle rank,< \rangle$, where $rank : 2^{\mathbb{C}} \times 2^{\mathbb{C}} \rightarrow A$ and $<$ is a binary relation over $A$.
These criteria can be lexicographically composed through the operator $\oplus$.
The overall optimisation criteria is then $crit_1 \oplus \ldots \oplus crit_n$, where $crit_i$ is a criteria $\langle rank_i,<_i \rangle$.

A pseudo-Boolean criteria is a tuple $\langle f, r \rangle$, where $f$ is a pseudo-Boolean function, and $r \in \{<,>\}$ which are orders over natural numbers.

The main problem with mapping between the functions $f$ and $rank$ is that $rank$ takes both current and proposed systems as parameters, where $f$ takes only the proposed.
However, as the current system $\alpha$ stays invariable for a given evolution problem, 
the function $f_{\alpha}$ is defined to mean, a psuedo-Boolean function with respect to the current solution $\alpha$.

\begin{defs}
A criteria $\langle rank,< \rangle$,
is mapped to a pseudo-Boolean criteria is a tuple $\langle f_{\alpha}, r \rangle$
such that $rank(\alpha,\beta) = f_{\alpha}(\beta)$, and $< = r$. 
\end{defs}

This definition limits the possible ranking functions to only pseudo-Boolean functions over natural numbers, 
and possible orders over natural numbers to only greater than or less than.
These restriction is necessary to use the iterative strengthening algorithm presented later in this chapter.

\begin{defs}
Lexicographically composed criteria $crit_1 \oplus \ldots \oplus crit_n$ is mapped to a tuple of pseudo-Boolean criteria $\langle crit_1,\ldots,crit_n\rangle$.
\end{defs}
This tuple of  pseudo-Boolean criteria is later used in the iterative strengthening algorithm.

\section{Algorithms}
\label{impl.algorithms}
The algorithms to find a solution to a SAT problem, and to find an optimal solution given a list of criteria that are defined using pseudo-Boolean functions are described.

\subsection{Davis-Putnam-Logemann-Loveland algorithm for SAT Solvers}
%%%A successful algorithm for solving SAT problems is the DPLL algorithm, here we describe it in overview
The Davis-Putnam-Logemann-Loveland (DPLL) algorithm \citep{Davis1960, davis1962machine} for solving SAT problems is a complete (meaning it will find a solution if one exists), 
backtracking-based search algorithm for SAT problems represented in conjunctive normal form (CNF).

It is defined to take a formula $F$ and a set of literals $P$ (described as a partial assignment), and return \verb+SATISFIABLE+ if $P$ is a partial solution, 
otherwise returning \verb+UNSATISIFABLE+.
By first calling DPLL with $P$ being the empty set, by adding literals to $P$ and recursively  calling itself, the DPLL function searches for whether a solution to the formula exists, 
i.e the formula is satisfiable.
The DPLL algorithm in defined in figure \ref{impl.DPLL} as presented in \citep{dixon2004automating}:
\begin{figure}[h]
\begin{center}
\begin{alltt}
function DPLL(\(F, P\))
   P = unit-propagate(\(F, P\))
   if \(P\) is not consistent:
       then return UNSATISIFABLE;
   if \(P\) is a solution to \(F\):
       then return SATISFIABLE;
   \(l\) = decide\((P)\);
   if DPLL\((F, P \cup \{l\})\)
       return SATISFIABLE
   else
       return DPLL\((F, P \cup \{\neg l\})\);
\end{alltt}
  \caption{Recursive DPLL algorithm}
  \label{impl.DPLL}
\end{center}
\end{figure}

The first \verb+if+ branch returns \verb+UNSATISIFABLE+ if the partial assignment $P$ is inconsistent.
The second \verb+if+ branch determines if the partial assignment $P$ is a solution to $F$, this would end the search by returning \verb+SATISFIABLE+.
The \verb+unit-propagation+ function finds literals that must be in $P$ for $P$ to be a partial solution,
and the \verb+decide+ function returns a literal to be added to $P$.
These functions are described in the following sections. 

\subsubsection{Unit Propagation}
The first line in the  DPLL algorithm calls the \verb+unit-propagation+ function.
This function uses the clauses in the formula and the partial assignment and adds literals to $P$ that must be included if $P$ is to be a partial solution.

A clause $c$ is called unit given a partial assignment $P$, if $c$ is not satisfied by $P$, and $P$ contains all but one of its literals negations.
For example, a clause $\{a,b,c\}$ is unit if the partial assignment contains $\neg b$ and $\neg c$ but neither $a$ or $\neg a$.
A unit literal is then defined as the literal in a unit clause whose negation is not in the partial assignment.
In the above example, the literal $a$ is the unit literal.

For a formula to be satisfiable given partial assignment, each unit literal must be included in the partial assignment,
because if their negation is included the clause is not satisfied by the partial assignment.
For example, given a formula $\{c\}$, where $c = \{a,b\}$;
given the assignment $\{\neg a\}$ the clause $c$ is unit and unit literal is $b$.
If $\neg b$ where in the partial assignment, $c$ would not be satisfied by the partial assignment,
therefore $b$ must be in the partial assignment for it to be a partial solution.

The process of unit propagation is defined in figure \ref{impl.propagation}.
\begin{figure}[htp]
\begin{center}
\begin{alltt}
unit-propagate(\(F, P\)):
  while P is consistent and there exists a \(c \in F\) that given \(P\) is unit:
    \(l\) = unit literal in \(c\)
    \(P\) = \(P \cup \{l\}\)
\end{alltt}
  \caption{Pseudo code of Unit Propagation}
  \label{impl.propagation}
\end{center}
\end{figure}

\subsubsection{Literal Order}
The function \verb+decide+ returns a literal who is not in the partial assignment, nor whose negation is in the partial assignment.
That is, if $l =$ \verb+decide+$(P)$, then $\{l\} \not \in P$ and $\{\neg l\} \not \in P$.
This literal is then added to the partial assignment as an assumption that it might be in the final solution.
This assumption is checked to be correct or not by recursively calling the DPLL function again with the new partial assignment.

If the \verb+decide+ function selects literals that are in the final solution then the search for whether the formula is satisfiable is quickly resolved.
However, if it returns literals that are not in the final solution, the search can be extensive and costly.
Therefore, the efficacy of this selection has a great impact on the overall efficiency of the DPLL algorithm.

\subsection{Advancements in SAT Solvers}
Though the DPLL algorithm is the basis of most modern SAT solvers, the actual implementations have been significantly altered to increase efficiency.
Some changes, including the use of conflict learning, backjumping, and watched literals, are briefly described here.
This section should give a broad overview of the techniques used in current SAT solvers, 
in order to show that their application to problems like component evolution is justified. 

\subsubsection{Conflict Learning and Backjumping}
Conflict learning \citep{stallman1976} is a technique to cache previously tried sets of assignments in order to stop re-solving the same sub-problems.
This is accomplished by remembering what unit clauses, also known as reasons, caused literals to be added to the partial assignment. 
This process works by identifying a variable both inferred to be true and false,
then creating a new clause, known as the learnt clause, which stops that inconsistency being reached again.
This learnt clause is derived by disjoining the two reason clauses after removing both the references to the inconsistent variable.
This process is shown in figure \ref{impl.clauselearning}. 

\begin{figure}[htp]
\begin{center}
$\begin{array}{c}
\{a_1,\ldots,a_k, l\} \\
 \{b_1,\ldots,b_m,\neg l\}\\
\hline
\{a_1,\ldots,a_k, b_1,\ldots,b_m \}
\end{array}$
  \caption{Clause Learning. Where $\{a_1,\ldots,a_k, l\}$ is the reason for $l$, and $\{b_1,\ldots,b_m,\neg l\}$ is the reason for $\neg l$
   are used to create the learnt clause $\{a_1,\ldots,a_k, b_1,\ldots,b_m \}$.}
  \label{impl.clauselearning}
\end{center}
\end{figure}

For example, if the reason for the inferred literal $a$ is clause $\{a, b\}$, and the reason for $\neg a$ is clause $\{\neg a, c\}$,
then the learnt clause derived is $\{b,c\}$, and added to the formula.

Backjumping \citep{Gaschnig1979} is the technique which determines how far to up the search tree to backtrack when a conflict is found.
The higher up the tree the technique ``jumps'' to, the greater reduction of the search space.
The level at which the algorithm backjumps is typically the point at which the learnt clause becomes unit. 

More advanced methods of conflict learning occur by minimising the size of the learnt clauses, as presented in \citep{sorensson2009}.
This research describes search methods that use other reason clauses to find smaller more succinct learnt clauses.
The smaller the clause, the more of the search tree is pruned and the more levels are backjumped through the search.

\subsubsection{Watched Literals}
As noted by studies into the efficiency of DPLL-based SAT solvers \citep{dixon2004automating}, unit propagation is where the bulk of the computation occurs.
Attempts to increase the efficiency of this task was initially to find better heuristics \citep{JamesMCrawford1996} for the literal order, to encourage cascades of unit propagation.
These attempts were shown to work well on random SAT problems but be less efficient for large structured problems \citep{dixon2004automating}.

It was noted that within unit propagation most of the time was spent on identifying the unit clauses.
The naive approach to unit propagation was to examine every clause, and then every literal in the clause to find if it is unit or not.
A more efficient approach was proposed using watched literals \citep{Madigan2001}, where instead of having the clauses examined, 
the clauses maintain an index of the necessary literals and notify the algorithm when they become unit.
This ``don't call us, we will call you'' concept makes the efficiency of the unit propagation function less dependent on the amount of clauses in the formula.

Advances on watched literals have occurred through algorithms to maintain the index of literals, like that presented in \citep{Moskewicz2001}.
Such algorithms enable larger formulae to be solved without necessarily increasing the time to solve them.


\subsection{Iterative Strengthening}
The algorithm known as iterative strengthening, presented in \citep{calistri1994iterative} and \citep{le2010sat4j}, 
describes an anytime algorithm using constraint satisfaction that iteratively finds better solutions.
This algorithm can be used to find optimal solutions to an evolution problem, where only one criterion is optimised.
This is done by first finding a solution, then iteratively adding constraints to the formula that ensure the next solution found will be better than the previous.
This continues until either the strengthened formula is found to be unsatisfiable, or the algorithm is interrupted, at which point the best solution currently found is returned. 
This algorithm is defined in figure \ref{impl.strength}.

\begin{figure}[htp]
\begin{center}
\begin{alltt}
find-solution(\(F\),strengthen):
    \(answer\) = DPLL(\(F\))
    if \(answer\) = UNSAT:
        return UNSAT
    do:
        \(model\) = \(answer\)
        \(constraints\) = strengthen(\(model\))
        \(F\) = \(F \cup constraints\)
        \(answer\) = DPLL(\(F\))
    while not interrupted() and \(answer\) != UNSAT
    return \(model\) 
\end{alltt}
  \caption{Pseudo code of Iterative Strengthening Algorithm}
  \label{impl.strength}
\end{center}
\end{figure}

The first aspect to note in this algorithm is that the \verb+DPLL+ function has been altered slightly.
This function now takes the formula $F$ as a parameter and returns a solution, $answer$, that satisfies $F$. 
If the formula is unsatisfiable, however, \verb+DPLL+ returns \verb+UNSAT+.

The first action in this algorithm is to check if the formula is satisfiable, by assigning the output of the \verb+DPLL+ function to the variable $answer$.
If $answer$ is \verb+UNSAT+, then the algorithm stops and returns that the formula is unsatisfiable.

The main loop of this function first assigns the contents of the variable $answer$ to the variable $model$.

The function \verb+strengthen+ is then called on $model$, to generate a set of constraints that when added to the formula ensure that only solutions more optimal are satisfiable.
Therefore, given some function $d$ measures how optimal a solution is, $model_1$ satisfies $F$, and $constraints = $ \verb+strengthen+$(model_1)$;
any $model_2$ that satisfies $F \cup constraints$ must be of the relation $d(model_1) < d(model_2)$.
That is, that given some measurement of optimal, adding the constraints ensure that only more optimal solutions satisfy the formula with the strengthened constraints. 

The constraints from the \verb+strengthen+ function are then added to the formula, a more optimal solution is searched for with \verb+DPLL+.

To ensure that this algorithm will end, and not just have \verb+DPLL+ repeatedly return the same answer to the formula, 
the found $model$ must not satisfy the formula with the strengthened constraints.
If this is the case, the number of answers that satisfy $F$ will decrease by at least one per iteration.

This main loop will continue until the function \verb+interrupted+ returns true, which may happen given some external cause (like a user stopping the algorithm or a timer running out),
or the strengthened formula is found to be unsatisfiable.
At this point the previously best solution discovered stored in the variable $model$ will be returned.

\subsubsection{Lexicographic Optimisation}
%%%Basic iterative strengthing can be modified towards handling lexicographically ordered crtiera, and allowing this criteria to effect DPLL's efficiency
The basic iterative strengthening algorithm, presented in figure \ref{impl.strength}, can be modified to handle lexicographically related criteria, so as to solve component evolution problems.
By adding an additional outer loop, including other necessary alterations, the original algorithm has been modified to handle lexicographical criteria.

The original iterative strengthening algorithm with the lexicographical extension has been implemented in GJSolver. 
This altered algorithm is presented in figure \ref{impl.lexstrength}.

\begin{figure}[htp]
\begin{center}
\begin{alltt}
find-lex-solution(\(F\), \(\langle crit_1, \ldots ,crit_n \rangle \)):
    
    \(F\) = \(F \cup \) initial-constraints(\(\langle crit_1, \ldots ,crit_n \rangle \))
    \(answer\) = DPLL(\(F\))
    if \(answer\) = UNSAT:
        return UNSAT
    \(i\) = 0
    do:
        \(i = i + 1\)
        do:
            \(model\) = \(answer\)
            \(constraints\) = \(crit_i\).strengthen(\(model\))
            \(F\) = \(F \cup constraints\)
            \(answer\) = DPLL(\(F\))
        while not interrupted() and \(answer\) != UNSAT
        \(F\) = \(F \backslash constraints\)
        \(F\) = \(F \cup \) \(crit_i\).lock(\(model\))
        \(answer\) = \(model\)
    while not interrupted() and \(i\) != \(n\)
    return \(model\) 
\end{alltt}
  \caption{Pseudo code of the Lexicographic Iterative Strengthening Algorithm}
  \label{impl.lexstrength}
\end{center}
\end{figure}

The first alteration to the original algorithm is that this version takes two parameters, firstly the formula denoted as $F$;
secondly a stack of criteria, ordered so the top of the stack is the criteria most lexicographically important.

The first action of this algorithm is to include initial criteria constraints,
these constraints may define new literals that can be used in other criteria functions.
These constraints are just a union of sets of initial constraints given by individual criteria, and they should not effect a solutions satisfiability. 

The next action of this algorithm is to check whether the formula is satisfiable,
and if it is not, return unsatisfied.

A new outer loop is defined, that loops over the criteria until either there is none left or it is interrupted.
This loop find optimal solution for each criteria, then moves to the next including the constraints stopping a less optimal solution to be found.
The first action is to $\verb+pop+$ off the (current) most important criteria and store it in the variable \verb+current-criterion+.

Then the inner loop based from original iterative strengthening algorithms main loop is then defined.
This main difference between this and the originals inner loop is the \verb+strengthen+ function is now defined per criteria from the function \verb+current-criterion.strengthen+.
This means the \verb+strengthen+ function changes with the outer loop, as it optimises for different criteria.
As this inner loop iterates more optimal solutions are found, until there exist no more to be found.

When the \verb+current-criterion+ has strengthened the formula to make it unsatisfiable, the constraints that made it unsatisfiable are removed.
This is necessary to continue the search for more optimal solutions judged by the next criteria.
The formula then has constraints from the function \verb+current-criterion.lock+ added to it, this ensures no worse solution will be found on the next iteration.

These loops will iterate until either all the criteria have been optimised, or the function is interrupted.
The result is a set of literals that satisfy the constraints in the variable $model$.
These literals can be easily mapped to a set of components that are non-negative literals in the model.
That is, the resulting system of components is equal to $[c \mid c \in \mathbb{C}$ and $c \in model]$.

\subsubsection{Criteria}
%%%This algorithm shows the use of some methods that criteria must implement
This new algorithm allows us to describe the necessary requirements a criteria must define in order to be used.
Each criteria must define the functions \verb+strengthen+ and \verb+lock+ to be used in this algorithm.
Further more, through altering the \verb+decide+, or literal order, function in the DPLL algorithm, each criteria could make the search for optimal solutions more efficient.
This will help guide the search to near optimal solutions, which will lower the upper bound of the number of loops necessary to find the most optimal solution.

%%%Initial constraints
Some criteria also require the ability to add constraints to the formula before the outer loop starts,
this is implemented in the function \verb+criteria.initial-constraints+.
These constraints are typically used to measure quantities that have no direct variable available.
A typical case for this is when a criteria needs to measure component by their names.
For instance, if component $\langle a,2\rangle$ was removed and component $\langle a,3 \rangle$ was added. 
Although a component was removed and a new one added, the component name $a$ still exists in the system, so no change to component names were made.

Further description of how initial constraints are used is given in chapter \ref{strategies}. 

\subsubsection{Formal Criteria}
Each criterion in the stack \verb+criteria+ contains a pseudo-Boolean function that either requires minimisation or maximisation.
This equation, represented by a list of literals and coefficients, is used to define the \verb+strengthen+ function.

The ranking function \verb+criteria+.$f$ is the pseudo-Boolean function used by the criteria.
Given a criterion's pseudo-Boolean function $f$,
the function \verb+strengthen+ the criterion is defined to return a pseudo-Boolean constraint such that:
\begin{itemize}
  \item if the function is being minimised then the returned constraint is $\langle f,<,f(l_2) \rangle$
  \item if the function is being maximised then the returned constraint is $\langle f,>,f(l_2) \rangle$
\end{itemize}
That is, the strengthen function for a criteria is created from its pseudo-Boolean function.
The function \verb+lock+ is similarly defined except the inequalities $>$ and $<$ are replaced respectively with $\geq$ and $\leq$.

\subsection{Drawbacks of this Optimisation approach}
%%%There are a few drawbacks to this mapping;
GJSolver's implementation tries to address the initial set of requirements by using an SAT solver extended with pseudo-Boolean constraints as its core,
and the lexicographic iterative strengthening algorithm for optimisation.
This has created some drawbacks that will be discussed here.

%%%The optimisation must be represented linearly
As GJSolver uses a SAT solver at its core, 
and it has been noted in \citep{le_berre_dependency_2009} and \citep{leBerre2010} there is no easy solution to extending a SAT solver to handle non-linear constraints.
There is no practical way of creating criteria that can optimise have non-linear functions within GJSolver.
This will stop many criteria from being defined, though the trade-off is that GJSolver will remain relatively simple and efficient.

%%%Real numbers must be truncated to fit the integer representation. 
The pseudo-Boolean constraints use only integer and not real coefficients.
This stops the definition of criteria that optimise values like percentages.
A practical solution to this is to multiply any real number by a large value and truncate the product to an integer representation.
Scaling a value in such a manner, may remove a small amount of resolution of the real value, though it does allow for criteria to use real values numbers.

\section{Verification of GJSolver}
\label{impl.verif}
%%%We entered this solver into two Mancoosi MISC competitions
The GJSolver implementation was verified by entering it into the Mancoosi International Solver Competition, whose requirements are described above.
This process was taken twice, firstly in a MISC Live event, which is an interim competition held during the year;
secondly at the MISC 2011 event, 
which is the main competition, whose results are announced at the Workshop on Logics for Component Configuration\footnote{http://www.pps.jussieu.fr/~treinen/lococo/2011/ accessed 6/3/2012}.

\subsection{Tracks and Scoring}
As MISC Live and MISC are competitions to compare solvers entered by different developers and researchers,
criteria to select the ``best'' solver must be defined through a scoring system.

Each competition is broken down into three possible tracks, each defined by the criteria used to solve the set of problems.
The first basic track, is 'paranoid', the second more advanced track is 'trendy', and third track is 'user'.
Both 'paranoid' and 'trendy' have pre-defined criteria in set lexicographic order for the solvers to use, 
and the 'user' track uses pre-defined set of criteria whose configurations are unknown to the solvers before the competition.
This means that 'paranoid' and 'trendy' can have solutions tailored to their specific criteria that is required, where the 'user' track cannot.
The exact criteria and how they are defined, are discussed in the next chapter.

For each track, a set of solvers is entered.
Each track has a set of problems to solve, and all participating solvers are executed to return solutions to these problems.
Each solution is scored and all are summed to get a final score of a solver for a given track, where the solver with the lowest amount of points is victor.

The way in which a solution is scored is by first giving it one of three classes; a real solution; no solution; a incorrect solution.
A real solution is any solution that is correct; no solution occurs when a solver finished without output, this can happen because of error, timeout, or there not being a satisfiable solution;
an incorrect solution is the worst class, as it can cause an incorrect system to be created.

If $m$ is the number of solvers entered into the competition,
a real solution is given $1$ point if it is the best solution returned by any solver, and $1$ plus how ever many solvers found better solutions.
For example, if solver $s1$ found a solution where solvers $s2,s3,s4$ found a more optimal solution, then $s1$ gets $4$ points for that solution.
If no solution is returned then $2\times m$ are the points given, and if an incorrect solution is returned then $3 \times m$ points are given.

If more than one solver has the same amount of points at the end of a track, then the time it took for them to find each solution is summed and the solver that took the least time wins.
This is also a lexicographical order of points, where it is infinitely better to return good solutions, than to return solutions quickly.

\subsection{MISC Live}
%%%In the first competition we had only partially implemented much of the functionality, so we did not expect great results.
The first competitions, MISC Live, was entered without GJSolver being fully implemented.
Therefore, the only track that was possible to enter was the basic track 'paranoid'.
The results for this track\footnote{http://mancoosi.org/misc-live/20101126/paranoid/ accessed 6/3/2012} where promising, though there where some clearly necessary improvements.
Due to the competition openly distributing the solutions and the output from the entered solvers, solution to problems from other solvers, 
as well as GJSolver, were able to be analysed to search for improvement opportunities.

\subsection{MISC}
The main verification of GJSolver was through the MISC 2011 event.
In this event GJSolver was entered into all tracks, where the 'paranoid' track had a total of 5 solvers, the 'trendy' track had a total of 6 solvers, 
and the 'user' track had a total of 4 solvers.
Each track was also entered by the solver which GJSolver is based on, Eclipse P2, and another very efficient solver aspuncud.

The scores and the times for each of the track compared to that from Eclipse P2 and aspuncud in table \ref{impl.misc2011}.
\begin{table}
\begin{tabular}{| l | c | c | c | c |}\hline
Track & \# of Problems & GJSolver & P2 & aspuncud\\ \hline
paranoid & 129 & (190 : 5,294) & (181 : 4,646) & (147 : 1,035) \\ \hline
trendy & 129 & (197 : 13,073) & (232 : 13,435) & (151 : 1,767) \\ \hline
user & 400 & (656 : 73,522) & (1392 : 87,956) & (1215 : 39,905) \\ \hline
\end{tabular}
\caption{Results from MISC 2011, results are (score:time in seconds)}
\label{impl.misc2011}
\end{table}

The total winners for each track where for both 'paranoid' and 'trendy' aspuncud won, and for the 'user' track GJSolver won.

\subsection{Analysis}
The main analysis of the results from the MISC competition is accomplished through comparing them to the previously stated requirements that
GJSolver should be \textbf{Mancoosi International Competition Ready}, should implement an \textbf{Anytime Algorithm} and should be \textbf{Easily Creatable Criteria}.

The results from MISC show that GJSolver was ready for its entrance, during the competition it had very consistent results.
These results allowed it to compete with the other solvers, and even win the 'user' track.

The implementation of the anytime algorithm was shown through the many hard problems required to be solved.
This can be seen in many of the problems in the 'trendy' track, these required a lengthy search and many required interruption of the algorithm to return the currently best found solution.
Looking at many of these timed out problems, reveals that many of them returned optimal solutions regardless of the fact they were stopped before finishing the search.
This leads to the presumption that the finding of the optimal solution is not necessarily the time consuming part of the search, but the proof that there are no better solution remaining.

The criteria that were defined in to enter this competition were wide ranging, and will be specifically discussed in chapter \ref{strategies}.
It may be said that the reason for GJSolvers win of the 'user' track could be put down to the easily defined and tuning criteria that allowed the quick testing of different heuristics.


%\section{Other Methods}
%This \citep{Stuckenholz2007} study looks at using boolean optimisation with branch and bound as a solution, as does \citep{Jenson2010a}.
%\subsection{Integer Programming}
%Discussion of this method as the best MISC solver uses this, it has a very complex implementation
%\subsection{SMT Solvers}
%SMT Solver, a slightly higher logic than SAT uses; it has to broad a definition when SAT suffices
%\subsection{Constraint Solver}
%We could just use Prolog, like SMT I think it is too broad when there are good SAT solvers


\section{Summary}
In this chapter first Boolean satisfiability solvers were introduced, 
describing the DPLL algorithm and its extensions through pseudo-Boolean constraints as a means to solve component evolution problems.
Then the implementation produced through this research, GJSolver, was introduced with the modified iterative strengthening algorithm to find optimal solutions.
Finally, the description of the MISC competition and GJSolver's verification through it were given to show the requirements were satisfied.
