\chapter{Implementation}
\label{implementation}
\epigraph{What I cannot create, I do not understand.}
{\textit{Richard Feynman, 1988}}

%%%CDR Has been shown to be NP-Complete to find a solution and NP-Hard to find an optimial one.
As presented in the previous chapter, the problem of evolving a component system was shown to be NP-Complete, and the problem of finding an optimal evolution was shown to be NP-Hard.
Therefore, to automate this complex evolution process it will require an efficient algorithm and implementation to not only find satisfiable solutions, but also optimal ones. 

%%%The efficiency of the implementation is important as the problem can combinatorially grow into a hard problem 
The efficiency of the implementations is necessary for many instances of CDR problems.
The goal of many component models is to have large amounts of components with support creating multiple versions of each component.
This directly makes the complexity of evolution grow in a combinatorial manner, making any ad-hoc or overly simple CDR implementation quickly overwhelmed.
The CDR algorithm should then be implemented to be robust and fast.

%%%Formally it resembles the SAT problem, which already has efficient implementations, and has been used to solve this problem
The formal representation of this problem was presented as a set of constraints that must all be satisfied in order for a component relationships and the user request to be fulfilled.
This structure closely resembles the Boolean Satisfiability Problem (SAT), which has fast, robust solver implementations that can be used.
This has been noticed by other researchers \citep{leberre2008,Mancinelli2006} 
and the use of SAT solvers to implement CDR in both academia \citep{abate2011} and industry \citep{leBerre2010} has become common.

%%%Finding the optimial solutions requires extensions to SAT solvers
The finding of an optimal solution though is difficult when using a ``pure'' SAT solver, as it cannot easily represent constraints relating to criteria. 
However, through extending the SAT solver implementations to also handle other types of constraints, such as pseudo Boolean constraints, optimisation becomes significantly easier.

%%%In this chapter\ldots
In this chapter the algorithms and implementation of CDR using SAT solvers extended with pseudo Boolean constraints, are discussed.
First the Davis-Putnam-Logemann-Loveland algorithm \citep{Davis1960, davis1962machine} and its extensions, which are the basis for many current SAT solvers, are described.
This will give a basic understanding of the internal workings of current SAT solvers.
The specific implementation of Eclipse P2 and P2CUDF by \cite{leBerre2010} is then described,
this is given as a comparison of a current technology used by CDR implementations.
Then the description of this research's CDR implementation, GJSolver, is given.
The specific reason for the necessary changes, the differences from other solutions, and the advantages to using it.
This also describes the process in which it was validated and compared against other solvers in the MISC 2011 competition.

\section{Boolean Satisfiability Solvers}
\label{impl.SAT}
Boolean satisfiability (SAT) is the problem of determining if the variables in a Boolean equation can be assigned in such a way that the equation return true.
This problem was the first identified NP-Complete problem, meaning that no known algorithm exists that efficiently solves all instances of SAT problems.
This also means that many proofs of NP-Completeness is through the reduction to SAT, and therefore many difficult problems have been expressed in the terms of a Boolean equation.
The benefit of this is that algorithms and implementations of programs that solve SAT problems have received significant attention and research to provide solutions to these various problems.
Taking advantage of these algorithms will help with the creation of a CDR implementation that can robustly evolve a component system.

%%%SAT solvers are used to solve many problems, creating a community dedicated to creating fast implenmenations
SAT solvers have been used in various domains to tackle problems such as electronic design automation \citep{Marques-Silva2000}, 
model verification \citep{dennis2006}, and, of course, component system evolution \citep{leBerre2010}.
These uses have caused a great need to create efficient solvers, and this has spawned a community dedicated to creating efficient implementation.
This community, creates and improves their solvers, 
and as a means of validating their creations, regularly compare them against one another in a series of competitions \footnote{http://www.satcompetition.org/}.
Using a competition for progress not only encourages the improvement of solvers for a particular problem set, but also general improvements to the algorithms and heuristics used. 

%%%In this section a common implementation of SAT solvers is discussed, the DPLL algorithm
In this section, first a brief description of the DPLL algorithm is given, this shows the basis of most current SAT solvers.
Current implementations of SAT solvers have altered the DPLL algorithm to become conflict driven,
this alteration will give a more realistic view of the internal workings of current SAT solvers.
Both the conflict analysis and the literal order of such conflict driven SAT solvers are briefly described.
as these are key parts of the algorithm and require further description.
The extension of this algorithm with Pseudo-Boolean constraints is then described,
as this extension allows for more complicated optimisation problem to be represented.
Finally, other methods of solving SAT problems are briefly discussed to give a broader description of the domain. 


\subsection{Davis-Putnam-Logemann-Loveland algorithm for SAT Solvers}
%%%A successful algorithm for solving SAT problems is the DPLL algorithm, here we describe it in overview
The Davis-Putnam-Logemann-Loveland (DPLL) algorithm \citep{Davis1960, davis1962machine} for solving SAT problems is a complete, meaning it will find a solution if one exists, 
backtracking-based search algorithm for SAT problems represented in conjunctive normal form (CNF).

It runs by recursively selecting a variable from the Boolean formula, then assumes that it is true, calculates the implications of this assumption.
If the implication is that this assumption leads to a formula that is unsatisfiable (determined by calling itself), then it assumes that the variable is false.
A formula can be determined to be satisfiable if after the manipulation through the algorithm occurs only unit clauses are left (that is clauses with only one literal).
The formula is determined to be unsatisfiable if either there exists an empty clause, which means all possible assignments making that clause true have been eliminated.

Specifically, given a set of variables $V$, 
each variable $v$ in $V$ having two literals $v$ and $\neg v$
a clause is a set of literals, 
and a CNF formula $F$ which is a set of clauses,
$DPLL(F)$ will return true if there exists a set of literals that satisfy all clauses in $F$. 
The DPLL algorithm in defined in figure \ref{impl.DPLL}:

\begin{figure}[h]
\begin{center}
\begin{alltt}
function DPLL(\(F\))
   if \(F\) is a consistent set of unit clauses:
       then return true;
   if \(F\) contains an empty clause
       then return false;
   for every unit literal \(l\) in \(F\)
      \(F\) = unit-propagate\((l, F)\);
   for every literal \(l\) that occurs pure in \(F\)
      \(F\) = pure-literal-assign\((l, F)\);
   \(l\) = decide\((F)\);
   return DPLL\((F \cup \{l\})\) or DPLL\((F \cup \{\neg l\})\);
  		
\end{alltt}
  \caption{Recursive DPLL algorithm}
  \label{impl.DPLL}
\end{center}
\end{figure}

The first \verb+if+ branch of the DPLL algorithm finds if the formula is satisfiable only if it consists of consistent unit clauses.
The second \verb+if+ branch determines if the formula is unsatisfiable by looking for a clause that contains no literals, thus it is unsatisfied.

The first loop of the algorithm aims to remove literals from clauses in $F$ whose negation are in unit clauses.
For example, if $F = \{c1,c2,c3\}$ where $c1 = \{v\}$ $c2 = \{\neg v, a\}$ and $c3 = \{\neg a , x\}$,
the set of unit clauses in $F$ is $\{c1\}$ making $v$ the only unit literal.
Therefore, \verb+unit-propagate+$(v,F)$ removes all $\neg v$ literals, making $c2 = \{a\}$.
This propagation then made the clause $c2$ and literal $a$ unit, continuing the process with \verb+unit-propagate+$(a,F)$ which makes $c3 = \{x\}$, making $x$ unit.
So this process of unit propagation turned $F$ into a set of unit clauses, making it satisfiable.

The second loop simplifies the formula by removing the clauses which are known to be true, as they are superfluous to the formula.
This is done by identifying literals which have no negation in the formula, and then removing any clause which contains that literal.
For example, if the formula $F = \{c1,c2,c3\}$ where $c1 = \{v\}$ $c2 = \{\neg v, a\}$ and $c3 = \{\neg a , x\}$, 
the set of pure literals equals $\{x\}$ as it has no negation in any clause.
Therefore, \verb+unit-propagate+$(x,F)$ would remove the clause $c3$ from the formula, so $F = \{c1,c2\}$, this in turn makes $a$ pure.
Once \verb+unit-propagate+$(a,F)$ then eliminates $c2$ from the formula to leave $F= \{c1\}$ making all clauses unit in the formula thus it is satisfiable.

The function \verb+decide+ returns a literal that is not unit, nor whose negation is unit, i.e. given $l =$ \verb+decide+$(F)$, $\{l\} \not \in F$ and $\{\neg l\} \not \in F$.
This literal is then added to the formula and checked for satisfiability, 
and if it is not, the satisfiable of the formula where its negation is added is checked.  

\subsection{Conflict driven Solvers}
Though the DPLL algorithm is the basis of most modern SAT solvers, the actual implementations have altered to increase efficiency.
The core change is from being a recursive propagation based algorithm to a conflict driven algorithm,
where the reason for conflicts are calculated in order to prune the search tree.

\begin{figure}[htp]
\begin{center}
\begin{alltt}
conflictDrivenDPLL:
	c = propagate(F)
	if c = null then
		if all variables assigned then
			return true
		else
			v = decide
			F = F cup {c}
	else
		c = analyse(c)
		if c conflicts with F
			return Unsatisfiable
		else
			backtrack()
			
\end{alltt}
  \caption{Conflict Driven SAT solver algorithm}
  \label{impl.conflictSAT}
\end{center}
\end{figure}

%%%Some improvements over the core algorithm have been appended over the years since DPLL was first proposed.

\subsection{Conflict Analysis}
%%%Simple resolution as proposed by Stallman was an effecient means to stop thrashing(looking at the same subtree repeatedly)

%%%Improvement of this resolution can be made by finding smaller constraints, this can be accomplished by eliminating literals in a constraint

%%%Further improvement can be made by searching for the smallest constraint through such variable elimination

\subsection{Literal Order}
%%%The order in which the literals are assumed in the DPLL algorithm, is heuristically accomplished

%%%The selection of a literal can be broken into two stages for modularity;
%%%first selecting a variable that is very volatile,
%%%second selecting a phase (negative or positive)

\subsection{Pseudo-Boolean Optimisation of SAT Solvers}
%%%Optimisation of SAT solvers is typically done through extending their possible constraints to include Psuedo Boolean inequalities

%%%Then through repeatdly finding a solution then adding a constraint to ensure the next solution is at least as good, the best solution is found

%%%Further optimisation can be made through quickly finding a satisfactory solution, this reduces the space in which to find a better solution

\subsection{Other Methods}
This \cite{Stuckenholz2007} study looks at using boolean optimisation with branch and bound as a solution, as does \cite{Jenson2010a}.


\subsubsection{Integer Programming}
%%%Discussion of this method as the best MISC solver uses this, it has a very complex implementation
\subsubsection{SMT Solvers}
%%%SMT Solver, a slightly higher logic than SAT uses; it has to broad a definition when SAT suffices
\subsubsection{Constraint Solver}
%%%We could just use Prolog, like SMT I think it is too broad when there are good SAT solvers



\section{GJSolver}
%%%My Implementation, cut the fat, straight forward

\subsection{Product and Lexicographic orders in Pseudo-Boolean Optimisation}
%%%Here we descirbe the mapping from our Optimisation framework to PBO, product order is easy, lexicographic requires some muddling

\subsection{Drawbacks of this Optimisation approach}
%%%There are a few drawbacks to this mapping;
%%%These simplifications are mostly necessary for this problem as it keeps the problem manageable,
%%%and including all aspects of our formal optimisation framework would be practically impossible

%%%The optimisation must be represented linearly
As noted in \cite{le_berre_dependency_2009} and \cite{leBerre2010} there is no easy solution to extending a SAT solver to handle non-linear constraints.

%%%Real numbers must be truncated to fit the integer representation. 

%%%Only one solutin can be chosen so product orders are randomly selected to a degree, this can be mitigated by having no stochastic elements in the algorithm

%%%No criterion can be of a partial order though through product composition the problem can be a partial order

%%%The only aggregation of numbers between integers is addition.

%%%It is too expencive to recalculate cardinalities for each component given a specific solution, therefore the cardinalities of a component should be solution independant

\subsection{Verification}
%%%We entered this solver into two Mancoosi MISC competitions

%%%In the first competition we had only partially implemented much of the functionality, so we did not expect great results.

%%%In some instances we where returing non optimal solutions, The bug where we had to add all package versions of all components. 

%%%These problems where fixed, through comparing the results we had gotten from the competition with those from other solvers.
%%%We created our own miny competition which we ran to also ensure that time limits where adhered to.

%%%In the second competition we had fully implemented the solver and this allowed us to enter all tracks with EXCELLENT results

%%%The paranoid track

%%%The trendy track

%%%The user track




