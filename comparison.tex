\chapter{Component Model Comparisons w.r.t. Component Dependency Resolution}
\label{comparison}
%%%These investigations outline the first investigations into the structure of component dependencies
{}Two of the leading component models that CDR is currently used with are OSGi and the Debian Package system.
{}The differences that occur because of their differences will effect the dependencies and relationships between components,
{}and then affect the properties of CDR or the way it is used.
{}To generalise any results, or to analyse the effects of different CDR criteria we must first compare them.
{}We first give specific differences and then use dependency graph and constraint analysis to explore their consequences.
{}We find \ldots%TODO   


\section{Differences}
{}Different aspects of a component model and its use can effect the dependencies and relationships between components,
{}therefore affect the properties of CDR or the way it is used.
{}Aspects such as, developer communities and conventions, what is allowed in the syntax of the meta-data expressing the dependencies,
{}and the semantics of the component model will affect CDR.  

%%%Developer conventions like in eclipse depending explicitly on component, vs. OSGi where it is prefeered to depend on packages

%%%Metadata syntax will allow certain constraints to be defined, in eclipse the only conflict is in a singleton relationship.

%%%Component model Semantics, how dependencies are manifested, 
	
	

The semantics of a component model will effect how the components are defined and used within that model.
This will effect the relationships between components and therefore CDR will be effected by these changes.

\section{Detailed comparison of OSGi and Debian}
%%%Here we will discuss various static graph analysis, between the OSGi component model and the Debian component model

\section{Dependency Graph analysis}
%%%Distributions of dependencies...

\section{Constraint analysis}
%%%Here we analyise the size of the possible search spaces of both OSGi and Debian, through the simple mechanism of brute force search

%TODO This maybe
%\section{SAT solver optimisation}
%We change some of the different variables in the SAT solver (constraint resolution, oredering, ...) and analize the domain specific results

\section{Conclusions}
%%%Here we can go over the similarities and differences between the two component models (hopefuly the are similar enough)
