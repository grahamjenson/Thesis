\chapter{Component Model Comparisons w.r.t. Component Dependency Resolution}
\label{comparison}
%%%An important issue when looking at this problem of CDR can information gained through studying one instance be generalisable to another.\\
{}Different aspects of a component model and its use can effect the dependencies and relationships between components,
{}therefore effect the properties of CDR or the way it is used.
{}Aspects such as, developer communities and conventions, what is allowed in the syntax and the semantics of the component model to express dependencies,
{}and goals of the component system will affect CDR.
{}This will clearly affect the study of CDR and its results, and whether they are generalisable.
{}Developer conventions created by communities, component model semantics, goals of the component model can all effect the constraints on the system thus CDR's solutions.

%%%Developer conventions like in eclipse depending explicitly on component, vs. OSGi where it is preferred to depend on packages

%%%Component model semantics, how dependencies are manifested, some can have multiple version of the same component installed with complex wiring standards,
%%%some certain constraints to be defined, in eclipse the only conflict is in a singleton relationship.

%%%The goals of the component model, whether adaptability is the core goal or a precisely defined system

%%%We use two methods to compare and analyse these possible effects on the component models, graph and constraint analysis, 
%%%to ultimately determine if through studying one of these component models the results can be generalised.

%%%We study and compare two leading component model, OSGI and Debian, using thee methods.
Two of the leading component models that CDR is currently used with are OSGi and the Debian Package system.
The differences that occur because of their differences will effect the dependencies and relationships between components,
and then affect the properties of CDR or the way it is used.
To generalise any results, or to analyse the effects of different CDR criteria we must first compare them.
We first give specific differences and then use dependency graph and constraint analysis to explore their consequences.
We find \ldots%TODO   

%%%Clearly these results themselves cannot be generalised, but through this method we hope that other component models can be checked to see if our results will be consistent.

\section{OSGi and Debian}
%%%Here we will discuss various differences between the OSGi component model and the Debian component model, their goals communities of developers, specifications\ldots

\subsection{Debain}
%%%Debian Developer Manual


\subsection{OSGi}
%%%OSGi specification and compendium 

\section{Static Research}
%%%Discuss information gained through repositories that does not require CDR 
%%%e.g. often versions are released, how often versions change their dependencies, size and granularity of components 

%%%We can look at popular packages in both with app-install-data and also eclipse repository

\section{Dependency Graph analysis}
%%%Distributions of dependencies, graph properties, small world? What does their internal graph structure look like?

\subsection{Criteria}
%%%We can use some of the previously defined functions to analyse the graphs

%%%Page rank distributions

%%%HITs distributions

%%%Coupling and instability

\section{Constraint analysis}
%%%The graph removes some of the details, e.g. conflicts, thus it is a broad view of their structure

%%%By analysing the constraints created by the components, e.g. number of conflicts, distributions of dependency sizes\ldots, we can gain finer grained information

\subsection{Search Space Size}
%%%Here we analyse the size of the possible search spaces of both OSGi and Debian, through the simple mechanism of brute force search

%%%We can analyse their distributions of search space

\section{Conclusions}
%%%Just because a single solution can solve two problems, does not make the problems the same.
%%%Here we look at the 2 problems of

%%%Here we can go over the similarities and differences between the two component models, hopefuully they are the same

%%%If they are very different, accounting for their relative sizes and maturity, we can eliminate the idea that our other results are generalisable (which is useful but not preffered)
