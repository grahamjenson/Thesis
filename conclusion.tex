\chapter{Conclusion}
\label{conclusion}
%%%Answers to our initial questions
This research has investigated the evolution of component systems using a developed simulation.
To build this simulation, first the background of software evolution, component-based software engineering and component systems were explored (chapter \ref{background}).
The models \modelname (chapter \ref{formal}) and \usermodel (chapter \ref{simulation}) were then developed 
to describe a component system evolving through repetitive change and the user that requests the changes.
The implementation GJSolver (chapter \ref{implementation}) was then created to calculate the exact changes made to a system during evolution.
The verification and validation of these artifacts were discussed, in accordance with our methodology \cite{Law2005}.

Experiments were defined that use these artifacts to simulate the evolution of many different systems.
The primary focus of these experiments were the out-of-dateness and changes made to the systems during evolution.
The experiments have provided empirical evidence for the discoveries:
\begin{itemize}
  \item Upgrading a system causes the majority of change 
  and installing new components only adds to this change by increasing the number of components to be upgraded.
  \item Upgrading frequently during most of the year is unnecessary as it has little impact on the out-of-dateness of systems.
  However, during periods where new component versions are released quickly, e.g.\ during Ubuntu releases, 
  it is important to upgrade frequently. 
  \item Regular updates cannot ensure that a component system stays up-to-date.
  This is in part due to the criteria used by \texttt{apt-get} that restricts adding new components when upgrading.
  \item Frequently upgrading a system may cause additional change 
  when many versions of a component are released quickly, forcing it to be upgraded multiple times.
\end{itemize} 

This research also provides empirical evidence and quantitative measurements of the benefit of reuse, especially during CSE.
The main benefit during CSE is the reduced change due to lowering the amount of components to be installed,
that in turn lowers the amount of components that are upgraded.
Reuse was estimated to save 650 changes on a system during the year. 

To reduce the out-of-dateness of systems during evolution, the progressive criteria was proposed.
This criteria allows new components to be installed to enable the upgrading of components.
This alteration was shown to be most effective for users that upgrade frequently.
It was shown to decrease the out-of-dateness of a system by as much as 49\% at the cost of increasing change by 34\%.
For users that prefer up-to-date systems, this trade-off might be favourable for their system.

To reduce the change made to a system during CSE the unstable criterion was proposed.
This criterion waits for a period of days for a component to cease releasing newer version and become stable.
By using this criterion and waiting a week for a component to become stable before upgrading it, it was estimated to save some types of users more than 30 changes over a year.
These changes could introduce bugs into the system and are seen as unnecessary. 
Using this criterion comes at the cost of the system always being out-of-date by the number of days that are waited for components to become stable.
For users that prefer to reduce change to their systems, this may be a tolerable trade-off.

The remainder of this chapter will examine the proposed thesis in chapter \ref{introduction},
the potential future research,
and provide a closing remark.
\section{Thesis Validation}
This research started with the thesis:
\begin{quote}
\textit{It is possible to reduce the negative effects of component system evolution by altering the mechanisms by which systems are changed.} 
\end{quote}

The steps taken to validate this thesis were:
\begin{itemize}
  \item To develop a reproducible and controllable environment in which to measure the effects of CSE.
  \item To use this environment to study how systems evolve.
  \item To alter the mechanisms by which systems are changed and study their impact on CSE.
  \item To demonstrate a reduction on change and out-of-dateness using such alterations.
\end{itemize}

To address these steps:
\begin{itemize}
  \item A simulation was developed.
  \item It was used to study CSE through simulating the evolution of many Ubuntu systems.
  \item The novel unstable criterion and progressive criteria were proposed and their impacts on CSE studied.
  \item The unstable criterion and progressive criteria were then simulated to approximate the evolution of real component systems
  and they were demonstrated to reduce change and out-of-dateness.
\end{itemize}  

It is therefore possible to conclude that the thesis is supported.
 
\section{Future Research}
\label{conclusion.futurework}
The largest area of future research is in the simulation of additional systems, such as Eclipse, 
so they can be studied as they evolve. 
Through such simulation the conclusions of this research can be shown to be either invariant between component systems 
or to be Ubuntu-specific.
The core hurdle of accomplishing this is that much of the collected information in this study does not exist for other component models.
Component models like Eclipse do not have their history archived as precisely as Ubuntu.
Information, like the Ubuntu repository, does not exist or does not contain the detailed information for other component models.
Resources of information are available, but can be incomplete.
For example, the Usage Data Collector\footnote{http://www.eclipse.org/epp/usagedata/faq.php} for the Eclipse project (similar to the Ubuntu Popcon) collected information on popular Eclipse components. 
However, this project was shut down as it was deemed not to have a significant return on investment. 
 
Another area of research involves altering and extending the \modelname and \usermodel models to more precisely describe CSE.
An aspect of CSE that is not included in \modelname is that a component system may only view a certain subset of components that exist.
This is described in section \ref{sim.modelvalidation} as the view Ubuntu has of its repository.
The \usermodel has many aspects that could be added to make the simulated users closer to reality.
Constraints such as only upgrading during the work week, 
or additional types of request like removal of components that are no longer required, could create more valid simulations.

The development of criteria that use other properties of components, 
could be used to more accurately represent the user's preferences.
Such properties could be collected from additional sources
, e.g. component meta-data, component contracts \citep{Watkins1999}, repositories \citep{Guo2000}, 
composition testing \citep{XuejieZhang2008}.
These could be used to create criteria that optimise for size, reliability, licence, even validity.
Such criteria could be simulated and their effect, positive or negative, 
could be measured using the simulation developed in this research.

%%%Further stufy of the interaction between users and their component systems. We found it hard to find information, maybe some research would be interesting,
%%%which we could include to make the simulations more valid.
The part of the simulation that has the most room for improvement is the selection of the requests made by the user.
This is when they select to upgrade, when they select to install, and what components they select to install.
By studying real users changing their systems, either through collecting more \texttt{apt-get} logs or through direct observation,
a more valid simulation could be created.

%%%Integration with MPM then test on real systems the implications of this research
Comparing the results from the simulations to the evolution of real systems would be able to validate the conclusions drawn from the simulation.
This could be accomplished using the MPM solver \citep{abate2011} from Mancoosi.
This solver has already been proposed for the modification of real systems.
Through modification or extension of MPM, the proposed criteria could be used to evolve real Ubuntu systems.
Such a study would require significant resources, but its results and conclusions would be worthwhile as they could provide a direct benefit to a very large number of users.

\section{Closing Remark}
This research has given an understanding of how component systems evolve and can be used to provide users and developers with insights into the effects of their choices.
Additionally, this research has proposed two novel ways to reduce negative effects during CSE and the tools in with which to measure the effectiveness of these techniques.
This research, therefore, has the potential of impacting the millions of Ubuntu and Eclipse systems that are currently evolving.

