\chapter{Background}
\label{background}
\epigraph{
In order to agree to talk, we just have to agree we are talking about roughly the same thing.
}
{\textit{The Feynman Lectures on Physics, Motion, Richard Feynman, 1961.}}

%%%What is Software Evolution
Software evolution \citep{lehman1980} is the process of change made to a software system to maintain it, or to extend its functionality over the systems lifetime.
This evolution process is necessary as the system must adapt to the changing software environment, accommodate new user requirements, 
fix errors, and/or prevent errors from occurring in the future \citep{IsoIec2006}.
Software maintenance and evolution are often used interchangeably \citep{Godfrey2008}, though some differences exist.
One difference is that maintenance has connotations of a planned activity, where evolution is the gradual refinement of a system \citep{lehman1980}.
Throughout this thesis software maintenance is referred to as the range of activities through which software is changed; this is similar to the definition given in \citep{lehman1980}.

The concept of software evolution has been discussed since the 1960's \citep{Lehman1969}, 
however some still see it at a young field \citep{Godfrey2008} as many open questions remain to be answered.
One such open question is: \textit{how can software engineers lower the cost of software evolution?}
The cost of evolving a system was estimated by \cite{Brooks1975} to be as high as 90\% of the total cost of the system.
More empirical studies that explore the cost of software evolution are summarised by \cite{Grubb2003}, showing that the costs of software maintenance range from 49\% to 75\%, 
and that these costs have not fallen since the 1970's. 

To lower the cost of software evolution, various methods and tools have been proposed.
For example, agile software development methodologies \citep{beck2001manifesto} are defined to encourage rapid and flexible responses to change,
and refactoring tools \citep{fowler1999refactoring} have been developed to restructure code to decrease complexity and increase maintainability. 
Another way to lower costs of the software evolution process is by creating systems from encapsulated units called \textit{software components} \citep{Szyperski2002}.

%%%Composition
A component system is created out of a set of components that are combined into a functioning system by a composer (or assembler) \citep{Szyperski2002}.
Composer's are also the ``evolvers'' of a system, as they have the power to alter the composition of components that make the system.
Originally it was assumed that developers were the composers \citep{Parnas1972,PrietoDiaz1987}, where they would compose, verify, then release a system.
However, recently the user has started to fill the composer role, where they compose their own systems to satisfy their requirements.

The situation where the user is the composer of the system has been called system tailoring \citep{Morch1997} and end-user assembly \citep{Szyperski2002}. 
It has been noted that the potential, and fragility, of component systems is increased with this user composition \citep{Szyperski2002}.
A system has more potential as the user can craft solutions without requiring expert assistance.
A system is also more fragile, as the quality of a system cannot be verified by the user, as they do not have the tools or knowledge required.
This means that the system is tested by the user running it. 

The decoupling of the developer and composer roles also breaks the developer's control over the evolution of the component system.
The developer is no longer in charge of evolving the deployed system.
This means that a component can be composed into a system without prior testing to check if that specific system will work.
This leaves the evolution of the component system in the hands of its user.

If the user is to be the composer, and they are not a technical expert, then there must be guidance when composing or evolving a component system \citep{Szyperski2002}.
The core technical knowledge required to compose a component system comes from the complex relationships between components.
These relationships create constraints that must be satisfied in order for the systems components to be functional.
The guidance to satisfy these constraints comes in the form of a process called Component Dependency Resolution (CDR) in this thesis.
CDR takes a user's request to change the composition of their system, and attempts to create a functioning system 
that satisfies all component relationship constraints and the users request.

%%%Defining components
How ``software component'' is defined will impact how component system evolution and CDR are represented.
It is difficult to find a precise definition of what a software component is, as the intuitive concept may be quite different from any model or implementation \citep{Crnkovic2011}.
Finding a definition that satisfies every party may be an impossible task.
However, by defining a software component only with respect to component system evolution and CDR, 
this process can be studied without the paralysis of finding a complete definition.
In this thesis a software component is defined to have explicitly declared relationships and include mechanisms to automatically alter the components composition.
These two attributes allow CDR to evolve a component system and component system evolution to be studied. 

%%%Examples of component models
With respect to this definition, various component models can be described.
These component models may differ by functionality, platform, or their objectives for being created.
They may also have their own CDR implementations, created to assist in their component systems evolution. 

%%%In this chapter\ldots
This chapter illustrates the histories and the present day picture of software evolution and component-based software engineering are examined in section \ref{background.histories}.
In the next section, component evolution and component system evolution are discussed.
The definitions of software components and component models with respect to component system evolution are then presented in section \ref{background.components}.
To conclude this chapter,  examples of various component models and CDR implementation are then given in section \ref{background.models}.
This includes an introduction to the Debian model, and CUDF model that are described in more detail in the following chapters.

\section{Software Evolution and Component-Based Software Engineering}
\label{background.histories}
The foundations for software evolution, software components, and component systems were laid in 1969.
Software evolution, the process of changing a system to adapt the environment it exists in, 
was first described by \cite{Lehman1969}.\footnote{though it was not referred to as evolution at that time}
\cite{McIlroy1969} proposed his concept of software components that were created by a ``subindustry'' of suppliers.
Furthermore, that year the operating system Unix \citep{raymond2003art} was developed, which at its core philosophy is a modular system.

The domains of software engineering, software evolution and component-based software engineering, have gone through many advancements since their inceptions.
This section breifly describes these advancements, building up to a discussion about component evolution, and component system evolution.

\subsection{Software Evolution}
\cite{Brooks1975} states that over 90\% of the cost of a system occurs after deployment in the maintenance phase,
and that any successful piece of software will inevitably be maintained.
This realisation, that software requires significant expense to maintain, lead researches to study how software behaved after deployment.
In this research software maintenance is a range of activities that evolve software \citep{lehman1980}.

In 1980, two fundamental empirical studies on the emerging domain of software evolution were published.
The first study by \cite{Lientz1980} explored the activities that occur during software maintenance (later formalised in ISO/IEC 14764 \citep{IsoIec2006});
\begin{enumerate}
	\item \textit{Adaptive Maintenance}: adapting to new system or technical requirements.
	\item \textit{Perfective Maintenance}: adapting to new user requirements.
	\item \textit{Corrective Maintenance}: fixing errors and bugs.
	\item \textit{Preventive Maintenance}\footnote{later added in taxonomies such as \citep{iee1990ieee}}: adapting to prevent future problems.
\end{enumerate} 

This study showed that around 75\% of the maintenance effort was on the first two types, and corrective maintenance took about 21\% of the effort.

The second study by \cite{lehman1980} explored how the evolution affected the software.
He described a set of laws, similar to the laws of nature like gravity, that characterise software evolution:
\begin{enumerate}
  \item \textit{Continuing Change:} Software systems\footnote{E-type systems: software implemented in a real-world computing context \citep{lehman1980}} must be continually adapted 
  otherwise, they become progressively less satisfactory.
  \item \textit{Increasing Complexity:} As the system evolves its complexity increases unless work is done to reduce it.
  \item \textit{Self Regulation:} The system evolves with statistically determinable trends and invariences.
  \item \textit{Conservation of Organisational Stability:} The average effective activity rate to evolve a system is invariant over its lifetime.
  \item \textit{Conservation of Familiarity:} As the system evolves, its incremental growth remains invariant to ensure users maintain mastery over the system.
  \item \textit{Continuing Growth:} The system must continually grow to maintain user satisfaction.
  \item \textit{Declining Quality:} The quality of the system will decline unless rigorously maintained.
  \item \textit{Feedback System:} The function a system performs is changed by the effect it has on its environment.
\end{enumerate}

%%%These laws describe the fact that it is impossible to create a single satisfactory system, and an iterative approach is best to strive for continual satisfaction
Both these studies from \cite{Lientz1980} and the laws from \citep{lehman1980} argue that
the software engineer's objective of creating a satisfactory system is difficult, expensive, and not often achievable.
They state that the continual evolution of a software system is necessary, and this evolution reduces quality, increases complexity, and is costly.

From the perspective of software evolution, the software engineer's goal is then to create a system that can be quickly altered to adapt to a changing environment, 
while working to reduce the inevitable complexity caused by changing software.
Towards such goals, iterative development processes have been created, such as the spiral development method presented by \cite{Boehm1988}.
This process describes the stages of development as communication, planning, modeling, construction, and deployment. 
These stages are continually iterated until the software project is no longer actively maintained. 

The practical problems of software evolution can be seen in the struggle with legacy software \citep{Bennett1995}.
Legacy software is functional software that is old and outdated, but it has not been replaced due to its critical status, it not being well understood, or the cost to redesign it.
A piece of software is described as ``legacy'' if it cannot be evolved  (due to its complexity or size) within an acceptable cost \citep{Bisbal1999}.
As a legacy system cannot be evolved, new user and technical requirements cannot be fulfilled, and the satisfaction with the system will decrease over time.
This has lead the problem of legacy software to be described as enduring \citep{Bennett2000}. 

Current explorations of the history, and state-of-the-art, of software evolution are presented in \citep{Bennett2000}, \citep{Lehman2003}, and \citep{Godfrey2008}.
In all these papers, the importance of software evolution is emphasised, and the need for more knowledge about the evolution process and its properties is discussed.

\subsection{Component-Based Software Engineering}
The concept of Component-Based Software Engineering (CBSE) was first outlined by \cite{McIlroy1969}, 
by describing the idea of a software components subindustry which created components to be used to in software.
This report is an expansion on an earlier idea for \textit{pipes}, presented in \citep{mcilroy1964pipes}, 
where he described designing software to fit together, like screwing a hose to a tap.

The reuse of code in order to decrease development time, was originally the major perceived benefit of using software components.
Later, other benefits of constructing systems from modular components were identified by \cite{Parnas1972}:
\begin{itemize}
  \item \textit{Managerial Separation}: develop components in separate groups with little communication
  \item \textit{Product Flexibility}: make drastic changes to one component, without changing others
  \item \textit{Comprehensibility}: be able to study the system one module at a time
\end{itemize}

The software component concept was soon picked up by other researchers such as \cite{Yourdon1976}, where they described their ideas on structured design as:
\begin{itemize}
  \item the art of designing the components of a system and the interrelationship between those components in the best possible way.
  \item the process of deciding which components are interconnected in which way to solve some well-specified problem.
\end{itemize}
\cite{Yourdon1976} list the goals of such a design as efficiency, maintainability, modifiability, generality, flexibility, and utility.
These goals are aimed to be achieved by dividing the system into functional units that can be treated independently.
Each unit corresponds to exactly one small, well-defined piece of the system, and the units relationship corresponds to a relationship between pieces of the system.

A problem soon emerged when using software components, that of which set of components to compose together to create a system.
\cite{PrietoDiaz1987} describe this as the \textbf{selection problem}, where a composer can have many alternative compositions of components to select from.
This increases the effort required to use software components, as each possible combination must be examined and ranked based on how well they match the composers specifications.
Research into this problem continues by trying to answer fundamental questions, such as how to describe a component and its attributes \citep{treinen2009common,Xinjuan2007},
how to search for a compositions fits a set of requirements \citep{abate2011,Kwong2010,Treinen2009,DeAlmeida2004}, 
and how to rank a particular composition \citep{Chen2011,Aleti2009}.

A current perspective on component-based software engineering (CBSE) is that software component's main driving force is increasing the evolvability of a system \citep{Szyperski2000}.
However, the concept of a software component subindustry as originally described by \cite{McIlroy1969}, has never truly come to fruition \citep{Szyperski2002}.
A possibility for this is the unsure definitions of what a software components are \citep{Crnkovic2011}.
The current state of CBSE is fractured, where there are many different component frameworks (some presented in section \ref{background.models}),
each with different goals and attributes.
The hope for CBSE, as described by \cite{Crnkovic2011}, is that the technology and research will converge, and terms and concepts in the software component domain will become standardised.
This problem is later described in section \ref{background.components}, where the software component definition in this thesis this research will be discussed.

\subsubsection{Unix and GNU/Linux Modular Operating Systems}
The research into software components by \cite{McIlroy1969} coincided with his help in the development of the operating system Unix \citep{raymond2003art}. 
McIlroy had significant impact not only on the implementation of Unix, where many of his ideas like pipes where included,
but also on Unix's philosophy. 
The Unix philosophy is summarised by McIlroy as \citep{Salus1994}:
\begin{quote}
Write programs that do one thing and do it well. Write programs to work together. 
Write programs to handle text streams, because that is a universal interface.
\end{quote}

This philosophy lead to the first two rules (of fifteen) of Unix are \citep{raymond2003art}:
\begin{itemize}
  \item \textit{Rule of Modularity}: Write simple parts connected by clean interfaces.
  \item \textit{Rule of Composition}: Design programs to be connected to other programs.
\end{itemize} 

To eliminate the perceived problems of the proprietary Unix system, Richard Stallman, in 1983, created the GNU project \citep{stallman1985gnu} in order to create a free Unix-like operating system.
With a kernel developed by Linus Torvalds based on the MINIX \citep{tanenbaum1989minix} operating system, the GNU/Linux \citep{torvalds2002just} operating system was created.
GNU/Linux is seen as a return to the original philosophy of Unix by \cite{Gancarz2003}, where the creation of small modular programs that interact is central.
Aligned with this philosophy, a distribution of GNU/Linux called Debian \citep{Barth2005} was announced in 1993.
This release came with the Debian Manifesto \citep{murdock1994brief} 
that stated that Debian would be constructed from high quality components (or packages) which can be maintained by experts.

Initially, a significant amount of technical expertise was required for a user to be a composer of packages to create or change a Debian system.
This was because the selection problem (as discussed above) had to be resolved by the user themselves.
If a user wanted to build or change a Debian system, they would have to find, select, then install all the packages they wanted in the system.
The user would also have to make sure that all the constraints caused by the relationships between components were satisfied.

With the release of the application \textit{apt-get}, the necessary technical knowledge for the user to compose a Debian system was significantly reduced.
With this application, the desired change to a system could be stated, and finding, selecting and installing a solution was automated.
This is one of the first CDR implementations to enable the user to become the composer of a component system.

\section{Component Evolution vs. Component System Evolution}
\label{background.componentsystemevolution}
%%%Evolution of a component system is made up of two cycles, the component evolution and system evolution.
In this section, the differences between the evolution of components and the evolution of component systems are discussed.

\subsection{Component Evolution}
When evolving an individual component, the encapsulation from other components and their composition into a system must be considered.

Component encapsulation allows the simultaneous and separate development of components.
As a developer of a component may not have control over the development of related components, each component may have different evolution cycles.
Therefore the evolution of component must also consider evolution of components it relates to.
For example, if a component $a$ is changed, and component $b$ uses $a$'s functionality, this change may effect $b$.
As the developers of $a$ do not have control over the development of $b$, the changes must be discussed, and co-ordinated to ensure that they are not detrimental to either component.

Given a set of components, a combinatorial amount of systems could be created.
Ensuring the correctness of every possible composition could be an impractical task.
To ensure a component can be composed into a system, its relationships are described and these form constraints on a system.
It is assumed then, that if these constraints are satisfied the component will function correctly.
Component contracts \citep{Watkins1999} have been created as a way of checking to see if the system is correctly composed.
By formally defining contracts and integrating them into composition of components, it is hoped to verify component systems \citep{Dietrich2010}.

The atomic unit of the evolution of a component is the \textit{version}.
A version is marker assigned to a component to mark its evolved state.
To be used to reason about components, the component model defines a version model which states the syntax of a version.
This version model also describes the comparison of versions, which should create a total order over a set of components, where the highest value should be the most evolved state.
In general, the version does not convey any meaning \citep{Stuckenholz2005}, but attempts have been made to add semantics to versions \cite{Bauml2009}.
In most current component models, loosely enforced conventions are used to associate meanings to versions \cite{Barth2005,osgicore2007}.
A discussion on the versioning of components is given by \cite{Stuckenholz2005},
this describes the problems associated with the versioning and evolution of components.

A proper methodology for the development and evolution of software components it still being searched for \citep{Szyperski2002}.
Discussions on types of evolutionary changes with comparisons to other domains are given in \citep{Papazoglou2011},
and empirically explored by \cite{vasa2007patterns}.
There are many aspects of component evolution left unexplored, including how these evolutions effect the systems they are deployed in.

\subsection{Component System Evolution}
A component system is evolved by the composer by continually altering the set of components contained within the component system.
This is opposed to the evolution of individual component, where a developer alters an individual component.
This architectural level view of evolution does not concern itself with the internals of the constituent components, only the relationships between the components.

The core functions to evolve a component system are \textbf{install} and \textbf{remove}, these add or remove a component to or from the system.
The additional function of \textbf{upgrade} can also be defined to remove then install a higher version of the same component. 
This limited set of possible actions when changing a component system does not require much technical knowledge. 
To evolve a single component requires programming, software development knowledge, as well as some technical knowledge of the component model.
However, the evolution of a piece of software from a higher level only requires some knowledge of the component model and tools to change it.
For example, to upgrade the components in a Debian GNU/Linux component system, the command \texttt{apt-get upgrade} is all that is needed to be executed.
To extend the system to install a component \texttt{comp} the command \texttt{apt-get install comp} can be executed.
This simplicity gives the user the ability to be the composer of their own system.

A problem arises when trying to measure the evolved state of the system,  of a system as there is no way to assign a ``version'' to a component system.
A component system is a set of components, where each component can have various versions.
This may make a component system impossible to compare to another in a way in which versions allow component to be compared.
For instance, a system that has version 1 of component $a$ and version 2 of component $b$, is neither more or less evolved than a system with version 2 of $a$ and version 1 of $b$.
This can get even more complicated when considering some component models allow multiple versions of a single component installed, e.g. 
is a system with version 1 and 2 of $a$ installed less evolved that a system with only version 2 of $a$ installed?

This process of component system evolution is empirically studied by \cite{fortuna2011} 
who look at the first ten releases of the Debian and compare it to the evolution within biology.
Methods to evolve component systems are discussed in \citep{Ryan2005} and \citep{Luo2004},
and the mitigation of the negative effects caused by such evolution is discussed in the paper \citep{Stuckenholz2007}.

A formal definition of the component system evolution is given in chapter \ref{formal}.

\subsection{Component Dependency Resolution}
For a non-technical user to become the composer, many of the aspects of component composition must be automated.
In this research the function that automates these aspects is called  \textbf{Component Dependency Resolution} (CDR).

%%%The tedious nature of satisfying component dependencies during component system evolution can be automated into a function called Component Dependency Resolution.
CDR takes requests to change a system from the composer and constraints created by the component relationships and searches for systems that satisfy both.
For example, a user wants to install a new text editor component into their system, and the selected text editor has a dependency on a spell checker.
This means a spell checker component must be installed for the text editor to be functional.
As the spell checker can have its own relationships such as dependencies, the process of satisfying all constraints is a recursive problem. 
Therefore, CDR tries to find a valid system that satisfies all constraints.

For CDR to be usable, components must explicitly state their relationships.
If they do not, then CDR cannot analyse the constraints that are needed to be satisfied and a solution cannot be searched for.
This search is a selection problem, where CDR can select from many different possible systems that satisfy all constraints.
To select an acceptable solution, CDR must consider the user's goals for the system.
These goals are part of the user's strategy; the user's plan to evolve the system.

In other areas CDR, or applications that contain CDR, can be referred to by various names; provisioning system or dependency management in Eclipse domain \citep{leBerre2010},
or package manager in the GNU/Linux \citep{Barth2005} domain.
Many other component models use CDR to compose and evolve component systems.
CDR can be used at design time to determine the required dependencies to build and test a project (as in Apache Maven \citep{casey_better_2008}),
at run time to evolve or extend a component-based system (as in Eclipse P2 \citep{leBerre2010}),
or it can be used to build and restructure software product lines \citep{savolainen_analyzing_2007}.
CDR is also used in package management systems, like apt-get \citep{Barth2005},
for adding, removing and updating components in package based operating systems like Ubuntu\footnote{http://www.ubuntu.com accessed 6/3/2012}. 

An implementation of CDR is described in chapter \ref{implementation}

\subsection{Evolution Strategies}
%%%The decisions that lead to an evolution strategy
Lehman's laws of software evolution \citep{lehman1980,lehman1997} describe two opposing forces,
the need to change and the resistance to change.
When change occurs it is inevitable that it will introduce more complexity and the quality of the system will be reduced.
It is also inevitable that without change the user will become dissatisfied with their system.
A user's strategy to evolve their system must take into account both these forces.
As each user and each system have different requirements and objectives, each strategy can be different for any given user.

Further discussion on strategies and how they are defined is presented in chapter \ref{strategies}.

\section{What is a Software Component?}
\label{background.components}
As described above, software component ideals can be traced to \citep{McIlroy1969}.
The discussion of the definition and nature of such software components is described in this section.

%%%What is a component in a larger context
A component is a natural concept that describes a part or element of a larger system or process.
A broad characterisation of a component is ``components can be composed together''.
They can be physical, as in electrical or mechanical components, where their properties and how they are composed are documented to be used in construction.
Typically, components can be used in many different contexts, e.g. a resistor component, they can be used in electrical systems from space stations to cellphones.   
This natural concept of what a component is leads to problems when defining the concept of a software component.

%%%Discussion of articles where Szyperski and Meyers discuss give their versions of what a component is.
A discussion between two researchers in component software, Bertrand Meyer and Clemens Szyperski, highlight the difficultly of defining ``software component''. 
They describe across multiple articles \citep{Meyer1999,Szyperski2000a, Szyperski2000, Meyer2000} their definitions of what a software component is and the importance of its features.

%%%Szyperski's definition
Szyperski defines components in \citep{Szyperski2002} as having three characteristic properties:
\begin{enumerate}
  \item a unit of independent deployment
  \item a unit of third party composition
  \item has no externally observable state
\end{enumerate}

%%%Meyers Definition
Meyer's definition of software components is enumerated as:
\begin{enumerate}
  \item May be used by other software elements (clients)
  \item May be used by clients without the intervention of the components developers
  \item Includes a specification of all dependencies (hardware and software platform, versions, other components)
  \item Includes a precision specification of the functionality it offers
  \item Is usable on the sole basis of that specification
  \item Is composable with other components
  \item Can be integrated into a system quickly and smoothly
\end{enumerate}

%%%Using a component model
Others, like \citep{heineman2001component} have stated that components must conform to a component model: 
\begin{quotation}
A software component is a software element that conforms to a component model and can be independently deployed and composed without modification according to a composition standard.
\end{quotation}

Defining a component model as:
\begin{quotation}
A component model defines a set of standards for component implementation, naming, interoperability, customization, composition, evolution, and deployment.
\end{quotation}

%%%Other differences, architectural/implementation
Exactly what is, and what is not a software component is in dispute amongst the community,
and a definitive description of a software component is elusive \citep{vasa2007patterns}.
As such, many different components models have been developed, each targeting various domains with different functionality and technical aspects.
This diversity has inspired a classification approach from \citep{Crnkovic2011}, where components and component models are classified into a scheme.
This effort highlights the difficulty in creating an exact definition of a software component. 

%%%It is difficult to define a natural concept such as component with a set of fixed enumerable features.
These problems with exactly defining a software component may stem from the fact that ``component'' is a natural concept.
The problem, as observed from the area of formal concept analysis \citep{Ganter1999} by \citep{Szyperski2002}, is that it is impossible to

\begin{quotation} 
``enumerate a fixed agreeable set of features that is necessary and sufficient for a natural concept such as component.'' 
\end{quotation}

%%%Our egotistical view of components is defined with relation to Component Dependency Resolution.
However, a definition can be found, not by feature enumeration but through stating the intention for the concept and exploring the technically inevitable consequences \citep{Szyperski2002}. 
As the intention of this study is to investigate component system evolution, CDR, and evolution strategies
the definition of software component will be with respect to these concepts.

%%%This will not fit all views of what a component is (no definition will), but it eliminates unnecessary arguments so we can focus on CDR
This view of a software component's definition will not agree with every persons view of what a software component is, but (as described above) no precise view can.
By only looking at aspects that are necessary for this research, the arguments over unnecessary semantics can be eliminated. 

\subsection{Software Component}
The definition of a software component is given with respect to the evolution of a component system using component dependency resolution.
Both these areas have already been discussed in this chapter and will be used to define the concept of ``software component''.
This definition specifies the type of components and component models this research can be applied to.

In this research a software component is a unit of independent deployment, and third party composition,
and a component model must:
\begin{enumerate}
  \item require the explicit definition of relationships between components
  \item include mechanisms in which to programmatically compose or change a component system
\end{enumerate}

To automate the evolution of a system with CDR, as discussed above, the relationships between components must be explicitly defined and computer readable.
This automation also requires an interface for CDR to evolve the system.
This means that CDR should have the access to mechanisms to add, remove and replace components in a system.
As CDR would evolve a system automatically, these actions must be able to be enacted either through a function call, or well defined command interface.

This definition leaves undefined many aspects of a component model, as can be seen when compared to the classification from \citep{Crnkovic2011}.
Most aspects of a component model are ignored as they are superfluous to the core topic of this research.
This makes the definition in this research broadly applicable, while also making it focused.

This definition of software component will be examined through examples of different component models that conform to it.

\section{Component Models}
\label{background.models}
%%%Some core component models are described, and their dependencies and evolution are discussed
Given the definition of a software component for this research, some current component models are described and discussed.
These models come from industry (OSGi, Eclipse Plugins, Fractal, Maven), open source community (Debian, RPM) and academia (SOFA2).

%%%Meta data of component models is discussed
The typical mechanism in which components from these frameworks express their explicit relationships is through meta-data files.
These files contain what a component requires and provides, as well as other information like versions and provider details.
These meta-data files are then a core element in the description of a component model.

%%%How to change a system is discussed
A component model must also provide a mechanism to alter the system of components.
This is usually a low level interface, where the addition or removal of a component is an atomic step.
A CDR implementation can then use these interfaces to evolve a system.

%%%Exisiting CDR implementations are discussed
Some component models have a CDR implementation already available.
These implementations are described and the possible user strategies that they allow will be discussed. 

To compare these component models, the example of a text editor component that depends on a spell checker component is used.
It is hoped this simple situation will highlight the similarities and differences between the various component models.

\subsection{OSGi}
%%%What OSGi is? Bundle Service
OSGi is a mature component model from the OSGi Alliance.
It has implementation from organizations like the Eclipse Foundation with their Equinox framework \citep{mcaffer2010osgi}, 
and the Apache foundation with their framework Felix\footnote{http://felix.apache.org/ accessed 6/3/2012}.

OSGi components are refereed to as bundles, each contains a meta-data file describing the bundle's relationships, and a set of Java packages and classes as implementation.
A notable difference between OSGi and other component frameworks is that it separates the unit of deployment, the bundle, and the run-time unit, the service.
These services exist on a separate layer to the bundles, each service is created at run-time and is represented by a Java object.
This service layer can also have explicit relationship through frameworks like Spring Dynamic Modules\footnote{http://www.springsource.org/osgi accessed 6/3/2012}.
Under the definition of component in this research, this makes both the bundle layer and the service layer software component models.

\subsubsection{Bundle Layer}
The bundle layer of OSGi is implemented to allow dynamic addition and removal of bundles.
This can be done either through an implemented console, like the one in the Equinox framework, or through a programmatic interface.

An example of the meta-data which defines a text editor bundle that depends on a spell checker is described in figure \ref{osgibundle}.

\begin{figure}[htp]
\begin{center}
\begin{alltt}
Bundle-Name: TextEditor
Bundle-Vendor: Graham Jenson
Bundle-SymbolicName: nz.geek.textEditor
Bundle-Version: 0.0.1.alpha
Bundle-RequiredExecutionEnvironment: J2SE-1.4
Export-Package: nz.geek.textEditor;version="0.0.1.alpha"
Require-Bundle: nz.geek.fonts
Import-Package: nz.geek.spellchecker;version>"0.0.1"
\end{alltt}
  \caption[OSGi Meta-data]{Example of OSGi Meta-data}
  \label{osgibundle}
\end{center}
\end{figure}

%%%OSGI Meta data description
This meta-data shows the name, version and vendor of the component, as well as the exported packages (referring to Java packages) which are able to be used by other components.
The \texttt{Require-Bundle} key expresses the direct dependence on another bundle.
The \texttt{Import-Package} key expresses the dependence on a package provided by a bundle and not on the bundle itself.

\subsubsection{Service Layer}
This bundle meta-data only contains information necessary for the execution of a component.
However, for the component to be functional the service layer relationships must be considered.

This service layer is defined in the core OSGi specification \citep{osgicore2007}, however it does not describe declarative support with meta-data.
Therefore, a number of frameworks have emerged that define service relationships, e.g
Spring Dynamic Modules\footnote{http://www.springsource.org/osgi accessed 6/3/2012}.
OSGi's compendium specification \citep{osgicompendium2007} also defines service layer meta-data format called Declarative Services(DS).
An example of a DS component's description file is given in figure \ref{dsmetadata}.

\begin{figure}[htp]
\begin{center}
\begin{alltt}
<?xml version="1.0"?>
<component name="textEditor">
    <implementation class="nz.geek.textEditor.TextEditorImpl"/>
    <service>
        <provide interface="nz.geek.textEditor.TextEditor"/>
    </service>
    <reference name="spellChecker"
        interface="nz.geek.spellchecker.SpellChecker"
        bind="setSpellChecker"
        unbind="unsetSpellChecker"
        cardinality="0..1"
        policy="dynamic"/>
</component>
\end{alltt}
  \caption[OSGi Declarative Services]{Example of OSGi Declarative Services meta-data}
  \label{dsmetadata}
\end{center}
\end{figure}

This meta-data includes references to implementation elements like interfaces that are provided and required, and methods to interact with the services.
These relationships can have cardinalities, e.g. a text editor can use multiple spell checkers, and other programmatic features.
The \texttt{service} tag describes the services provided, and the \texttt{reference} tag expresses a dependence on another services.

%%%DS provides no versioning
One aspect of DS that is lacking, is the ability to explicitly version services.
The version of the bundle that is providing the service and the version of the package that is provided by the bundle, implicitly give a version to the service being provided.
However, for the evolution of a component model, some type of versioning is typically defined.

\subsubsection{Evolution of OSGi}
The programmatic evolution of an OSGi system is defined in the interfaces created by the OSGi alliance.\footnote{http://www.osgi.org/javadoc/r4v43/ accessed 6/3/2012}
The installation and removal of both the bundles and services from the OSGi system are defined as such:
 
\begin{itemize}
  \item To install a bundle:\\ \texttt{org.osgi.framework.BundleContext}\verb+#+\texttt{install}
  \item To uninstall a bundle:\\ \texttt{org.osgi.framework.Bundle}\verb+#+\texttt{uninstall}
  \item To register a service: \\ \texttt{org.osgi.framework.BundleContext}\verb+#+\texttt{registerService}
  \item To unregister a service: \\ \texttt{org.osgi.framework.ServiceRegistration}\verb+#+\texttt{unregister}
\end{itemize}

These methods can sometimes be implemented in a console, this allows a user to directly execute them to add or remove bundles.

\subsubsection{OSGi Bundle Repostiory}
%%%The OBR proposed by Peter Kriens and Richard Hall, is a meta data format for OSGi bundle resolution
A CDR implementation for OSGi was proposed by Peter Kriens and Richard S. Hall in RFC-0112 \citep{the_osgi_alliance_rfc-0112_2006}.
This is called the OSGi Bundle Repository and consists of collection of components described in an XML format and a function to reason about them.
An example of this document is shown in figure \ref{obrmetadata}.

\begin{figure}[htp]
\begin{center}
\begin{alltt}
<repository name='OBR REP' time='123'> 
 <resource version='0.0.1' name='nz.geek.textEditor' 
 uri='nz.geek.textEditor.0.0.1.jar'> 
  <require optional='false'  multiple='false'  name='package' 
    filter='(&amp;(package=nz.geek.spellChecker)(version&gt;=1.0.0))'> 
   Import package nz.geek.spellChecker ;version=1.0.0 
  </require> 
 </resource> 

 <resource version='1.0.0' name='nz.geek.spellChecker' 
 uri='nz.geek.spellChecker-1.0.0.jar'> 
  <capability name='package'> 
    <p v='nz.geek.spellChecker' n='package'/> 
    <p v='1.0.0' t='version' n='version'/> 
  </capability> 
 </resource> 

</repository>
\end{alltt}
  \caption[OSGi Bundle Repository markup example]{Example of OSGi Bundle Repository meta-data}
  \label{obrmetadata}
\end{center}
\end{figure}

This meta-data was designed so that it can merge the bundle and service meta-data as described in figures \ref{osgibundle} and \ref{dsmetadata}.
It also ignore many of the implementation aspects of OSGi bundles, as they are superfluous to the CDR probelm.
OBR is then an abstract complete model that represented the necessary elements from both OSGi bundle and service layers in order to evolve a system with CDR. 

OBR has been seen as a solution to to simplify deployment of OSGi applications \citep{Jung2007}, 
distribution and deployment to embedded ubiquitous systems \citep{Jung2006}, 
smart home applications \citep{Gouin-Vallerand2007} and 
dynamic distribution of drivers \citep{Kriens2008}. 

The most mature implementation of OBR client is offered by the Apache foundation, which is bundled with their core OSGi framework Apache Felix. 
This can be used with any of the large public or private OBR collections of bundles.
An example of one such public repository is the Paremus repository\footnote{http://www.osgi.org/Repository/ accessed 6/3/2012} which contains (as of December 2011) over 2000 bundles.

The specification of OBR does not define a specific strategy, or a mechanism to define strategies to be used during evolution.
The implementation specific strategy that is used by the Apache OBR\footnote{http://felix.apache.org/site/apache-felix-osgi-bundle-repository.html accessed 6/12/2011} 
is described on its help page as:

\begin{quotation}
OBR might have to install new bundles during an update to satisfy either new dependencies or updated dependencies that can no longer be satisfied by existing local bundles. 
In response to this type of scenario, the OBR deployment algorithm tries to favor updating existing bundles, if possible, as opposed to installing new bundles to satisfy dependencies.
\end{quotation}

This shows that when updating a system of bundles, newer bundles may be installed although currently installed bundles are preferred.

\subsection{Eclipse Plugins}
%%%What is Eclipse
Eclipse is a widely used IDE and an extensible plugin platform for creating Java applications.
It is built on top of the OSGi framework, but ignores OSGi's service layer and re-implements it with its own Eclipse plugin runtime.
Therefore, the distrusted components are OSGi bundles and the run time elements are plugin services.

These plugins are defined using extensions and extension points, where extensions provide a service for an extension point.
An example of the meta-data describing a plugin is given in figure \ref{eclipseplugin}. 

\begin{figure}[htp]
\begin{center}
\begin{alltt}
<?xml version="1.0"?>
<plugin
    name="Text Editor"
    id="nz.geek.textEditor"
    version="0.0.1.alpha"
    provider-name="Graham Jenson">
    
    <requires>
        <import plugin="nz.geek.fonts"/>
    </requires>

    <runtime>
        <library name="texteditor.jar"/>
    </runtime>
    
    <extension-point id="nz.geek.spellchecker" 
        name="Spell Checker" 
        schema="spellchecker.exsd"/>
</plugin>
\end{alltt}
  \caption[Eclipse Plugin meta-data]{Example of an Eclipse Plugin plugin.xml meta-data file}
  \label{eclipseplugin}
\end{center}
\end{figure}

This plugin defines the name, version and provider of the plugin, and using the tags \texttt{requires} defines the requirements of this plugin to function.
The \texttt{extension-point} tag defines not only what the plugin provides, but also the required information in order to provide it, described in a schema.
This is a special feature of the Eclipse plugin framework, as other component models generally do not specify these parameters on this type of relationship. 

The schema of an extension point, as shown in figure \ref{eclipseextensionschema}, describes the elements to use an extension.

\begin{figure}[htp]
\begin{center}
\begin{alltt}
<schema xmlns="http://www.w3.org/2001/XMLSchema">
 <element name="spellchecker">
  <complexType>
   <attribute name="spellchecker" type="string" use="required">
    <annotation>
     <appinfo>
      <meta.attribute kind="java" basedOn=":nz.geek.ISpellChecker"/>
     </appinfo>
    </annotation>
   </attribute>
  </complexType>
 </element>
</schema>
\end{alltt}
  \caption[Eclipse Plugin extension point schema]{Example of an Eclipse Plugin extension point schema file}
  \label{eclipseextensionschema}
\end{center}
\end{figure}

In this description, this extension point requires a string that describes a java class that implements \texttt{nz.geek.ISpellChecker}.
This is the only requirement for this extension point, but other parameters of types, e.g. Boolean, integer, can be defined.

\subsubsection{Eclipse Evolution}
The programmatic evolution of an Eclipse system is through the use of the previously OSGi methods and with interaction with the plugin registry.

The extensions and extension points, for an Eclipse system, can be altered in the plugin registry.
This registry contains all references to the extensions and extension points in a system.
The methods used to add and remove these are:

\begin{itemize}
  \item add extensions and extension points described in an XML file: \\ \texttt{org.eclipse.core.runtime.IExtensionRegistry}\verb+#+\texttt{addContribution}
  \item remove extension: \\ \texttt{org.eclipse.core.runtime.IExtensionRegistry}\verb+#+\texttt{removeExtension}
  \item remove extension point:\\ \texttt{org.eclipse.core.runtime.IExtensionRegistry}\verb+#+\texttt{removeExtensionPoint}
\end{itemize}

The Eclipse plugin system is a component system in this study, 
as it contains explicit relationships, and the programmatic evolution requirements necessary for CDR to be used.

\subsubsection{Eclipse P2}
%%%The Eclipse P2 Solver, by Le Berre and Rapicault
Eclipse P2 \citep{le_berre_dependency_2009}, \citep{leBerre2010} is the provisioning system for the Eclipse IDE platform.
It provides the interaction to alter an Eclipse based component system. 
It is implemented using a Boolean Satisfiability solver, called SAT4J \citep{le2010sat4j}.
This method of implementation has become a common way of implementing CDR, explored in \citep{Berre2008}, and further discussed in chapter \ref{implementation}.

Eclipse P2 is mainly accessed through the Eclipse user interface, where the user can select to update the entire system, or install a component.

Some of the criteria used by Eclipse P2 to evolve the system are described in \citep{le_berre_dependency_2009} and \citep{leBerre2010}:
\begin{enumerate}
  \item minimise the amount of components installed that have no dependency to them 
  \item minimise the removal of already installed components
  \item minimise the age of the installed components
  \item minimise changing an installed component if they are unrelated to the request being made
\end{enumerate} 

These criteria represent the strategy that it employed to choose a solution when multiple are available. 

\subsection{Fractal}
Fractal \citep{Quma2006} is a component model developed by France Telecom R\&D and INRA.
It is a specification that is designed to be programming language independent, unlike OSGi or Eclipse that both depend on Java specific elements.
The most notable aspect, and the reason for its name, is that a components can be composed together to make a new component.
This recursive nature of composition, actually means the system of components can itself be a component.

An example of meta-data to define a Fractal component is given in figure \ref{fractalmetadata}.

\begin{figure}[htp]
\begin{center}
\begin{alltt}
<?xml version="1.0" encoding="ISO-8859-1" ?>
<!DOCTYPE definition PUBLIC 
    "-//objectweb.org//DTD Fractal ADL 2.0//EN" 
    "classpath://org/objectweb/fractal/adl/xml/basic.dtd">

<definition name="textEditorComponent">
  <interface name="textEditor" role = "server" 
   signature = "nz.geek.textEditor"/>
  <interface name="spellChecker" role = "client" 
   signature = "nz.geek.spellChecker"/>
  <content class="nz.geek.textEditorImpl"/>
</definition>
\end{alltt}
  \caption[Fractal ADL Example]{Example of a simple Fractal ADL file}
  \label{fractalmetadata}
\end{center}
\end{figure}

This meta-data describes the relationship between components as a metaphor between a client and a server.
The interface tag with the role attribute assigned to ``server'', defines the provided functionality of the component.
The interface tag with the role attribute assigned to ``client'' then defines the required functionality of the component.  

An example of the definition of a recursive component can be seen in figure \ref{fractalrecursve}.

\begin{figure}[htp]
\begin{center}
\begin{alltt}
<?xml version="1.0" encoding="ISO-8859-1" ?>
<!DOCTYPE definition PUBLIC 
    "-//objectweb.org//DTD Fractal ADL 2.0//EN" 
    "classpath://org/objectweb/fractal/adl/xml/basic.dtd">

<definition name="textEditorWithSpellCheckerComponent">
  <interface name="textEditorWithSpellChecker" role="server" 
   signature="nz.geek.textEditorWSC"/>
  <component name="textEditorComponent">
    <interface name="textEditor" role="server" 
     signature="nz.geek.textEditor"/>
    <interface name="spellChecker" role="client" 
     signature="nz.geek.spellChecker"/>
    <content class="nz.geek.textEditorImpl"/>
  </component>
  <component name="spellCheckerComponent">
    <interface name="spellChecker" role="server" 
     signature="nz.geek.spellChecker"/>
    <content class="nz.geek.spellChecker.SpellChecker"/>
  </component>
  <binding client="this.textEditorWithSpellChecker" 
   server="textEditorComponent.textEditor"/>
  <binding client="textEditorComponent.spellChecker" 
   server="spellCheckerComponent.spellChecker"/>
</definition>
\end{alltt}
  \caption[Fractal ADL Example]{Example of a simple Fractal ADL file}
  \label{fractalrecursve}
\end{center}
\end{figure}

This description of a recursive component structure defines both the text editor and spell checker components, as well as their relationships.
The tag \texttt{definition} is then used to define a composition of components that provides a service of \texttt{textEditorWithSpellChecker}.
To provide this the spell checker must be bound (described in the tag \texttt{binding}) to the text editor,
and the text editors provided interface bound to the output interface. 

\subsubsection{Fractal Evolution}
The programmatic evolution of a Fractal component system can differ between implementations.
To simplify this, only the Java implementation is described. 
In this implementation the methods to edit the configuration are\footnote{http://fractal.ow2.org/current/doc/javadoc/fractal/ accessed 6/3/2012}:
\begin{itemize}
  \item to create a component: \\ \texttt{org.objectweb.fractal.api.factory.GenericFactory}\verb+#+\texttt{newFcInstance}
  \item to add a component to a composite: \\ \texttt{org.objectweb.fractal.api.control.ContentController}\verb+#+\texttt{addFcSubComponent}
  \item to remove a component from a composite: \\ \texttt{org.objectweb.fractal.api.control.ContentController}\verb+#+\texttt{removeFcSubComponent}
\end{itemize}

There are two points that may make the evolution of Fractal component model particularly difficult,
the recursive nature of Fractal, and the lack of required version information about components and composites.

%%%Fractal requires not only the selection of components but the configuration
Given the recursive nature of the components described, any system may provide the same functionality with the same components in a combinatorial number of configurations.
Therefore, not only would a CDR algorithm require to select a set of components but also group those component into composites.
The simplest solution to this problem is to ignore the recursive nature of this component model,
and use a CDR algorithm to create exactly one composite with all components in it.
More difficult solutions could be created by reusing user composites,
or analyzing the graph structure to extract relationship by using algorithms like the one presented in \citep{dietrich2008cluster}.

Another aspect of Fractal that will make the automatic evolution of systems difficult will be the lack of versioning information.
Like DS, this component model does not explicitly require the version of the component and the interfaces they provide.
As versions are typically the atomic unit of evolution, the process of selecting a newer version will be made more difficult if not impossible. 

\subsection{Maven}
Maven is a build automation tool, designed to be programming language agnostic, but primarily used with Java.
The core component aspect of Maven is that it can dynamically select and download files from a repository to be used to build a system.
Although Maven is built using a plugin architecture, the focus in this section is on the use of Maven as a component model and not Maven itself.

A system is defined using a Project Object Model (POM) file, an example given in \ref{mavenmetadata}. 

\begin{figure}[htp]
\begin{center}
\begin{alltt}
<project xmlns="http://maven.apache.org/POM/4.0.0"
  xmlns:xsi="http://www.w3.org/2001/XMLSchema-instance"
  xsi:schemaLocation="http://maven.apache.org/POM/4.0.0
                      http://maven.apache.org/xsd/maven-4.0.0.xsd">

  <groupId>nz.geek</groupId>
  <artifactId>textEditor</artifactId>
  <version>0.0.0.alpha</version>

  <dependencies>
    <dependency>
      <groupId>nz.geek</groupId>
      <artifactId>spellChecker</artifactId>
      <version>[0.0.1,1.0.0)</version>
      <type>jar</type>
     </dependency>
  </dependencies>
</project>
\end{alltt}
  \caption[Maven POM file]{Example of a Maven POM file}
  \label{mavenmetadata}
\end{center}
\end{figure}

This model defines the components namespace using the \texttt{groupId} tag, the name using the tag \texttt{artifactId}, and the version.
It also defines the dependencies on other projects through the \texttt{dependency} tag,
where it states that it depends on the artifact \texttt{spellChecker} from versions 0.0.1 to (but excluding) version 1.0.0.

\subsubsection{Maven Evolution}
The use of Maven primarily during development does not exclude it from the definition of being a component model in this study.
Maven both explicitly declares relationships and provides an interface to evolve its component systems. 
This unique characteristic of Maven does mean that evolving a component system is done in two stages where first it is altered, then it is recompiled and replaced.
The evolution of a Maven based systems are therefore done simply by selecting the projects to build, and using Maven to compile them into a system. 

Another core difference from the other presented component models is that Maven POM objects do not declare what they provide.
For example, an OSGi bundle can declare provided packages and a Fractal component declares provided interfaces, where a Maven project can only provide itself.
This will simplify a CDR algorithm by not having to resolve another layer of dependencies, 
though this means it goes against component principles like dependency on interface or specification \citep{Szyperski2002}.

\subsection{Debian Packages}
Debian is a GNU/Linux based operating system provided by the Debian Project \citep{Barth2005}.
This operating system most prominent feature is that it is composed of packages, a unit of deployment with explicit relationships.

Each package is defined using a control file, an examples presented in figures \ref{debianmetadatate} and \ref{debianmetadatasc}.

\begin{figure}[htp]
\begin{center}
\begin{alltt}
Package: textEditorPackage
Version: 0.0.1.alpha
Depends: spellChecker
Conflicts: otherTextEditorPackage
\end{alltt}
  \caption[Debian Control file for Text Editor]{Example of a Debian Control File for Text Editor}
  \label{debianmetadatate}
\end{center}
\end{figure}

\begin{figure}[htp]
\begin{center}
\begin{alltt}
Package: spellCheckerPackage
Version: 1.0.0
Provides: spellChecker
\end{alltt}
  \caption[Debian Control files for Spell Checker]{Example of a Debian Control File for Spell Checker}
  \label{debianmetadatasc}
\end{center}
\end{figure}

These control files describe two packages, a text editor and spellChecker, they are organised into key/value pairs separated by the ``:'' character.
The text editor states using the \texttt{Depends} key that is requires a spellChecker to be installed, 
and through the \texttt{Conflicts} key states that it cannot be installed with another text editor.
The spell checker package defines that it provides a spell checker using the \texttt{Provides} key.
This spell checker is of a type called a virtual package, which has some specific semantics, discussed later in chapter \ref{ubunutsimulation}.

The semantics of these control files is further described in the chapter \ref{ubunutsimulation}.

One aspect that differentiates the Debian package model from the many other models, is that it does not specify the rules of composition between components.
That is, there is no formal specification of the semantics of the communication and relationships between packages.
For instance, OSGi bundles exist in a very constrained environment that manages their interactions, depending on another bundle has run-time implications.
For Debian, the dependence has no prescribed run-time effects, only whether the package is functional or not.
The way in which a package, uses or communicates with a depended upon package is then defined between the packages and not in the component model.

\subsubsection{Debian Evolution}
The application ``dpkg'' is the command line tool use to modify a Debian package system.

The commands to modify a Debian package system are:
\begin{itemize}
  \item to add a package: dpkg --install <package file>
  \item to remove a package: dpkg --remove <package>
\end{itemize}

These commands define the atomic actions to change a Debian system.
They can be executed from other applications through the command line, this makes them programming language independent and easily accessible. 

\subsubsection{Apt-get}
The application dpkg can only remove and add a single component at a time, it does not resolve the dependencies of the component.
There are many applications that can resolve these dependencies on a Debian system, the default and most popular solution is through the ``apt-get'' application.
This tool is built on top of dpkg, to provide an interface to simplify and extend its functionality.
The apt-get application is a command line tool, which provides the necessary tools to efficiently allow the user to evolve their system.

Apt-get follows some basic criteria when altering the system, for example when upgrading:
\begin{itemize}
  \item under no circumstances will apt-get remove an already installed package
  \item under no circumstances will apt-get retrieve a package that is not installed
\end{itemize}

Another application that extends APT further is Aptitude \citep{Burrows2005}.
This application adds a basic user interface to APT and also increases functionality in aspects like locating and removing redundant packages.
Aptitude allows some of its criteria to be user defined, for example when the flag \verb+--safe-resolver+ is used:
\begin{itemize}
  \item it attempts to preserve as many of your choices as possible.
  \item it will never remove a package
  \item it will never install a version of a package other than the package's default candidate version.
\end{itemize}

\subsection{SOFA 2.0}
Software Appliance (SOFA) is a component model developed at Charles University in Prague.
A component in the SOFA framework is defined by it's frame, which contains the meta-data of what the component requires and provides.
Like Fractal, SOFA is a recursive component model, allowing for a composite of components to be treated as a single component.

An example definition of a frame is given in figure \ref{SOFAmetadata}.

\begin{figure}[htp] 
\begin{center}
\begin{alltt}
<?xml version="1.0"?>
<frame name="nz.geek.textEditor">
  <requires name="spellChecker" itf-type="sofatype://nz.geek.spellChecker"/>
</frame>

<?xml version="1.0"?>
<frame name="nz.geek.spellChecker">
  <provides name="spellChecker" itf-type="sofatype://nz.geek.spellChecker"/>
</frame>
\end{alltt}
  \caption[SOFA ADL files]{Example of a SOFA ADL Files}
  \label{SOFAmetadata}
\end{center}
\end{figure}

This meta-data describes two components, a text editor and a spell checker, using the tag \texttt{frame}.
The sub elements \texttt{requires} and \texttt{provides} describes their relationship through an externally defined interface with the name-space \texttt{sofatype://nz.geek.spellChecker}.
This interface is defined using an Architecture Definition Language (ADL) and added to a repository to be re-used. 


\subsubsection{SOFA Evolution}
SOFA 2.0 is an extension of the SOFA framework, with new services including dynamic reconfiguration for dynamic evolution of an architecture at run-time. 
Unlike other component systems,
it only allows for controlled evolution; 
Many component models allow the change of the component system on a fine grained level of adding or removing a component.
However this model only allows more granular changes to be made to the system, these changes must conform to specified evolution patterns.
This restriction is meant to increase the manageability and predictability of a component systems evolution. 

Three evolution patterns are predefined: factory pattern, removal pattern, and service access pattern.
As its name suggests, in factory pattern a designated component serves as a component factory. 
The removal pattern serves for destroying of a component previously dynamically created.
The service access pattern allows for the access to external services.
These patterns are further discussed in \citep{Hnetynka2006}.


\subsection{Common Upgradeability Description Format}
%%%CUDF model created for compeition
Many component models share similar properties:
\begin{itemize}
  \item a component name
  \item a component version
  \item description of what the component provides
  \item description of what the component requires
\end{itemize} 

By creating an abstraction of these models, that only contains the information that is necessary for CDR, different CDR implementations can be compared.

The Common Upgradeability Description Format (CUDF) model is defined as an abstract representation of the problem found when evolving a package based system, e.g. Debian.
It was created by the Mancoosi project to foster interest and competition in creating solvers for upgrading problems faced by Free and Open Source Software (FOSS).
This specification contains the necessary detail to express the core aspects of evolution, and an extensible syntax to allow for component model specific detail. 
CUDF allows the description of many CDR problems in a single format, and the use of various CDR implementations to solve them. 

%%%No concrete components, but fits our definition
There is no concrete component representation, as it was not designed to be an executable component model.
This means that there is no detail of how to execute the component system, or other implementation focused details that are necessary for the other presented component models.
In this respect it is similar to the OBR component model, as it has been created to explicitly represent the CDR problem to be solved.

In figure \ref{CUDFmetadata}, an example of a CUDF problem is presented.

\begin{figure}[htp] 
\begin{center}
\begin{alltt}
package: textEditor
version: 1
depends: spellCheckerService

package: spellChecker
version: 2
provides: spellCheckerService

request:
install:textEditor

\end{alltt}
  \caption[CUDF meta-data file]{Example of a CUDF problem file}
  \label{CUDFmetadata}
\end{center}
\end{figure}

The CUDF file is defined using key/value pairs, where each component is defined using the two tags \texttt{package} and \texttt{version}.
This file does not only represent the components but also the evolutionary change with the tag \texttt{request}.
Further definition of the syntax and semantics of CUDF is in chapter \ref{formal}, as it is used extensively throughout this study.

\subsubsection{Mancoosi MPM}
%%%The mancoosi Modular Package Manager uses CUDF, currently going through the process in the Debian component repository
The CUDF component model is defined especially to compare CDR implementations, therefore many various implementations exist.
Some of these will be discussed in the latter chapter \ref{implementation}.
However, here only the default implementation, Modular Package Manager \citep{abate2011} (MPM), is discussed.
This implementation was created by the Mancoosi organisation to explore the possibility of using CUDF to evolve real systems.

In the study by \cite{abate2011}, this package manager was compared against other component solvers, such as apt and aptitude, and shown to be an improvement.
This CDR implementation is the first to allow the user to completely specify the criteria by which the solver selects a system.
In the study the solver first minimised the amount of packages removed from the system, and then minimised the amount of packages changed in the system.
Further discussion of these criteria and how they are combined is discussed in chapter \ref{strategies}. 

\subsection{Comparison}
Each of these component models provide an explicit description of their components relationships, and a mechanism to alter a component systems composition.
These relationships fall into three groups:
\begin{enumerate}
  \item dependencies, where one component requires what another component or what it provides.
  \item conflicts, where one component conflicts with another component or what it provides. 
\end{enumerate} 

Typically these relationships are between pre-defined types in the component model, e.g. a service or a package.
OBR, Maven, and SOFA allowed the definition of the dependencies type, typically as a string.
This allowed these component models to have multiple types of dependencies and express them in their meta-data. 

The representation of the relationships between components in meta data is defined in many different files, though the only two formats are either XML of a set of key/value pairs.

The comparison between the relationships in these component models are summarised in the tables  \ref{background.comparisonrel} and \ref{background.comparisonmeta}.  

\begin{table}
\begin{tabular}{|l || c | c | c | }
\hline \textbf{Component Model}& \textbf{Component}& \textbf{Provides}	& \textbf{Provide/Require}\\ \hline

OSGi					& Bundle			& Package 					& Export/Require Import \\
Declarative Services	& Component			& Service 					& Provide/Reference\\
OBR						& Resource			& \textit{typed}			& Capability/Require\\
Eclipse					& Plug-in			& Extension-Point			& Extension/Requires\\
Fractal					& Component			& Interface 				& Server/Client\\
Maven					& Artifact			& \textit{typed} 			& NA/Dependency\\
Debian					& Package			& Virtual-Package 			& Provide/Depends\\
SOFA 2.0				& Frame				& \textit{typed} 			& Provide/Require\\
CUDF					& Package			& Package 					& Provides/Depends\\\hline
\end{tabular}
\caption{Summary of presented component models' relationships. Note: \textit{typed} refers the ability to type a requirement.}
\label{background.comparisonrel}
\end{table}


\begin{table}
\begin{tabular}{|l || c | c | c | }
\hline \textbf{Component Model}& \textbf{Relationship Description File} & \textbf{Format}	\\\hline

OSGi					& Meta-Data				& Key/Value pairs 					\\
Declarative Services	& Component Description	& XML 								\\
OBR						& Repository			& XML								\\
Eclipse					& plugin.xml			& XML								\\
Fractal					& ADL					& XML 								\\
Maven					& POM					& XML 									\\
Debian					& control				& Key/Value pairs							\\
SOFA 2.0				& ADL					& XML 							\\
CUDF					& CUDF					& Key/Value pairs									\\\hline
\end{tabular}
\caption{Summary of presented component models' relationship description formats}
\label{background.comparisonmeta}
\end{table}

These component models also provide mechanisms to alter a component systems configuration with the exception of OBR, Maven, and CUDF.
In OBR and CUDF, the component systems are represented by the meta-data files themselves, therefore altering the component system is a matter of altering the meta-data. 
Maven component systems are also represented by the meta-data, though once the meta-data is altered to change the system the Maven application must be executed.

With the exception of SOFA 2.0, all the mechanisms to alter a component system in the remaining component models have the actions of either adding, removing and/or replacing a component.
SOFA only allows altering the system by pre-defined evolution patterns.

Most of these models have some form of CDR implementation to help evolve their component systems.
Fractal and SOFA are the only systems to not include a CDR implementation, this may be due to the difficulty in evolving hierarchical component models.
A comparison between these component model properties is presented in table \ref{background.comparisonprop}.

\begin{table}
\begin{tabular}{|l || c | c | c | }
\hline \textbf{Component Model}& \textbf{Hierarchical}& \textbf{Evolution Mechanism}	& \textbf{CDR}\\\hline

OSGi					& No			& Method/Command Line 					& OBR\\
Declarative Services	& No			& Method 								& OBR\\
OBR						& No			& NA									& OBR \\
Eclipse					& No			& Method								& Eclipse P2\\
Fractal					& Yes			& Method 								& None\\
Maven					& No			& NA 									& Maven\\
Debian					& No			& Command Line 							& apt-get\\
SOFA 2.0				& Yes			& Evolution Patterns 					& None\\
CUDF					& No			& NA 									& MPM\\\hline
\end{tabular}
\caption{Summary of presented component models' properties.}
\label{background.comparisonprop}
\end{table}

\section{Summary}
In this chapter, first the evolution of software was discussed, and the nature and laws of this process described.
Component evolution and component system evolution are then described and compared as separate software evolution processes.
Automating the evolution of component systems is then defined through the function component dependency resolution.
The strategies that a user employs to evolve a system is then discussed, and the forces that are considered in the strategy.
Software components are defined with respect to the objectives of this research as having explicit relationships and an interface to evolve a component system.
Various component models that conform to this definition are then discussed and their properties explored.

In the next chapter concepts such as component system evolution are formalised and further discussed.  

