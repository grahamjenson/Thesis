%%This is a very basic article template.
%%There is just one section and two subsections.
\documentclass{report}

\usepackage[T1]{fontenc}
\usepackage{ae,aecompl}

\usepackage{amsmath}
\usepackage{amssymb}
\usepackage{graphicx}
\usepackage{listings} 
\usepackage{alltt}

\newtheorem{defs}{Definition}
\newtheorem{prof}{Proof}
\newtheorem{prop}{Proposition}

\title{Component Dependency Resolution}

\begin{document}

%%%This document is an outline of the thesis.\\
%%%All words in italic are summarries (generally of a paragraph) and will not be in the final thesis.\\ 
%%%All non-italic words (generally introductory paragraphs) will be included in the final thesis.\\
%%%\hline

{}\tableofcontents

\chapter{Introduction}


\begin{quotation}
\textbf{Man}: When you take apart a Lego house and mix the pieces into the bin, where does the house go?\\
\textbf{Woman}: It's in the bin.\\
\textbf{Man}: No, those are just pieces. They could become spaceships or trains. 
The house was just an arrangement. The arrangement doesn't stay with the pieces and it doesn't go anywhere else. It's just gone\\
--XKCD Lego
\end{quotation}

%%%The start of this thesis leads the reader to the idea that when building a complex system, breaking the probelm into parts is a natural and beneficial instinct\\
{}When confronted with building a complex system, our first instinct in an attempt to construct the system is often to break it into smaller, simpler components. 
{}These components are defined by their contribution to the overall system and their relationships to other components.
{}Through this ``divide and conquer'' method, the design and construction of the complex system becomes a manageable task.

%%%We use the metaphor of building a car to explain the benefits of breaking systems into components
When building a complex system like a car, you would not attempt to build the car as a whole, but break it into parts for design and construction.
A car requires a body, electrical system, interior, suspension and steering, engine \ldots. 
The engine can be further broken into other parts like cooling, oil system, exhaust and intake systems, fuel, and so on.
Each of these parts have relationships that lead to complex dependencies, for instance the carburettor blends the air from the intake with petrol from the fuel system.

%%%The benefits of breaking apart systems are gained during design, implementation, maintenance and evolution. 
The benefits of breaking a system apart are gained through the entire product process; design, implementation, deployment, maintenance and evolution.

%%%During Design; planning (better plan the build), understanding the overall structure ("big picture"), risk analysis

%%%During implementation; specialization of people to tasks, concurent development, testing each part separtly, risk mitigation (if an individual component fails, the entire system is salvageable)

%%%During maintenance with well defined interactions a part can be replace with known side effects,
%%%and a problem can be detected and solved easier 
The apocryphal story of George Washington's axe which has three times had its handle replaces and twice its head replaced deomstrates the power of maintence on a system of parts.

%%%During Evolution, replacing a component with a superior part can be accomplished as long as it fits the required specifications,
%%%extra parts can be added to improve and add functionality along with specifications

\section{Software Components}
{}The use of components in software systems has additional benefits over their use in physical systems.
{}Software components can be reused in different contexts, replaced ``on the fly" without interruption to the system, validated for correctness before being used, 
{}and automatically composed into a system through resolving their defined dependencies.

%%%A software component can provide its services to many other components all for different purposes.
In a physical car there are pumps which accomplish many different tasks, pumping oil, air, water.
In a software system such replication is unnecessary as a single component could provide a service that pumps, 
and other components can use this service for many different purposes.
A software component, not bound by physical limitations, can be in many places at once, doing many similar jobs simultaneously.
A component providing a http server can be used for web site hosting, communication, internal and external web-sites, REST and SOAP communication.
This is done all by the same component, as the abstraction of the http service is broad enough to accomplish many tasks.

%%%Being able to change components while running the system is difficult but possible in software systems.
Replacing a tire or repairing a windshield while continuing to drive the car is at the very least difficult, if not impossible.
Inside a software system, where changes in state are accomplished so quickly that they are unperceivable, the replacement of parts of a system are easily accomplished while the system is running.

%%%Given the software component relationships are well defined, with contracts and requirements a system or component can be validated to be correct before use
If an enthusiast adds a turbo to a car, it may not work correctly in that context, it may create too much pressure and blow the engine.
A component can have complex mechanisms like contracts or strictly defined specifications to automatically detect these incompatibilities before they cause problems.

%%%Given the relationships in a software component system are strictly defined, we can use these to then find valid combinations with user requirements
Imagine being able to define the car you want by describing the functionality you require, and then a custom car, one that may exist no where else in the world,
is automatically built for you. 
While then driving this car your requirements change, you want to drive off road, then the car changes parts to adapt to your new requirements, new tires, engine, gearbox\ldots
This is the power that dependency resolution offers component systems, automatic resolution of a users requirements.

\section{Research Goals}
{}The aspect of the vast research areas of software components we will look at throughout this thesis is the final point made in the previous section, 
{}the automatic composition of software components through resolving component dependencies.
{}Specifically we will try to answer the question:\\
{}\begin{quote}
{}What are the \textbf{consequences to a component system} while \textbf{optimising for different criteria} during evolution when using \textbf{Component Dependency Resolution (CDR)}?
{}\end{quote}

{}This question leads to three lines of inquiry; first component dependency resolution, the optimisation with different criteria, and the consequences of CDR optimisation on component systems.

%%%The questions we aim to answer related to CDR:\\
%%%What is necessary of a component model for CDR?; Background chapter \ref{background}\\
%%%What is CDR and why use it?; CDR chapter \ref{cdr}\\
%%%How do we formallt define CDR?; CDR chapter \ref{cdr}\\
%%%How do we implement CDR?; Implementation chapter \ref{implementation}

%%%The questions we aim to answer related to the optimisation of CDR with different criteria are:\\
%%%How do we represent optimisation in CDR?; CDR chapter \ref{cdr}\\
%%%How do we optimise for criteria?; Implementation chapter \ref{implementation}\\
%%%What criteria are important?; Criteria chapter \ref{criteria}

%%%The questions we aim to answer related to the consequences of such optimisation:\\
%%%What consequences are we looking for?; Criteria chapter \ref{criteria}\\
%%%How similar are component systems?; Investigations chapter \ref{comparison}\\
%%%Are they similar enough to generalise our results?; Investigations chapter \ref{comparison}\\
%%%What consequences are important to measure?; Simulation chapter \ref{simulation}\\
%%%How can we measure the consequences of criteria?; Simulation chapter \ref{simulation}

\section{Thesis Overview}
%%%A list and breif explination of each of the chapters and how they relate to our Research Goals 

\chapter{Background}
\label{background}
{}Component-based software engineering is considered by many to be the future for development of software systems \cite{Szyperski2002},
{}where instead of developing a monolithic system, 
{}many small reusable units of execution are developed and composed into a single system.
{}The specification of how these components are defined and used to create a system is given in the \textbf{component model}.
{}This model can also support the evolution of the component systems by defining mechanisms to add, remove and replace components in that system.

%%%Manually evolving a system can be tedious, so we use component dependency resolution
Evolving a system through individually adding and removing components can be tedious task 
as components can have dependencies on other components which also need to be satisfied.
Therefore, after the user has identified the wanted components, 
those components' dependencies must be satisfied with other components, which themselves can have dependencies.
Satisfying all these dependencies manually is a recursive, time-consuming and error-prone task to accomplish.
However, as most modern component models require components to explicitly state their dependencies; 
this enables component dependency resolution to build systems where all dependencies are satisfied.

%%%A small example of such a problem.

In this chapter we go over the what a component is, and the difficulty in defining it.
We then discuss what a component model is, giving and comparing current examples of them.
Then discuss the evolution of individual components and component-based systems.
We then discuss component dependency resolution, giving a broad view of its goals and use cases for component systems.
Finally we give and compare different examples of component dependency resolvers for design-time and run-time, used in academia 
 
 
\section{What is a Component?}
%%%Some history to software components mentioning McIlroy's 1969 paper \cite{McIlroy1969}
Software components ideals can be traced to M. D. McIlroy's paper \cite{McIlroy1969}, where he described his ideas industrializing software engineering through a software component sub-industry. 
These components are made through the decomposition of software into discreet units with explicitly stated interfaces and behaviours.
However there is still much debate over how to define what a software component is, and what features it should have.\\

%%%Discussion of articles where Szyperski and Meyers discuss give their versions of what a component is.
Dr Dobb's magazine published a set of articles by two leading researchers in software engineering, Bertrand Meyer, originator of the design by contract methodology, 
and Clemens Szyperski, author of ``Component Software: Beyond Object-Oriented Programming''. 
These articles \cite{szyperski_components_2000, szyperski_point_2000, meyer_what_2000,meyer_significance_1999} discuss the
definitions of what a software component is as and how to use them.\\ 
Szyperski defined components in the first edition of his book, \cite{szyperski_component_2002} as having three characteristic properties:
\begin{enumerate}
  \item a unit of independent deployment
  \item a unit of third party composition
  \item has no externally observable state
\end{enumerate}

In the Dr Dobb's articles Szyperski defended his position against Meyers own enumeration of what a software components should be:
\begin{enumerate}
  \item May be used by other software elements (clients)
  \item May be used by clients without the intervention of the components
  developers
  \item Includes a specification of all dependencies (hardware and software
  platform, versions, other components)
  \item Includes a precision specification of the functionality it offers
  \item Is usable on the sole basis of that specification
  \item Is composable with other components
  \item Can be integrated into a system quickly and smoothly
\end{enumerate}

%%%It is difficult to define a natural concept such as component with a set of fixed enumerable features.
%%%Our egotistical view of components is defined with relation to Component Dependency Resolution.\\
{}As noted by Szyperski in \cite{Szyperski2002}, 
{}formal concept analysis \cite{Ganter1999} is the observation that it is impossible to
{}``enumerate a fixed agreeable set of features that is necessary and sufficient for a natural concept such as component.'' 
{}A definition can be found however, not by feature enumeration, but through stating the intention for the concept and exploring the technically inevitable consequences. 
{}As our intention is to investigate component dependency resolution, we will define what a component is with respect our goal of using component dependency resolution.

%%%This will not fit all views of what a component is (no definition will), but it eliminates unnecessary arguments so we can focus on CDR
This view of a software components definition will not agree with every persons view of what a software component is, but then no view can.
By only looking at aspects that are necessary for our topic we eliminate arguments over unnecessary semantics to focus on the core aspects of component dependency resolution. 

\subsection{A Software Component with respect to Component Dependency Resolution}
For component dependency resolution to be used the component model must require the explicit definition of dependencies of all components
and include mechanisms in which the composition of components can be altered.
This definition leaves undefined many aspects in the space of component software, however these are superfluous when discussing dependency resolution therefore are ignored.

%%%A component must have explict (computer readable) requirements on the context in which it can be used. Different from COTS requirements which are meant to be human readable.
The core requirement of a component that enables the use of component dependency resolution, is the explicit definition of what context a component can be executed under.
 
%%%The component model should specifiy mechanisms to support evolution of the system; e.g. the addition and removal of components from a composition.
Being able to use dependencies in order to find a system that fulfils all component requirements is not entirely useful unless there exists mechanisms in which to build or change a system.

%%%Aspects such as semantics of composition, distribution, development, are outside the scope of CDR, therefore ignored in this definition (that is not to say they are unimportant).    
Composition can be ignored because although the dependencies express a semantic purpose, that is not necessary for the resolver to know.
How the components are distributed is also superfluous, as the selection of the component is where we live

\section{Component Models}
%%%We go over some component models with their goals, syntax, semantics but end focusing on the component models, OSGi, Debian and CUDF
With respect to our definition of a software component, we examine some current component models, focusing on to popular models OSGi and Debian.

%%%Fractal

%%%Maven POM

%%%JavaBeans

%%%RPM

\subsection{Debian}
%%%The debian component model is used in Debain based GNU/Linux distributions

%%%Some feel that this is not truly a component model as it has no formal definition of composition.\\

%%%However it fits our criteria of a component as given in previous sections

\subsection{OSGi}
{}OSGi is a mature component model from the OSGi Alliance that has a significant presence in industry \cite{Kriens2008} 
{}from organizations like the Eclipse Foundation with their Equinox framework and Apache with Felix.

\subsection{CUDF}
%%%Here we give a brief introduction to CUDF, later in chapter \ref{implementation} we describe it more in depth

\subsection{Comparison}
%%%The comparison of component models focusses on their semantic definitions of dependencies and requirements,
%%%and the mechanisms by which a composition change.

\section{Evolution}
%%%Evolution is an important core aspect of Software Engineering and examining it in realtion to Component Systems is beneficial to the community\\
{}A software engineers goals towards software is to have continual satisfactory execution with maximum confidence at minimal cost and delay in this changing world \cite{Lehman2006}, 
{}while maintaining intellectual control over the complexity of design \cite{Brooks1975}.
{}Component systems aim to manage complexity through decomposition of systems into well defined components, and the evolution of such components through well defined mechanisms.

%%%Evolving a component system requires less change when compared to evolving a monolithic system.
Evolving component-based software requires only altering a minimal set of parts of the system, 
where as evolving a monolithic piece of software is achieved by replacing the entire system.
This comparison is described by Szyperski \cite{Szyperski2002} as ``evolution not revolution''.
As introducing change to a system introduces significant risk,
minimising change during evolution lowers the risk of component-based systems compared to monolithic systems. 
This risk occurs due to side effects that are caused by change, 
and changing large parts of a system incurs more side effects versus altering only specific parts of a system.
Therefore, lessening the change lowers the risk, and this is a very attractive feature of using and developing component-based software.

%%%A component model should support the evolution of individual components, and the evolution and change of the entire system

{}In this section we first discuss the support provided by a component model through the evolution of individual components and component systems.

\subsection{Component Evolution}
%%%Component evolution, i.e. the modification of components to create newer versions or branch existing versions.\\

%%%What is a version?
A version is the mechanism through which individual components evolve.
It is a unique marker which is compariable to other versions such that they are ordered.
As with the evolution of entire software systems, evolution of components requires constant maintenance.
Bugs can be found, features can be added, and code can be refactored all which make the component better.

%%%How are components versioned?
The syntax and semantics of a version is usually defined within the component model.
This forces all components that are developed for this model to follow the same general guidelines.

%%%Versions between components?
Although component must be versioned to conform to a versioning model defined by the component model, 
the way in which a developer uses this model to version their component is not usually strictly defined.
Each component is designed to be an independent unit, as such developers of components are separate from one another,
and the way in which they version their components can be different.
Therefore, version comparison is only useful between different versions of the same component;
e.g. a spell-checker component can be version 10 but a separate and superior spell-checker component may be only version 1.
This makes using component versions to decide between components impossible.

%%%A version of a component that is greater than another version of that component is assumed to be better in some way.
A newer version exists because the component has changed to be better in some way.
This may be through a bug being fixed, a security hole being patched, or the functionality being extended.
Having a newer version of a component in the system, may also make all of the other components that depend on it better as well,
causing a propagation of system improvement with the upgrading of one component.

In Belief Revision, one must maintain the newest set of information while only changing the minimal amount of previous knowledge. 

The change to the system that is caused when a newer version is installed may cause harm to your system.
In the same way that a version upgrade can propagate benefits through the system, 
friction or errors can propagate and cause a faulty or not functional system.

%%%Drawbacks of versions, they don't tell what changes
A common approach to selecting an optimal solution is to
maximise the versions of components while minimising the number of components.
The problem with the first criterion is that the version information does not tell us what quality improves, 
how much it improves or if that quality is relevant to the user or system.
Optimising for fewer components in a solution can be against many component system goals, like reuse or extensibility.
For example, this heuristics optimal solution would be a monolithic component with redundant functionality, over 
a set of more specific components. 


Versioning is the main mechanism in which components evolve.
There is also branching, and environment specific altering.

Versioning components has significant research behind it.

Branching is when a difference in opinion or idealogy on how a component should proceed into the future
causes the project to take two different routes given the same base code or idea.

Spring DM takes libraries and makes them OSGi compatable, 
Debian package can be compiled for different CPU architectures.


\subsection{Component System Evolution}
%%%Changing the component in the component systems by adding removeing components to support evolution\\
Version and Dynamic dependence, with focus on interface give extensive tools to the composer 
Add, remove, upgrade, downgrade components in the system while it is running.

%%%How is this achieved?

%%%Problems with versioning a component system is different combinations of components, with different versions.

There are some problems that must be solved within the component model like making sure that during the change,
the system is never invalid or unstable. 
This is a scheduling problem as described by %TODO

The composer must also make sure that all dependencies are meet

\section{Component Dependency Resolution}
%%%The tedious nature of satisfying component dependencies during component system evolution can be automated into a function called Component Dependency Resolution.
Manually satisfying all component dependencies while evolving a component system can be a tedious, time consuming task.
This process can be automated into the function Component Dependency Resolution (CDR) \cite{Jenson2010}.
This takes advantage of the explicitly declared requirements of components to find a system where all component constraints are satisfied.

%%%CDR can be used at design-time and run-time on component models.
CDR can be used at design time to determine the required dependencies to build and test a project (as in Apache Maven \cite{casey_better_2008}),
at run time to evolve or extend a component-based system (as in Eclipse P2 \cite{leBerre2010}),
or it can be used to build and restructure software product lines \cite{savolainen_analyzing_2007}.
CDR is also used in package management systems, like apt-get \cite{Barth2005},
for adding, removing and updating components in package based operating systems like Ubuntu\footnote{http://www.ubuntu.com}. 

%%%Talk about dependency hell and how that it is largely mitigated via dependency resolvers

\section{Boolean Satisfiability (SAT) Solvers}
%%%The definition of SAT has been around for a while, it was the first problem to be given the complexity NP-Complete

%%%SAT solvers have been investigated and found as a common good solution to CDR (EDOS PAPER)
Dependency Resolution is a NP-Complete problem, and NP-Hard given criteria to find an optimal solution.
This problem has been found in many areas, and there have been efforts to standardise the solution in an extensible way.
The EDOS project did a review of using SAT solvers, modified for optimisation, to solve dependecy resolution, and suggested that they were a satisfactory solution.
Today, most solvers use a SAT solver as they represent a quick extendible way to implement this functionality.

%%%SAT solvers are used to solve many problems, scheduling, model validity\ldots

%%%SAT solvers are very good at solving these kinds of problems because they are tested for performance against each other, to find what works empirically   

%%%We will breifly discuss their implementation and the algorithms associated, as it is the mechanism which we use to solve CDR problems

\subsection{Davis-Putnam-Logemann-Loveland algorithm for SAT Solvers}
%%%A successful algorithm for solving SAT problems is the DPLL algorithm, here we describe it in overview

%%%Some improvements over the core algorithm have been appended over the years since DPLL was first proposed.

\subsection{CHAFF Watched Literals}
%%%Through having clauses watch the literals, rather than the other way around, the most expencive step, the propogation, can be reduced significantly.
%TODO cite

\subsection{Conflict Learning}
%%%Simple resolution as proposed by Stallman was an effecient means to stop thrashing(looking at the same subtree repeatedly)

%%%Improvement of this resolution can be made by finding smaller constraints, this can be accomplished by eliminating literals in a constraint

%%%Further improvement can be made by searching for the smallest constraint through such variable elimination

\subsection{Literal Order}
%%%The order in which the literals are assumed in the DPLL algorithm, is heuristically accomplished

%%%The selection of a literal can be broken into two stages for modularity;
%%%first selecting a variable that is very volatile,
%%%second selecting a phase (negative or positive)

\subsection{Pseudo-Boolean Optimisation of SAT Solvers}
%%%Optimisation of SAT solvers is typically done through extending their possible constraints to include Psuedo Boolean inequalities

%%%Then through repeatdly finding a solution then adding a constraint to ensure the next solution is at least as good, the best solution is found

%%%Further optimisation can be made through quickly finding a satisfactory solution, this reduces the space in which to find a better solution

\subsection{Other Methods}
\subsubsection{Integer Programming}
%%%Discussion of this method as the best MISC solver uses this, it has a very complex implementation
\subsubsection{SMT Solvers}
%%%SMT Solver, a slightly higher logic than SAT uses; it has to broad a definition when SAT suffices
\subsubsection{Constraint Solver}
%%%We could just use Prolog, like SMT I think it is too broad when there are good SAT solvers

\section{Common Upgradeability Description Format}
%%%Here we give an introduction to the CUDF format from Mancoosi, this format is used as a standard for Dependency Resolution problems
{}\cite{treinen2009common}

%%%It was made to standardise the problem, for two reasons, to make building solvers generalisable, and to compete solvers in the MISC competetions to compare solutions.

%%%In both regards it has succeded, there have been many competitions run, 
%%%and their creation of a Modular Package Manager(MPM) is currently moving through the Debian stages into their core repository

\subsection{CUDF Syntax}
%%%The CUDF syntax is made to resemble similar CDR meta-data through using key-value pairs (not XML, too verbose?)

%%%Package Formulas are the standard mechanism of dependence

%%%Versions as Integers

%%%The Syntax has been defined to be extensible within the language, so additional model specific information can be included without increasing the size of the standard

\subsection{CUDF Semantics}
%%%As CUDF was a format the was supposed to be open to allow people to build differnt solvers for it, and be used as a real package manager, it walked a fine line.

%%%It must be simple enough so that developers can quickly understand it to build or modify existing solvers capable of competing in MISC competitions

%%%It must be broad enough to define and solve real world problems


\section{Design Time Dependency Resolvers}
%%%Design time dependency resolvers benefits the software engineer building a large complex system that can involve many libraries with complex relationships.

\subsection{Apache Maven}
%%%Maven is a design time project and build automation tool, that can resolve library dependencies from a repository (https://oss.sonatype.org/content/repositories/)

\subsection{Nuget} 
%%%A new distribution model from microsfot for .Net libraries (http://nuget.codeplex.com/)

\subsection{RubyGems bundler}
%%% Bundler is a package dependency manager for RUBY gems (http://gembundler.com/)

\section{Run Time Dependency Resolvers}
%%%Run time dependency resolvers benefit users of the component system, who would rather use a systems functionality than maintain a complex system of interacting components.

\subsection{OSGi Bundle Repostiory}
%%%The OBR proposed by Peter Kriens and Richard Hall, is a meta data format for OSGi bundle resolution
Peter Kriens \cite{kriens_osgi_2008}, 
a primary author of the OBR \cite{the_osgi_alliance_rfc-0112_2006}, described a use of a component repository within a resource constrained environment.
When a device registers, 
through the repository, the driver for that device is loaded and then used, until the device is no longer available
then to conserve resources the driver is uninstalled dynamically. 
This demonstrates two requirements for a component repository, dynamic installation and uninstallation and being scalable to small devices.

For the OBR, we reviewed the standard \cite{the_osgi_alliance_rfc-0112_2006}, 
contacted both authors \cite{kriens_osgi_2008, richard_s._hall_obr_2008} 
and downloaded the open source Apache Felix OBR implementation and reviewed the source code.
OBR was our most completely reviewed repository.

OSGi is a 
mature standard component model from the OSGi Alliance that has seen a recent uptake in industry \cite{kriens_osgi_2008} 
from organizations like the Eclipse Foundation with their Equinox framework and Apache with Felix. This has multiplied the 
number of available bundles \cite{the_osgi_alliance_rfc-0112_2006}. OSGi Bundle Repository (OBR) was initially defined by 
Richard S. Hall, and later Peter Kriens, to help with the distribution of components, and is proposed in 
RFC-0112 \cite{the_osgi_alliance_rfc-0112_2006}. The design follows a federation of repositories each describing 
the capabilities and requirements of it's components and providing a URL to access it. Search is done using syntax with OSGi filter 
based expressions. This repository can also resolve dependencies and install these composites into a system. \\
OBR has been seen as a solution to to simplify deployment of OSGi applications
\cite{jung_integrating_2007}, distribution and deployment to embedded
ubiquitous systems \cite{jung_embedded_2006}, smart home applications \cite{gouin-vallerand_managing_2007} and 
dynamic distribution of drivers \cite{kriens_osgi_2008}. It is the only easily available deployment 
architecture available for OSGi, others have been developed to either extend or compete \cite{frnot_component_2005} with OBR, 
but are either in development, have been abandoned or are proprietary.\\ 
The most mature implementation of OBR client is offered by Apache, it is bundled with their core OSGi framework, Felix. 
The largest public OBR implementation is the federated repository provided by the OSGi alliance,
currently (as of 6 January 2009) aggregating components from Apache (32 bundles) and Eclipse (1534 bundles).
There are also many private repositories, some discussed with as much as 5000 bundles involved \cite{kriens_osgi_2008}.\\  

\subsection{OSGi Declarative Services}
%%%DS is standardised in the OSGi standards

\subsection{Spring Dynamic Modules}
%%%Spring is a competitor to DS, such that it resolves service level dependencies

\subsection{yum}
%%%Red hat package manager

\subsection{Apt-get}
%%%Apt-get is the core application for package management in Debian based GNU/linux distributions

Debian is a free and open source GNU/Linux based operating system. 
Debian's package system (dpkg) is used to install, remove and provide information about the operating system packages. 
APT is a repository client built on top of dpkg and simplifies it's functionality.\\
Aptitude is an application that adds a basic user interface to APT and also increases functionality in aspects like 
locating and removing redundant packages. 
Search of a package repository is done using regular expressions mixed with a
custom set of search terms that increase expressiveness. Using a simple
constraint-directed best-first search algorithm with some basic conflict
tracking Aptitude resolves dependencies between packages, and generates
composites to install.\\ 
There are many public and private implementations of this repository system. 
The largest public implementation is the Debian public repository, with 28078 packages in the ``Lenny'' stable repository.
With the most popular Debian distribution Ubuntu\footnote{http://www.ubuntu.com} having ten million users \cite{vance_software_2009} the user base for this repository is massive.
This repository has many regulations to publish and distribute, enumerated in the Debian Developers Reference \cite{andreas_barth_debian_2008}. 
User and automated testing is recommended, and if any critical bugs are found they are removed from the repository. 
This very manual way of publication and
distribution has its drawbacks, mostly the long community process that must be followed.\\

\subsection{Aptitude}
%%%Aptitude is a competitor to apt-get which has some more advanced features
Aptitude with APT, has immense amounts of information free on-line, the hardest part was reviewing as much as possible.
We also used and read the help documentation, contacted the author of the software \cite{burrows_modelling_2005} and
the Debian developers reference for the specific implementation of the Debian repository.

\subsection{Mancoosi MPM}
%%%The mancoosi Modular Package Manager uses CUDF, currently going through the process in the Debian component repository

\subsection{Eclipse P2}
%%%The Eclipse P2 Solver, by Le Berre and Rapicault

For the Eclipse P2 repository system there was extensive literature provided by the Eclipse foundation in the form of a developer forum.
We also reviewed an early implementation through use of the software within the Eclipse environment.
Due to its recent release there is no literature reviewing or discussing this software.

Eclipse P2
is the new provisioning system for the Eclipse IDE platform.
It provides an update manager, the functionality to component pool, and a new dependency resolver.\\
The update manager is what most users will be familiar with, being the primary interface to the P2 functionality. 
A public repository, called a site, is added, and can be searched for components and their dependencies to install, and run.
This simplifies the evolution of Eclipse based applications to a simple user interface.\\
Component or bundle pooling uses a centralised set of components to run different applications.
This comes from the common problem of running multiple Eclipse based applications on the same system with different combinations of components.
A file is used to describe what components should be used in each particular application and P2 builds the composition into an application.\\
The dependency resolver of P2 is based on SAT4J\footnote{http://www.sat4j.org/} technology, a SAT resolver, which finds an optimal solution based on
a provided fitness function, if a solution exists it is guaranteed to find it.\\  
Eclipse already is
the core of many applications with its Rich Client Platform. This is a large base of applications that already support and can integrate with P2.

\subsubsection{P2CUDF implementation}
%%%Our implementation is based off of Daniel Le Berre and Rapicult P2CUDF, which uses SAT4J a SAT and PB solver, tested and validated in competitions.
The Mancoosi organisation who have created the CUDF standard have also organised competitions for the solvers.
One implementation of note is the P2CUDF implementation, that uses SAT

\section{Comparison of functionality}
%%%The comparisons of these resolvers will focus on the functionality that they provide given their component model (we will not compare their implementations)

%%%Multiple package installations 
%Maven Debain No, OSGi Eclipse DS and CUDF yes

%%%Focus on interface not implementation dependence 
%Maven no, Debian Community defined virtual packages, 

%%%Version Control

%%%Run time replacement Dynamic Dependence
%Maven no (no runtime), Eclipse Kind of a bit buggy requires restarts,
%OSGI DS Debian Yes, CUDF is abstract so has no real implementation

%%%Conflicts
%Maven?, Spring and DS no,  OSGi Singleton, Debian and CUDF are Expressive

%%%Reccommendation System, soft dependency (is this even a good idea)


\chapter{Component Dependency Resolution}
{}The strict definitions of the relationships between components can not only be used to find  

\section{Formal Definition}


\section{Current Implementations}
\subsection{Maven}

\subsection{Nuget} 
%%%-- C\#

\subsection{bundles}
%%% RUBY

\subsection{Nimble}
\subsection{OSGi Bundle Repostiory}
%%%The OBR proposed by Peter Kriens and Richard Hall, is a meta data format for OSGi bundle resolution

\subsection{Apt-get}
\subsection{Aptitude}


\section{Implementation}

\section{Boolean Satisfiability (SAT) Solver}
%%%In this chapeter we describe an effecient and popular solution to the CDR problem, the SAT solver (EDOS PAPER)

%%%First we define the SAT problem

%%%Then discuss its Complexity (NP complete if it the problem satisfies some criteria)


\subsection{Davis-Putnam-Logemann-Loveland algorithm for SAT Solvers}
%%%A successful algorithm for solving SAT problems is the DPLL algorithm, here we describe it in overview

%%%Some improvements over tie

\subsection{CHAFF Watched Literals}

\subsection{Conflict Learning}
%%%Simple resolution as proposed by Stallman was an effecient means to stop thrashing(looking at the same subtree repeatedly) constraints

%%%Improvement of this resolution can be made by finding smaller constraints, this can be accomplished by eliminating literals in a constraint

%%%Further improvement can be made by searching for the smallest constraint through such variable elimination

\subsection{Literal Order}
%%%The order in which the literals are assumed in the DPLL algorithm, is heuristically accomplished



\section{Optimisation Representation}



\chapter{Implementation}
\label{implementation}
%%%The implementation of CDR is also a very important aspect of the problem, as the queries can be very large and complex to solve given its NP-Hard nature.

%%%The implementation to these solvers is sometimes ad-hoc, this can be bad as the problem is similar across different domains, 
%%%ad-hoc solution decrease the possibility to share knowledge and solutions, or extend your system

%%%An efficient mechanism to find optimal solutions that is gaining popularity is through using a Boolean Satisfiability Solver (SAT)

%%%We have already reduced the problem to a SAT problem in the previous chapter to show the problem is NP-Complete

%%%Through a slight modification of these solvers to allow constraints with inequalities and cardinalities, Psuedo Boolean constraints,
%%%we can include optimisation for the required criteria using Pseudo Boolean Optimisation (PBO)

%%%There are other methods to find a solution, some of them faster, but SAT was chosen because of its simplicity and clear mapping to the core problem.

%%%We discuss the format specified by the Mancoosi project, Common Upgradeability Description Format (CUDF),
%%%which gives us a unified syntax and semantics to map many real world problems, like Debian and OSGi, to our formal definition.

%%%We look at P2CUDF, an implementation by LeBerre and Rapicult using SAT with PBO to solve CUDF problems. 

%%%Finally, using this knowledge SAT solvers with PBO, and the format CUDF we were able to implement our own solver inspired by P2CUDF.
%%%We discuss its implementation, its differences from P2CUDF, and its validation in the MISC competition (August TODO)

%%%Here we discuss our method of implementation we used to solve CDR

%%%The back end model for this solver though was OSGi specific, we decided to replace the back model with a more CUDF specific one.

%%%Our model, expressed as a UML class diagram

%%%We have altered the parser to fit our model

%%%We have changed the mapping of their model to SAT, to our model to SAT

%%%We also refactored SAT4J, to SlimSAT, this is identicle to SAT4J, except we removed significant portions of unused code, retained for competitions 
%%%(i.e. MaxSat instances, and unused constraint implementations)


\subsection{Debian Model to CUDF mapping}
%%%The modelling of Debain packages to CUDF is very similar 
The mapping of the Debian dpkg component meta-data to CUDF is mostly a direct process as the meta data is very similar. 
However, there are a few instances where the Debian semantic causes complication with the CUDF format, and in this section we describe these in detail.
The first of these complications occuurs with versioning, as in Debian the version model is different to the CUDF model.
The second instance is because of the special relationships that virtual packages have in the Debian model, 
these must be intergrated into the CUDF mapping. 
Lastly the priority ranking, recommends, suggests and other extra information that is not standard in the CUDF format,
but with CUDF's extencibility can be represented and used in the resolution process.
The one difference we have from the Debian semantic is the ability to have multiple versions of the same package installed, this is impossible in debian, but we see it as a simplification of the problem.

%%%How versions are handeled in this mapping, ordering all versions in a list and giving them the version of their index
The versions in Debian follow a model that first describes the epoch,
the upstream version, then the Debian revision (further described in the Debian policy manual). %TODO cite debian package manager
When comparing two versions these values are compared lexicographically such that
if the epoch is equal then the upstream versions are compared, 
and if these are equal then the Debian revision is compared.
As the CUDF version model is merely a single integer number, 
the Debian versions are not easily mapped.
To map these versions we order all Debian versions that are referenced in a repository into a single list, 
such that the least version has the index 0, and the last element in the list is the greatest version.
The index of the a Debian version in this list is then used as the CUDF version.

%%%Virtual Packages must be handeled, there are some specific debian semantics that are particularly important
A virtual package in

%%%Priority ranking also must be handeled, as some packages are absolutly required to be in a system, these can be maked similarly in CUDF

%%%We allow multiple versions of the same package to be installed, this is different from the debian 

\subsection{OSGi Component Model to CUDF Mapping}

%%%How versions are handeled, similar to the Debian model



\subsection{Product and Lexicographic orders in Pseudo-Boolean Optimisation}
%%%Here we descirbe the mapping from our Optimisation framework to PBO, product order is easy, lexicographic requires some muddling

\subsection{Drawbacks of this Optimisation approach}
%%%There are a few drawbacks to this mapping;
%%%These simplifications are mostly necessary for this problem as it keeps the problem manageable,
%%%and including all aspects of our formal optimisation framework would be practically impossible

%%%Real numbers must be truncated to fit the integer representation. 

%%%Only one solutin can be chosen so product orders are randomly selected to a degree, this can be mitigated by having no stochastic elements in the algorithm

%%%No criterion can be of a partial order though through product composition the problem can be a partial order

%%%The only aggregation of numbers between integers is addition.

%%%It is too expencive to recalculate cardinalities for each component given a specific solution, therefore the cardinalities of a component should be solution independant



\section{Validation}
%%%We entered this solver into a Mancoosi MISC competition, speed doesn't matter for these results, but the returned and quality of the solutions does.
%%%MISC competitions delibratly create very difficult problems, so I doubt we will get the best solutions in most, and may fail a few. 
%%%What I expect are consistent results that are comparable to other solvers.

\section{Criteria}
The two main criteria considered during component system evolution are to maximise the versions of individual components, 
while minimising the change caused by the evolution to the system. 
These two criteria are conflicting as increasing the version of a component is a change to the system.
They are also not strictly defined, as the version of the components must be aggregated to measure the entire system,
and the amount a system is changed can be measured in different ways.
The way in which the criteria relate to each other is different for different users.
For example, a server administrator will likely change as little as possible rather than risk a fault, 
where a developer/debugger will always wants the newest versions of components to stay up to date.

\subsection{Version vs. Minimal Change}
\quote{Why is having the newest version of a component important?}

A newer version exists because the component has changed to be better in some way.
This may be through a bug being fixed, a security hole being patched, or the functionality being extended.
Having a newer version of a component in the system, may also make all of the other components that depend on it better as well,
causing a propagation of system improvement with the upgrading of one component.

\quote{Why would you not want the newest version then?}

The change to the system that is caused when a newer version is installed may cause harm to your system.
In the same way that a version upgrade can propagate benefits through the system, 
friction or errors can propagate and cause a faulty or not functional system.

\subsection{Definitions}
\quote{How can we measure the version of a system from the versions of installed components?}

To answer this question over a set of components that are probably versioned differently, 
we need to explore functions that can aggregate the versions of different components together in a meaningful way.

Each component must be versioned to conform to a versioning model defined by the component model, 
however the way in which a developer uses this model to version their component is not usually strictly defined.
Each component is designed to be an independent unit, as such developers of components are separate from one another,
and the way in which they version their components can be different.
Therefore, version comparison is only useful between different versions of the same component;
e.g. a spell-checker component can be version 10 but a separate and superior spell-checker component may be only version 1.
This makes using component versions to decide between components impossible.


The Mancoosi organisation uses a metric that minimises the number of components in a system that have better versions, 
e.g. a system $a-1,b-1$ if $a-2$ exists is $1$ out of date. 
The Eclipse P2 implementation counts the amount of versions that are better for each component,
e.g. a system $a-1,b-1$ if $a-2$ and $a-3$ exist, is $2$ out of date.
Both of these take into account only the component and not the components that depend on it.

The measurement of change that a system goes through during evolution is also difficult to define.
The added, removed, updated, total changed, have all been considered as metrics before. 
As with the version metrics, none of these consider the dependencies when looking at the change that is performed on the system.


\subsection{The Gap}
Each time a user decides to evolve a component system, the decision must be made about the risks of the evolution.
In an environment which is mission critical, all risk is eliminated and an unecessary change to the system is too risky.
In a development environment where the user may be trying to fix potential problem, 
or test different packages, then the risk is accepted as the system is essentially disposable, and a complete re-installation is not out of the question.
These two strategies are represented in by the current strategies used, however very little middle ground is avaiable for the customisation of applications.
The user may want 

\chapter{Component Model Comparisons w.r.t. Component Dependency Resolution}
\label{comparison}
%%%An important issue when looking at this problem of CDR can information gained through studying one instance be generalisable to another.
{}Different aspects of a component model and its use can effect the dependencies and relationships between components,
{}therefore affect the properties of CDR or the way it is used.
{}Aspects such as, developer communities and conventions, what is allowed in the syntax of the meta-data expressing the dependencies,
{}and the semantics of the component model will affect CDR.
{}This will clearly affect the study of CDR and its results, and whether they are generalisable.
{}Developer conventions created by communities, component model semantics, goals of the component model can all effect the constraints on the system thus CDR's solutions.

%%%Developer conventions like in eclipse depending explicitly on component, vs. OSGi where it is prefeered to depend on packages

%%%Component model semantics, how dependencies are manifested, some can have multiple version of the same component installed with complex wiring standards,
%%%some certain constraints to be defined, in eclipse the only conflict is in a singleton relationship.

%%%The goals of the component model, whether adpatability is the core goal or a precisly defined system

%%%We use two methods to compare and analyise these possible effects on the component models, graph and constraint analysis, 
%%%to ultimatly determine if through studying one of these component models the results can be generalised.

%%%We study and compare two leading component model, OSGI and Debian, using thee methods.
Two of the leading component models that CDR is currently used with are OSGi and the Debian Package system.
The differences that occur because of their differences will effect the dependencies and relationships between components,
and then affect the properties of CDR or the way it is used.
To generalise any results, or to analyse the effects of different CDR criteria we must first compare them.
We first give specific differences and then use dependency graph and constraint analysis to explore their consequences.
We find \ldots%TODO   

%%%Clearly these results themselves cannot be generalised, but through this method we hope that other component models can be checked to see if our results will be consistent.

\section{OSGi and Debian}
%%%Here we will discuss various differences between the OSGi component model and the Debian component model, their goals communities of developer

\subsection{Debain}
%%%repackaging

%%%Upstream developers

\subsection{OSGi}
%%%Rebundling libraries


\section{Dependency Graph analysis}
%%%Distributions of dependencies, graph propeties, small world? What does their internal graph structure look like?

\subsection{Critieria}
%%%We can use some of the previously defined criteria to analyise the graphs

%%%Page rank distributions

%%%HITs distributions

%%%Coupling and instability

\section{Constraint analysis}
%%%The graph removes some of the details, i.e. constraints, thus it is a broad view of their structure

%%%By analyising the constraints created by the components, e.g. number of conflicts, distributions of dependency sizes\ldots, we can gain finer grained information

\subsection{Search Space Size}
%%%Here we analyise the size of the possible search spaces of both OSGi and Debian, through the simple mechanism of brute force search

%%%Relative to their size, we can analyise their distributions of search space

\section{Conclusions}
%%%Here we can go over the similarities and differences between the two component models, hopefuully they are the same

%%%If they are very different, accounting for their relative sizes and maturity, we can eliminate the idea that our other results are generalisable (which is useful but not preffered)


\chapter{Simulation}
\label{simulation}
\epigraph{A model is a physical, mathematical, or logical representation of a system entity, phenomenon, or process. 
A simulation is the implementation of a model over time. 
A simulation brings a model to life and shows how a particular object or phenomenon will behave.}
{\textit{Systems Engineering Fundamentals. Defense Acquisition University Press, 2001}}

To study the CSE process is the main purpose of this thesis.
This study could be in the form of either looking at real systems and users, studying a system in a controlled environment or simulating the environment in an abstract manner.
Each of these options is discussed and weighed for their costs and benefits.
The simulation method is selected as it has the best attributes required for this research.

The methodology outlined by \cite{Law2005} is used to create a valid and credible simulation.
The basic artifact of this methodology is a ``conceptual model'', which is a simplified abstraction of reality.
A conceptual model can be used to create a CUDF* documents that can then be simulated to study the CSE process.
These relationships are described in figure \ref{sim.modeldiagram}.



\begin{figure}[htp]
\begin{center}
\digraph[scale=.5]{simmodeldiagram}{
rankdir=BT;
CMS[label=<<TABLE BORDER="0" CELLBORDER="1" CELLSPACING="5"><TR><TD WIDTH="150">CUDF*</TD></TR></TABLE>> shape=none];
subgraph {
	rank=same;
	CMI[label=<<TABLE BORDER="0" CELLBORDER="1" CELLSPACING="5"><TR><TD WIDTH="150">:CUDF*</TD></TR></TABLE>> shape=none];
	ConMI[label=<<TABLE BORDER="0" CELLBORDER="1" CELLSPACING="5"><TR><TD WIDTH="150">:Conceptual Model</TD></TR></TABLE>> shape=none];
}
ConM[label=<<TABLE BORDER="0" CELLBORDER="1" CELLSPACING="5"><TR><TD WIDTH="150">Conceptual Model</TD></TR></TABLE>> shape=none];
CMI -> CMS [ label="instantiates"];
ConMI -> ConM [ label="instantiates"];
ConMI -> CMI [ label="creates"];
}
  \caption[labelInTOC]{figureCaption}
  \label{sim.modeldiagram}
\end{center}
\end{figure}



\section{The study of Component System Evolution}
To study the CSE process, one must examine and investigate how a component system evolves in response to a request to change. 
This requires access to many component systems state before an after a change, and the requests that instigate the change.

To obtain this information three methods could be used:
\begin{enumerate}
  \item field data could be collected from real systems evolving ``in the wild''.
  \item a controlled environment could be created, in which systems are monitored as users are asked to request changes.
  \item a simulation of a systems evolution could be run, where variables of the simulation are altered to represent different environments and situations.
\end{enumerate} 

In this section the positive and negative aspects of each of these approaches are discussed,
and the reasons for the selection of the simulation method for this research are presented.

\subsection{Using Real Users and Systems}
Studying real systems that are evolved by real users is one method of studying CSE.
Users could participate by installing monitoring software in their component systems that collects data.
This data could then be used to study the process by which their systems evolve.

This approach  has significant drawbacks:
\begin{itemize}
  \item it will take long periods of time to find meaningful results.
  \item the monitoring of user's actions is an invasion of their privacy, which may limit the pool of willing participants.
  \item finding and convincing enough users to participate to generate useful results will require significant effort.
\end{itemize}

\subsection{Users in a Controlled Environment}
To increase the efficiency of collecting results and to eliminate privacy issues of users, a controlled environment may be created.
This environment is a component system where the parameters of its evolution could be controlled and monitored.
A user would then be asked to evolve the system to approximate their real actions, the resulting systems could then be studied.

Such a controlled environment could very quickly simulate  many interactions that would of occurred over a long period.  
It also removes the necessity user trust as the user is altering a system that is not theirs.
However, it will also produce less valid results, as it is still an approximation of a real system, and the users know this.
Any controlled environment would alter how the user interacts with their controlled system, making the results less reliable.

However, the largest problem with this method is the effort required to conduct the experiment on enough users to find significant results.
The core difficulty with this method is finding and convincing enough users to participate, and then monitoring and supporting them while they are in the controlled environment.

\subsection{Simulating the Problem}
CSE can be studied by simulating the users actions and the component system using reasonable approximations.
A simulation is a surrogate of the real system, 
such that it represents the core aspects of the reality while simplifying and abstracting away unnecessary detail.
The goal of a simulation is not to represent every aspect of the real world, 
but to make a ``close enough'' approximation so that the conclusions drawn from it are valid while minimising complexity.

When analysing the results and forming conclusions from a simulation, 
the assumptions and abstractions made, must be taken into account.
Therefore, the majority of the effort of creating a simulation is making sure the results from the simulation are valid.

\subsection{Why Simulate?}
The advantages of simulating are the speed at which ideas can be tested and evaluated;
the cost to test and get results; and the control over variables and configuration of the environment.

The core steps to generate new knowledge is the iterative process of creating a hypothesis, then testing its accuracy.
As the speed of this process increases, the greater amount of knowledge and meaningful information that can be generated. 
Using real users, or real systems, requires long periods of waiting and preparation where little or no progress is made.
However, if the actions and environment can be modeled and executed computationally, these times of no progress are eliminated.

%%%Cost
The cost of testing a hypothesis can be measured in the amount of time it takes to define an experiment and to collect the results.
Using real users requires significant amount of time to collect and convince users to participate.
However, with a simulation the main effort is ensuring that the returned results are valid.
Once the simulations validity is established, it takes very little time to test a hypothesis.

%%%Control
The control over the variables in a simulation allows for testing extreme and normal situations.
It also allows for the sensitivity analysis of different variables, finding how important they are to the simulated system.

%%%Final words on why we simulate
A simulation of CSE is selected in this research as it has a cost/benefit ratio that is desirable,
and provides an acceptable level of accuracy to draw meaningful conclusions.

\section{Methodology}
%%%We use the methodology from `` Build a Valid and Credible Simulation''
The core hurdle in creating a simulation is validating the returned are similar to reality.
To create a valid and credible simulation the methodology that \citep{Law2005} outlines is followed.
This methodology is a guideline for defining the study, collecting information, creating and validating models, and running the simulation.
In this section the methodology is described and aligned to this research's objectives. 

\subsection{Validation and Credibility}
%%%Why do we use this methodology and how is it relevant?
This methodology was created after the observation that validation was often ``attempted after the simulation models had already been developed'' \citep{Law2005}.
That is even if validation was attempted, it may only occurred if there was money and time left at the end of the project.
However, such simulations, that are not validated, can produce erroneous information that leads to bad, possibly costly decisions being made.
This reduces the credibility of the simulation to be used in future as a tool.

A simulation is an abstraction and simplification of reality, often created as using an actual system can be disruptive, not cost-effective, or simply impossible.
In this context,

\begin{quotation}
``\textit{Validation} is the process of determining whether a simulation is an accurate representation of the system, for the particular object of the study.'' \citep{Law2005}
\end{quotation}

The latter part is an important aspect of validation, as the accuracy of the simulation is directly dependent on the problem and questions the study addresses.
Therefore, the definition of the problem will directly lead to the modelling and scope of the simulation.

%%%What is credible
A simulation, and by extension its results, have \textit{credibility} if key stakeholders accept them as ``correct''.
A credible model is not necessarily valid, and vice-versa, as it involves the input of a person who decides if the goals of the simulation have been obtained.
Credibility of a simulation is then only attainable if the key personale from the project understand and are involved directly with the project.

The simulation produced by this study to identify the effect of different strategies on the evolution of component systems must be validated to produce meaningful results,
and must be credible for these results to be trusted.

\subsection{Seven Step Method}
The methodology presented in \citep{Law2005} has a seven step approach to creating a valid and credible simulation.
These steps are; formulating the problem; collecting information and data to construct a conceptual model; validating the conceptual model;
implementing (programming) the model; validating the programmed model; designing, conducting and analysing experiments; and documenting and presenting the simulation results.

\subsubsection{Step 1: Formulate the problem}
The first step is to formulate the problem as clearly as possible, this is usually done with core stakeholders in a ``kick-off meeting''.
The core artifacts from this step are the overall objectives of the study, specific questions wanting to be answered, scope of the study,
 and different configurations of the simulation with the measures used to evaluate their performance. 

\subsubsection{Step 2: Collect information and data to construct conceptual model}
The conceptual model is a description of how the simulation and system work relative to the problems earlier defined.
It is the most important artifact of the simulation.
It should be high level enough to be understood by the core stakeholders but detailed enough to be reused in future simulations.
It is created through interviews with subject matter experts and collecting relevant data like results from similar exiting systems.
Problems like the data not being representative of the model, not being in the appropriate format or type, and containing errors must be handled before use.

The conceptual model also contains all of the variables that can be configured including their documented assumptions. 
It is defined to the level of detail with respect to project objectives, performance measures, data availability, computer constraints, and resource constraints.

\subsubsection{Step 3: Conceptual model validation}
The conceptual model is the most important aspect of the simulation, thus its validation must be thorough.
The core method used to validate this model, is to discuss it with core-stakeholders and subject matter experts.
This provides feedback as to the direction of the conceptual model, ensuring that it will answer the questions posed in the study.

\subsubsection{Step 4: Implement the conceptual model}
The implementation of the conceptual model must also be executed and documented in a way that allows others to replicate and repeat the process.
The artifacts created during this process must be verified to work correctly, this can be accomplished through test-cases and debugging \citep{Pressman1992}.

\subsubsection{Step 5: Validate implementation}
There is no completely definitive approach to validating the simulation,
however, the most definitive test of a simulations validity is established by closely looking at the outputted results compared to that from an actual system \citep{Law2005}.

This is done through:
\begin{itemize}
  \item \textbf{Results validation: } a comparable system is used to create results and compared with the results from the simulation for validation.
  \item \textbf{Face Validation: } experts are given output of the simulation model and checked to see if it is consistent with how they perceive the system should operate.
\end{itemize}

Further validation of the implementation can be accomplished with sensitivity analysis,
which is performed on the simulation to find the factors with the greatest impact on the performance and results.

\subsubsection{Step 6: Design, conduct and analyse experiments}
For each of the experiments that are run, the time and number of independent runs must be defined.
If the results are inconclusive, or other aspects of interest have arisen, it may be necessary to run additional experiments.

\subsubsection{Step 7: Document and present results}
This step involves the presentation of the conceptual model, simulation, and results to the core-stakeholders.
This presentation is critical for the future re-use of the model, as it should promote credibility through describing the validation process.

\subsection{Differences in methodology}
A core goal of creating this simulation study is to produce valid and credible results.
However, this specific methodology has been created for large scale industrial projects with substantial resources.
As this study is smaller in scale and resources some of the procedures recommended in this methodology have been restricted and some removed.

The most significant difference between this study and the described methodology is the clear definition between decision-maker and simulation designer.
In larger projects a simulation designer is contracted by a decision-maker who has a problem to solve or question to answer.
In this project both these people are the same person, therefore meetings between them are not required.

Other people in the project including subject-matter experts, core-stakeholders, simulation analysts consist of survey participants and project supervisors.
This is because the limits of the projects resources excludes the employment of experts for validation.
This may reduce the validity of the end model, but these restrictions have been made only when necessary,
and done so in a manner that attempts to minimise negative effects.



\section{User Survey}
\label{strat.usersurvey}
The previous chapters have left the role a user plays in the evolution of their component system unexplored.
This chapter describes the motivations and strategies for users to evolve their component systems.

A strategy is the plan or pattern employed by a user to evolve a system.
A fundamental influence on the user's strategy is their perception of risk.
Each change to the system the user requests, and the criteria used to accomplish this change is effected by the risk of that change.
For example, upgrading a system may introduce new bugs and problems, yet not upgrading may let old vulnerabilities compromise the system.

To explore the users role in CSE a survey was given to users, the results are discussed in section \ref{strat.usersurvey}.
This survey was conducted in order to illuminate general user strategies and discuss aspects like interactions while evolving component systems.
The criteria a user judges solutions by are then discussed, the reasons and system objectives they fulfil are described.
Each criteria is mapped to the pseudo-Boolean function to be either minimised or maximised as discussed in chapter \ref{implementation}.
This chapter is finished with a brief discussion on how criteria can be composed to fulfil different user objectives. 

%%%The survey used to validate and refine the model
To describe and define the strategies that different users employ to evolve their component systems an online survey was conducted.
This survey targeted users of GNU/Linux distributions (specifically Ubuntu), and server administrators through the online forum reddit\footnote{http://reddit.com accessed 6/3/2012}.
In this context the way in which component systems are evolved is through package managers, as described in chapter \ref{background}.
It was completed by 59 users, who answered questions about themselves, the system they are evolving, and the way in which they evolve it.
It also involved the submission of package manager log files, which through analysis can provide greater detail of their behaviours.
These logs are not analysed here, but are used further discussed in chapter \ref{ubunutsimulation}.

In this section a brief description of the survey and an analysis of responses is given.
The full set of questions in the survey is presented in section \ref{apx.survey}. 

\subsection{Questions}
%%%The goals of the survey:
There are two types of questions in the survey; a set to identify the type of user, and a set to describe their interactions with package managers.

A set of questions used to identify the user are:
\begin{itemize}
  \item How experienced with package managers are you? 
  \item What system are you using?
  \item What package manager are you using?
\end{itemize}
These can be used to categorise the type of user and put into context the answers that they give.
They can also be used to weigh their answers for credibility and potentially exclude results from novice or inexperienced users. 

A set of questions asked about their use of package mangers are:
\begin{itemize}
  \item Describe your initial interaction with the package manager.
  \item Describe your day to day interactions with the package manager.
  \item What usage life cycle do you have with the package manager?
  \item Describe an unusual scenario in which the package manager was used.
\end{itemize}
The main focus of these questions is to identify the life-cycle of interactions that the user has when evolving a system.
The responses will provide reasons for these actions to occur, and illuminates the users objectives and strategies.

The survey finished with the question:
\begin{itemize}
  \item What other questions should I have asked?
\end{itemize}
This question tried to illicit comments and possible information that could of been gathered.
It also provided a small verification of the survey, as to see if anything was missing that should of been asked.

\subsection{Results}
%%%The results are summarised as:
The type of user that submitted the survey is important to be able to put in context their replies.
Most users (29/59) used the dpkg component model, either in Debian or Ubuntu systems.
The majority of these users used apt-get as their package manager.
The next largest component model used was ArchLinux\footnote{http://www.archlinux.org/ accessed 6/3/2012} (13/59),
all using the package manager pacman\footnote{http://www.archlinux.org/pacman/ accessed 6/3/2012}.
Other represented component models and package mangers include, Fedora with the rpm component model and the yum package manger,
Slackware with pkgtools, and Gentoo with portage.

The users mean experience was recorded as 3.9/5.
This is of course a subjective measure but describes the confidence in which the participants answered the questions. 

The life cycle questions have been summarised into the frequency of each action in table \ref{strat.tblaction}.

\begin{table}[htp]
\begin{tabular}{l | c | c | c | c |}
Request & Init. & Daily & Weekly & Monthly \\
Upgrade  & 45 & 27 & 16 & 0 \\
Install & 49 & 6 & 17 & 3 \\
Remove & 6 & 4 & 1 & 0\\
\end{tabular}
\caption{Summary of the frequency of user actions with package managers}
\label{strat.tblaction}
\end{table}

This table shows the amount of users from the survey who perform the actions of upgrading their system, installing or removing a package 
initially (when the system is first created), daily, weekly or monthly.
It shows that the vast majority of users when they create a system upgrade the system and install required packages.
This also shows that most users upgrade their system daily, and if they will likely install a package weekly.

The unusual scenarios which the user described using the package manager included 
11 of the 59 users stating that they had installed a set of packages to fulfil a specific task only to remove them the next day.
This is described by the users response 
\begin{quote}
``I sometimes install lots of packages very quickly as I need to accomplish a task, then remove them once that task has finished.''
\end{quote}

Also, 3 of the users stated that the package manager broke their system at some point, which then required to be fixed through the package manager.

In the final question asked if there is any more questions the general themes of responses include:
\begin{itemize}
  \item the mitigation of failed actions
  \item installing multiple versions of packages
  \item using a graphical user interface v.s. console based package manager
\end{itemize} 
The majority (37/59) users did not fill out this option in the survey, a potential indicator that the survey was complete.
One user even remarked to this question:
\begin{quote}
``Quite an all encompassing survey!''
\end{quote}

\subsection{Progressive vs. Conservative Users}
The main outcome of this survey was the identification of different user attitudes towards evolving a component system.
These attitudes have been used to define user stereotypes, which are described as either \textbf{conservative} or \textbf{progressive}.
These terms come from the domain of politics where conservatism is the philosophy that emphasises minimal and gradual change in society,
where progressivism promotes change and reform.
These ideologies can be caused because of a conflicting perception of risk; the risk of change introducing new problems, and the risk of not changing having old problems persist.

These ideologies are analogous to component system evolution where there is a force that wishes to reduce the risk to a system by not altering it,
and one to reduce risk by installing new versions of components.
The majority of users from the survey found some middle ground in which to mitigate both types of risk.

For example, a users response that express a slight conservative attitude:
\begin{quotation}
In production I rarely remove packages (easier to leave software as-is than risk breaking stuff).
\end{quotation}
That is they are more likely not to change the system, even though it may be beneficial as it may lead to a fault. 

Another example of a users response expressing a slight progressive attitude:
\begin{quotation}
I update my packages whenever I log in each day
\end{quotation}
Upgrading the system each day will ensure that no packages with fixed bugs will be installed, though may require significant change.

Both of these ideologies have extremes though, where users who try to eliminate all of a risk.

For example, a user's response in the survey with an extreme progressive attitude:
\begin{quotation}
I do run an unstable system all the time, I help mitigate this with some redundancy in my most frequently used components, 
using packages which perform the same function, but have different dependencies, since it's less likely to have multiple packages break at the same time. 
If something is rather buggy for me, I tend to update on a more frequent basis to check for the next stable point to jump into.
\end{quotation}
This user has components in their system that are so new as to not be fully tested, which are described as unstable packages.
To lower the risk that such packages have on the function of the system, this user has redundant functionality installed in their system to ensure that it can be used.
This users system will almost certainly be completely up to date, as any new component versions will be immediately integrated into their system.

An example of an extremely conservative user's response when asked about the frequency with which they interact with the package manager:
\begin{quotation}
As little as possible. I like build my box into whatever I'll need in the first couple of weeks after an install. 
Following the configuration and construction, only the occasional upgrade is necessary. 
Unless, of course, I receive a security notice about something.
\end{quotation}
So after the setting up of a system, this user will not evolve the system unless there is a direct security risk.
This lack of change will result in the system quickly becoming out of date.

%%%The reasons for keeping a system up-to-date; the fixing of bugs security, the adding of functionality, 
The reasons for these progressive attitudes of users is that maintaining an up-to-date system will eliminate known security exploits against the system,
and also introduce more efficient or new functionality into the system.
%%%The reasons for minimal change; through change unknown bugs could be introduced, the act of changing can create bugs, the effort of changing can consume resources  (e.g. network traffic).
Where the conservative attitudes come from the wither the perceived risk that unknown bugs could be introduced into the system, the act of changing a system can cause faults,
or the consumed resources (e.g. time, network traffic) are to expensive compared to the benefits.

Both concerns are valid assessments of the risk of component system evolution, the user must decide with respect to the system objectives what they require. 
For example, a server administrator that is running a mission critical system will likely change the system begrudgingly as any fault could cause massive damage.
However, a user running a desktop personal computer system may be more interested in new functionality than the up-time of the computer, so would change more readily.

\section{Conceptual Model}
In this section the ``conceptual model'' of the component system evolution process is presented.
The c the conceptual model is defined to represent \textbf{realistic} component system evolutions.

This relation to reality is the core difference between the conceptual model and the \modelname model.
For example, \modelname could describe a situation where a user may requests to upgrade their system every minute for a year.
This situation is clearly unrealistic, and not able to be represented by the conceptual model.

In this section, the conceptual models of the user and repository are described.
How an instance of the conceptual model can be used to create a realistic component system evolution, described by a CUDF* document, is also described.

\subsection{Conceptual User}
The important aspects to the conceptual user are the initial component system, the user requests to be made to the system, and the times at which these requests are made. 

Installing a component system is the first action a user takes during the CSE process.
This initial system is represented in the conceptual model as a CUDF* document.

In CUDF* three different actions are defined that a user can request to alter their system; installing, removing or upgrading a component.
However, as shown in the user survey presented in chapter \ref{strategies} the two main actions of users are the installation of a component, and the upgrading of the entire system.

The times at which the user requests these changes is also important.
For example, a user will typically not request to upgrade their system every minute of the day as this would be seen as superfluous.
Reasonably, if an action occurs its max

%%%We have limited it to two actions, install and update, as they are the core actions a user executes
The variables that make up the conceptual user are: 
\begin{itemize}
  \item the initial system as a CUDF* document
  \item the probability a user selects to update the system per day
  \item the probability a user selects to install a component per day
  \item the probability a component is selected to be installed
  \item the MOF criteria used to select an optimal system for an update request
  \item the MOF criteria used to select an optimal system for an install request
  \item The number of days they exist
\end{itemize}

The initial system the user installs is also modelled here, as this is the starting point of any evolution.

\subsubsection{Install}
A user typically requests to install a component to extend the functionality of their component system.
How often does a user requests to install a package, and what packages is the user likely to request to install are interesting questions. 
simulating, the information of what component a user will install and when will a user select to install it are needed. 

%%%What probability a package will be selected to be installed
Determining the probability a user will select any component to be installed is difficult given the enormous amount of factors this relies on.
The users job, location, current tasks, previously installed software, favourite colour and numerous other aspects can determine what the user will select to install.
Given that all this information is impractical to simulate, this problem is abstracted into the form of two questions;
\begin{itemize}
  \item What components may a user select to install?
  \item How likely would a user select to install any of these?
\end{itemize}
This is represented by a weighted list of components, where the weights represent popularity;
e.g. component ``A'' has a 10\% chance of being selected as the component to install, or component ``B'' has a 5\% chance.

%%%The core problem with this is correlation between packages and user
The core problem with this weighting of components is that it ignores the correlation between probability a component will be installed. 
Two or more components may complement their functionality, thus making their installation together more likely.
This is impossible to represent with a weighted list, however this type of information is very difficult to calculate.
It would require the analysis of a significant amount of systems, which would probably cost more than the benefits gained.

The next question is what is the probability that a user will install a component, and how many components?
This is represented by a probability for the amount of components that a user may install on a day,
e.g. they have a 80\% chance of installing nothing, a 15\% chance of installing one component.

\subsubsection{Initial System}
%%%The initial system to start from is important, typically there are many to choose from
The starting system will effect how the system is evolved.
Therefore, selecting an initial system to start evolving is an important issue.

%%%What initial systems are there to chose from
Many component frameworks will release various configurations of components that satisfy many use cases.
This is analogous to software product lines \citep{clements2001software}, which increases the re-use of components and satisfy many different users with minimal effort.
The Eclipse framework offers more than ten types of initial Eclipse installs for different users.

The Ubuntu distribution offers three different systems for use, server, desktop and alternative.
Each of these also includes the choice of either amd64 or i386 chipsets.

The selection of the initial system can depend on many aspects of the simulation, which should be considered when defining the questions to be asked.


\subsection{Repository Model}
The set of components that exist at a given time.

\subsubsection{Time Range}
%%%The range of time in which to look at
The range of time over which the simulation is run will determine the repositories that are required.
It is also important as it should be long enough to draw conclusions from,
but as the simulation can take considerable resources to execute, too long and it may make it impractical to execute all the iterations necessary.
Other external aspects such as policy changes in the way in which a repository is run, or release cycles of the component system, 
must be considered when defining the time range.
These may have an effect on the results therefore should be considered in the conclusions drawn.

\subsubsection{Variables}

%%%this model contains a record of packages in repository over time
This model then contains one set of information:
\begin{enumerate}
  \item A daily record of components stored in the repository
  \item A time frame, start and finish, over which the simulation is run
\end{enumerate}

\subsection{Configuration}
The simulation configuration is a set of variables in which the simulation can be altered through.
To treat these variables as a set of inputs, their format and constraints must be defined.

The format for the initial system and criteria have been defined in previously chapters.
The initial system and the repository information will be store in the CUDF format, as described in chapter \ref{background}. 
The criteria to update and install will be defined in our modified MANCOOSI format defined in chapter \ref{criteria}.

Some inputs are trivially defined;
the update probability is a number between 0 and 1;
the time frame is defined as the dates of the days (given in seconds since the epoch at Jan. 1st 1970) between a given start date and the number of days that the simulation should run for.

These final inputs, the probability distributions of whether a component will be selected to be installed and how many components a user will select to install on a given day,
are represented using a set of pairs, each pair containing the action and the probability the action will be taken.



\subsection{Processes}
These processes are attempts to abstract and simplify the reality of the component system evolution.
The process that simulates the evolution of a component system takes a configuration and first generates a set of ``user actions" a user may take.
These user actions are then iteratively applied to the initial system though generating a CUDF problem,
then passing it to the resolver with the appropriate criteria to find the resulting solution.

The results of this process is then a set of systems that are created through the user actions.
These will be different along the dimensions of the configuration.



\paragraph{Initialisation}
%%%Initialize the algorithm
The \verb+generateUserActions+ algorithm first initializes three variables, \verb+userActions+, \verb+keeps+ and \verb+days+.
The \verb+userActions+ list is returned at the end of the algorithm.

The variable \verb+keeps+ is an initially empty list, that is contains all the components the user has previously selected to install.
This list is later translated into a ``keep: package'' property of the installed component with defined semantics in the CUDF specification.
This ensures that after a component has been selected to be installed by a user a component is not later removed by an action, like an update.




\paragraph{Differences to Reality}
The core differences from this model of a user to a real user are life cycle actions and correlation between actions.

A user when first installing a system will likely have components that they require that are not in the default system, and therefore when starting will install a lot of components.
This type of reasoning leads to life cycle actions, actions that are caused by events in the life cycle of the system.
Another example of this may be that at the end of each month a user decides to look through what they have installed and remove unneeded packages.
These actions add an extra dimension to the user model and process and will clearly change some results.
However, it has been opted out of the simulation as a precise set of possible actions and associated parameters are difficult to define.
Also as stated in their name they occur infrequently, if at all, and therefore it has been decided that they may not be significant.  

One action may also impact the occurrence of another action, so that they are correlated to happen together.
For example, a user may select to always update before they install, or to only remove a component directly after it has been installed.
The complexity that can be introduced by these correlations could make this simulation significantly more complex.
As these values may be different for each type of user and is probably more related to the person than the strategy they employ these have been deemed 



\section{Simulation Validation}
%%%Validation of this simulation, what needs to be validated/why it should be validated
The conceptual model presented is a simplified abstraction of the reality in which users evolve component systems.
It describes the variables that effect component system evolution as a configuration,
and the processors used to execute the simulation given a configuration.

%%%What if it is wrong
If some significant aspect of the system was missed, or if some aspect was incorrectly defined, the simulation may produce results that are incomprehensible,
or worse, misunderstood.
Therefore, the validation of these artifacts is essential to move forward. 
This validation was accomplished though regular stakeholder meetings, and an online survey with subject matter experts (as described in chapter \ref{strategies}).

\subsection{Stakeholder Meetings}
%%%Weekly meetings with stakeholders (i.e. supervisors)
As described in the methodology; one effort to validate these artifacts is done through meetings and a structured walk-through with the core stakeholders.
In this simulation the core stakeholders are the project researcher and supervisors.
These are the people who are asking the questions and are also impacted by the outcome, therefore they are directly effected by the validity of the results.
Meetings where held at regular intervals to ensure the projects progress and direction where correct.

\subsection{Subject Matter Expert Survey}
%%%Results from what else should be asked, install stuff not from repository, installs break
The survey described in chapter \ref{strategies}, 
was conducted at a point in the project when the conceptual model was just being developed, so had considerable impact on these artifacts.
The questions asked in this survey helped gauge the necessity and frequency of user actions,
so that only the most important aspects of the problem can be selected to be simulated.
It also filled in gaps of what was missing from the survey and model, giving direction for exploration.

\subsubsection{Frequency of User Actions}
%%%How often do users do these actions
The more frequently a user selects an action to evolve their system, the more important it is to the evolution of their system.
The information gained from the survey provides confidence that not including the action to remove a component and abstract the requirement multiple repositories, 
would not damage the validity of the results.
It also helped us define and represent the update and install actions in the configuration.

The remove action can be ignored as it seemed many of the users do not use this frequently.
When they do select to remove a component it is usually directly after installing it, if the selection to install a component was seen as a mistake.
Although it is clearly an important function to be included when evolving a system, the assumption is made that it is unnecessary for this simulation.

This survey also clearly shows that the main actions of a user is to update their system,
with the installation of a component the second most used action.
This survey also showed that the update occurs at more regular intervals than the install components.
Therefore, only the update and install user actions were included in this simulation and it also defined their representation. 

\subsubsection{Component Fault Exploration}
Another aspect of this problem that was mentioned by the subject matter experts through the survey was that a system may break during a change.
This typically then requires a reversion of the system components to a previously stable state.
Simulating this effect was ultimately deemed outside the scope of the project as it is seen as a rare occurrence with modern systems.
However, before it was eliminated, it was explored for possible inclusion.

The core problem with including simulated faults in a component systems evolution, is that each component has a different likelihood of causing a fault.
Different properties of a component like development process or complexity can impact this value.
Therefore, the function to calculate the likelihood of a component failing could rely on many different properties.

Instead of creating a function that tried to calculate the probability of failure per component,
an effort to measure it was attempted through a small study of component bug reports was attempted.

The feedback generated when a component causes a fault should ideally be a bug report, for the Ubuntu distribution these are filed on the project hosting server LaunchPad.
By using the Launchpad API to extract bug information, the amount of bugs per component in the system was able to be measured.
It was initially assumed that the number of systems that a component was installed on would increase the number of bug reports generated,
as more users means more eyes and systems to find bugs.
The Ubuntu Debian popularity contest was used then to see if this relationship existed the graph in figure \ref{bugsvspop} was created.

\begin{figure}[htp]
\begin{center}
  \includegraphics[width=\textwidth]{simulationpics/bugsvspopularity}
  \caption[Bugs v.s. Popularity]{A plot of the bugs a package has compared to its popularity, with notable outliers labeled}
  \label{bugsvspop}
\end{center}
\end{figure}

The first thing to note is that there seems to be very little relationship between the two variables, other than the package with the most bug reports is also one of the most used components.
The second thing to note is that the packages with the largest amount of bugs are ``apt'' and ``aptitude'' the two most popular package managers.
This could be because those packages are very buggy, or it could be because problems caused by apt, e.g. trying and failing to install a faulty package, may be reported as a problem with apt.

The last thing to notice is that there are many less popular components that have many bug reports.
When identifying the purposes of these packages many are used by developers, e.g. emacs21 is a popular text editor to program in.
The reason for their increased amount of bug reports may be that the users have prior experience and appreciation for the bug reports and the maintenance process, so file more bugs.

Measuring a components likely hood of failure using bug reports is then hypothesised to be impractical if not impossible,
as a component purpose and a components users may affect the results of the best measurement method available.
Other methods of finding a components likelihood of failure have not been further explored since this variable was eliminated from the configuration.
Though, it is expected that this is an intractable problem that is likely impossible to simulate accurately.

\subsection{Further Validation}
The assignment of the configuration variables is a different stage in the validation of this simulation.
Clearly if you create a configuration that is completely unrealistic, the saying ``garbage in, garbage out'' applies to the results.
However, this is not a concern when validating the conceptual model, or the abstract processes.
Further discussion of the validation of the assignment of the configuration variables is in chapter \ref{ubunutsimulation}.

\section{Summary}
{}In this chapter possible options were discussed for studying various strategies employed when evolving component systems.
{}Simulation, through the methodology \citep{Law2005} describes, was selected, and the steps involved were described.
{}The central artifact of this methodology, the conceptual model, was broken down into models of the user, repository and solver, and the processes of simulation.
{}These models were validated through regular meetings with the core stakeholders, and a survey conducted on subject matter experts.
{}In the next chapter the configuration of the simulation is further defined, and the questions about component system evolution are attempted to be answered.


\chapter{Conclusion}
\label{conclusion}
%%%Answers to our initial questions
There are an estimated 20 million Ubuntu GNU/Linux systems users\footnote{http://www.ubuntu.com/, accessed 16/5/2012}.
These users are repeatedly making changes to their systems by installing new components and upgrading components to newer versions.
This causes their systems to evolve over time.

This research investigates the evolution of such component systems using a developed simulation.
To build this simulation, first the background of software evolution, component-based software engineering and component systems were explored (chapter \ref{background}).
The models \modelname (chapter \ref{formal}) and \usermodel (chapter \ref{simulation}) were then developed 
to describe a component system evolving through repetitive change and the user that requests the changes.
The implementation GJSolver (chapter \ref{implementation}) was then created to calculate the exact changes made to a system during evolution.
The verification and validation of these artifacts were discussed, in accordance with our methodology \cite{Law2005}, 
as this is the primary concern when creating a simulation.

Experiments were defined that use these artifacts to simulate the evolution of many different systems.
The primary focus of these experiments were the out-of-dateness and changes made to the systems during evolution.
Some discoveries were:
\begin{itemize}
  \item Upgrading a system causes the amjority of change 
  and installing new comopnents only adds to this change by increasing the number of comopnents to be upgraded.
  \item Upgrading frequently during most of the year is unnecessary as it has little impact on the out-of-dateness of systems.
  However, during periods where new component versions are released quickly, e.g. during Ubuntu releases, 
  it is important to upgrade frequently. 
  \item Regular updates cannot ensure that a component system stays up-to-date.
  This is in part due to the criteria used by \texttt{apt-get} that restricts adding new components when upgrading.
  \item Frequently upgrading a system may cause additional change 
  when many versions of a comopnent are released quickly, forcing it to be upgraded multiple times.
\end{itemize} 

This research also provides empirical evidence and quantitative measurements of reuses benefits, especially during CSE.
The main benefit during CSE is the reduced change due to lowering the amount of components to be installed,
that in turn lowers the amount of components that are upgraded.
The benefits reuse has when using software components has been explored before in \cite{Szyperski2002}.

To reduce the out-of-dateness of systems during evolution the progressive criteria was proposed.
This criteria allows new components to be installed to enable the upgrading of components.
This alteration was shown to be most effective for users that upgrade frequently.
It was shown to decrease the out-of-dateness of a system by as much as 49\% at the cost of increasing change by 34\%.
For users that prefer up-to-date systems, this trade-off might be favourable for their system.

To reduce the change made to a system during CSE the unstable criterion was proposed.
This criterion waits for a period of days for a component to cease releasing newer version and become stable.
By using this criterion and waiting a week for a component to become stable before upgrading it, it was estimated to save some types of users more than 30 changes over a year.
These changes could introduce bugs into the system and are seen as unnecessary. 
Using this criterion comes at the cost of the system always being out-of-date by the number of days that are waited for components to become stable.
For users that prefer to reduce change to their systems, this may be a tolerable trade-off.

The remainder of this chapter will examine the proposed thesis in chapter \ref{introduction},
the potential future research,
and provide a closing remark.
\section{Thesis Validation}
This research started with the thesis:
\begin{quote}
\textit{It is possible to reduce the negative effects of component system evolution by altering the mechanisms by which systems are changed.} 
\end{quote}

The steps taken to validate this thesis were:
\begin{itemize}
  \item To develop a reproducible and controllable environment in which to measure the effects of CSE.
  \item To use this environment to study how systems evolve.
  \item To alter the mechanisms by which systems are changed and study their impact on CSE.
  \item To demonstrate a reduction on change and out-of-dateness using such alterations.
\end{itemize}

To address these steps:
\begin{itemize}
  \item A simulation was developed.
  \item It was used to study CSE through simulating the evolution of many Ubuntu systems.
  \item The novel unstable criterion and progressive criteria were proposed and their impacts on CSE studied.
  \item The unstable criterion and progressive criteria were then simulated to approximate real CSE
  and demonstrated to reduce change and out-of-dateness.
\end{itemize}  

It is therefore possible to conclude that the thesis is supported.
 
\section{Future Research}
\label{conclusion.futurework}
The largest area of future research is into the generalisation of the results of this thesis.
This is where additional component systems, such as Eclipse, can be studied as they evolve. 
Through such generalisation the conclusions of this research can be shown to be invariable between component systems or be Ubuntu specific.
The core hurdle of accomplishing this is that much of the collected information in this study does not exist for other component models.
Component models like Eclipse do not have their history archived as precisely as Ubuntu.
Information, like the Ubuntu repository, does not exist or does not contain the detailed information for other component models.
Resources of information are available, but can be incomplete.
For example, the Usage Data Collector\footnote{http://www.eclipse.org/epp/usagedata/faq.php} for the Eclipse project (similar to the Ubuntu Popcon) collected information on popular Eclipse components. 
However, this project was shut down as it was deemed not to have a significant return on investment. 
 
Another area of research involves altering and extending the \modelname and \usermodel models to more precisely describe CSE.
An aspect of CSE that is not included in \modelname is that a component system may only view a certain subset of components that exist.
This is described in section \ref{sim.modelvalidation} as the view Ubuntu has of its repository.
The \usermodel has many aspects that could be added to make the simulated users closer to reality.
Constraints such as only upgrading during the work week, 
or additional types of request like removal of components that are no longer required, could create more valid simulations.

The development of criteria that use other properties of components, 
could be used to more accuratly represent the users preferences.
Such properties could be collected from additional sources
, e.g. component meta-data, component contracts \citep{Watkins1999}, repositories \citep{Guo2000}, 
composition testing \citep{XuejieZhang2008}.
These could be used to create criteria that optimise for size, reliability, licence, even validity.
Such criteria could be simulated and their effect, positive or negative, 
could be measured using the simulation developed in this research.

%%%Further stufy of the interaction between users and their component systems. We found it hard to find information, maybe some research would be interesting,
%%%which we could include to make the simulations more valid.
One of the least valid parts of the simulation is the selection of the requests made by the user.
This is both when they select to upgrade, when they select to install, and what components they select to install.
By studying real users changing their systems, either through collecting more \texttt{apt-get} logs or through direct observation,
a more valid simulation could be created.

%%%Integration with MPM then test on real systems the implications of this research
Comparing the results from the simulations to the evolution of real systems would be able to confirm the conclusions drawn from the simulation.
This could be accomplished using the MPM solver \citep{abate2011} from Mancoosi.
This solver has already been proposed for the modification of real systems.
Through modification or extension of MPM, the proposed criteria could be used to evolve real Ubuntu systems.
Such a study would require significant resources, but its results and conclusions would be valid.

\section{Closing Remark}
Understanding how component systems evolve can inform users and developers about the effects of their choices.
It can also provide new methods to change systems that better suit a user's requirements.
This research has proposed novel ways to reduce negative effects during CSE and the tools in which to measure the effectiveness of these proposals.
 



\chapter{Appendix}

\section{Glossary}
MISC
CUDF
Software Component
Component System Evolution
Component Evolution
Component Model
Strategy
install
remove
upgrade

Composer: The user who creates or alters component systems (compositions)

Versioning Model: A method to define and compare versions

\section{Survey}
\label{apx.survey}

\section{CUDF Examples}

\section{Code Samples}
\subsection{Scripts for Simulation}
\subsection{SAT Algorithms}

{}\bibliographystyle{plain}
{}\bibliography{/home/gdjenson/PhD/References/Bibtex/--Thesis--}

\end{document}
