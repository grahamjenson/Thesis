%%This is a very basic article template.
%%There is just one section and two subsections.
\documentclass{report}

\title{Component Dependency Resolution}

\begin{document}

\chapter{Introduction}


\begin{quotation}
\textbf{Man}: When you take apart a Lego house and mix the pieces into the bin, where does the house go?\\
\textbf{Woman}: It's in the bin.\\
\textbf{Man}: No, those are just pieces. They could become spaceships or trains. 
The house was just an arrangement. The arrangement doesn't stay with the pieces and it doesn't go anywhere else. It's just gone\\
--XKCD Lego
\end{quotation}

%%%The introduction leads to the idea that when building a complex system, breaking the probelm into parts is a natural instinct
{}When confronted with building a complex system, our first instinct in an attempt to understand and build the system is often to break it into smaller, simpler components. 
{}These components are defined by their goals in the overall system and their relationships to other components.
{}Through this ``divide and conquer'' method, the design and construction of a complex system is broken down into a manageable problems to solve.

%%%We use the metaphor of building a car to explain the benefits of breaking systems into components
When building a complex system like a car, you would not attempt to build the car as a whole, but break it into parts for design and construction.
A car requires a body, electrical system, interior, suspension and steering, engine \ldots. 
The engine can be further broken into other parts like cooling, oil system, exhaust and intake systems, fuel, and so on.
Each of these parts have relationships that lead to complex dependencies, for instance the carburetor blends the air from the intake with petrol from the fuel system.

%%%Generally we break apart systems for the benefits gained during design, implementation, maintenance and evolution. 
The benefits of breaking a system apart are gained through the entire product process; design, implementation, deployment, maintenance and evolution.

%%%During Design, planning, understanding the big picture,

%%%During implementation, specialization, concurent development, testing each part separtly

%%%During maintenance with well defined interactions a part can be replace without side effects, 
%%%or a problem can be detected and solved given its location in the system 
The apocryphal story of George Washington's axe which has three times had its handle replaces and twice its head replaced deomstrates the power of maintence on a system of parts.

%%%During Evolution, replacing an inferrior part with a superior part can be accomplished as long as it fits the required specifications

\section{Software Components}
{}There are more reasons for the use of components in software systems, additional benifits over their use in physical systems.
{}Software components can be reused, replaced ``on the fly" without interruption to the system, validated for correctness, 
{}and automatically building or modifying a system through resolving their defined dependencies.

%%%The reuse of services supplied by software components is a key driving force for use
In a physical car there are pumps which accomplish many different tasks, in a software system such replication is uneccesary so a single component can provide the servie to pump.

%%%Being able to change components while running the system is difficult but possible in software systems.
Replacing a tire or repairing a windshield while continueing to drive the car is impossible, but inside a software system it can be done.

%%%Given the software component relationsships are well defined, with contracts and requirements a system or component can be validated to be correct before use
If an enthusiast adds a turbo to a car, it may not work correctly in that context, it may create too much pressure and blow the engine.
A component can have complex mechanisms like contracts or strictly defined specifications to automatically detect these problems.

%%%Given the relationships in a software component system are strictly defined, we can use these to then find valid combinations given user requirements
Imagine being able to define the car you want by describing the functionality you require, and then a custom car, one that may exist no where else in the world,
is automatically built for you. 
While then driving this car your requirements change, then the car changes parts to adapt to your new requirements.
This is the power that dependency resolution offers component systems, automatic resolution of a users requirements.

\section{Research Goals}
{}The part of this vast research area we look at is the final point made in the previous section, 
{}the automatic composition of parts through resolving component dependencies and it's effects on a component system.


\section{Overview of Thesis}
%%%This section just has an outline, and breif description of each of the chapters.

\chapter{Background}
\label{background}
\epigraph{
In order to agree to talk, we just have to agree we are talking about roughly the same thing.
}
{\textit{The Feynman Lectures on Physics, Motion, Richard Feynman, 1961.}}

%%%What is Software Evolution
Software evolution \citep{lehman1980} is the process of change made to a software system to maintain it, or to extend its functionality over the systems lifetime.
This evolution is a necessary process as the system must adapt to the changing software environment, accommodate new user requirements, 
fix errors, and/or prevent errors from occurring in the future \citep{IsoIec2006}.
Software maintenance and evolution are often used interchangeably \citep{Godfrey2008}, though some semantic differences exist.
One difference is that maintenance has connotations of a planned activity, where evolution is the gradual refinement of a system \cite{lehman1980}.
Throughout this research software maintenance is refereed to as a process through which software is evolved, this is similar to the .

The concept of software evolution has been discussed since the 1960's \citep{Lehman1969}, 
however some still see it at a young field \citep{Godfrey2008} as many open questions left to be answered.

%%%Ways to mitigate evolution costs are
One such open question is: \textit{how can software engineers lower the cost of software evolution?}
The cost of evolving a system was estimated by \cite{Brooks1975} to be as high as 90\% the total cost of the system.
More empirical studies that explore the cost of software evolution are summarised by \cite{Grubb2003}, these show the costs of software maintenance are ranging from 49\% to 75\%, 
and that these have not fallen since the 1970's. 

To lower the cost of software evolution various methods and tools have been proposed.
For example, agile software development \citep{beck2001manifesto} methodologies are defined to encourage rapid and flexible responses to change,
and refactoring tools \citep{fowler1999refactoring} have been developed to restructure code to decrease complexity and increase maintainability. 
Another way to lower costs of the software evolution process is through creating systems from encapsulated units, called software components \citep{Szyperski2002}.

%%%Composition
A component system is created out of a set of components that are combined into a functioning system by a composer (or assembler) \citep{Szyperski2002}.
Composer are also the ``evolvers'' of a system, as they have the power to alter the composition of components that make the system.
Originally it was assumed that developers were the composers \citep{Parnas1972,PrietoDiaz1987}, where they would compose, verify, then release a system.
However, recently the user has started to fill the composer role, where they compose their own systems to satisfy their requirements.

The situation where the user is the composer of the system has been called system tailoring \citep{Morch1997} and end-user assembly \citep{Szyperski2002}. 
It has been noted that the potential, and fragility, of component systems is increased with this user composition \citep{Szyperski2002}.
A system has more potential as the user can craft solutions without requiring expert assistance,
and the system is more fragile as the quality of a system cannot be verified as the user does not have the tools or knowledge required.
This means that the system is tested by the user running it. 

The decoupling of the developer and composer roles also breaks the developer's control over the evolution of the component system.
The developer is no longer in charge of evolving the deployed system.
This means, a component can be composed into a system without prior testing to check if that specific system will work.
This leaves the evolution of the component system in the hands of its user.

If the user is to be the composer, and they are not a technical expert, then there must be guidance when composing or evolving a component system \citep{Szyperski2002}.
The core technical knowledge required to compose a component system comes from the complex relationships between components.
These relationships create constraints that must be satisfied in order for the systems components to be functional.
The guidance to satisfy these constraints comes in the form of a process called in this research Component Dependency Resolution (CDR).
CDR is  a tool that takes a user's request to change the composition of their system, and attempts to create a functioning system 
that satisfies all component relationship constraints and the users request.

%%%Defining components
How ``software component'' is defined will impact how component system evolution and CDR are modeled.
It is difficult to find a precise definition of what a software component is, as the intuitive concept may be quite different from any model or implementation \citep{Crnkovic2011}.
Finding a definition that satisfies every party may be an impossible task.
However, by defining a software component only with respect to component system evolution and CDR, 
this process can be studied without the paralysis of finding a complete definition.
In this research a software component is defined to have explicitly declared relationships and with mechanisms to automatically alter the components composition.
These two attributes allow CDR to evolve a component system and component system evolution to be studied. 

%%%Examples of component models
With respect to this definition various component models can be described.
These component models can differ by functionality, platform, or their objectives for being created.
They may also have their own CDR implementations, created to assist in their component systems evolution. 

%%%In this chapter\ldots
In this chapter, the histories of software evolution and component-based software engineering are examined in section \ref{background.histories}, to create the present day picture.
In the next section \ref{background.componentsystemevolution}, component evolution and component system evolution are discussed.
This section includes descriptions of component dependency resolution and the evolution strategies of users when evolving their systems.
The definition of software components and component models with respect to component system evolution is then presented in section \ref{background.components}.
To conclude this chapter  examples of various component models and CDR implementation are then given in section \ref{background.models}.
This includes an introduction to the Debian model, and CUDF model that are described in more detail in the following chapters.

\section{Software Evolution and Component-Based Software Engineering}
\label{background.histories}
The foundations for software evolution, software components, and component systems can all be traced to the year 1969.
Software evolution, the process of changing a system to adapt the environment it exists in, 
was first described by \cite{Lehman1969}.\footnote{though it was not refereed to as evolution at that time}
\cite{McIlroy1969} proposed his concept of software components that were created by a ``subindustry'' of suppliers.
Furthermore, that year the the operating system Unix \citep{raymond2003art} was developed, which at its core philosophy is a modular system.

The domains of software engineering, software evolution and component-based software engineering, have seen massive advancement since their inceptions in 1969.
In this section these advancements are briefly described, leading up to the descriptions of component evolution, and component system evolution.

\subsection{Software Evolution}
\cite{Brooks1975} stated that over 90\% of the cost of a system occurs after deployment in the maintenance phase,
and that any successful piece of software will inevitably be maintained.
This realisation, that software requires significant expense to maintain, lead researches to study how software behaved after deployment.
In this research software maintenance and software evolution are synonyms, though some consider them different processes \citep{Lehman2006,Godfrey2008}.

In 1980 two of the fundamental empirical studies on the emerging domain of software evolution where published.
The first study by \cite{Lientz1980} explored the activities that occur during software maintenance (later formalised in \citep{IsoIec2006});
\begin{enumerate}
	\item \textit{Adaptive Maintenance}: adapting to new system or technical requirements.
	\item \textit{Perfective Maintenance}: adapting to new user requirements.
	\item \textit{Corrective Maintenance}: fixing errors and bugs.
	\item \textit{Preventive Maintenance}\footnote{later added in taxonomies such as \citep{iee1990ieee}}: adapting to prevent future problems.
\end{enumerate} 

In this study, it was shown that around 75\% of the maintenance effort was on the first two types, and corrective maintenance took about 21\% of the effort.

The second study by \cite{lehman1980} explored how the evolution effected the software.
He discovered a set of laws, similar to the laws of nature like gravity, that characterise software evolution:
\begin{enumerate}
  \item \textit{Continuing Change:} Systems\footnote{E-type systems: software implemented in a real-world computing context} must be continually adapted 
  otherwise, they become progressively less satisfactory.
  \item \textit{Increasing Complexity:} As the system evolves its complexity increases unless work is done to reduce it.
  \item \textit{Self Regulation:} The system evolves with statistically determinable trends and invariences.
  \item \textit{Conservation of Organisational Stability:} The average effective activity rate to evolve a system is invariant over its lifetime.
  \item \textit{Conservation of Familiarity:} As the system evolves, its incremental growth remains invariant to ensure users maintain mastery over the system.
  \item \textit{Continuing Growth:} The system must continually grow to maintain user satisfaction.
  \item \textit{Declining Quality:} The quality of the system will decline unless rigorously maintained.
  \item \textit{Feedback System:} The function a system performs is changed by the effect it has on its environment.
\end{enumerate}

%%%These laws describe the fact that it is impossible to create a single satisfactory system, and an iterative approach is best to strive for continual satisfaction
Both the activities from \cite{Lientz1980} and the laws from \citep{lehman1980} show the view that
the software engineer's objective of creating a satisfactory system is difficult, expensive, and not often achievable.
Therefore, the continual evolution of a software system is necessary, and this evolution will reduces quality, increase complexity, and be costly.

From the perspective of software evolution, the software engineer's goal is then to create a system that can be quickly altered to adapt to a changing environment, 
while working to reduce the inevitable complexity caused by this change.
Towards such goals iterative development processes have been created, such as the spiral development method presented in \citep{Boehm1988}.
This process describes the stages of development as communication, planning, modeling, construction, and deployment. 
These stages are continually iterated until the software project reaches the end of its life. 

The practical problems of software evolution can be seen in the struggle with legacy software \citep{Bennett1995}.
Legacy software is functional software that is old and outdated, but due to its critical status, it not being well understood, or the cost to redesign it, 
it has not been replaced.
A piece of software is described as ``legacy'' if it cannot be evolved  (due to its complexity or size) within an acceptable cost \citep{Bisbal1999}.
As a legacy system cannot be evolved, new user and technical requirements cannot be fulfilled, and the satisfaction with the system will decrease over time.
This inevitable consequence of software evolution has lead the problem of legacy software to be described as enduring \citep{Bennett2000}. 

Current explorations of the past and state-of-the-art in software evolution are presented in \citep{Bennett2000}, \citep{Lehman2003}, and \citep{Godfrey2008}.
In all these papers, the importance of software evolution is stressed, and the need for more knowledge about the evolution process and its properties is discussed.

\subsection{Component-Based Software Engineering}
The concept of Component-Based Software Engineering (CBSE) was first outlined by \cite{McIlroy1969}, 
by describing the idea of a software components subindustry, which created components to be used to in software.
This report is an expansion on an earlier idea for \textit{pipes}, presented in \citep{mcilroy1964pipes}, 
where he described having software fit together, like screwing a hose to a tap.

Originally, using software components had one major perceived benefit, the reuse of code in order to decrease development time.
Later, other benefits of constructing systems from modular components where identified by \cite{Parnas1972}:
\begin{itemize}
  \item \textit{Managerial}: develop components in separate groups with little communication
  \item \textit{Product Flexibility}: make drastic changes to one component, without changing others
  \item \textit{Comprehensibility}: be able to study the system one module at a time
\end{itemize}

The software component concept was soon picked up by other researchers such as \cite{Yourdon1976}, where he described his ideas on structured design as:
\begin{itemize}
  \item the art of designing the components of a system and the interrelationship between those components in the best possible way.
  \item the process of deciding which components are interconnected in which way to solve some well-specified problem.
\end{itemize}
\cite{Yourdon1976} lists the goals of such a design as efficiency, maintainability, modifiability, generality, flexibility, and utility.
These goals are aimed to be achieved by dividing the system into functional units that can be treated independently.
Each unit corresponds to exactly one small, well-defined piece of the system, and the units relationship corresponds to a relationship between pieces of the system.

A problem soon emerged when using software components, that of which composition of components to use.
\cite{PrietoDiaz1987} describe this as the \textbf{selection problem}, where a composer can have many alternative compositions of components to select from.
This increases the effort required to use software components, as each possible combination must be examined and ranked based on how well they match the composers specifications.
How to describe a component and its attributes \citep{treinen2009common,Xinjuan2007},
how to search for a compositions fits a set of requirements \citep{abate2011,Kwong2010,treinen2009,DeAlmeida2004}, 
and how to rank a particular composition \citep{Chen2011,Aleti2009} are still themes of current research.

A current perspective on component-based software engineering (CBSE) is that software components main driving force is increasing the evolvability of a system \citep{Szyperski2000}.
However, the concept of a software component subindustry as originally described by \cite{McIlroy1969}, is seen as never truly come to fruition \cite{Szyperski2002}.
A possibility for this is the unsure definitions of what a software components are \citep{Crnkovic2011}.
The current state of CBSE is fractured, where there are many different component frameworks (some presented in section \ref{background.models}),
each with different goals and attributes.
The hope for CBSE, as described by \cite{Crnkovic2011}, is that the technology and research will converge, and terms and concepts in the software component domain will become standardised.
This problem is later described in section \ref{background.components}, where the software component definition in this research is discussed.

\subsubsection{Unix and GNU/Linux Modular Operating Systems}
The research into software components by \cite{McIlroy1969} coincided with his help in the development of the operating system Unix \citep{raymond2003art}. 
McIlroy had significant impact not only on the implementation of Unix, where many of his ideas like pipes where included,
but also on Unix's philosophy. 
The Unix philosophy is summarised by McIlroy as \citep{Salus1994}:
\begin{quote}
Write programs that do one thing and do it well. Write programs to work together. 
Write programs to handle text streams, because that is a universal interface.
\end{quote}

This philosophy lead to the first two rules of Unix, as listed in \citep{raymond2003art}:
\begin{itemize}
  \item \textit{Rule of Modularity}: Write simple parts connected by clean interfaces.
  \item \textit{Rule of Composition}: Design programs to be connected to other programs.
\end{itemize} 

To eliminate the perceived problems of the proprietary Unix system, Richard Stallman, in 1983, created the GNU project \citep{stallman1985gnu} in order to create a free Unix-like operating system.
With a kernel developed by Linus Torvalds based on the MINIX \citep{tanenbaum1989minix} operating system, the GNU/Linux \citep{torvalds2002just} operating system was created.
GNU/Linux is seen as a return to the original philosophy of Unix by \cite{Gancarz2003}, where the creation of small modular programs that interact is central.
Aligned with this philosophy, a distribution of GNU/Linux called Debian \citep{Barth2005} was announced in 1993.
This release came with the Debian Manifesto \citep{murdock1994brief} 
that stated that Debian would be constructed from high quality components (or packages) which can be maintained by experts.

Initially, a significant amount of technical expertise was required for a user to be a composer of packages to create or change a Debian system.
This was because the selection problem (as discussed above) was left to be resolved by the user.
If a user wanted to build or change a Debian system, they would have to find, select, then install all the packages they wanted in the system.
The user would also have to make sure that all the constraints caused by the relationships between components were satisfied.

With the release of the application apt-get, the necessary technical knowledge for the user to compose a Debian system was significantly reduced.
With this application, the wanted change to the system could be stated, and finding, selecting and installing a solution was automated.
This is one of the first CDR implementations to enable the the user to become the composer of a component system.

\section{Component Evolution vs. Component System Evolution}
\label{background.componentsystemevolution}
%%%Evolution of a component system is made up of two cycles, the component evolution and system evolution.
In this section, the differences between the evolution of components and the evolution of component systems are discussed.

\subsection{Component Evolution}
When evolving an individual component, the encapsulation from other components and their composition into a system must be considered.

Component encapsulation allows the simultaneous and separate development of components.
As a developer of a component may not have control over the development of related components, each component may have different evolution cycles.
Therefore the evolution of component must also consider evolution of components it relates to.
For example, if a component $a$ is changed, and component $b$ uses $a$'s functionality, this change may effect $b$.
As the developers of $a$ do not have control over the development of $b$, the changes must be discussed, and co-ordinated to ensure that they are not detrimental to either component.

Given a set of components, a combinatorial amount of systems could be created.
Ensuring the correctness of every possible composition could be an impractical task.
To ensure a component can be composed into a system, its relationships are described and these form constraints on a system.
It is assumed then, that if these constraints are satisfied the component will function correctly.
Component contracts \citep{Watkins1999} have been created as a way of checking to see if the system is correctly composed.
By formally defining contracts and integrating them into composition of components, it is hoped to verify component systems \citep{Dietrich2010}.

The atomic unit of the evolution of a component is the \textit{version}.
A version is marker assigned to a component to mark its evolved state.
To be used to reason about components, the component model defines a version model which states the syntax of a version.
This version model also describes the comparison of versions, which should create a total order over a set of components, where the highest value should be the most evolved state.
In general, the version does not convey any meaning \citep{Stuckenholz2005}, but attempts have been made to add semantics to versions \cite{Bauml2009}.
In most current component models, loosely enforced conventions are used to associate meanings to versions \cite{Barth2005,osgicore2007}.
A discussion on the versioning of components is given by \cite{Stuckenholz2005},
this describes the problems associated with the versioning and evolution of components.

A proper methodology for the development and evolution of software components it still being searched for \citep{Szyperski2002}.
Discussions on types of evolutionary changes with comparisons to other domains are given in \citep{Papazoglou2011},
and empirically explored by \cite{vasa2007patterns}.
There are many aspects of component evolution left unexplored, including how these evolutions effect the systems they are deployed in.

\subsection{Component System Evolution}
The evolution of a component system is accomplished by the composer through altering the set of components contained within the component system.
This is opposed to the evolution of individual component, where a developer alters an individual component.
This architectural level view of evolution does not concern itself with the internals of the constituent components, only the relationships between the components.

The core functions to evolve a component system are \textbf{install} and \textbf{remove}, these add or remove a component to or from the system.
The additional function of \textbf{upgrade} can also be defined to remove then install a higher version of the same component. 
This limited set of possible actions when changing a component system does not require much technical knowledge. 
To evolve a single component requires programming, software development knowledge, as well as some technical knowledge of the component model.
However, the evolution of a piece of software from a higher level only requires some knowledge of the component model and tools to change it.
For example, to upgrade the components in a Debian GNU/Linux component system, the command \verb+apt-get upgrade+ is all that is needed to be executed.
To extend the system to install a component \verb+comp+ the command \verb+apt-get install comp+ can be executed.
This simplicity gives the user the ability to be the composer of their own system.

A problem arises when trying to measure the evolved state of the system,  of a system as there is no way to assign a ``version'' to a component system.
A component system is a set of components, where each component can have various versions.
This may make a component system impossible to compare to another in a way in which versions allow component to be compared.
For instance, a system that has version 1 of component $a$ and version 2 of component $b$, is neither more or less evolved than a system with version 2 of $a$ and version 1 of $b$.
This can get even more complicated when considering some component models allow multiple versions of a single component installed, e.g. 
is a system with version 1 and 2 of $a$ installed less evolved that a system with only version 2 of $a$ installed?

This process of component system evolution is empirically studied by \cite{fortuna2011} 
who look at the first ten releases of the Debian and compare it to the evolution within biology.
Methods to evolve component systems are discussed in \citep{Ryan2005} and \citep{Luo2004},
and the mitigation of the negative effects caused by such evolution is discussed in the paper \citep{Stuckenholz2007}.

A formal definition of the component system evolution is given in chapter \ref{formal}.

\subsection{Component Dependency Resolution}
For a non-technical user to become the composer, many of the aspects of component composition must be automated.
In this research the function that automates these aspects is called  \textbf{Component Dependency Resolution} (CDR).

%%%The tedious nature of satisfying component dependencies during component system evolution can be automated into a function called Component Dependency Resolution.
CDR takes requests to change a system from the composer and constraints created by the component relationships and searches for systems that satisfy both.
For example, a user wants to install a new text editor component into their system, and the selected text editor has a dependency on a spell checker.
This means a spell checker component must be installed for the text editor to be functional.
As the spell checker can have its own relationships such as dependencies, the process of satisfying all constraints is a recursive problem. 
Therefore, CDR tries to find a valid system that satisfies all constraints.

For CDR to be usable, components must explicitly state their relationships.
If they do not, then CDR cannot analyse the constraints that are needed to be satisfied and a solution cannot be searched for.
This search is a selection problem, where CDR can select from many different possible systems that satisfy all constraints.
To select an acceptable solution, CDR must consider the user's goals for the system.
These goals are part of the user's strategy; the user's plan to evolve the system.

In other areas CDR, or applications that contain CDR, can be referred to by various names; provisioning system or dependency management in Eclipse domain \citep{leBerre2010},
or package manager in the GNU/Linux \citep{Barth2005} domain.
Many other component models use CDR to compose and evolve component systems.
CDR can be used at design time to determine the required dependencies to build and test a project (as in Apache Maven \citep{casey_better_2008}),
at run time to evolve or extend a component-based system (as in Eclipse P2 \citep{leBerre2010}),
or it can be used to build and restructure software product lines \citep{savolainen_analyzing_2007}.
CDR is also used in package management systems, like apt-get \citep{Barth2005},
for adding, removing and updating components in package based operating systems like Ubuntu\footnote{http://www.ubuntu.com}. 

An implementation of CDR is described in chapter \ref{implementation}

\subsection{Evolution Strategies}
%%%The decisions that lead to an evolution strategy
Lehman's laws of software evolution \citep{lehman1980,lehman1997} describe two opposing forces,
the need to change and the resistance to change.
When change occurs it is inevitable that it will introduce more complexity and the quality of the system will be reduced.
It is also inevitable that without change the user will become dissatisfied with their system.
A user's strategy to evolve their system must take into account both these forces.
As each user and each system have different requirements and objectives, each strategy can be different for any given user.

Further discussion on strategies and how they are defined is presented in chapter \ref{strategies}.

\section{What is a Software Component?}
\label{background.components}
As described above, software component ideals can be traced to \citep{McIlroy1969}.
The discussion of the definition and nature of such software components is described in this section.

%%%What is a component in a larger context
A component is a natural concept that describes a part or element of a larger system or process.
A broad characterisation of a component is ``components can be composed together''.
They can be physical, as in electrical or mechanical components, where their properties and how they are composed are documented to be used in construction.
Typically, components can be used in many different contexts, e.g. a resistor component, they can be used in electrical systems from space stations to cellphones.   
This natural concept of what a component is leads to problems when defining the concept of a software component.

%%%Discussion of articles where Szyperski and Meyers discuss give their versions of what a component is.
A discussion between two researchers in component software, Bertrand Meyer and Clemens Szyperski, highlight the difficultly of defining ``software component''. 
They describe across multiple articles \citep{Meyer1999,Szyperski2000a, Szyperski2000, Meyer2000} their definitions of what a software component is and the importance of its features.

%%%Szyperski's definition
Szyperski defines components in \citep{Szyperski2002} as having three characteristic properties:
\begin{enumerate}
  \item a unit of independent deployment
  \item a unit of third party composition
  \item has no externally observable state
\end{enumerate}

%%%Meyers Definition
Meyer's definition of software components is enumerated as:
\begin{enumerate}
  \item May be used by other software elements (clients)
  \item May be used by clients without the intervention of the components developers
  \item Includes a specification of all dependencies (hardware and software platform, versions, other components)
  \item Includes a precision specification of the functionality it offers
  \item Is usable on the sole basis of that specification
  \item Is composable with other components
  \item Can be integrated into a system quickly and smoothly
\end{enumerate}

%%%Using a component model
Others, like \citep{heineman2001component} have stated that components must conform to a component model: 
\begin{quotation}
A software component is a software element that conforms to a component model and can be independently deployed and composed without modification according to a composition standard.
\end{quotation}

Defining a component model as:
\begin{quotation}
A component model defines a set of standards for component implementation, naming, interoperability, customization, composition, evolution, and deployment.
\end{quotation}

%%%Other differences, architectural/implementation
Exactly what is, and what is not a software component is in dispute amongst the community,
and a definitive description of a software component is elusive \citep{vasa2007patterns}.
As such, many different components models have been developed, each targeting various domains with different functionality and technical aspects.
This diversity has inspired a classification approach from \citep{Crnkovic2011}, where components and component models are classified into a scheme.
This effort highlights the difficulty in creating an exact definition of a software component. 

%%%It is difficult to define a natural concept such as component with a set of fixed enumerable features.
These problems with exactly defining a software component may stem from the fact that ``component'' is a natural concept.
The problem, as observed from the area of formal concept analysis \citep{Ganter1999} by \citep{Szyperski2002}, is that it is impossible to

\begin{quotation} 
``enumerate a fixed agreeable set of features that is necessary and sufficient for a natural concept such as component.'' 
\end{quotation}

%%%Our egotistical view of components is defined with relation to Component Dependency Resolution.
However, a definition can be found, not by feature enumeration but through stating the intention for the concept and exploring the technically inevitable consequences \citep{Szyperski2002}. 
As the intention of this study is to investigate component system evolution, CDR, and evolution strategies
the definition of software component will be with respect to these concepts.

%%%This will not fit all views of what a component is (no definition will), but it eliminates unnecessary arguments so we can focus on CDR
This view of a software component's definition will not agree with every persons view of what a software component is, but (as described above) no precise view can.
By only looking at aspects that are necessary for this research, the arguments over unnecessary semantics can be eliminated. 

\subsection{Software Component}
The definition of a software component is given with respect to the evolution of a component system using component dependency resolution.
Both these areas have already been discussed in this chapter and will be used to define the concept of ``software component''.
This definition specifies the type of components and component models this research can be applied to.

In this research a software component is a unit of independent deployment, and third party composition,
and a component model must:
\begin{enumerate}
  \item require the explicit definition of relationships between components
  \item include mechanisms in which to programmatically compose or change a component system
\end{enumerate}

To automate the evolution of a system with CDR, as discussed above, the relationships between components must be explicitly defined and computer readable.
This automation also requires an interface for CDR to evolve the system.
This means that CDR should have the access to mechanisms to add, remove and replace components in a system.
As CDR would evolve a system automatically, these actions must be able to be enacted either through a function call, or well defined command interface.

This definition leaves undefined many aspects of a component model. 
This can be seen though some of the aspects of a component model from the classification in \citep{Crnkovic2011}:
\begin{itemize}
  \item Lifecycle 
  		\begin{itemize}
	  		\item Modelling
	  		\item Implementation
	  		\item Packaging
	  		\item Deployment
		\end{itemize}
	\item Construction
		\begin{itemize}
	  		\item Interface Specification
	  		\item Binding
	  		\item Interactions 
		\end{itemize}
	\item Extra Functional Specifications
		\begin{itemize}
	  		\item Management
	  		\item Specification
	  		\item Composability
		\end{itemize}	
\end{itemize}  
These categories show the lacking complexity of the component model definition in this research.
The only two categories within the scope of this research are ``deployment'' and ``construction'', which both focus on the evolution of a system.
All the other aspects are ignored as their definition is superfluous to the core topic of this research.
This may make the definition in this research broad, though it also makes it focused.

This definition of software component will be examined through examples of different component models that conform to it.

\section{Component Models}
\label{background.models}
%%%Some core component models are described, and their dependencies and evolution are discussed
Given the definition of a software component for this research, some current component models are described and discussed.
These models come from industry (OSGi, Eclipse Plugins, Fractal, Maven), open source community (Debian, RPM) and academia (SOFA2).

%%%Meta data of component models is discussed
The typical mechanism in which components from these frameworks express their explicit relationships is through meta-data files.
These files contain what a component requires and provides, as well as other information like versions and provider details.
These meta-data files are then a core element in the description of a component model.

%%%How to change a system is discussed
A component model must also provide a mechanism to alter the system of components.
This is usually a low level interface, where the addition or removal of a component is an atomic step.
A CDR implementation can then use these interfaces to evolve a system.

%%%Exisiting CDR implementations are discussed
Some component models have a CDR implementation already available.
These implementations are described and the possible user strategies that they allow will be discussed. 

To compare these component models, the example of a text editor component that depends on a spell checker component is used.
It is hoped this simple situation will highlight the similarities and differences between the various component models.

\subsection{OSGi}
%%%What OSGi is? Bundle Service
OSGi is a mature component model from the OSGi Alliance.
It has implementation from organizations like the Eclipse Foundation with their Equinox framework \citep{mcaffer2010osgi}, 
and the Apache foundation with their framework Felix\footnote{http://felix.apache.org/}.

OSGi components are refereed to as bundles, each contains a meta-data file describing the bundle's relationships, and a set of Java packages and classes as implementation.
A notable difference between OSGi and other component frameworks is that it separates the unit of deployment, the bundle, and the run-time unit, the service.
These services exist on a separate layer to the bundles, each service is created at run-time and is represented by a Java object.
This service layer can also have explicit relationship through frameworks like Spring Dynamic Modules\footnote{http://www.springsource.org/osgi}.
Under the definition of component in this research, this makes both the bundle layer and the service layer software component models.

\subsubsection{Bundle Layer}
The bundle layer of OSGi is implemented to allow dynamic addition and removal of bundles.
This can be done either through an implemented console, like the one in the Equinox framework, or through a programmatic interface.

An example of the meta-data which defines a text editor bundle that depends on a spell checker is described in figure \ref{osgibundle}.

\begin{figure}[htp]
\begin{center}
\begin{alltt}
Bundle-Name: TextEditor
Bundle-Vendor: Graham Jenson
Bundle-SymbolicName: nz.geek.maori.textEditor
Bundle-Version: 0.0.1.alpha
Bundle-RequiredExecutionEnvironment: J2SE-1.4
Export-Package: nz.geek.maori.textEditor;version="0.0.1.alpha"
Require-Bundle: nz.geek.maori.fonts
Import-Package: nz.geek.maori.spellchecker;version>"0.0.1"
\end{alltt}
  \caption[OSGi Meta-data]{Example of OSGi Meta-data}
  \label{osgibundle}
\end{center}
\end{figure}

%%%OSGI Meta data description
This meta-data shows the name, version and vendor of the component, as well as the exported packages (referring to Java packages) which are able to be used by other components.
The \verb+Require-Bundle+ key expresses the direct dependence on another bundle.
The \verb+Import-Package+ key expresses the dependence on a package provided by a bundle and not on the bundle itself.

\subsubsection{Service Layer}
This bundle meta-data only contains information necessary for the execution of a component.
However, for the component to be functional the service layer relationships must be considered.

This service layer is defined in the core OSGi specification \citep{osgicore2007}, however it does not describe declarative support with meta-data.
Therefore, a number of frameworks have emerged that define service relationships, e.g
Spring Dynamic Modules\footnote{http://www.springsource.org/osgi}.
OSGi's compendium specification \citep{osgicompendium2007} also defines service layer meta-data format called Declarative Services(DS).
An example of DS meta-data is in figure \ref{dsmetadata}.

\begin{figure}[htp]
\begin{center}
\begin{alltt}
<?xml version="1.0"?>
<component name="textEditor">
    <implementation class="nz.geek.maori.textEditor.TextEditorImpl"/>
    <service>
        <provide interface="nz.geek.maori.textEditor.TextEditor"/>
    </service>
    <reference name="spellChecker"
        interface="nz.geek.maori.spellchecker.SpellChecker"
        bind="setSpellChecker"
        unbind="unsetSpellChecker"
        cardinality="0..1"
        policy="dynamic"/>
</component>
\end{alltt}
  \caption[OSGi Declarative Services]{Example of OSGi Declarative Services meta-data}
  \label{dsmetadata}
\end{center}
\end{figure}

This meta-data includes references to implementation elements like interfaces that are provided and required, and methods to interact with the services.
These relationships can have cardinalities, e.g. a text editor can use multiple spell checkers, and other programmatic features.
The \verb+service+ tag describes the services provided, and the \verb+reference+ tag expresses a dependence on another services.

%%%DS provides no versioning
One aspect of DS that is lacking, is the ability to explicitly version services.
The version of the bundle that is providing the service and the version of the package that is provided by the bundle, implicitly give a version to the service being provided.
However, for the evolution of a component model, some type of versioning is typically defined.

\subsubsection{Evolution of OSGi}
The programmatic evolution of an OSGi system is defined in the interfaces created by the OSGi alliance.\footnote{http://www.osgi.org/javadoc/r4v43/}
The installation and removal of both the bundles and services from the OSGi system are defined as such:
 
\begin{itemize}
  \item To install a bundle: \verb+org.osgi.framework.BundleContext#install+
  \item To uninstall a bundle: \verb+org.osgi.framework.Bundle#uninstall+
  \item To register a service: \verb+org.osgi.framework.BundleContext#registerService+
  \item To unregister a service: \verb+org.osgi.framework.ServiceRegistration#unregister+
\end{itemize}

These methods can sometimes be implemented in a console, this allows a user to directly execute them to add or remove bundles.

\subsubsection{OSGi Bundle Repostiory}
%%%The OBR proposed by Peter Kriens and Richard Hall, is a meta data format for OSGi bundle resolution
A CDR implementation for OSGi was proposed by Peter Kriens and Richard S. Hall in RFC-0112 \citep{the_osgi_alliance_rfc-0112_2006}.
This is called the OSGi Bundle Repository and consists of collection of components described in an XML format and a function to reason about them.
An example of this document is shown in figure \ref{obrmetadata}.

\begin{figure}[htp]
\begin{center}
\begin{alltt}
<repository name='OBR REP' time='123'> 
  <resource version='0.0.1' name='nz.geek.maori.textEditor' 
  uri='nz.geek.maori.textEditor.0.0.1.jar'> 
    <require optional='false'  multiple='false'  name='package' 
        filter='(&amp;(package=nz.geek.maori.spellChecker)(version&gt;=1.0.0))'> 
      Import package nz.geek.maori.spellChecker ;version=1.0.0 
    </require> 
  </resource> 

  <resource version='1.0.0' name='nz.geek.maori.spellChecker' 
  uri='nz.geek.maori.spellChecker-1.0.0.jar'> 
    <capability name='package'> 
      <p v='nz.geek.maori.spellChecker' n='package'/> 
      <p v='1.0.0' t='version' n='version'/> 
    </capability> 
  </resource> 

</repository>
\end{alltt}
  \caption[OSGi Bundle Repository markup example]{Example of OSGi Bundle Repository meta-data}
  \label{obrmetadata}
\end{center}
\end{figure}

This meta-data was designed so that it can merge the bundle and service meta-data as described in figures \ref{osgibundle} and \ref{dsmetadata}.
This then would be a complete model at both these OSGi layers, which can be used by CDR to evolve a system. 

OBR has been seen as a solution to to simplify deployment of OSGi applications \citep{Jung2007}, 
distribution and deployment to embedded ubiquitous systems \citep{Jung2006}, 
smart home applications \citep{Gouin-Vallerand2007} and 
dynamic distribution of drivers \citep{Kriens2008}. 

The most mature implementation of OBR client is offered by the Apache foundation, which is bundled with their core OSGi framework Apache Felix. 
This can be used with any of the large public or private OBR collections of bundles.
An example of one such public repository is the Paremus repository\footnote{http://www.osgi.org/Repository/} which contains (as of December 2011) over 2000 bundles.

The specification of OBR does not define a specific strategy, or a mechanism to define strategies to be used during evolution.
The implementation specific strategy that is used by the Apache OBR\footnote{http://felix.apache.org/site/apache-felix-osgi-bundle-repository.html accessed 6/12/2011} 
is described on its help page as:

\begin{quotation}
OBR might have to install new bundles during an update to satisfy either new dependencies or updated dependencies that can no longer be satisfied by existing local bundles. 
In response to this type of scenario, the OBR deployment algorithm tries to favor updating existing bundles, if possible, as opposed to installing new bundles to satisfy dependencies.
\end{quotation}

This shows that when updating a system of bundles, newer bundles may be installed although currently installed bundles are preferred.

\subsection{Eclipse Plugins}
%%%What is Eclipse
Eclipse is a widely used IDE and an extensible plugin platform for creating Java applications.
It is built on top of the OSGi framework, but ignores OSGi's service layer and re-implements it with its own Eclipse plugin runtime.
Therefore, the distrusted components are OSGi bundles and the run time elements are plugin services.

These plugins are defined using extensions and extension points, where extensions provide a service for an extension point.
An example of the meta-data describing a plugin is given in figure \ref{eclipseplugin}. 

\begin{figure}[htp]
\begin{center}
\begin{alltt}
<?xml version="1.0"?>
<plugin
    name="Text Editor"
    id="nz.geek.maori.textEditor"
    version="0.0.1.alpha"
    provider-name="Graham Jenson">
    
    <requires>
        <import plugin="nz.geek.maori.fonts"/>
    </requires>

    <runtime>
        <library name="texteditor.jar"/>
    </runtime>
    
    <extension-point id="nz.geek.maori.spellchecker" 
        name="Spell Checker" 
        schema="spellchecker.exsd"/>
</plugin>
\end{alltt}
  \caption[Eclipse Plugin meta-data]{Example of an Eclipse Plugin plugin.xml meta-data file}
  \label{eclipseplugin}
\end{center}
\end{figure}

This plugin defines the name, version and provider of the plugin, and using the tags \verb+requires+ defines the requirements of this plugin to function.
The \verb+extension-point+ tag defines not only what the plugin provides, but also the required information in order to provide it, described in a schema.
This is a special feature of the Eclipse plugin framework, as other component models generally do not specify these parameters on this type of relationship. 

The schema of an extension point, as shown in figure \ref{eclipseextensionschema}, describes the elements to use an extension.

\begin{figure}[htp]
\begin{center}
\begin{alltt}
<schema targetNamespace="null" xmlns="http://www.w3.org/2001/XMLSchema">
 <element name="spellchecker">
  <complexType>
   <attribute name="spellchecker" type="string" use="required">
    <annotation>
     <appinfo>
      <meta.attribute kind="java" basedOn=":nz.geek.maori.ISpellChecker"/>
     </appinfo>
    </annotation>
   </attribute>
  </complexType>
 </element>
</schema>
\end{alltt}
  \caption[Eclipse Plugin extension point schema]{Example of an Eclipse Plugin extension point schema file}
  \label{eclipseextensionschema}
\end{center}
\end{figure}

In this description, this extension point requires a string that describes a java class that implements \verb+nz.geek.maori.ISpellChecker+.
This is the only requirement for this extension point, but other parameters of types, e.g. Boolean, integer, can be defined.

\subsubsection{Eclipse Evolution}
The programmatic evolution of an Eclipse system is through the use of the previously OSGi methods and with interaction with the plugin registry.

The extensions and extension points, for an Eclipse system, can be altered in the plugin registry.
This registry contains all references to the extensions and extension points in a system.
The methods used to add and remove these are:

\begin{itemize}
  \item add extensions and extension points described in an XML file: \\ \verb+org.eclipse.core.runtime.IExtensionRegistry#addContribution+
  \item remove extension: \\ \verb+org.eclipse.core.runtime.IExtensionRegistry#removeExtension+
  \item remove extension point:\\ \verb+org.eclipse.core.runtime.IExtensionRegistry#removeExtensionPoint+
\end{itemize}

The Eclipse plugin system is a component system in this study, 
as it contains explicit relationships, and the programmatic evolution requirements necessary for CDR to be used.

\subsubsection{Eclipse P2}
%%%The Eclipse P2 Solver, by Le Berre and Rapicault
Eclipse P2 \citep{le_berre_dependency_2009}, \citep{leBerre2010} is the provisioning system for the Eclipse IDE platform.
It provides the interaction to alter an Eclipse based component system. 
It is implemented using a Boolean Satisfiability solver, called SAT4J \citep{le2010sat4j}.
This method of implementation has become a common way of implementing CDR, explored in \citep{Berre2008}, and further discussed in chapter \ref{implementation}.

Eclipse P2 is mainly accessed through the Eclipse user interface, where the user can select to update the entire system, or install a component.

Some of the criteria used by Eclipse P2 to evolve the system are described in \citep{le_berre_dependency_2009} and \citep{leBerre2010}:
\begin{enumerate}
  \item minimise the amount of components installed that have no dependency to them 
  \item minimise the removal of already installed components
  \item minimise the age of the installed components
  \item minimise changing an installed component if they are unrelated to the request being made
\end{enumerate} 

These criteria represent the strategy that it employed to choose a solution when multiple are available. 

\subsection{Fractal}
Fractal \citep{Quma2006} is a component model developed by France Telecom R\&D and INRA.
It is a specification that is designed to be programming language independent, unlike OSGi or Eclipse that both depend on Java specific elements.
The most notable aspect, and the reason for its name, is that a components can be composed together to make a new component.
This recursive nature of composition, actually means the system of components can itself be a component.

An example of meta-data to define a Fractal component is given in figure \ref{fractalmetadata}.

\begin{figure}[htp]
\begin{center}
\begin{alltt}
<?xml version="1.0" encoding="ISO-8859-1" ?>
<!DOCTYPE definition PUBLIC 
    "-//objectweb.org//DTD Fractal ADL 2.0//EN" 
    "classpath://org/objectweb/fractal/adl/xml/basic.dtd">

<definition name="textEditorComponent">
  <interface name="textEditor" role="server" signature="nz.geek.maori.textEditor"/>
  <interface name="spellChecker" role="client" signature="nz.geek.maori.spellChecker"/>
  <content class="nz.geek.maori.textEditorImpl"/>
</definition>
\end{alltt}
  \caption[Fractal ADL Example]{Example of a simple Fractal ADL file}
  \label{fractalmetadata}
\end{center}
\end{figure}

This meta-data describes the relationship between components as a metaphor between a client and a server.
The interface tag with the role attribute assigned to ``server'', defines the provided functionality of the component.
The interface tag with the role attribute assigned to ``client'' then defines the required functionality of the component.  

An example of the definition of a recursive component can be seen in figure \ref{fractalrecursve}.

\begin{figure}[htp]
\begin{center}
\begin{alltt}
<?xml version="1.0" encoding="ISO-8859-1" ?>
<!DOCTYPE definition PUBLIC 
    "-//objectweb.org//DTD Fractal ADL 2.0//EN" 
    "classpath://org/objectweb/fractal/adl/xml/basic.dtd">

<definition name="textEditorWithSpellCheckerComponent">
  <interface name="textEditorWithSpellChecker" role="server" signature="nz.geek.maori.textEditorWSC"/>
  <component name="textEditorComponent">
    <interface name="textEditor" role="server" signature="nz.geek.maori.textEditor"/>
    <interface name="spellChecker" role="client" signature="nz.geek.maori.spellChecker"/>
    <content class="nz.geek.maori.textEditorImpl"/>
  </component>
  <component name="spellCheckerComponent">
    <interface name="spellChecker" role="server" signature="nz.geek.maori.spellChecker"/>
    <content class="nz.geek.maori.spellChecker.SpellChecker"/>
  </component>
  <binding client="this.textEditorWithSpellChecker" server="textEditorComponent.textEditor"/>
  <binding client="textEditorComponent.spellChecker" server="spellCheckerComponent.spellChecker"/>
</definition>
\end{alltt}
  \caption[Fractal ADL Example]{Example of a simple Fractal ADL file}
  \label{fractalrecursve}
\end{center}
\end{figure}

This description of a recursive component structure defines both the text editor and spell checker components, as well as their relationships.
The tag \verb+definition+ is then used to define a composition of components that provides a service of \verb+textEditorWithSpellChecker+.
To provide this the spell checker must be bound (described in the tag \verb+binding+) to the text editor,
and the text editors provided interface bound to the output interface. 

\subsubsection{Fractal Evolution}
The programmatic evolution of a Fractal component system can differ between implementations.
To simplify this, only the Java implementation is described. 
In this implementation the methods to edit the configuration are\footnote{http://fractal.ow2.org/current/doc/javadoc/fractal/}:
\begin{itemize}
  \item to create a component: \\ \verb+org.objectweb.fractal.api.factory.GenericFactory#newFcInstance+
  \item to add a component to a composite: \\ \verb+org.objectweb.fractal.api.control.ContentController#addFcSubComponent+
  \item to remove a component from a composite: \\ \verb+org.objectweb.fractal.api.control.ContentController#removeFcSubComponent+
\end{itemize}

There are two points that may make the evolution of Fractal component model particularly difficult,
the recursive nature of Fractal, and the lack of required version information about components and composites.

%%%Fractal requires not only the selection of components but the configuration
Given the recursive nature of the components described, any system may provide the same functionality with the same components in a combinatorial number of configurations.
Therefore, not only would a CDR algorithm require to select a set of components but also group those component into composites.
The simplest solution to this problem is to ignore the recursive nature of this component model,
and use a CDR algorithm to create exactly one composite with all components in it.
More difficult solutions could be created by reusing user composites,
or analyzing the graph structure to extract relationship by using algorithms like the one presented in \citep{dietrich2008cluster}.

Another aspect of Fractal that will make the automatic evolution of systems difficult will be the lack of versioning information.
Like DS, this component model does not explicitly require the version of the component and the interfaces they provide.
As versions are typically the atomic unit of evolution, the process of selecting a newer version will be made more difficult if not impossible. 

\subsection{Maven}
Maven is a build automation tool, designed to be programming language agnostic, but primarily used with Java.
The core component aspect of Maven is that it can dynamically select and download files from a repository to be used to build a system.
Although Maven is built using a plugin architecture, the focus in this section is on the use of Maven as a component model and not Maven itself.

A system is defined using a Project Object Model (POM) file, an example given in \ref{mavenmetadata}. 

\begin{figure}[htp]
\begin{center}
\begin{alltt}
<project xmlns="http://maven.apache.org/POM/4.0.0"
  xmlns:xsi="http://www.w3.org/2001/XMLSchema-instance"
  xsi:schemaLocation="http://maven.apache.org/POM/4.0.0
                      http://maven.apache.org/xsd/maven-4.0.0.xsd">

  <groupId>nz.geek.maori</groupId>
  <artifactId>textEditor</artifactId>
  <version>0.0.0.alpha</version>

  <dependencies>
    <dependency>
      <groupId>nz.geek.maori</groupId>
      <artifactId>spellChecker</artifactId>
      <version>[0.0.1,1.0.0)</version>
      <type>jar</type>
     </dependency>
  </dependencies>
</project>
\end{alltt}
  \caption[Maven POM file]{Example of a Maven POM file}
  \label{mavenmetadata}
\end{center}
\end{figure}

This model defines the components namespace using the \verb+groupId+ tag, the name using the tag \verb+artifactId+, and the version.
It also defines the dependencies on other projects through the \verb+dependency+ tag,
where it states that it depends on the artifact \verb+spellChecker+ from versions 0.0.1 to (but excluding) version 1.0.0.

\subsubsection{Maven Evolution}
The use of Maven primarily during development does not exclude it from the definition of being a component model in this study.
Maven both explicitly declares relationships and provides an interface to evolve its component systems. 
This unique characteristic of Maven does mean that evolving a component system is done in two stages where first it is altered, then it is recompiled and replaced.
The evolution of a Maven based systems are therefore done simply by selecting the projects to build, and using Maven to compile them into a system. 

Another core difference from the other presented component models is that Maven POM objects do not declare what they provide.
For example, an OSGi bundle can declare provided packages and a Fractal component declares provided interfaces, where a Maven project can only provide itself.
This will simplify a CDR algorithm by not having to resolve another layer of dependencies, 
though this means it goes against component principles like dependency on interface or specification \citep{Szyperski2002}.

\subsection{Debian Packages}
Debian is a GNU/Linux based operating system provided by the Debian Project \citep{Barth2005}.
This operating system most prominent feature is that it is composed of packages, a unit of deployment with explicit relationships.

Each package is defined using a control file, an examples presented in figures \ref{debianmetadatate} and \ref{debianmetadatasc}.

\begin{figure}[htp]
\begin{center}
\begin{alltt}
Package: textEditorPackage
Version: 0.0.1.alpha
Depends: spellChecker
Conflicts: otherTextEditorPackage
\end{alltt}
  \caption[Debian Control file for Text Editor]{Example of a Debian Control File for Text Editor}
  \label{debianmetadatate}
\end{center}
\end{figure}

\begin{figure}[htp]
\begin{center}
\begin{alltt}
Package: spellCheckerPackage
Version: 1.0.0
Provides: spellChecker
\end{alltt}
  \caption[Debian Control files for Spell Checker]{Example of a Debian Control File for Spell Checker}
  \label{debianmetadatasc}
\end{center}
\end{figure}

These control files describe two packages, a text editor and spellChecker, they are organised into key/value pairs separated by the ``:'' character.
The text editor states using the \verb+Depends+ key that is requires a spellChecker to be installed, 
and through the \verb+Conflicts+ key states that it cannot be installed with another text editor.
The spell checker package defines that it provides a spell checker using the \verb+Provides+ key.
This spell checker is of a type called a virtual package, which has some specific semantics, discussed later in chapter \ref{ubunutsimulation}.

The semantics of these control files is further described in the chapter \ref{ubunutsimulation}.

One aspect that differentiates the Debian package model from the many other models, is that it does not specify the rules of composition between components.
That is, there is no formal specification of the semantics of the communication and relationships between packages.
For instance, OSGi bundles exist in a very constrained environment that manages their interactions, depending on another bundle has run-time implications.
For Debian, the dependence has no prescribed run-time effects, only whether the package is functional or not.
The way in which a package, uses or communicates with a depended upon package is then defined between the packages and not in the component model.

\subsubsection{Debian Evolution}
The application ``dpkg'' is the command line tool use to modify a Debian package system.

The commands to modify a Debian package system are:
\begin{itemize}
  \item to add a package: dpkg --install <package file>
  \item to remove a package: dpkg --remove <package>
\end{itemize}

These commands define the atomic actions to change a Debian system.
They can be executed from other applications through the command line, this makes them programming language independent and easily accessible. 

\subsubsection{Apt-get}
The application dpkg can only remove and add a single component at a time, it does not resolve the dependencies of the component.
There are many applications that can resolve these dependencies on a Debian system, the default and most popular solution is through the ``apt-get'' application.
This tool is built on top of dpkg, to provide an interface to simplify and extend its functionality.
The apt-get application is a command line tool, which provides the necessary tools to efficiently allow the user to evolve their system.

Apt-get follows some basic criteria when altering the system.
For example, when upgrading the system under no circumstances will apt-get remove an already installed package, or retrieve a package that is not installed. 
Such criteria exist for each different function, such as install, and remove.

Another application that extends APT further is Aptitude \citep{Burrows2005}.
This application adds a basic user interface to APT and also increases functionality in aspects like locating and removing redundant packages. 

\subsection{SOFA 2.0}
Software Appliance (SOFA) is a component model developed at Charles University in Prague.
A component in the SOFA framework is defined by it's frame, which contains the meta-data of what the component requires and provides.
Like Fractal, SOFA is a recursive component model, allowing for a composite of components to be treated as a single component.

An example definition of a frame is given in figure \ref{SOFAmetadata}.

\begin{figure}[htp] 
\begin{center}
\begin{alltt}
<?xml version="1.0"?>
<frame name="nz.geek.maori.textEditor">
  <requires name="spellChecker" itf-type="sofatype://nz.geek.maori.spellChecker"/>
</frame>

<?xml version="1.0"?>
<frame name="nz.geek.maori.spellChecker">
  <provides name="spellChecker" itf-type="sofatype://nz.geek.maori.spellChecker"/>
</frame>
\end{alltt}
  \caption[SOFA ADL files]{Example of a SOFA ADL Files}
  \label{SOFAmetadata}
\end{center}
\end{figure}

This meta-data describes two components, a text editor and a spell checker, using the tag \verb+frame+.
The sub elements \verb+requires+ and \verb+provides+ describes their relationship through an externally defined interface with the name-space \verb+sofatype://nz.geek.maori.spellChecker+.
This interface is defined using an Architecture Definition Language (ADL) and added to a repository to be re-used. 


\subsubsection{SOFA Evolution}
SOFA 2.0 is an extension of the SOFA framework, with new services including dynamic reconfiguration for dynamic evolution of an architecture at run-time. 
Unlike other component systems,
it only allows for controlled evolution; 
Many component models allow the change of the component system on a fine grained level of adding or removing a component.
However this model only allows more granular changes to be made to the system, these changes must conform to specified evolution patterns.
This restriction is meant to increase the manageability and predictability of a component systems evolution. 

Three evolution patterns are predefined: factory pattern, removal pattern, and service access pattern.
As its name suggests, in factory pattern a designated component serves as a component factory. 
The removal pattern serves for destroying of a component previously dynamically created.
The service access pattern allows for the access to external services.
These patterns are further discussed in \citep{Hnetynka2006}.


\subsection{Common Upgradeability Description Format}
%%%CUDF model created for compeition
Many component models share similar properties:
\begin{itemize}
  \item a component name
  \item a component version
  \item description of what the component provides
  \item description of what the component requires
\end{itemize} 

By creating an abstraction of these models, that only contains the information that is necessary for CDR, different CDR implementations can be compared.

The Common Upgradeability Description Format (CUDF) model is defined as an abstract representation of the problem found when evolving a package based system, e.g. Debian.
It was created by the Mancoosi project to foster interest and competition in creating solvers for upgrading problems faced by Free and Open Source Software (FOSS).
This specification contains the necessary detail to express the core aspects of evolution, and an extensible syntax to allow for component model specific detail. 
CUDF allows the description of many CDR problems in a single format, and the use of various CDR implementations to solve them. 

%%%No concrete components, but fits our definition
There is no concrete component representation, as it was not designed to be an executable component model.
This means that there is no detail of how to execute the component system, or other implementation focused details that are necessary for the other presented component models.

In figure \ref{CUDFmetadata}, an example of a CUDF problem is presented.

\begin{figure}[htp] 
\begin{center}
\begin{alltt}
package: textEditor
version: 1
depends: spellCheckerService

package: spellChecker
version: 2
provides: spellCheckerService

request:
install:textEditor

\end{alltt}
  \caption[CUDF meta-data file]{Example of a CUDF problem file}
  \label{CUDFmetadata}
\end{center}
\end{figure}

The CUDF file is defined using key/value pairs, where each component is defined using the two tags \verb+package+ and \verb+version+.
This file does not only represent the components but also the evolutionary change with the tag \verb+request+.
Further definition of the syntax and semantics of CUDF is in chapter \ref{formal}, as it is used extensively throughout this study.

\subsubsection{Mancoosi MPM}
%%%The mancoosi Modular Package Manager uses CUDF, currently going through the process in the Debian component repository
The CUDF component model is defined especially to compare CDR implementations, therefore many various implementations exist.
Some of these will be discussed in the latter chapter \ref{implementation}.
However, here only the default implementation, Modular Package Manager \citep{abate2011} (MPM), is discussed.
This implementation was created by the Mancoosi organisation to explore the possibility of using CUDF to evolve real systems.

In the study by \cite{abate2011}, this package manager was compared against other component solvers, such as apt and aptitude, and shown to be an improvement.
This CDR implementation is the first to allow the user to completely specify the criteria by which the solver selects a system.
In the study the solver first minimised the amount of packages removed from the system, and then minimised the amount of packages changed in the system.
Further discussion of these criteria and how they are combined is discussed in chapter \ref{strategies}. 

\section{Summary}
In this chapter, first the evolution of software was discussed, and the nature and laws of this process described.
Component evolution and component system evolution are then described and compared as separate software evolution processes.
Automating the evolution of component systems is then defined through the function component dependency resolution.
The strategies that a user employs to evolve a system is then discussed, and the forces that are considered in the strategy.
Software components are defined with respect to the objectives of this research as having explicit relationships and an interface to evolve a component system.
Various component models that conform to this definition are then discussed and their properties explored, these are briefly compared in tables \ref{background.comparisonvocab} and \ref{background.comparisonprop}.

\begin{table}
\begin{tabular}{|l | c | c | c | }
\textbf{Component Model}& \textbf{Component}& \textbf{Component Provides}	& \textbf{Provide/Require}\\

OSGi					& Bundle			& Package 					& Export/Require or Provide\\
Declarative Services	& Component			& Service 					& Provide/Reference\\
Eclipse					& Plug-in			& Extension-Point			& Extension/Requires\\
Fractal					& Component			& Interface 				& Server/Client\\
Maven					& Artifact			& \textit{typed} 			& NA/Dependency\\
Debian					& Package			& Virtual-Package 			& Provide/Depends\\
SOFA 2.0				& Frame				& \textit{typed} 			& Provide/Require\\
CUDF					& Package			& Package 					& Provides/Depends\\
\end{tabular}
\caption{Summary of presented component models' vocabulary. Note: \textit{typed} refers to the fact that what the component provides it typed in the meta-data.}
\label{background.comparisonvocab}
\end{table}

\begin{table}
\begin{tabular}{l | c | c | c  }
\textbf{Component Model}& \textbf{Hierarchical}& \textbf{Evolution Mechanism}	& \textbf{CDR}\\

OSGi					& No			& Method/Command Line 					& OBR\\
Declarative Services	& No			& Method 								& OBR\\
Eclipse					& No			& Method								& Eclipse P2\\
Fractal					& Yes			& Method 								& None\\
Maven					& No			& NA 									& Maven\\
Debian					& No			& dpkg 									& apt-get\\
SOFA 2.0				& Yes			& Evolution Patterns 					& None\\
CUDF					& No			& NA 									& MPM\\
\end{tabular}
\caption{Summary of presented component models' properties.}
\label{background.comparisonprop}
\end{table}


In the next chapter concepts such as component system evolution are formalised and further discussed.  



\chapter{Formal Component System Evolution}
\label{cdr}
The exact definition of component system evolution is required in order to express and study it.

The evolution of a system is made through iterative evolutionary steps over a period of time. 
As any evolutionary step can have many outcomes, the optimisation of the process is necessary.
This optimisation may also include many various criteria that must be combined and optimised for.
The overall evolution of a system can then be defined, and the process discussed in a formal light.
This evolution then requires a representation, a format in which to describe then evolve a system.

In this chapter, the formal definition of an evolutionary step is given and shown to be NP-Hard problem in section \ref{formal.step}.
Then the definition of an optimisation framework for this process is described in section \ref{formal.opt}.
The overall formal description of an evolving system is then defined in section \ref{formal.evo}.
Finally, the CUDF description language (previously discussed in chapter \ref{background}) is defined for this problem in section \ref{formal.cudf}.

\section{Formal Definition}
\label{formal.step}
Suppose a user wants to install a new text editor component into their component system.
The selected text editor has a dependency on a spell checker;
this means a spell checker component must be installed for the text editor to be functional.
There are three spell checker components available that satisfy this dependency,
this means that one of these spell checkers should be selected for installation as well.

Or suppose, the use already has the text editor installed, and wants to update it to a newer version to increase stability and functionality.
There are two newer versions available, however the newest version of the text editor requires the system to install a new spell checker,
where the lesser up to date text editor can use the current spell checker.
This requires a selection of which to install, based on the users objectives for the system.

The first step in solving these problems is to represent them formally, this allows the study of the problems properties.
This definition must include some way to choose between solutions, an optimisation framework that lets the users criteria be expressed.

In this section, a formal definition for a evolutionary step of a component based system is given.
This definition includes the ability to be optimised to choose a solution using a distance metric.
This problem is then shown to be NP-Complete in finding a solution and NP-Hard when finding the best solution.

\subsection{Formal Definition}
%%%The formal definition of CDR is an expansion from our CBSE 2011 paper.
First the space of components is denoted as $\mathbb{C}$, each described by a name and version, such that no two components can have the same name and version.
The evolutionary step function is then defined as $C_R: 2^\mathbb{C} \rightarrow 2^\mathbb{C}$, 
such that it takes a set of components and returns a set of components that satisfy all constraints $R$.
For instance, given the set of components $X = \{a,b\}$ and a set of component constraints $R$,
$C_R(X)$ would try to evolve the set $X$ to a set of components that satisfy $R$.

%%%The four constraints can be keep constraints, dependency requirements, or conflict constraints, and request constraints//
The constraints in $R$ can be either: 
\begin{itemize}
  \item keep requirement in the form $a \vee \ldots \vee b$, e.g. $a \vee b$ means the $a$ or $b$ must be in the system
  \item dependency requirements in the form $a \rightarrow c_1 \vee \ldots \vee c_n$, e.g. $a \rightarrow b \vee c$ means $a$ depends on $b$ or $c$
  \item conflict constraints in the form $a \rightarrow \neg c$, meaning $a$ conflicts with $c$
  \item request constraints in the form of $(a_1 \vee \ldots \vee a_i) \wedge \ldots \wedge (z_1 \vee \ldots \vee z_j)$ is a conjunction of clauses, each clause a disjunction of components.
\end{itemize}
%%%Semantics of the constraints are then defined.
These constraints are defined semantically such that
a keep requirement  $a \vee \ldots \vee b$ means $a \in C_R(X)$ or $b \in C_R(X)$ (note: this can be unit i.e. $a$ is a valid keep constraint),
a dependency constraint $a \rightarrow c_1 \vee \ldots \vee c_n$ means if $a \in C_R(X)$ then $c_1 \in C_R(X)$ or \ldots or $c_n \in C_R(X)$,
and a conflict constraint $a \rightarrow \neg c$ means if $a \in C_R(X)$ then $c \not \in C_R(X)$.
The request constraint describes the changes that are necessary for the system.
These types of constraints are expressive enough to describe many component system evolution problems.

%%%We extend this definition of CDR to include optimisation through some measurement of improvement over the previous system.\\
CDR is expressed w.r.t. a heuristic using a distance function $d(\mathbb{C}_1,\mathbb{C}_2)$ 
that returns a measure of distance between two sets of components as a non-negative real number.
Given a set of user-requests $X$, a system $C_R(X)$ and new user-request $\Delta$,
$C_R(X \cup \Delta) = \mathbb{C}_1$ iff there does not exist another system $\mathbb{C}_2$ such that $X \subset \mathbb{C}_2$, $\mathbb{C}_2$ satisfies $R$, 
and $d(C_R(X),\mathbb{C}_2) < d(C_R(X),\mathbb{C}_1)$.
That is, if two possible systems exist that $C_R$ could return, the system with the least distance from the original system is returned. 
By defining $d$ we can alter CDR w.r.t. our heuristics.

%%%This optimisation function can be simplified to a function where the previous solution is not compared, but some static metric of the system is measured.

\subsection{NP-Complete Proof}
%%%Here we reduce the problem to SAT, which is simple as the constraints are very similar.

%%%We show that it is not a 'simple' SAT problem, due to conflict constraints, therefore it is NP-Complete.

%%%This proof has been published before; list of papers.

\section{Optimisation}
\label{formal.opt}
%%%This section is based on our APSEC 2011 paper, though we expand the ranking function to include an earlier system (dynamic).\\
%%%CDR can return potentially enormous amount of solutions, each has different properties that may make it more or less desirable than another, we need criteria to find an optimal solution.
Any given CDR problem can be resolved to a potentially enormous number of solutions \cite{Jenson2010},
and as each solution has different properties, suitable criteria should be included in the selection of an optimal solution.

%%%Information from differnet loactions and meta data can be used to define the criteria.
Information from component meta-data, component contracts \cite{Watkins1999}, repositories \cite{Guo2000}, 
composition testing \cite{XuejieZhang2008} and user preferences can be used to calculate these optimal solutions.
Additional criteria, such as size, reliability or licence, may change a solutions' desirability, 
therefore optimising for them will help return a better solution.

%%%This problem is multi-criteria as users will likely have more than one criteria to judge a solution by.

\subsection{Multi-Criteria CDR}
%%%Combining criteira is a challenge because criteria have different domains and orders, some have partial orders, arbitrary rankings, criterion orderings.
Optimising additional criteria proposes some challenges; firstly criteria may have different domains and orders, 
e.g., minimising the size in bytes while maximising percentages of reliability. 
Some criteria may have partial orders where two solutions can be incomparable,
e.g., one licence may not be better or worse than another.
Criteria may have arbitrary rankings, 
e.g., preferences for licences can differ significantly between users.
A criterion could also be more important than another, 
e.g., correct functionality of a system will always be more important than its size.

%%%A naive solution is to map all criteria to real numbers, then aggregate them through addition multiplication with weights.
A naive solution for this problem is to create a function to map each of the criteria to a real number.
Then by aggregating these numbers (through addition or multiplication) with user defined weightings 
(to express their preferences on properties), we create a single fitness function, 
and using this function we can search for optimal solutions.

%%%This has significant draw backs, some criteria are infinitly more important, it is not intuitive for users to weight.
This method has some significant drawbacks;
we lose some information in the conversion to the totally ordered real numbers,
making aspects like Pareto fronts \cite{pareto1964cours} (where multiple solutions are incomparable) impossible to represent.
The relationships between criteria are also difficult to define 
as complicated functions must be created to maintain proper ordering of solutions.
Asking users to assign preferences of criteria with arbitrary weights is not intuitive,
e.g., stating ``size is $.25$ more important than vendor'' does not convey meaningful information to the user about their weighting decision.

%%%We approach this problem formally using orders over partially ordered sets, where any sub set is bounded, this is a complete lattice.
The formal approach to solve this problem is to use orders over sets, 
where each criterion is a set ($A$) with an order ($\leq$).
Each criterion is then a partially ordered set (the order must be reflexive, antisymmetric and transitive).
Furthermore we require that any set of elements in a criterion must be bounded, 
i.e. have a least upper bound and a greatest lower bound.
A complete lattices satisfies these requirements, therefore is used as our representation. 

%%%We can then compose these complete lattice criteria using the product and lexicographic orders.

%%%This framework can then be used to describe and discuss the multicritera problem of CDR.
This framework defines criteria as a ranking system over a complete lattice.
The ranking system describes a mechanism in which to rank a solution to a complete lattice. 
The complete lattice gives an order to the solutions based on their ranking.
By combining these criteria (ranking systems and complete lattices) multi-criteria problems can be composed.
Then using a general algorithm that can search these structures, 
optimal solutions for any defined problem can be found.

\subsection{Complete Lattice}
%%%Complete Lattices have been used in other optimisation frameworks.
Complete lattices are an effective abstraction to represent rankings,
and have been used before when describing such criteria \cite{Bistarelli1997,Fernandez}.
They can represent problems that involve partial orders,
be combined into new lattices,
and be easily visualised and presented using Hasse diagrams \cite{davey1990introduction}.
They also have strict bounding, which enables general methods, such as branch and bound \cite{Land1960}, to find solutions.

%%%The formal definition for comlpete lattices is a partially ordered set where all subsets have a unique least upper bound and unique greatest lower bound
A complete lattice is a partially ordered set where all subsets have a unique least upper bound and unique greatest lower bound.
Formally we define:
\begin{defs}
A complete lattice is a pair $L = \langle A, \leq \rangle$ where $A$ is a non empty set 
and $\leq \subseteq A \times A$ is a relation such that:
\begin{itemize}
  \item $\leq$ is a partial order
  \item for all subsets of $A$ there exists a unique least upper bound (join or $\vee$) and a unique greatest lower bound (meet or $\wedge$)
\end{itemize}
\end{defs}

%%%The domiain of the complete lattice represents the value domain of the criteria.
The domain of the complete lattice ($A$) defines the value space (e.g., probability, cost, \ldots) 
that is used to rank solutions. 
The core attribute of a complete lattice is that all subsets of the lattice have both a unique least upper bound (join or $\vee$) 
and a unique greatest lower bound (meet or $\wedge$).
Using these operators we can define the greatest ($1$) and least ($0$) elements as
the join of the entire domain ($\bigvee A = 1$) and the meet of the domain ($\bigwedge A = 0$).

\subsection{Ranking and Solution}
%%%We define a ranking system and function, which returns a value for the complete lattice domain.
An optimal solution can now be defined through a ranking system with respect to complete lattices.

%%%A ranking system is a lattice a finite domain of variables, and their assignment domain
\begin{defs}
Consider a complete lattice $L = \langle A, \leq \rangle$; we define 
a ranking system over this lattice as a triple $RS = \langle L, D, V \rangle$, 
where $V$ is a finite set of variables and $D$ is their finite domain.
\end{defs}

%%%Given we specifically look at components, the variables are components and their domain is \{T,F\}, whether they are installed or not
Given we are focused on the component dependency resolution problem,
this abstract definition of a ranking system can be further defined.
Within dependency resolution, 
each variable in this problem is a component and their domain is whether they are included or excluded from the final solution.
Therefore, $D = \{F,T\}$ and $V = C$, where $C$ is the set of components in the problem.

%%%A solution (system) is a mapping of variables to the domain
\begin{defs}
A solution is a variable binding $\gamma: V \rightarrow D$, a mapping of each variable to an element in the domain,
and $\Gamma$ is the set of all solution functions.
\end{defs}

This definition of solution does not take order into account, where a solution could be the worst possible solution.

%%%The ranking function, takes two solutions, and returns the distance between them 
%TODO increase the ranking function to take two solutions and return a distance between them
\begin{defs}
Consider a complete lattice $L = \langle A, \leq \rangle$ and a ranking system $RS = \langle L, D, V \rangle$.
A ranking function ($Rank$) maps the distance from the current solution and a new solution to elements from $A$,
$Rank: \Gamma \times \Gamma \rightarrow A$.
\end{defs}

%%%The set of optimal solutions is then the set of solutions where there are no better solutions
\begin{defs}
Consider a complete lattice $L = \langle A, \leq \rangle$, a ranking system $RS = \langle L, D, V \rangle$, a previous solution $\gamma$ and a ranking function $Rank$.
The set of optimal solutions ($OSol \subseteq \Gamma$) is the set of solutions where there is no other solution whose rank is better,
such that $OSol = \{ s \in \Gamma : \nexists s' \mbox{ such that } Rank(\gamma,s) < Rank(\gamma,s')\}$
\end{defs}

%%%The set of optimial solutions may still be undesirable, as it is only solutions that are available
The set of optimal solutions only includes the highest ranked solutions,
as with the definition of solutions, these optimal solutions could be undesirable.
For instance given a problem which is ranked with either $Correct$ or $Incorrect$, 
the optimal solution could be $Incorrect$ if there exists no solutions that are $Correct$. 

\subsection{Composition of Complete Lattices}
%%%This section is mostly identicle to the section in our APSEC paper, it just proves that combining two complete lattices with lexicographic or product order returns a complete lattice
\label{sec:composition}
As a complete lattice with a ranking system is the representation of a criterion,
the composition of criteria is the way of expressing their relationships to one another. 
We first define two separate composition operators of complete lattices resulting in the product and lexicographic orders.

Here we prove that the lexicographical and product compositions of complete lattices result in a complete lattice,
this enables any combination of these orders to be used and remain within our framework.
This is proven by deriving the join (unique least upper bound) and meet (unique greatest lower bound) operations,
and showing there exists a least element ($0$) within the lattice \cite{davey1990introduction}. 

We then define how the ranking systems are composed with these orderings.

\begin{defs}
\label{def:product_order}
Consider two complete lattices $L_1 = \langle A_1, \leq_1 \rangle$ and $L_2 = \langle A_2, \leq_2 \rangle$;
$L_1 \otimes L_2$ creates a product composition $L_P = \langle A_1 \times A_2, \leq_P \rangle$
where $\leq_P$ is a product order; 
$(a,b) \leq_{P} (a',b') \mbox{ iff } a \leq_{1} a' \mbox{ and } b \leq_{2} b'$.
\end{defs}

The product order of a classical lattice $L = \langle \{T,F\}, \leq \rangle$ 
composed with its self is given in the diagram in Figure \ref{fig:product}.
%\begin{figure}[htp]
%\begin{center}
%  \includegraphics[height=.1\textheight]{productorder.png}
%  \caption{Hasse diagram of the product composition of classical lattices}
%  \label{fig:product}
%\end{center}
%\end{figure}

The product order is trivially reflexive, anti-symmetric and transitive, therefore it is a partial order.
\begin{prop}
The product composition of two complete lattices is a complete lattice.  
\end{prop}
\begin{prof}
For any two points $(a,b)$ and $(a',b')$;
their join is the join of each element $(a \vee a', b \vee b')$, 
inversely their meet is the meet of each element $(a \wedge a', b \wedge b')$.
This lattice is also complete as its least bound is the least bound of each set $(0_1,0_2)$.
$\Box$\end{prof}

\begin{defs}
\label{def:lex_order}
Consider two complete lattices $L_1 = \langle A_1, \leq_1 \rangle$ and $L_2 = \langle A_2, \leq_2 \rangle$;
$L_1 \odot L_2$ creates a lexicographical composition  $L_L = \langle A_1 \times A_2, \leq_L \rangle$
where $\leq_L$ is a lexicographic order;
$(a,b) \leq_{L} (a',b')$ iff $a <_{1} a'$ or $(a = a'$ and $b \leq_{2} b')$.
\end{defs}

The lexicographic order of a classical lattice $L = \langle \{T,F\}, \leq \rangle$ 
composed with its self is given in the diagram in Figure \ref{fig:lex}.
%\begin{figure}[htp]
%\begin{center}
%  \includegraphics[height=.1\textheight]{lexicograhicorder.png}
%  \caption{Hasse diagram of the lexicographical composition of classical lattices}
%  \label{fig:lex}
%\end{center}
%\end{figure}

The lexicographical order is trivially reflexive, anti-symmetric and transitive, therefore it is a partial order.
\begin{prop}
The lexicographical composition of two complete lattices is a complete lattice.
\end{prop}
\begin{prof}
Consider two points $(a,b)$ and $(a',b')$; 
the join of $(a,b)$ and $(a',b')$ is $(a',b')$ if $a <_1 a'$,
$(a,b)$ if $a' <_1 a$,
$(a, b \vee b')$ if $a = a'$,
or if $a$ and $a'$ are incomparable ($a||a'$) it is $(a \vee a', 0_2)$
as this is an upper bound, and any other upper bound must be bigger than this.
The meet is defined dually where the meet of $(a,b)$ and $(a',b')$ is $(a',b')$ if $a' <_1 a$,
$(a,b)$ if $a <_1 a'$,
$(a, b \wedge b')$ if $a = a'$,
or if $a$ and $a'$ are incomparable ($a||a'$) it is $(a \wedge a',1_2)$
as this is a lower bound, and any other lower bound must be smaller than this.

This lattice is also complete as its least bound is the least bound of each set $(0_1,0_2)$.
$\Box$\end{prof}

There are two things to note about the lexicographical composition; 
first it does not create a distributive lattice i.e. $\wedge_L$ does not distribute over $\vee_L$.
\begin{prop}
The lexicographic composition does not result in a distributive lattice.
\end{prop}
\begin{prof}
Consider two pairs $(a,b)$ and $(a', b') \in L_1 \odot L_2$ where $a || a'$ and $a = a''$ and $a' || a''$,
then $(a,b) \wedge_{L} ((a', b') \vee_{L} (a'', b'')) = (a,b)$,
where under the same conditions $((a,b) \wedge_{L} (a', b')) \vee_{L} ((a,b) \wedge_{L} (a'', b'')) = (a,b \wedge_2 b'')$.
$\Box$\end{prof}

Secondly, two lattices (not complete lattices) lexicographically combined together may not be a lattice. 
This is another reason for the choice of complete lattices as the structure of representation.
\begin{prop}
The lexicographic composition of two lattices may not result in a lattice.
\end{prop}
\begin{prof}
Suppose a lattice $K$ has a non linear order, and a lattice $L$ has no least bound ($0_L$). 
For two pairs $(a,b)$ and $(a',b)$ in $K \odot L$ have no join if $a$ and $a'$ are incomparable.
This would be defined as $(a,b) \vee (a',b) = (a \vee a',0_L)$, as $0_L$ does not exist this join does not exist.
$\Box$\end{prof}

In figure \ref{fig:stacking} is presented some examples of how our composition operators can be stacked
by using a classical lattice ($L = \langle \{T,F\}, \leq \rangle$) composed in different ways.
In these examples we have simplified the domain for brevity, 
e.g., $\langle T, \langle T,T\rangle\rangle$ is represented with $\langle T,T,T\rangle$
%\begin{figure}[htp]
%\begin{center}
%  \includegraphics[width=.45\textwidth]{all.png}
%  \caption{Hasse diagrams of the composition of a classical complete lattice}
%  \label{fig:stacking}
%\end{center}
%\end{figure}

The composition of the ranking systems is defined as:
\begin{defs}
Consider two complete lattices $L_1 = \langle A_1, \leq_1 \rangle$ and $L_2 = \langle A_2,\leq_2 \rangle$ and
the ranking systems over them $RS_1 = \langle L_1,D,V \rangle $ and $RS_2 = \langle L_2,D,V \rangle$,
with defined ranking functions as $Rank_1: \Gamma \rightarrow A_1 $ and $Rank_2: \Gamma \rightarrow A_2$.
The combination $RS_1 \odot RS_2$ (resp. $RS_1 \otimes RS_2$) results in the ranking system 
$RS = \langle L_1 \odot L_2, D, V \rangle$ (resp. $RS = \langle L_1 \otimes L_2, D, V \rangle$), 
and the combination of the ranking functions is 
$Rank: \Gamma \rightarrow A_1 \times A_2$ where $Rank(\gamma) = \langle Rank_1(\gamma), Rank_2(\gamma) \rangle$.
\end{defs}

The main point in this definition is that to combine ranking systems they must have the same set of variables 
and the same domain.

\subsection{Semi-ring constraint incompatibility}
%%%Discussion of the related yet incompatible work C-semirings; how they inspired this work but due to lexicographic order we are unable to use it.
Instances of different complete lattices used for constraint solving can be found in a similar framework based on c-semirings, 
by Bistarelli et al. \cite{Bistarelli1997}.
They prove that all instances of their framework are complete lattices, and their $+$ operator coincides with join
and if their $\times$ operator is idempotent, it coincides with meet.
Their examples include fuzzy, weighted, probabilistic and valued constraint problems which can be used in our framework.
In their framework $\times$ must distribute over $+$, this gives a distributive lattice,
since lexicographic orders may create non-distributive lattices, our overall frameworks are incompatible.
The c-semiring framework is still greatly useful as c-semirings are complete lattices all instances of this framework will work in ours.


\section{Component System Evolution}
\label{formal.evo}

\section{Common Upgradeability Description Format}
\label{formal.cudf}

%%%Here we give an introduction to the CUDF format from Mancoosi, this format is used as a standard for Dependency Resolution problems
{}\cite{treinen2009common}

%%%It was made to standardise the problem, for two reasons, to make building solvers generalisable, and to compete solvers in the MISC competetions to compare solutions.

%%%In both regards it has succeded, there have been many competitions run, 
%%%and their creation of a Modular Package Manager(MPM) is currently moving through the Debian stages into their core repository

\subsection{CUDF Syntax}
%%%The CUDF syntax is made to resemble similar CDR meta-data through using key-value pairs (not XML, too verbose?)

%%%Package Formulas are the standard mechanism of dependence

%%%Versions as Integers

%%%The Syntax has been defined to be extensible within the language, so additional model specific information can be included without increasing the size of the standard

\subsection{CUDF Semantics}
%%%As CUDF was a format the was supposed to be open to allow people to build differnt solvers for it, and be used as a real package manager, it walked a fine line.

%%%It must be simple enough so that developers can quickly understand it to build or modify existing solvers capable of competing in MISC competitions

%%%It must be broad enough to define and solve real world problems





\chapter{Resolving \modelname}
\label{implementation}
\epigraph{What I cannot create, I do not understand.}
{\textit{Richard Feynman, 1988}}
The previous chapter presented the \modelname model that describes the evolution of a component system as a series of evolution steps.
Each step evolves the component system by satisfying a user request and system constraints with a new component system that is optimal with respect to some criteria.
Calculating the resulting systems of these evolution steps is described as \textbf{resolving} a \modelname instance.
Resolving a \modelname instance can require a significant effort due to the nature of the problem (as discussed in the previous chapter).

This chapter turns its attention a way of resolving a \modelname instance by mapping each of its evolution steps to, then solving, a \modelimpl problem.
\modelimpl uses lexicographically ordered pseudo-Boolean criteria 
to find an optimal solution to a Boolean satisfiability (SAT) problem extended with pseudo-Boolean (PB) constraints (SAT+PB) \citep{dixon2004automating}.
These relationships are presented in figure \ref{impl.modelsatdiagram}.

\begin{figure}[htp]
\begin{center}
\digraph[scale=0.5]{implsatgraph}{
rankdir=BT;
SAT [label=<<TABLE BORDER="0" CELLBORDER="1" CELLSPACING="5"><TR><TD WIDTH="150">SAT problem</TD></TR></TABLE>> shape=none];
PBCrit [label=<<TABLE BORDER="0" CELLBORDER="1" CELLSPACING="5"><TR><TD WIDTH="150">PB criteria</TD></TR></TABLE>> shape=none];
SATPB [label=<<TABLE BORDER="0" CELLBORDER="1" CELLSPACING="5"><TR><TD WIDTH="150">SAT+PB problem</TD></TR></TABLE>> shape=none];
SS [label=<<TABLE BORDER="0" CELLBORDER="1" CELLSPACING="5"><TR><TD WIDTH="150">\modelimplwx</TD></TR></TABLE>> shape=none];
SSI [label=<<TABLE BORDER="0" CELLBORDER="1" CELLSPACING="5"><TR><TD WIDTH="150">:\modelimplwx</TD></TR></TABLE>> shape=none];
\modelnamewx [label=<<TABLE  BORDER="0" CELLBORDER="1" CELLSPACING="5"><TR><TD WIDTH="150">\modelnamewx</TD></TR></TABLE>> shape=none];
FMI [label=<<TABLE BORDER="0" CELLBORDER="1" CELLSPACING="5"><TR><TD WIDTH="150">:\modelnamewx</TD></TR></TABLE>> shape=none ];
SS -> PBCrit [label=uses];
SS -> SATPB [label=uses];
SSI -> SS [label=instanciates];
subgraph {
rank=same
	FMI -> SSI [ label="maps to 1..*" ];
}
SATPB -> SAT [label=extends];
FMI -> \modelnamewx [ label = "instantiates"];
}
  \caption{A description of the relationships of \modelimpl and \modelname}
  \label{impl.modelsatdiagram}
\end{center}
\end{figure}

The algorithm employed to solve \modelimpl problems is the novel lexicographic-iterative-strengthening algorithm.
This algorithm uses the iterative strengthening \citep{calistri1994iterative, le2010sat4j} and Davis-Putnam-Logemann-Loveland (DPLL) \citep{Davis1960, davis1962machine} algorithms.

To create an instance of \modelname to be resolved, some criteria must be defined. 
Some selected criteria are presented and defined, and their mapping from MOF criteria names, to \modelname criteria, and to PB criteria is presented. 

This chapter also describes the implementation GJSolver.
GJSolver takes a CUDF* document parses it into a \modelname instance (as described in appendix \ref{apx.cudf}),
and resolves it by following the process described further in this chapter.
It will also be shown how GJSolver was verified through MISC.

This chapter, in section \ref{impl.mapping} describes the \modelimpl problem and its mapping to an evolution step from a \modelname instance.
In section \ref{impl.algorithms}, the algorithms used to solve a \modelimpl problem, and resolve a \modelname instance are described.
Section \ref{impl.criteria} presents and defies the mapping of selected criteria.
Finally, in section \ref{impl.gjsolver} GJSolver is discussed, followed by the description of its verification through MISC.

\section{\modelimpl}
\label{impl.mapping}
A \modelname instance may consist of many evolution steps.
Each of these steps involves satisfying an evolution problem with an optimal solution defined using an evolution preference order.
To resolve a \modelname instance each evolution step is mapped to a \modelimpl problem.
A \modelimpl problem involves satisfying a SAT+PB problem with an optimal solution defined using lexicographically ordered PB criteria.
By mapping each evolution problem to a SAT+PB problem, and each evolution preference order to lexicographically ordered PB criteria,
a \modelname instance can be resolved.

This section describes the SAT problem and its extension to the SAT+PB problem by including pseudo-Boolean constraints.
It also presents the \modelimpl problem as consisting of a SAT+PB problem with PB criteria. 
The mapping from an evolution step of a \modelname instance to a \modelimpl problem is then described.

\subsection{Boolean Satisfiability Problem (SAT)}
Boolean satisfiability (SAT)  is the problem of determining if the variables in a Boolean equation can be assigned in such a way that the equation returns true.
SAT was the first identified NP-Complete problem \cite{cook1971}, meaning there is no known algorithm that efficiently solves all instances of SAT problems.
The fundamental difficulty of SAT problems, 
combined with the ability to map many problems to SAT, has spawned a community\footnote{http://www.satcompetition.org/ accessed 6/3/2012} 
dedicated to creating, enhancing, and testing various SAT solver implementations. 
SAT solvers have been used in various domains to address problems such as electronic design automation \citep{Marques-Silva2000}, 
model verification \citep{dennis2006}, and component system evolution \citep{leBerre2010}.

A common representation of a SAT equation is in Conjunctive Normal Form (CNF) \cite{Moskewicz2001}.
CNF is defined as a conjunction of clauses, 
where each clause is a disjunction of literals, e.g. $(a \vee b) \wedge (\neg b \vee c)$.

A SAT problem in CNF is defined as:
\begin{defs}
\label{impl.defSAT}
{\ }
\begin{enumerate}
    \item Let $V$ be a set of variables
    \item A \textbf{literal} is a variable $v$ or its negation $\neg v$.
    \item A literal $\neg \neg v \equiv v$
    \item Given a set of literals $P$, $\neg P \equiv \{\neg v \mid v \in P\}$
    \item A \textbf{clause} is a set of literals 
    \item A \textbf{formula} is a set of clauses \label{impl.whatisaformula}
    \item A \textbf{SAT problem} is a set of variables $V$ and a formula $F$
    \item A set of literals is \textbf{consistent} if for any variable $v$, the set of literals does not contain both $v$ and its negation $\neg v$.
    \item A clause $\mathcal{C}$ is \textbf{satisfied} by a set of literals $P$ if there exists a literal in $\mathcal{C}$ that is also in $P$.
    \item A \textbf{solution} to a formula $F$ is a consistent set of literals $P$, such that for every clause $\mathcal{C}$ in $F$, $\mathcal{C}$ is satisfied by $P$.
    \item A \textbf{partial solution} to $F$ is a subset of any solution, i.e. given $P$ is a solution, $P'$ is a partial solution iff $P' \subseteq P$.
    \item A SAT problem is \textbf{satisfiable} if there exists a set of literals $P$ that is a solution to $F$, otherwise the instance of the SAT problem is \textbf{unsatisfiable}. 
\end{enumerate}
\end{defs}

For example, consider a SAT problem where $V = \{a,b,c\}$ and $F = \{\mathcal{C}_1,\mathcal{C}_2\}$, where clauses $\mathcal{C}_1 = \{a,b\}$ and $\mathcal{C}_2 = \{\neg b, c\}$.
A solution for this problem could be $\{a,\neg b,c\}$ as $a \in \mathcal{C}_1$ and $\neg b \in \mathcal{C}_2$.
However, $\{a, \neg b, b\}$ is not a solution because it is not consistent, and $\{a, b, \neg c\}$ is not a solution because it does not contain a literal in $\mathcal{C}_2$.  


\subsubsection{Pseudo-Boolean Extension of SAT to SAT+PB}
A typical extension of the SAT problem is the inclusion of pseudo-Boolean constraints \citep{dixon2004automating} into SAT formula.
This extends the SAT problem to a SAT+PB problem.
PB constraints are a linear relation over Boolean variables.

A pseudo-Boolean function takes a set of Boolean literals and returns an integer, e.g. given a set of literals $P$, $f(P) = 4$.
\begin{defs}
\label{impl.PBfunction}
Given a tuple of literals $\langle l_1,\ldots,l_n \rangle$ and a tuple of integers $\langle a_1,\ldots,a_n\rangle$,
a \textbf{pseudo-Boolean function} takes a set of literals $P$, and returns an integer such that:

$f(P) = \sum \limits_{i=0}^n f_i(P)$
where $f_i(P) = \begin{cases} a_i & l_i \in P \\ 0 & l_i \not \in P \end{cases}$
\end{defs}
For example, consider the pseudo-Boolean function $f$ defined with a tuple of literals $\langle x_1, \neg x_2\rangle$ and a tuple of integers $\langle 1,3\rangle$.
The value of $f(\{x_1\})$ will equal $1$, $f(\{\neg x_2\})$ will equal $3$, and $f(\{x_1,\neg x_2\})$ equals $4$.

A pseudo-Boolean constraint is a relation between a PB function and an integer, e.g. given a set of literals $P$, $f(P) \leq 3$.
\begin{defs}
\label{impl.PBConstraint}
A \textbf{pseudo-Boolean constraint} is a tuple of a pseudo-Boolean function $f$, a relationship  $R$ in $\{\geq,>,\leq,<,=\}$, and an integer $k$, i.e. $\langle f,R,k\rangle$.
\end{defs}

\begin{defs}
A pseudo-Boolean constraint $\langle f,R,k\rangle$ is \textbf{satisfied} by a set of literals $P$ iff $ f(P)$ $R$ $k$.
\end{defs}
For example, the pseudo-Boolean function $f$ described in the above example can be combined with a relation $>$ and number $2$ to create the constraint $\langle f, >, 2 \rangle$.  
This constraint will be satisfied by the set of literals $P_1 = \{ x_1, \neg x_2 \}$ as $f(P_1) = 4$ and $4 > 2$, 
but not with the set of literals $P_2 = \{ x_1,  x_2 \}$ as $f(P_2) = 1$  and $ 1 \not > 2$.

A SAT problem can be extended to include pseudo-Boolean constraints by modifying definition \ref{impl.defSAT}.
This extension is accomplished first by adding the definition of what a PB constraint is, and what satisfies a PB constraint:

\textit{
\begin{enumerate}
\setcounter{enumi}{12}
  \item A \textbf{pseudo-Boolean constraint} is a tuple of a pseudo-Boolean function $f$, a relation $R$ in $\{\geq,>,\leq,<,=\}$, and a number $k$, i.e. $\langle f,R,k\rangle$.
  \item A pseudo-Boolean constraint $\langle f,R,k\rangle$ is \textbf{satisfied} by a set of literals $P$ iff $ f(P)$ $R$ $k$.
\end{enumerate}
}

Secondly, by modifying the definition of what a formula can contain and what a solution to it is:
\textit{
\begin{enumerate}
\setcounter{enumi}{5}
    \item A \textbf{formula} is a set of clauses and pseudo-Boolean constraints
\end{enumerate}
\begin{enumerate}
\setcounter{enumi}{9}
    \item A \textbf{solution} to a formula $F$ is a consistent set of literals $P$, such that for every clause $\mathcal{C}$ in $F$, $\mathcal{C}$ is satisfied by $P$,
    and for every pseudo-Boolean constraint $pb$ in $F$, $pb$ is satisfied by $P$. 
\end{enumerate}
}

To illustrate how the modified definition from SAT to SAT+PB problem works, 
consider a SAT+PB problem where $V = \{a,b,c\}$ and $F = \{\mathcal{C}_1,\mathcal{C}_2, \langle f, <, 2 \rangle\}$. 
Clauses $\mathcal{C}_1 = \{a,b\}$ and $\mathcal{C}_2 = \{\neg b, c\}$, 
and the function $f$ is defined with the tuple of literals $\langle a, b\rangle$ and a tuple of integers $\langle 1,3\rangle$.
A solution for this problem could be $\{a,\neg b,\neg c\}$ as $a \in \mathcal{C}_1$, $\neg b \in \mathcal{C}_2$, and $f(\{a,\neg b,\neg c\}) < 2$.
However, $\{\neg a, b, c\}$ is not a solution because $f(\{\neg a, b, c\}) \not < 2$. 

Pseudo-Boolean constraints can be translated into CNF clauses.
However, with the extension to include pseudo-Boolean constraints problems can be exponentially more concise, 
and more quickly solved \citep{dixon2004automating}.
Additionally, only minor amendments may be required to allow SAT solving algorithms (e.g. DPLL) to also solve SAT+PB problems.
Some such amendments are described by \cite{Sheini2006}.

\subsection{\modelimpl Problem}
\label{impl.lexsatproblem}
In order to create a \modelimpl problem, some criteria used to define what properties an optimal solution have are required.
Pseudo-Boolean criteria use a PB function to define an optimal solution.

\begin{defs}
A \textbf{pseudo-Boolean criterion} $\mathfrak{crit}$ is a tuple consisting of a pseudo-Boolean function $f$,
a relation over integers $R$ that is either $<$ or $>$, and a SAT+PB formula $I$,
i.e. $\mathfrak{crit} = \langle f, R , I \rangle$.
\end{defs}
The formula $I$ is used to define the auxiliary variables that represent values within the problem, but do not alter the problem.
Auxiliary variables have been used before in similar implementations \citep{argelich2010solving}.


\modelimpl uses lexicographically ordered PB criteria to find an optimal solution.
The lexicographic order is defined $(a,b) > (a',b')$ iff $a$ is greater than $a'$ or $(a$ equals $a'$ and $b$ is greater than $b')$.
These terms are translated to the \modelimpl domain.
\begin{defs}
Given a SAT+PB formula $F$, a pseudo-Boolean criteria $\langle f,R, I \rangle$, and two sets of literals $P$ and $P'$, 
$P$ and $P'$ are \textbf{equal} w.r.t. $\langle f,R, I \rangle$
iff $P$ and $P'$ are solutions to $F \cup I$ and $f(P) = f(P')$.
\end{defs}
Note that both $P$ and $P'$ must be solutions to the formula $F \cup I$, not just the formula $F$.
This is done to ensure that the auxiliary variables required by the criteria are available.

\begin{defs}
Given a SAT+PB formula $F$, a pseudo-Boolean criteria $\langle f,R, I \rangle$, and two sets of literals $P$ and $P'$, 
$P$ is \textbf{greater than}  $P'$ w.r.t. $\langle f,R, I \rangle$
iff $P$ and $P'$ are solutions to $F \cup I$ and $f(P')$ $R$ $f(P)$.
\end{defs}

These definitions are used to define the lexicographic order:
\begin{defs}
Given a formula $F$, lexicographically ordered PB criteria $\langle \mathfrak{crit}_1,\ldots,\mathfrak{crit}_n \rangle$, and two sets of literals $P$ and $P'$,
the $P$ is \textbf{lexicographically greater than}  $P'$ w.r.t. to $\langle \mathfrak{crit}_1,\ldots,\mathfrak{crit}_n \rangle$
iff there exists an $i$ between $1$ and $n$ where for all $j < i$, $P$ is equal to $P'$ w.r.t. to $\mathfrak{crit}_j$ and $P$ is greater than $P'$ w.r.t. $\mathfrak{crit}_i$.
\end{defs}

This lexicographic order is then used to define an optimal solution:
\begin{defs}
Given a formula $F$ and lexicographically ordered PB criteria $\langle \mathfrak{crit}_1,\ldots,\mathfrak{crit}_n \rangle$
an \textbf{optimal solution} is a solution $P$ to $F$ 
where no other solution $P'$ to $F$ exists such that $P'$ is lexicographically greater than $P$ w.r.t. to $\langle \mathfrak{crit}_1,\ldots,\mathfrak{crit}_n \rangle$. 
\end{defs}

A \modelimpl problem consists of:
\begin{itemize}
  \item a tuple of lexicographically ordered PB criteria $\langle \mathfrak{crit}_1,\ldots,\mathfrak{crit}_n \rangle$
  \item a SAT+PB formula $F$
\end{itemize}
A solution to \modelimpl problem is the optimal solution to $F$ w.r.t. to its PB criteria.

\subsection{Mapping \modelname Instance to \modelimpl Problem}
\label{impl.mapping}
Following from the definition of \modelname in section \ref{formal.step}, 
an instance of \modelname consists of a series of evolution steps at time $t_i$, where $i$ is from $1$ to $n$.
Each step consists of:
\begin{itemize}
  \item a time $t_i$
  \item the set of components $\mathbb{C}_{t_i}$ 
  \item an evolution problem $\delta_{t_i} \cup \omega_{t_i}$
  \item an evolution preference order $\prec_{\alpha_{t_{i-1}}}$
  \item a previous system $\alpha_{t_{i-1}}$.
\end{itemize}

To resolve a \modelname instance each step is mapped to a \modelimpl problem which consists of a SAT+PB formula $F$ 
and lexicographically ordered PB criteria $\langle \mathfrak{crit}_1,\ldots,\mathfrak{crit}_n \rangle$.

The set of components $\mathbb{C}_{t_i}$ are variables in the problem, i.e $\mathbb{C}_{t_i} \subseteq V_{t_i}$.

\begin{defs}
	A component system $\alpha_{t_{i-1}}$ is mapped to a set of literals where
   $\alpha_{t_{i-1}} := \alpha_{t_{i-1}} \cup \{\neg c \mid c \in \mathbb{C}_{t_i}$ and $ c \not \in \alpha_{t_{i-1}}\}$.
\end{defs}
A component system $\alpha_{t_{i-1}}$ is mapped to a set of literals that is the union of $\alpha_{t_{i-1}}$ and the set of negative literals of components not in $\alpha_{t_{i-1}}$.

The reverse mapping, from a set of literals to a component system is:
\begin{defs}
A set of literals $P$ is mapped to a component system $\beta$ such that $\beta := \{c \mid c \in \mathbb{C}_t$ and $x \in P\}$
\end{defs}
A component system is the set of components that are not negative in the set of literals.  
Note: the mapping from literals to components is surjective, and the mapping from components to literals is injective.


The evolution problem $\delta_{t_i} \cup \omega_{t_i}$ is mapped to the formula $F$
by mapping each constraint to a SAT clause or PB constraint:
\begin{enumerate}
  \item \textbf{Exclusion}: $\neg a := \{\neg a\}$
  \item \textbf{Conflict}: $a \rightarrow \neg c := \{\neg a, \neg c\}$ 
  \item \textbf{Inclusive Disjunction}: $a_1 \vee \ldots \vee a_n := \{a_1, \ldots,  a_n\}$ 
  \item \textbf{Dependence}: $a \rightarrow c_1 \vee \ldots \vee c_n := \{\neg a, c_1, \ldots, c_n\}$
  \item \textbf{Exactly One}: $a_1 + \ldots + a_n = 1 := $ a pseudo-Boolean constraint $\langle f,=, 1 \rangle$, 
  where $f$ is defined with the tuple of literals $\langle a_1 ,\ldots , a_n\rangle$ and the tuple of natural numbers $\langle 1_1,\ldots,1_n \rangle$.
\end{enumerate}


\subsection{Evolution Preference Order Mapping}
The evolution preference order $\prec_{\alpha_{t_{i-1}}}$ defined with a lexicographic composition of criteria $crit_{1} \oplus \ldots \oplus crit_{n}$ 
can be mapped to a tuple of PB criteria $\langle \mathfrak{crit}_1,\ldots,\mathfrak{crit}_n\rangle$
by mapping each criterion $crit_{i}$ to a PB criterion $\mathfrak{crit}_i$.

A criterion $\langle rank_{\alpha} ,\leq \rangle$ maps to a PB criteria $\langle f, R , I \rangle$ iff:
\begin{itemize}
  \item $R$ equals the strict order of $\leq$.
  \item given a solution $P$ to formula $I$, and $P$ maps to the component system $\beta$, $f(P) = rank_{\alpha}(\beta)$. 
\end{itemize} 

To map an evolution preference order to the PB criteria,
it must be defined with the lexicographically composed criteria.
To map a criterion to a PB criterion, it must satisfy the above constraints.
This means that:
\begin{itemize}
  \item not all evolution preference orders can be described with lexicographically ordered PB criteria
  \item not all criteria can be mapped to PB criteria
\end{itemize} 
Therefore, the mapping from evolution preference orders to PB criteria is a partial mapping.

The partial mapping from \modelname criteria to PB criteria is not presented here.
This is because many of the \modelname criteria that can be defined are not useful, 
e.g. a criteria that maximises the number of components whose name starts with the letter \texttt{a}.
Only specific criteria are mapped to PB criteria, these are presented in the next section. 

To further illustrate the mapping from a \modelname criteria to a PB criteria, an example is presented.
Consider two components $a$ and $b$, a criterion $\langle rank_{\alpha},< \rangle$ whose ranking function $rank_{\alpha}$ is defined as:

$rank_{\alpha}(\beta) = \begin{cases} 1 & a \in \beta \text{ or } b \in \beta\\ 0 & \text{otherwise} \end{cases}$

This criterion expresses the preference of having either components $a$ or $b$ (or both) in the system.

Further consider the pseudo-Boolean criterion $\langle f, < , I \rangle$.
The auxiliary variable $x$ is defined such that $x \Leftrightarrow a \in \beta \vee b \in \beta$.
This variable must be  converted to the set of CNF clauses, and included in $I$,
i.e. $I = \{\{\neg x, a,b\}, \{\neg a,x\}, \{\neg b, x \}\}$.
The PB function $f$ is defined with the tuple of literals $\langle x \rangle$ and natural numbers $\langle 1 \rangle$.

The criterion $\langle rank_{\alpha}, \leq \rangle$ maps to $\langle f, < , I \rangle$,
as the strict order $<$ from the criterion equals the order $<$ in the PB criterion.
Given a solution $P$ to $I$, where $P$ maps to the component system $\beta$, $f(P) = rank_{\alpha}(\beta)$.
This can is shown in table \ref{impl.critmapexmp}.
\begin{table}[h!]
\centering
\begin{tabular}{| c | c | c | c |}
\hline
$P$                                &    $\beta$            & $rank_{\alpha}(\beta)$     & $f(P)$\\ \hline    
$\{\neg a, \neg b, \neg x\}$     & $\{\}$                & 0                        & 0 \\
$\{\neg a,  b, x\}$             & $\{b\}$                & 1                        & 1 \\
$\{ a,  \neg b, x\}$             & $\{a\}$                & 1                        & 1 \\
$\{ a,  b, x\}$                 & $\{a,b\}$                & 1                        & 1 \\ \hline
\end{tabular}
\caption{Values to show $f$  maps to $rank_{\alpha}$}
\label{impl.critmapexmp}
\end{table}

The above example describes how a \modelname criterion can be mapped to a PB criterion.
In the following section, some specific criteria are presented and mapped.

\section{Criteria}
\label{impl.criteria}
This section presents a selection of criteria and their mapping to between \modelname criteria, PB criteria, and MOF.
They are presented in two general categories, minimising change to the system, and maximising the versions of components.
The minimising change criteria were developed by Mancoosi and the criterion used to maximise versions was developed for Eclipse P2 solver \citep{leBerre2010}.

\subsection{Change Criteria}
Simple definitions of criteria to minimise change can often have negative effects during CSE.
For example, the most direct measurement of change of a component system is the measurement of the total changed components.
This is defined as a \modelname criterion:
\begin{defs}
	The \textbf{changed components} criteria is defined as $crit_{changec} = \langle rank^{changec}_{\alpha}, \leq \rangle$,
	where $rank^{changec}_{\alpha}(\beta) = |\alpha \Delta \beta|$.
\end{defs}
That is, the number of components in the symmetric difference between component systems is the minimised with this criterion.
This criterion's measurements are an inaccurate representation of the risks of changing a system.
For example, a component being replaced by another version of itself is seen as less risky than it being replaced with an entirely different component.  
Yet using the $crit_{changec}$ criterion, these would be seen as equivalent changes.

To define a change criterion that takes into account changing between components of the same name, is less risky, a components name can be considered.
For this purpose the function $V$ is defined as: 
\begin{defs}
The function $V: 2^{\mathbb{C}} \times \mathcal{N}$ takes a set of components $\alpha$ and a component name $n$, and returns a set of components with name $n$ that are in $\alpha$,
i.e. $V(\alpha,n) = \{\langle n',v \rangle \mid \langle n',v \rangle \in \alpha $ and $ n' = n \}$
\end{defs}

This function can be used to define a criterion that considers the name of a component:
\begin{defs}
	The \textbf{change} criterion is defined as $crit_{change} = \langle rank^{change}_{\alpha}, \leq \rangle$,
	where $rank^{change}_{\alpha}(\beta) = |\{n \mid n \in \mathcal{N}$ and $V(\alpha,n) \neq V(\beta,n) \}|$.
\end{defs}

This criterion can be altered to also consider new and removed components:
\begin{defs}
	The \textbf{new} criterion is defined as $crit_{new} = \langle rank^{new}_{\alpha}, \leq \rangle$,
	where $rank^{new}_{\alpha}(\beta) = |\{n \mid n \in \mathcal{N}$ and $V(\alpha,n) = \emptyset$ and $V(\beta,n) \neq \emptyset\}|$.
\end{defs}

\begin{defs}
	The \textbf{removed} criterion is defined as $crit_{removed} = \langle rank^{removed}_{\alpha}, \leq \rangle$,
	where $rank^{removed}_{\alpha}(\beta) = |\{n \mid n \in \mathcal{N}$ and $V(\alpha,n) \neq \emptyset$ and $V(\beta,n) =\emptyset\}|$.
\end{defs}

The change, new and remove criteria were taken directly from the MISC competitions definition of criteria.

The mapping between these criteria, PB criteria and MOF are presented in table \ref{impl.ccritmapping}.
A full description of the mapping is presented in appendix \ref{apx.critmapping}.
\begin{table}
\begin{tabular}{c | c | c}
\textbf{MOF name} 		& \textbf{\modelname criterion} & \textbf{PB criterion} \\
\texttt{-changed} 	& $crit_{change} = \langle rank^{change}_{\alpha}, \leq \rangle$ & $\langle f_{change}, <, I_{changed} \rangle$ \\
\texttt{-removed} 	& $crit_{removed} = \langle rank^{removed}_{\alpha}, \leq \rangle$ & $\langle f_{removed}, <, I_{removed} \rangle$ \\
\texttt{-new} 	& $crit_{new} = \langle rank^{new}_{\alpha}, \leq \rangle$ & $\langle f_{new}, <, I_{new} \rangle$ \\
\end{tabular}
\caption{Mapping between these elements}
\label{impl.ccritmapping}
\end{table}

\subsubsection{Version criteria}
%%%Maximise version, just sum up all versions and maximise their value
Defining how up-to-date a component system is, is a difficult task.
The most common way is to use a components version.
When using a components versions to measure how up-to-date a component system is, two properties of component versions must be considered:
\begin{itemize}
  \item A version of a component can only be compared to version of another component with the same name, 
  i.e. given two components $\langle n,v \rangle$ and  $\langle n',v' \rangle$ comparing $v$ and $v'$ is only useful if $n = n'$.
  \item The sum of versions is not a useful metric as two lesser versions will not be better than one greater version, 
  e.g. given components $\langle n,2 \rangle$,  $\langle n,3 \rangle$ and  $\langle n,4 \rangle$,
  a system with only $\langle n,2 \rangle$ and $\langle n,3 \rangle$ is not better than a system with just $\langle n,4 \rangle$. 
\end{itemize}

These problems can be seen in the simple criterion to maximise the versions in a system:
\begin{defs}
	The \textbf{maxversion} criterion is defined as $crit_{maxversion} = \langle rank^{maxversion}_{\alpha}, \geq \rangle$,
	where $rank^{maxversion}_{\alpha}(\beta) = \sum_{c \in \beta} c.$\texttt{version}.
\end{defs}
This criteria does not consider either of a versions properties, it sums together versions of components with different names.

A useful criterion has been defined in Eclipse P2 that minimises a measurement of the out-of-dateness of a component name, in a component system.
This out-of-dateness measurement is a relative measurement of a component given what other components are available.
It is defined as:
\begin{defs}
The function $uptodatedistance$ takes a component $\langle n, v \rangle$ and a set of components $\mathbb{C}_t$ and returns the number of components with the same name and a greater version,
i.e. $uptodatedistance(\langle n, v \rangle,\mathbb{C}_t) = |\{\langle n, v' \rangle \mid \langle n, v' \rangle \in \mathbb{C}_t $ and $ v' > v \}|$
\end{defs}

By creating a criterion with this measurement, the measurement of a system's up-to-dateness respects the two version criteria.
\begin{defs}
	Given the set of components $\mathbb{C}_t$, the \textbf{uptodate distance} criterion is defined as $crit_{utdd} = \langle rank^{utdd}_{\alpha}, \geq \rangle$,
	where $rank^{utdd}_{\alpha}(\beta) = \sum_{c \in \beta} uptodatedistance(c,\mathbb{C}_t)$.
\end{defs}
That is, the measurement to be minimised is the number of components that have the same name and a greater version than a component that is currently installed. 

The mapping between these elements is presented in table \ref{impl.ccritmapping}.
A full description of the mapping is presented in appendix \ref{apx.critmapping}.
\begin{table}
\begin{tabular}{c | c | c}
\textbf{MOF} 		& \textbf{\modelname criterion} & \textbf{PB criterion} \\
\texttt{-uptodatedistance} 	& $crit_{utdd} = \langle rank^{utdd}_{\alpha}, \geq \rangle$ & $\langle f_{utdd}, <, I_{utdd} \rangle$ \\
\end{tabular}
\caption{Mapping between these elements}
\label{impl.ccritmapping}
\end{table}


\section{Solving a \modelimpl problem}
\label{impl.algorithms}
Above a description is given how to map and instance of \modelname to \modelimpl problems.
This section describes the novel lexicographic iterative strengthening algorithms to solve \modelimpl problems.
This algorithm uses the DPLL algorithm \citep{Davis1960, davis1962machine} to find a solution to a SAT+PB formula,
and the iterative strengthening algorithm \citep{calistri1994iterative, le2010sat4j} to find an optimal solution to a SAT+PB problem given a  single PB criterion.
This section also describes how a \modelname instance is resolved by mapping to \modelimpl problems.

\subsection{Davis-Putnam-Logemann-Loveland algorithm for SAT Solvers}
%%%A successful algorithm for solving SAT problems is the DPLL algorithm, here we describe it in overview
The Davis-Putnam-Logemann-Loveland (DPLL) algorithm \citep{Davis1960, davis1962machine} is a complete (meaning it will find a solution if one exists), 
backtracking-based search algorithm for solving SAT and SAT+PB problems.

DPLL takes a formula $F$ and a set of literals $P$ (described as a partial assignment), and returns a solution to $F$ if $P$ is a partial solution, 
otherwise returning \verb+UNSATISIFABLE+.
When \texttt{DPLL} is called without a value $P$, $P$ is defaulted to equal the empty set .
By first calling \texttt{DPLL} with $P$ as the empty set, then adding literals to $P$ and recursively calling itself; 
the DPLL function searches for whether a solution to the formula is satisfiable.
The DPLL algorithm in defined in figure \ref{impl.DPLL} (a slight modification of the algorithm presented in \citep{dixon2004automating}):
\begin{figure}[h]
\begin{center}
\begin{alltt}
function DPLL(\(F, P\)):
   \(P\) = unit-propagate(\(F, P\))
   if \(P\) is not consistent:
       then return UNSATISIFABLE;
   if \(P\) is a solution to \(F\):
       then return \(P\);
   \(l\) = decide\((P)\);
   \(answer\) = DPLL\((F, P \cup \{l\})\)
   if answer != UNSATISIFABLE 
       return answer
   else
       return DPLL\((F, P \cup \{\neg l\})\);
\end{alltt}
  \caption{Recursive DPLL algorithm}
  \label{impl.DPLL}
\end{center}
\end{figure}

\texttt{DPLL} first calls the \texttt{unit-propagation} function (further described in section \ref{impl.unit}) which derives literals that must be in $P$ if it is a solution.
Next DPLL checks whether $P$ is inconsistent, which means it is not a partial assignment.
Then DPLL checks if $P$ is a solution to $F$, at which point $P$ is returned.
The \texttt{decide} function (further discussed in section \ref{impl.litorder}) returns a literal $l$ that is not, nor whose negation is in $P$.
The literal $l$ is added to $P$, which is then checked to be a partial solution by recursively calling \texttt{DPLL}.
If $P$ with $l$ is a solution then the found solution is returned,
otherwise the search continues by adding $\neg l$ to $P$ and checking if it is a partial solution by calling \texttt{DPLL}.

\subsubsection{Unit Propagation}
\label{impl.unit}
The first line in the  DPLL algorithm calls the \texttt{unit-propagation} function.
This function uses the clauses in the formula, and the partial assignment to identify and add literals to $P$ that must be included if $P$ is to be a partial solution.

\begin{defs}
Given a partial assignment $P$, a clause $\mathcal{C}$ is called \textbf{unit} iff $\mathcal{C}$ is not satisfied by $P$, and $P$ contains all but one of the literals in $\neg \mathcal{C}$.
The literal whose negation is not in $P$ is called a \textbf{unit literal}. 
\end{defs}

For example, a clause $\{a,b,c\}$ is unit if the partial assignment contains $\neg b$ and $\neg c$ but neither $a$ or $\neg a$.
The literal $a$ is then a unit literal.

For a formula to be satisfiable given partial assignment, each unit literal must be included in the partial assignment,
because if their negation is included the clause is not satisfied by the partial assignment.
For example, given a formula $\{\mathcal{C}\}$, where $\mathcal{C} = \{a,b\}$;
given the assignment $\{\neg a\}$ the clause $\mathcal{C}$ is unit and unit literal is $b$.
If $\neg b$ were in the partial assignment, $\mathcal{C}$ would not be satisfied by $P$.
Therefore, $b$ must be in the $P$ for $\mathcal{C}$ to be satisfied.

The process of unit propagation is defined in figure \ref{impl.propagation}.
\begin{figure}[htp]
\begin{center}
\begin{alltt}
unit-propagate(\(F, P\)):
  while \(P\) is consistent and there exists a \(\mathcal{C} \in F\) that given \(P\) is unit:
    \(l\) = unit literal in \(\mathcal{C}\)
    \(P\) = \(P \cup \{l\}\)
  return \(P\)
\end{alltt}
  \caption{Pseudo code of Unit Propagation}
  \label{impl.propagation}
\end{center}
\end{figure}

\subsubsection{Decide}
\label{impl.litorder}
The function \verb+decide+ takes a partial assignmnet $P$ and returns a literal $l$ such that $l \not \in P$ and $\neg l \not \in P$.
That is, if $l =$ \verb+decide+$(P)$, then $\{l\} \not \in P$ and $\{\neg l\} \not \in P$.
This literal is the point which the algorithm branches.
The order in which the \verb+decide+ function selects literals to branch on is also known as the literal order.
This order greatly impacts the efficiency of DPLL, as selecting literals that are in a solution (if one exists) would quickly return the result.

\subsection{DPLL Advancements}
Though the DPLL algorithm is the basis of many modern SAT solvers, the actual implementations have been altered to increase efficiency.
Some changes, including the use of conflict learning, backjumping, and watched literals, are briefly described here.
This section is aimed at giving a broad overview of techniques used in current solvers.

\subsubsection{Conflict Learning and Backjumping}
Conflict learning \citep{stallman1976} is a technique to cache previously tried sets of assignments in order to stop re-solving the same sub-problems.
This is accomplished by remembering what clauses, also known as reasons, caused literals to be added to the partial assignment through unit-propagation. 
This process works by identifying a variable both inferred to be true and false,
then creating a new clause, known as the learnt clause, which stops that inconsistency being reached again.
This learnt clause is derived by disjoining the two reason clauses after removing both the references to the inconsistent variable.
This process is shown in figure \ref{impl.clauselearning}. 

\begin{figure}[htp]
\begin{center}
$\begin{array}{c}
\{a_1,\ldots,a_k, l\} \\
 \{b_1,\ldots,b_m,\neg l\}\\
\hline
\{a_1,\ldots,a_k, b_1,\ldots,b_m \}
\end{array}$
  \caption{Clause Learning. Where $\{a_1,\ldots,a_k, l\}$ is the reason for $l$, and $\{b_1,\ldots,b_m,\neg l\}$ is the reason for $\neg l$
   are used to create the learnt clause $\{a_1,\ldots,a_k, b_1,\ldots,b_m \}$.}
  \label{impl.clauselearning}
\end{center}
\end{figure}

For example, if the reason for the inferred literal $a$ is clause $\{a, b\}$, and the reason for $\neg a$ is clause $\{\neg a, c\}$,
then the learnt clause derived is $\{b,c\}$, and added to the formula.

Backjumping \citep{Gaschnig1979} is the technique which determines how far to up the search tree to backtrack when a conflict is found.
The higher up the tree the technique ``jumps'' to, the greater reduction of the search space.
The level at which the algorithm backjumps is typically the point at which the learnt clause becomes unit. 

More advanced methods of conflict learning occur by minimising the size of the learnt clauses, as presented in \citep{sorensson2009}.
This research describes search methods that use other reason clauses to find smaller, more succinct learnt clauses.
The smaller the clause, the more of the search tree is pruned and the more levels are backjumped through the search.

\subsubsection{Watched Literals}
As noted by studies into the efficiency of DPLL-based SAT solvers \citep{dixon2004automating}, unit propagation is where the bulk of the computation in DPLL occurs.
Attempts to increase the efficiency of this task was initially to find better heuristics \citep{JamesMCrawford1996} for the literal order, to encourage cascades of unit propagation.
These attempts were shown to work well on random SAT problems but be less efficient for large structured problems \citep{dixon2004automating}.

It was noted that within unit propagation most of the time was spent on identifying the unit clauses.
The naive approach to unit propagation is to examine every clause, and then every literal in the clause to find if it is unit or not.
A more efficient approach was proposed using watched literals \citep{Madigan2001}, where instead of examining each clause, 
the clauses maintain an index of all necessary literals and notify the algorithm when they become unit.
This ``don't call us, we will call you'' concept makes the efficiency of the unit propagation function less dependent on the amount of clauses in the formula.

Advances on watched literals have occurred through algorithms to maintain the index of literals, like that presented in \citep{Moskewicz2001}.
Such algorithms enable larger formulae to be solved without necessarily increasing the time to solve them.

\subsection{Iterative Strengthening}
The iterative strengthening algorithm  \citep{calistri1994iterative, le2010sat4j} 
is an anytime algorithm using constraint satisfaction to iteratively find better solutions to a SAT+PB formula w.r.t. a PB criteria.
This algorithm can be used to find optimal solutions to an evolution problem, given a pseudo-Boolean criterion.
This is done by first finding a solution, then iteratively adding constraints (created using the criterion) to ensure the next solution found will be better than the previous solution.
This is the strengthening process.
Strengthening continues until either the strengthened formula is found to be unsatisfiable, or the algorithm is interrupted, at which point the best solution currently found is returned. 
This algorithm is defined in figure \ref{impl.strength}.

\begin{figure}[htp]
\begin{center}
\begin{alltt}
iterative-strengthening(\(F\),\(\langle f\sb{\alpha}, R, I \rangle\)):
    \(F\) = \(F \cup I\)
    \(answer\) = DPLL(\(F\))
    if \(answer\) = UNSATISIFABLE:
        return UNSATISIFABLE
    do:
        \(model\) = \(answer\)
        \(J\) = strengthen(\(model\),\(\langle f\sb{\alpha}, R, I \rangle\))
        \(F\) = \(F \cup J\)
        \(answer\) = DPLL(\(F\))
    while not (interrupted() or  \(answer\) == UNSATISIFABLE)
    return \(model\) 
\end{alltt}
  \caption{Pseudo code of Iterative Strengthening Algorithm}
  \label{impl.strength}
\end{center}
\end{figure}

The first action in the \texttt{iterative-strengthening} is to include the formula $I$ that defines auxiliary variables used by the criterion.
This ensures that any solution returned by DPLL is also a solution to $I$.

The next action is to check if the formula is satisfiable.
This is accomplished by passing the formula to the \texttt{DPLL} function, and assigning its output to the variable $answer$.
If the output from \texttt{DPLL} is \texttt{UNSATISIFABLE}, then the algorithm stops and returns \texttt{UNSATISIFABLE}, as there are no solutions.

The main loop of this algorithm is then defined.
The first action in this loop is to assign the contents of the variable $answer$ to the variable $model$.
The variable $model$ is a set of literals, used as a store of the currently best found solution.

The function \texttt{strengthen} is then called to create a formula $J$.
The formula $J$ ensures that if \texttt{DPLL} is called with formula $F \cup J$, either:
\begin{itemize}
  \item a solution is returned that is better (w.r.t. the criterion) than the currently best found solution $model$. 
  \item \texttt{UNSATISIFABLE} is returned, showing that no better solution exists.
\end{itemize}

\begin{defs}
Given a set of literals $model$, and a criterion $\langle f, R, I \rangle$,
the function \texttt{strengthen} returns a formula $J$ consisting of a single pseudo-Boolean constraint, i.e. $ J = \{ \langle f,R,f(model) \rangle \}$.
\end{defs}
That is, \texttt{strengthen} returns a formula $J$,
that ensures any solution $P$, to the formula $F \cup J$, must have a better ( w.r.t. $R$) value of $f(P)$ than the previously best solutions value of $f(model)$.

The next steps are then to add the formula $J$ to $F$, then search for a new solution using \texttt{DPLL}.

The main loop will end given the condition that either the \texttt{interrupted} function returns $true$, or the \texttt{DPLL} function returns \texttt{UNSATISIFABLE}.
The \texttt{interrupted} method is typically defined to return false until some external input (like a user stopping the algorithm or a timer running out) is encountered.
The \texttt{interrupted} method has the additional responsibility of stopping the \texttt{DPLL} function, 
so that if at any point \texttt{interrupted} returns true the \texttt{DPLL} function immediately returns.
When the main loop ends, currently best found solution $model$ is returned.

\subsection{Lexicographic Optimisation}
\label{impl.lexiterstre}
The iterative strengthening algorithm can be used find optimal solutions to lexicographically ordered criteria
by iteratively strengthening the each criteria in order.
This lexicographic iterative strengthening algorithm is presented in figure \ref{impl.lexstrength}.

\begin{figure}[htp]
\begin{center}
\begin{alltt}
lexicographic-iterative-strengthening(\(F\),\(\langle{}\mathfrak{crit}\sb{1}, \ldots ,\mathfrak{crit}\sb{n}\rangle\)):  
    \(answer\) = DPLL(\(F\))
    if \(answer\) = UNSATISIFABLE:
        return UNSATISIFABLE
    \(i\) = 0
    \(model\) = \(answer\)
    do:
        \(i = i + 1\)
        \(K\) = lock(\(model\),\(\mathfrak{crit}\sb{i}\))
        \(F\) = \(F \cup K\)
        \(model\) = iterative-strengthening(\(F\),\(\mathfrak{crit}\sb{i}\))
    while not (interrupted() or \(i\) == \(n\))
    return \(model\) 
\end{alltt}
  \caption{Pseudo code of the Lexicographic Iterative Strengthening Algorithm}
  \label{impl.lexstrength}
\end{center}
\end{figure}

The first action of the \texttt{lexicographic-iterative-strengthening} algorithm is to check if the formula is satisfiable using the \texttt{DPLL} algorithm.
If it is unsatisfiable, this algorithm returns \texttt{UNSATISIFABLE}.

The variable $i$ is then defined, this variable is a counter used to select the criterion to be strengthened.
Also, $model$ is assigned to be the set of literals $answer$, and is used to store the currently best found solution.

The main loop of this algorithm is then defined.
This loop first increments the counter $i$, to select the appropriate criterion to be strengthened.

The function \texttt{lock} is then called to return a formula $K$.
The formula $K$ ensures that if \texttt{DPLL} is called with $F \cup K$:
\begin{itemize}
  \item a solution is returned that is not worse (w.r.t. the criterion $\mathfrak{crit}_i$) than the solution $model$. 
\end{itemize}

\begin{defs}
Given a set of literals $model$, and a criterion $\langle f, R, I \rangle$,
the function \texttt{lock} returns a formula consisting of a single pseudo-Boolean constraint, $\{ \langle f,=,f(model) \rangle \}$,
\end{defs}
That is, \texttt{lock} returns a formula $K$,
that ensures any solution $P$, to the formula $F \cup K$, must have the best value of $f$ found so far.

Note that $K$ does not effect the satisfiability of $F$, as $model$ is still a valid solution to $F \cup K$.

The formula $K$ is then added to the formula $F$, i.e. $F = F \cup K$,
then the \texttt{iterative-strengthening} algorithm is called to find an optimal solution to $F$ given the criterion $\mathfrak{crit}_i$.
As the formula $F$ is known at this point in the algorithm to be satisfiable, the only possible returned value from \texttt{iterative-strengthening}
is a solution that is no worse than $model$.
This means that the returned value could be equivalent to the previously defined solution, if there exists no better solution than what has already been found.

The main loop will iterate until either all the criteria have been optimised, or the function is interrupted.
When the loop ends, the currently best found solution $model$ will be returned.

Some enhancements to the implementation of the lexicographic iterative strengthening algorithm can be made.
For example, when the \texttt{iterative-strengthening} is called, it is known that $F$ is satisfiable. 
Therefore, checking its satisfiability again within the \texttt{iterative-strengthening} function is not necessary.

The lexicographic iterative strengthening algorithm is an anytime algorithm.
It has been designed to return a solution to a \modelimpl problem, even if it is interrupted.
The reason for this anytime behaviour is that \modelimpl problems can take an impractical amount of time to solve.
It is therefore practically necessary to limit the time this algorithm searches for an optimal solution, 
and interrupt it when this time limit is reached. 

\subsection{Resolving a \modelname instance}
Resolving a \modelname instance involves finding the series of component systems $\alpha_{t_1},\ldots,\alpha_{t_n}$.
To do this, each evolution step starting at $t_1$ and ending at $t_n$ is mapped to an \modelimpl problem and solved using the lexicographic iterative strengthening algorithm.
This resolver algorithm is presented in figure \ref{impl.resolver}.
\begin{figure}[h!t!]
\begin{center}
\begin{alltt}
resolver:  
    for \(t\sb{i}\) in \(t\sb{1}\ldots,t\sb{n}\):
        \(F\) and \(\langle{}\mathfrak{crit}\sb{1}, \ldots ,\mathfrak{crit}\sb{n}\rangle\) mapped from evolution step at \(t\sb{i}\)
        \(answer\) = lexicographic-iterative-strengthening(\(F\),\(\langle{}\mathfrak{crit}\sb{1}, \ldots ,\mathfrak{crit}\sb{n}\rangle\))
        if \(answer\) equals UNSATISFIABLE:
            \(\alpha\sb{t\sb{i}}\) = \(\alpha\sb{t\sb{i-1}}\)
        else:
            \(\alpha\sb{t\sb{i}}\) mapped from \(answer\)
    return \(\alpha\sb{t\sb{1}},\ldots,\alpha\sb{t\sb{n}}\) 
\end{alltt}
  \caption{Pseudo code of the Lexicographic Iterative Strengthening Algorithm}
  \label{impl.resolver}
\end{center}
\end{figure}

In this algorithm, the systems are calculated starting at time $t_1$ and stopping at time $t_n$.
For each evolution step at time $t_i$ is mapped to a SAT + PB formula $F$ and PB criteria $\langle \mathfrak{crit}_{1}, \ldots ,\mathfrak{crit}_{n} \rangle$.
The algorithm \texttt{lexicographic-iterative-strengthening} is then used to find an optimal solution to $F$ with respect to the PB criteria, and return $answer$.
Either $answer$ equals \texttt{UNSATISFIABLE} at which point the system $\alpha_{t_i}$ is assigned as the previous system $\alpha_{t_{i-1}}$ (according to definition \ref{forml.stepdef}).
Otherwise, $answer$ is a set of literals that can be mapped back to the component system $\alpha_{t_i}$.
This algorithm, once completed, returns the set of component systems $\alpha_{t_1},\ldots,\alpha_{t_n}$.


\section{GJSolver}
\label{impl.gjsolver}
GJSolver is the implementation of the process from a CUDF* document to a resolved \modelname instance.
GJSolver grew through the course of this research to satisfy the need for an implementation to study CSE.
This implementation takes a CUDF* document, parses it to an instance of \modelname, which is then resolved by \modelimpl.

This process is described in figure \ref{impl.modelofgjsolver}.
\begin{figure}[htp]
\begin{center}
\digraph[scale=0.5]{implgjsolver}{
rankdir=LR;
subgraph cluster_0 {
label="GJSolver";
style=filled;
color=lightgrey;
FMI [label=<<TABLE  BORDER="0" CELLBORDER="1" CELLSPACING="5"><TR><TD WIDTH="150">:\modelnamewx</TD></TR></TABLE>> shape=none];
\modelimplwx [label=<<TABLE  BORDER="0" CELLBORDER="1" CELLSPACING="5"><TR><TD WIDTH="150">\modelimplwx</TD></TR></TABLE>> shape=none]
CMI[label=<<TABLE BORDER="0" CELLBORDER="1" CELLSPACING="5"><TR><TD WIDTH="150">:CUDF*</TD></TR></TABLE>> shape=none];
CMI -> FMI [ label="parsed to" ];
FMI -> \modelimplwx [ label="mapped to" ];
}
}
  \caption[labelInTOC]{GJSolver implementation description}
  \label{impl.modelofgjsolver}
\end{center}
\end{figure}

In this section the implementation and the validation of GJSolver are discussed.

\subsection{GJSolver Implementation}
\label{impl.gjsolverimpl}
The first decision made about the design of GJSolver was to base it on another similar implementation, Eclipse P2 \citep{le_berre_dependency_2009,leBerre2010}.
Basing the design on an existing implementation allowed the reuse of tools, and most importantly the reduction in risks during implementation.
The basis of GJSolver on Eclipse P2 lead to the following choices:
\begin{itemize}
  \item Java as the main implementation language.
  \item SAT4J as the core SAT+PB solver.
  \item Optimisation using PB criteria.
\end{itemize}

The main aspects where GJSolver does not reuse, or replicate Eclipse P2 is because P2 is designed especially for the OSGi and Eclipse component model, where GJSolver is designed for CUDF*.
Some of the differences between Eclipse P2 and GJSolver are:
\begin{itemize}
  \item No OSGi or Eclipse specific code in GJSolver.
  \item The internal representation of components is not based on OSGi.
\end{itemize}

Another important difference between Eclipse P2 and GJSolver is that Eclipse uses the iterative strengthening algorithm to optimise.
Eclipse P2 does this by compressing the tuple of PB criteria into a single criteria to be optimised (further described by \cite{leBerre2010}).  
How the optimisation approach used by GJSolver compares to the approach used by Eclipse P2 has not been explored in this research.

\subsubsection{SAT4J}
Given the use of SAT4J in GJSolver, a brief background of its development is presented here.

MiniSAT presented in \citep{een2003}, is a simple SAT solver implementation written in C, and designed for speed and extensibility.
It uses the DPLL based conflict driven algorithm as discussed in section \ref{impl.DPLL}.
This solver has become popular and is the basis of many other SAT solvers due to its open source distribution.
This has also lead to a track in the 2011 SAT competitions\footnote{http://www.satcompetition.org/2011/ accessed 6/3/2012} that deals with only altering MiniSAT to increase performance.
This means that MiniSAT has been repeatedly validated for performance by third parties across many different SAT problems. 

SAT4J \citep{le2010sat4j} is a Java re-implementation, and extension, of MiniSAT in the Java programming language.
SAT4J has been extended to efficiently solve a variety of related problems to SAT, including SAT+PB problems.
Due to the easily modifiable and transparent implementation of SAT4J, it has been able to be adapted to be used in various domains.

\subsection{Verification of GJSolver}
\label{impl.verif}
The verification of GJSolver is through the MISC competition held in 2011, and organised by Mancoosi.
This competition compares CUDF solvers by asking them to correctly parse and resolve hundreds of CUDF problems, and return optimal solutions with respect to various criteria.
MISC was created to promote interest in the component system evolution problems.

By viewing a CUDF problem as a single evolutionary step in a CUDF* problem, GJSolver has been modified to also solve CUDF problems.
Using such a modification, GJSolver could then be entered into the Mancoosi International Solver Competition (MISC) competition.

GJSolver was entered twice into MISC, firstly in a MISC Live event, which is an interim competition held during the year;
secondly at the MISC 2011 event\footnote{The results for MISC 2011 were announced at the Workshop on Logics for Component Configuration\footnote{http://www.pps.jussieu.fr/~treinen/lococo/2011/ accessed 6/3/2012}.}, 
which is the main competition. 

The main difference to GJSolver as described above, and the GJSolver entered into the MISC competition occurs when the evolution problem is unsatisfiable.
In the evolution step definition (\ref{formal.stepdef}) the component system $\alpha_{t_{i-1}}$ is returned if the evolution problem is unsatisfiable.
However, in this case, MISC requires that a file with only the text \texttt{FAIL} is written to state that no solution was found.
This is to correctly score a solver returning an incorrect solution to an evolution problem, 
and a solver finding a problem unsatisfiable.

\subsubsection{Mancoosi International Solver Competition}
\label{impl.MISCDEF}
Given a goal of the GJSolver implementation is to compete in the MISC, the interface and standards defined for this competition must be followed.
How the entered solvers are executed, what environment they are executed in, and the output required are all important aspects.

%%%They are executed on the command line
The way in which the entered solvers are executed is standardised to allow the automation of the competition.
This standard requires the entered solvers to be able to be executed on the command line with three arguments, \verb+cudfin+, \verb+cudfout+ and \verb+criteria+.
These arguments are defined as:
\begin{itemize}
  \item \verb+cudfin+: is a relative path to a CUDF document (as specified in section \ref{formal.cudf}) that describes the problem to be solver.
  \item \verb+cudfout+: is a relative path to a non-existent file, which is created by the solver to output the solution.
  \item \verb+criteria+: is a Mancoosi optimisation format (as described in section \ref{formal.mancoosioptimisationformat}) description of the criteria to select an optimal solution. 
\end{itemize}
The format of the output file, located at the path defined with \verb+cudfout+ argument, is a list of packages serialised as a list of stanzas with package and version properties.

%%%The environment POSIX, with 5minutes 1GB of memory
The environment in which the solver is executed is a virtual machine running a GNU/Linux system in a x86 architecture with 1GB of RAM.
It contains a Java runtime environment, allowing the use of Java as a primary language.
The time in which the solver is allowed to run is five minutes, after this time the solver will be forcibly stopped.
This time limit ensures that the competition can be run in an practical time frame.

\subsubsection{Tracks and Scoring}
Each MISC competition is broken down into three possible tracks, where each track is defined by the criteria used.
The first basic track, is 'paranoid', the second more advanced track is 'trendy', and third track is 'user'.
Both 'paranoid' and 'trendy' have pre-defined criteria defined in MOF.
The 'user' track uses a set of pre-defined criteria, though the exact optimisation criteria is unknown before the competition.
This means that 'paranoid' and 'trendy' can have solvers tailored to their specific criteria, where the 'user' track cannot.
The exact criteria that is used for these tracks can be found on the MISC website\footnote{http://www.mancoosi.org/misc-2011/criteria/ accessed 12/5/2012}.

For each track, a set of solvers is entered.
Each track has a set of evolution problems defined in CUDF.
The solvers for each track are then called to return optimal solutions for all problems given the tracks criteria.

As MISC competitions were created to compare various solvers, a scoring system was developed. 
When a solver is given a CUDF problem and some criteria, the returned solution falls into one of three classes:
\begin{itemize}
  \item a \textbf{real solution} is any solution to the CUDF problem.
  \item \textbf{no solution} occurs when a solver finished without returning a solution. This can happen because of error, timeout, or there not being a satisfiable solution.
  \item an \textbf{incorrect solution} occurs when the solver returns an answer that is not a solution to the CUDF problem.
\end{itemize}

Given $m$ is the number of solvers that entered into the track, the scoring of a solvers solution to an individual CUDF problem is as follows:
\begin{itemize}
  \item a \textbf{real solution} is given $1$ point, with an additional $1$ point for every solver that found a better solution.
  \item for \textbf{no solution} $2\times m$ points are given
  \item for an \textbf{incorrect solution} $3 \times m$ points are given.
\end{itemize}
This means, if a solver returns an optimal solution to a CUDF problem, it will receive $1$ point.
However, if other better solutions are returned, then the solver could be given up to $m$ points.

For each track, all solvers in that track are assigned points based on their solutions to all problems in that track.
For a given track a solvers points are summed to give a final score.
If more than one solver has the same amount of points at the end of a track, then the time it took for them to find each solution is summed.
This total time value is used as the tie breaker.

\subsubsection{MISC Live}
%%%In the first competition we had only partially implemented much of the functionality, so we did not expect great results.
The MISC Live was entered when GJSolver was only partially implemented.
Therefore, the only track that was possible to enter was the 'paranoid' track.
The results for this track\footnote{http://mancoosi.org/misc-live/20101126/paranoid/ accessed 6/3/2012} where promising, though some improvements were necessary.
Due to the competition openly distributing all competition data, GJSolver's deficiencies were identified and corrected.

\subsubsection{MISC}
The main verification of GJSolver was through the MISC 2011 event.
In this event GJSolver was entered into all tracks of this event.
The 'paranoid' track had a total of 5 solvers, the 'trendy' track had a total of 6 solvers, 
and the 'user' track had a total of 4 solvers.
Each track was also entered by the solver which GJSolver is based on, Eclipse P2, and another efficient solver aspuncud.
These two solvers will form the basis of GJSolver's comparison.

The scores and the times for each of the track compared to that from Eclipse P2 and aspuncud are in table \ref{impl.misc2011}.
\begin{table}
\begin{tabular}{| l | c | c | c | c |}\hline
Track & \# of Problems & GJSolver & P2 & aspuncud\\ \hline
paranoid & 129 & (190 : 5,294) & (181 : 4,646) & (147 : 1,035) \\ \hline
trendy & 129 & (197 : 13,073) & (232 : 13,435) & (151 : 1,767) \\ \hline
user & 400 & (656 : 73,522) & (1392 : 87,956) & (1215 : 39,905) \\ \hline
\end{tabular}
\caption{Results from MISC 2011, results are (score:time in seconds)}
\label{impl.misc2011}
\end{table}

The winner for both 'paranoid' and 'trendy' tracks was the aspuncud solver.
The winner for the 'user' track was GJSolver.

\subsubsection{Analysis}
The results from the MISC competition show:
\begin{itemize}
  \item During the competition GJSolver had very consistent results.
These results allowed it to compete with the other solvers, and win the 'user' track.
 \item No CUDF problem in the competition was incorrectly solved by GJSolver. 
 This could be used to argue that the parsing of CUDF to \modelname, mapping to \modelimpl problems, and the implementations of the algorithms are correct.
 \item When compared to the similar implementation of Eclipse P2, GJSolver produces similar results, in a similar time. 
 As GJSolver was based on Eclipse P2, this is seen as a validation that the differences between the two are not detrimental, and possibly improvements. 
\end{itemize}

Given these reasons GJSolver is seen as a verified implementation.

\section{Summary}
This chapter described the \modelimpl problem and how it mapped to the evolution steps from a \modelname instance.
The algorithms used to solve a \modelimpl problem where also discussed.
The implementation GJSolver was also presented.
GJSolver takes a CUDF* document, parses it to a \modelname instance, and resolves it by mapping it to \modelimpl problems and solving them.
GJSolver was verified through the MISC competition, and was shown to perform well when compared against other solvers.

The following chapter discusses the user, and their strategies for evolving component systems.
It also describes various criteria, and their mappings between MOF, \modelname, and PB criteria in \modelimpl.


\chapter{Initial Investigations}
%%%These investigations outline the first 

\section{Detailed comparison of OSGi and Debian}
%%%Here we will discuss various static graph analysis, between the OSGi component model and the Debian component model

\subsection{Graph analysis}
%%%Distributions of dependencies...

\section{Search Space Size}
%%%Here we analyise the size of the possible search spaces of both OSGi and Debian, through the simple mechanism of brute force search

\section{SAT solver optimisation}
%%%We change some of the different variables in the SAT solver (constraint resolution, oredering, ...) and analize the domain specific results

\section{Conclusions}
%%%Here we can go over the siilarities and differences between the two component models (hopefuly the are similar enough)


\chapter{Criteria}
{}The two core criteria considered during component system evolution are the maximisation of the versions of individual components, 
{}and the minimisation of the change caused by the evolution to the system.
{}Together these two criteria express that a system wants to be as up to date as possible, while changing as little as possible.
{}Although these criteria must be considered in a CDR algorithm,
{}they are not strictly defined, conflicting and for different users relate in different ways.
{}In this chapter we explore the nature and definitions of the two criteria,
{}and show that there is a significant gap between what is currently known and what we intend to explore. 

\section{Optimisation Notation}
%%%Pretext of how we are giong to note our optimistaion framework



\section{New Versions and Minimal Change}
\begin{quotation}
Motion or change, and identity or rest, are the first and second secrets of nature: Motion and Rest. 
The whole code of her laws may be written on the thumbnail, or the signet of a ring.

Ralph Waldo Emerson in "Nature", Essays, Second Series (1844)
\end{quotation}

%%%This is a general problem in nature, the force and resistance to change.
Changing to what is newer and better, but being wary of the change because of cost and risk is a conflicting problem in many domains.
For instance, in politics, conservitism is the philosophy that emphasises minimal and gradual change in society, %TODO cite
where progressivism promotes change and reform to governments.
These two conflicting ideaologies must be resolved, if a government is to function. 
As with component systems, 
the forces of change brought about by newer versions of components and the resistance to change brought about by the harm it may cause
are competing forces that must be resolved for a system to evolve.
In this section we discuss the nature of these criteria, why they are important to consider and how they conflict.


\subsection{Versions}
%%%What is a version
A version is the mechanism through which individual components evolve.
It is a unique marker which is compariable to other versions such that they are ordered.

%%%How are components versioned
The syntax and semantics of a version is usually defined within the component model.
This forces all components that are developed for this model to follow the same general guidelines.

%%%Versions between components
Although component must be versioned to conform to a versioning model defined by the component model, 
the way in which a developer uses this model to version their component is not usually strictly defined.
Each component is designed to be an independent unit, as such developers of components are separate from one another,
and the way in which they version their components can be different.
Therefore, version comparison is only useful between different versions of the same component;
e.g. a spell-checker component can be version 10 but a separate and superior spell-checker component may be only version 1.
This makes using component versions to decide between components impossible.

%%%Why does a version exist
As with the evolution of entire software systems, evolution of components requires constant maintenance.
Bugs can be found, features can be added, and code can be refactored all which make the component better.

%%%A version of a component that is greater than another version of that component is assumed to be better in some way
%%%Therefore installing it is seen as a good idea
A newer version exists because the component has changed to be better in some way.
This may be through a bug being fixed, a security hole being patched, or the functionality being extended.
Having a newer version of a component in the system, may also make all of the other components that depend on it better as well,
causing a propagation of system improvement with the upgrading of one component.

In Belief Revision, one must maintain the newest set of information while only changing the minimal amount of previous knowledge. 

The change to the system that is caused when a newer version is installed may cause harm to your system.
In the same way that a version upgrade can propagate benefits through the system, 
friction or errors can propagate and cause a faulty or not functional system.


\subsection{Minimal Change}
%%%How are these criteria currently defined, how can we define these criteria?
They are also not strictly defined as the mechanisms to either compare two systems versions or the change from a system,
can be measured in different ways.

The measure to use when determining version of a system from the versions of installed components varies.
As a set of components can be versioned differently, 
we need to explore functions that can aggregate the versions of different components together in a meaningful way.

The Mancoosi organisation uses a metric that minimises the number of components in a system that have better versions, 
e.g. a system $a-1,b-1$ if $a-2$ exists is $1$ out of date. 
The Eclipse P2 implementation counts the amount of versions that are better for each component,
e.g. a system $a-1,b-1$ if $a-2$ and $a-3$ exist, is $2$ out of date.
Both of these take into account only the component and not the components that depend on it.

The measurement of change that a system goes through during evolution is also difficult to define.
The added, removed, updated, total changed, have all been considered as metrics before. 
As with the version metrics, none of these consider the dependencies when looking at the change that is performed on the system.

\subsection{Versions, Minimality and Users}
%%%How do different users view these criteria

\section{The Middle Gap}
%%%Both extremes of criteria are repensented with current implementation, either very conservative or very progressive, but there are few implementations looking at the middle.
Each time a user decides to evolve a component system, the decision must be made about the risks of the evolution.
In an environment which is mission critical, all risk is eliminated and an unecessary change to the system is too risky.
In a development environment where the user may be trying to fix potential problem, 
or test different packages, then the risk is accepted as the system is essentially disposable, and a complete re-installation is not out of the question.
These two strategies are represented in by the current depenceny resolution implementations, however very little middle ground is available for the customisation of applications.
This was noted in the paper %TODO cite mancoosi CBSE 2011 paper




\chapter{User Model}
\label{simulation}
\epigraph{A model is a physical, mathematical, or logical representation of a system entity, phenomenon, or process. 
A simulation is the implementation of a model over time. 
A simulation brings a model to life and shows how a particular object or phenomenon will behave.}
{\textit{Systems Engineering Fundamentals. Defense Acquisition University Press, 2001}}

In the previous chapter, CUDF* documents were used to describe the evolution of component systems through a user's repeated requests to change.
However, such a document can be used to describe unlikely situations,
e.g. a user requesting to install the same component every second for a century.
Given that a goal of this research is study CSE in a practical context,
it is necessary to model \textit{realistic} users. 
For this purpose the \usermodel (\textbf{Sim}ulated \textbf{User}) model was developed.
This model describes how a user would realistically change their component system,
and it can be used to create CUDF* documents.
These relationships are described in figure \ref{sim.modeldiagram}.

\begin{figure}[htp]
\begin{center}
\digraph[scale=.5]{simmodeldiagram}{
rankdir=BT;
CMS[label=<<TABLE BORDER="0" CELLBORDER="1" CELLSPACING="5"><TR><TD WIDTH="150">CUDF*</TD></TR></TABLE>> shape=none];
subgraph {
	rank=same;
	CMI[label=<<TABLE BORDER="0" CELLBORDER="1" CELLSPACING="5"><TR><TD WIDTH="150">:CUDF* Document</TD></TR></TABLE>> shape=none];
	ConMI[label=<<TABLE BORDER="0" CELLBORDER="1" CELLSPACING="5"><TR><TD WIDTH="150">:\usermodelwx</TD></TR></TABLE>> shape=none];
}
ConM[label=<<TABLE BORDER="0" CELLBORDER="1" CELLSPACING="5"><TR><TD WIDTH="150">\usermodelwx</TD></TR></TABLE>> shape=none];
CMI -> CMS [ label="instantiates"];
ConMI -> ConM [ label="instantiates"];
ConMI -> CMI [ label="creates"];
}
  \caption{Relationships between the \usermodel model and CUDF*}
  \label{sim.modeldiagram}
\end{center}
\end{figure}

The \usermodel model was developed using the results of a survey conducted to determine how users change their component systems.
This survey is described in section \ref{strat.usersurvey}.

\usermodel contains variables to create a realistic scenario of CSE, e.g. a variable that describes the probability a user will upgrade their system on a given day.
In addition it contains information (e.g. the dates the simulation takes place) and assumptions (e.g. the probability a package will be selected to be installed) about the simulation.
\usermodel is presented in section \ref{sim.usermodel} and the variables and assumptions for the simulation are discussed.

In order to define parts of \usermodel it was necessary to collect and convert data into usable formats.
The sources this data was collected from and the methods used to convert it are described in section \ref{sim.collection}.

This chapter concludes in section \ref{sim.modelvalidation} with a discussion over the validity of \usermodel.
This section describes the differences between the reality of CSE and \usermodel.
These differences are important to explore, as they may reduce the validity of the results from the simulation.

\section{User Survey}
\label{strat.usersurvey}
To explore the user's role in CSE and to construct the \usermodel model, a survey was conducted on users of component systems.
This survey targeted users of GNU/Linux distributions (specifically Ubuntu) and server administrators through the online forum reddit\footnote{http://reddit.com accessed 6/3/2012}.
It was completed by 59 users, who answered questions about their background, the systems they use, and the ways they change their systems.
Typically the way in which GNU/Linux systems are changed is through interaction with package manager applications. 
Therefore, the questions were primarily about this interaction.

In this section, a description of the survey and an analysis of responses are given.
The results from this survey are used to describe and categorise the motivations and behaviours of users when changing their component systems.

\subsection{Questions}
%%%The goals of the survey:
The survey consisted of two groups of questions: questions that identified the type of user and questions that described their interactions with package managers.
The set of questions used to identify the user were:
\begin{itemize}
  \item How experienced with package managers are you (based on scale of 1 to 5 where 1 is no experience and 5 is highly experienced)? 
  \item What system are you using?
  \item What package manager are you using?
\end{itemize}
These can be used to categorise the type of user and weigh their answers for credibility based on their self rated experience.

The set of questions that asked about the use of package mangers were:
\begin{itemize}
  \item After you install a new system, what are your first interactions with the package manager?
  \item Describe your day to day interactions with the package manager?
  \item At what frequency are your typical interactions with the package manager?
\end{itemize}
These questions were focused on identifying the types and times the user's interacted with their package manager.
To answer these questions the user could input free text.

\subsection{Results}
%%%The results are summarised as:
As was already mentioned earlier, the survey was completed by 59 respondents.
The majority of these (29) were Debian based operating systems users.
Among these users apt-get was the most popular package manager.
Thirteen respondents used ArchLinux\footnote{http://www.archlinux.org/ accessed 6/3/2012} with pacman\footnote{http://www.archlinux.org/pacman/ accessed 6/3/2012} as the package manager.
Other systems and package managers represented in this survey include Fedora\footnote{http://fedoraproject.org/ accessed 18/7/2012} with the yum\footnote{http://yum.baseurl.org/ accessed 18/7/2012} package manger,
Slackware with the pkgtools\footnote{http://www.slackbook.org/ accessed 18/7/2012} package manager, and Gentoo with the portage\footnote{http://www.gentoo.org/doc/en/handbook accessed 18/7/2012} package manager.

The mean of the respondents self-ranked experience was 3.9/5.
While this can be considered a subjective measure, it is expected that the respondents were confident in their answers. 

The respondents responses to the questions about the types and the frequency of interactions with their package managers have been summarised in table \ref{strat.tblaction}.
\begin{table}[htp]
\centering
\begin{tabular}{l | c | c | c | c |}
Request & Set-up & Daily & Weekly & Monthly \\ \hline
Upgrade  & 45 & 27 & 16 & 0 \\
Install & 49 & 6 & 17 & 3 \\
Remove & 6 & 4 & 1 & 0\\
\end{tabular}
\caption{Summary of the survey respondents types and frequencies of interactions with their package managers.}
\label{strat.tblaction}
\end{table}

Based on this table, the vast majority of respondents upgrade and install new components when they first set-up their system.
This also shows that most users upgrade their system daily, and many install packages weekly.
Additionally, this shows that users do not often remove packages. 

\subsection{Progressive vs. Conservative Users}
One goal of this survey was to identify motivations of user's to change their systems.
Based on the responses the users attitudes towards changing their systems are affected by the following risks:
\begin{itemize}
  \item the potential risk of changing the system and introducing new problems.
  \item the potential risk of becoming out-of-date and having less functionality and old problems persist.
\end{itemize}
The behaviours of these users can be described with two user stereotypes, \textbf{conservative} and \textbf{progressive}.
These terms come from the domain of politics where conservatism philosophy that emphasises minimal and gradual change in society,
which is is contrasted by progressivism philosophy that promotes change and reform \citep{oed2010}.

Most users in the survey are partially motivated by both risks, e.g. a users response that express a slight conservative attitude:
\begin{quotation}
``In production I rarely remove packages ([it is] easier to leave software as-is than risk breaking stuff).''
\end{quotation}
The previous quote shows that this respondent is less likely to change their system, even though it may be beneficial. 

Another example of a user's response that expresses a progressive attitude:
\begin{quotation}
``I update my packages whenever I log in each day''
\end{quotation}
The reason this user upgrades their system every day is likely to ensure that packages do not become out-of-date.
This behaviour may increase the functionality of their system and allow bugs in packages to be fixed, though it has the cost of additional change.

Both of these stereotypes have extremes.
One such extreme is when a user tries to eliminate all of the risks associated with being out-of-date.
For example, a user responded in the survey with an extreme progressive attitude:
\begin{quotation}
``I do run an unstable system all the time, I help mitigate this with some redundancy in my most frequently used components, 
using packages which perform the same function, but have different dependencies, since it's less likely to have multiple packages break at the same time. 
If something is rather buggy for me, I tend to update on a more frequent basis to check for the next stable point to jump into.''
\end{quotation}
This user has not fully tested components in their system (described as unstable packages).
To lower the risk that such packages have on the function of the system, this user has redundant functionality installed.
This way the user tries to ensure that potential problems will not be too severe.
The system of this user will never be out-of-date, but will be changed frequently.

An example of a very conservative user's response on the frequency with which they interact with the package manager:
\begin{quotation}
``As little as possible. I like build my box into whatever I'll need in the first couple of weeks after an install. 
Following the configuration and construction, only the occasional upgrade is necessary. 
Unless, of course, I receive a security notice about something.''
\end{quotation}
After the setting up of a system, this user will not change it unless there is a direct security risk.
This lack of change will result in the system quickly becoming out-of-date.

The ``progressive'' and ``conservative'' terminology is useful to describe different types of users and their motivations.
It is used further when describing the simulated users in chapter \ref{experiments}.

\section{\usermodel model}
\label{sim.usermodel}
The \usermodel model consists of: 
\begin{itemize}
  \item a set of variables that describe the user's behaviour when changing their component system.
  \item a set of assumptions used by the simulation.
\end{itemize}
This section presents the variables and assumptions of \usermodel, and how an instance of this model can be used to create a CUDF* document.

\subsection{Variables and Assumptions}
The \usermodel entity consists of the following variables: 
\begin{itemize}
  \item $u$ is the probability a user requests to upgrade the system per day.
  \item $i$ is the probability a user requests to install per day.
  \item $U$ is the MOF criteria used to select an optimal system for an upgrade request.
  \item $I$ is the MOF criteria used to select an optimal system for an install request.
\end{itemize}
An instance of \usermodel is an assignment to these variables.

The variable $u$ is the probability per day that a user will upgrade their system, and the variable $i$ is the probability a user will request to install a component.
The reason for selecting the ``per day'' resolution came from the user survey responses where typically the most frequent a user interacted with a package manager was daily.
The $I$ and $U$ criteria to install a component and upgrade a system are represented in MOF.
These can be assigned criteria that is based from package managers like \texttt{apt-get} or novel criteria.


The assumptions made in the \usermodel model are:
\begin{itemize}
  \item The starting time of the evolution $t_0$ is October 30th 2009.
  \item The initial component system $\alpha_{t_0}$ is Ubuntu 9.10 (i386) released October 29th 2009.
  \item The user interacts with the system over a year (365 days).
  \item The available components to the evolving system are located at the central Ubuntu repository.
  \item Each component name has a probability that a user will select to install it. 
\end{itemize}

The reason for selecting Ubuntu as the system to simulate is discussed in chapter \ref{introduction}.
The time frame, October 2009 to 2010, was selected to start at the release of Ubuntu 9.10 until the release of Ubuntu 10.10.
A year was selected as 30/59 respondents of the survey stated that their systems are less than or about one year old.
The Ubuntu system has 6 monthly releases, therefore the simulation will be over the release of Ubuntu 10.04 in April 2010.

The Ubuntu central repository is the core location for where packages are distributed to Ubuntu systems.
Using this as the set of packages because it is the default choice on Ubuntu systems.

Typically a user would not select a specific version of a package to be installed, and let the package manager install the most recent version.
Additionally, a user would select packages for a purpose, to fulfil some requirement of their system.
Therefore, it is assumed that a user would more likely select some packages over others to be installed into their systems.
This assumes that all users will equally likely install the same packages, which is incorrect.
The impact on the validity of \usermodel caused bt such assumptions is further discussed in section \ref{sim.modelvalidation}.

\subsection{CUDF* Document Creation}
The process used to create a CUDF* document from a \usermodel instance attempts to create a description of how such a user would change their system.
The first part of this process is to represent the initial system Ubuntu 9.10 and the Ubuntu repository as a CUDF* document template.
This template is a complete CUDF* document except that it contains no requests.
By adding requests to this template the evolution of a component system can be described.
How an Ubuntu 9.10 system and the Ubuntu repository are used to create the CUDF* template document is described in section \ref{sim.collection}.

Given this template and a \usermodel instance the process to create a CUDF* document is described in figure \ref{sim.CUDF*}.
\begin{figure}[h]
\begin{center}
\begin{alltt}
createCUDF*(template, \( u\), \( i\), \( U\), \( I\)):
 for \(day\) in 1 to 365:
  if random-probability() \( u\):
    add upgrade request to template at time \(t\sb{0}+day\) using criteria \(U\)
  if random-probability() \( i\):  
   component = weighted selection without replacement from component names  
   add request to install component name at time \(t\sb{0}+day+10 minutes\) 
                                                      using criteria \(I\)
\end{alltt}
  \caption{The process to create a CUDF* document from a \usermodel instance.}
  \label{sim.CUDF*}
\end{center}
\end{figure}

This process takes the CUDF* template and the variables from a \usermodel instance and adds requests to the template to  create a complete CUDF* document.
Such a document describes the realistic evolution of an Ubuntu system.

The main loop iterates over the 365 days in which the simulation occurs.
Each day the ``user'' randomly selects to upgrade their system and/or install a component.

The function \texttt{random-probability()} is defined to return a random value between 0 and 1.
This function is used to randomly select what requests the user will make on a given day.

If a user requests to upgrade their system, the upgrade request is added to the CUDF template at time initial time $t_0$ plus the number of elapsed days.
This request uses the MOF criteria $U$ to find an optimal system.
If the user requests to install a component, first the component to be installed is selected.
This selection is: 
\begin{itemize}
  \item \textit{Weighted}: names with a higher weight are more likely to be requested to be installed.
  \item \textit{Without replacement}: a user cannot select to install the same component name more than once.
\end{itemize} 
The request to install the selected component is then added to the template ten minutes after the initial time $t_0$ plus the number of elapsed days.
By adding ten minutes it ensures if the user requests to upgrade the installation request will be after the system is upgraded.
The install request will use the criteria $I$ from the user model.

The discussion over the validity of the resulting CUDF* documents is presented in section \ref{sim.modelvalidation}.

\section{\usermodel Data Collection and Conversion}
\label{sim.collection}
In order to create CUDF* documents from \usermodel instances, data about the Ubuntu system must be collected and converted into useful formats.
The collected data and their sources are:
\begin{enumerate}
  \item the set of Ubuntu packages and the times they became available, collected from the Ubuntu repository.
  \item the initial component system Ubuntu 9.10 collected from a virtual install.
  \item the probabilities different component names will be selected to be installed, from the \texttt{app-install-data} package and weighted using the Ubuntu popularity contest.
\end{enumerate}

This section describes the data collection and conversion to create \usermodel.

\subsection{Collecting the Components}
The Ubuntu repository located at \texttt{http://archive.ubuntu.com/} is the default location that Ubuntu packages are distributed from.
This repository contains a history of all packages that have ever been included with the precise minute that the package was uploaded.

To collect these packages and information from the repository first the repositories web site was parsed and all the packages were downloaded.
These Debian/Ubuntu packages are compressed with meta-information, code and binary files.

Second, each package was decompressed and the main meta-data file, the \textit{control} file, was extracted.
This control file was tagged with the upload time of the package to the Ubuntu Repository.

Third, all control files that did not have the key \verb+architecture+  include the value \verb+i386+ or \verb+all+ where removed.
This removes any package that would not work on the simulated Ubuntu system due to architecture incompatibility.

Fourth, the control files are converted into a single CUDF* document.
The conversion from Debian control files to a CUDF document is described by \cite{abatetrenen2010}.
This report describes the handling of virtual packages and the conversion of versions. 
This process is followed to create the CUDF* template with the exceptions that:
\begin{itemize}
  \item \texttt{apt-pinning} priorities that force the use of only particular versions of a package are ignored.
  \item If a package has the value \texttt{require} for key \texttt{priority}, the CUDF* package is given the property \texttt{keep: package} to ensure the package remains between requests.
  \item In the \texttt{preamble} the time $t_0$ is added.
  \item Each CUDF package has the key \texttt{time} added with the tagged time of their control file.
  \item A self conflict is not added to each package to simulate the restraint that only one version of each package is allowed to be installed.
\end{itemize} 
The final difference is significant as it differs from how real Ubuntu systems evolve and could invalidate the simulations results.
However, the restriction for these systems to only allow one version to be installed is an interesting point of study.
By allowing multiple packages to be installed means that this specific constraint can be studied.
For example, questions like ``how often does an install request require the installation of multiple packages?'' 
or ``how much does does this restriction effect the evolution of Ubuntu systems?'' can be answered by not enforcing this restriction.
It is still preferable for a user to not have multiple versions of a package.
Therefore, this restriction can be ``softly'' enforced through using a criterion to minimise the number of packages with multiple versions.
Such a criterion is defined in section \ref{impl.ipp}.
It is shown in chapter \ref{experiments} that this alteration has almost no impact on the simulation.

\subsection{Probability a component will be selected}
%%%The probability a component will be selected
Different users will likely select different components to install.
For example, a user who is a graphic designer will more likely select graphics editor tools to be installed, and less likely select programming tools.
However, simulating the type of each user and their individual preferences of what they would install is impractical.
For this simulation each component name's probability to be installed will be the same for all users.

To define the probability a user will select a component to install, the problem is broken into two questions:
\begin{enumerate}
  \item What component would a user select to install?
  \item How often are these components selected to be installed?
\end{enumerate}
For many available packages in the Ubuntu repository a user would not likely select to install.
Packages that provide libraries, background daemons, interfaces between services are usually installed because other packages depend on them, not because the user selected to install them.
By answering the first question a set of packages can be found that a user may install.
Further, by looking at current systems and what packages they have installed, the probability that these packages are selected to be installed can be estimated.
These questions are answered using data from the package \texttt{app-install-package} and the Ubuntu popularity contest\footnote{http://popcon.ubuntu.com/ accessed 6/3/2012}..

The package \texttt{app-install-package} contains a list of 2399 packages\footnote{as of May 24th 2011} with meta-data like icons and descriptions.
This data is used by other applications, like the Ubuntu Software Center, to provide a list of packages the user may wish to install.
Some of these packages are installed by default in Ubuntu systems, and some are not available in the Ubuntu repository.
After filtering out such packages from the provided list there remain 2087 packages that the user may select to install in their system. 

%%%PopCon
The probability a package from the \texttt{app-install-package} list is actually selected to be installed can be calculated using the Ubuntu popularity contest.
The Ubuntu popularity contest is a broad data-set of information of the popularity of Ubuntu packages.
Each week an automated survey is submitted by nearly two million users, that contains information on what packages a user has installed.
The results from this survey are processed and the number of systems that have a package installed is presented in the Ubuntu popularity contest.
By dividing the the number of systems each package from the \texttt{app-install-data} packages  by the total survey respondents, 
the probability a package is selected to be installed is estimated.

\section{\usermodel Validation}
\label{sim.modelvalidation}
The validation of \usermodel has been accomplished through:
\begin{itemize}
  \item discussions with project supervisors and other stakeholders.
  \item comparing \usermodel to the responses from the user survey.
  \item compare generated CUDF* documents with \texttt{apt-get} logs collected from 19 
  respondents of the survey\footnote{Comparing these logs to the resulting simulated systems is described in section \ref{impl.validation}}.
  \item creating a virtual Ubuntu 11.10 system to study its perspective of the Ubuntu repository changing over the month of November 2011.
\end{itemize}

Using these methods differences between the simulation and the reality of CSE can be identified and discussed.
These differences largely come from the randomness in \usermodel, e.g. randomly selecting packages to install,
and from the limitations of the model, e.g. the limited types of request.
Additional differences exist in the way in which \usermodel uses the Ubuntu repository.
This section discusses these differences and their impacts on the validity of \usermodel.

\subsection{Randomness of \usermodel}
\label{sim.randomness}
When generating a CUDF* document, each simulated day there are three points of randomness in \usermodel:
\begin{enumerate}
  \item the probability a user requests upgrade their system, the $u$ variable.
  \item the probability a user requests to install a component, the $i$ variable.
  \item if the user requests to install, what package is selected to be installed.
\end{enumerate}
In reality, a user's request to upgrade the system or install a component are not random.
These requests are reasoned about using gathered information, preferences and external constraints.
Even the most insignificant aspect of a user may impact requests, e.g. the favourite colour of a user may impact their choices of components to install.
To model a user completely would be difficult, if not impossible, and would require significant effort.
\usermodel instead simulates the user with random behaviour.
The impact on validity of the randomness of these behaviours is discussed here.

The variables $u$ and $i$ introduce randomness into the simulation that does not exist in the evolution of real component systems.
For example, a user who upgrades their system each work day (Monday to Friday) would have a probability to upgrade their system as $5/7$.
Describing such a user with \usermodel and then generating a CUDF* document may create a situation where they will upgrade their system on a Saturday.
This is because users will not randomly select days to request changes, they have constraints (like the work week) and preferences that are not expressed in this model.

The most significant randomness in \usermodel comes from the selection of a components to be installed.
This randomness causes many differences to the reality of CSE:
\begin{itemize}
  \item A real user would typically not randomly select a component to be installed. a user would likely research a component before deciding to install.
  \item Each user would have different preferences of which components to install, e.g. a software developer will prefer components to aid in software development.
  \item Installing one package may change the preference of installing another, e.g. installing of browser \texttt{firefox} would decrease the probability to install another browser such as \texttt{chromium}.
\end{itemize}
To address these issues would require a more data to be collected (or created) and the modifications would make \usermodel significantly more complex.
The reduced effort and cost during data collection and the simplicity of the model was decided to take priority in this matter.

To mitigate the randomness in \usermodel many CUDF* documents will be created from a single \usermodel instance, then simulated.
The results from these simulations can then be aggregated and analysed.
Additionally, any conclusions derived from the results will be tempered by being aware of the randomness in \usermodel.

An important caveat to note is that there exist two assignments to the variables $u$ and $i$ that result in users that are not at all random:
\begin{enumerate}
  \item \textbf{Control} user: $u = 0$ and $i = 0$
  \item \textbf{Always Upgrade} user: $u = 1$ and $i = 0$
\end{enumerate}
The ``Control'' user never requests to change their system, therefore their system will always be the initial system and never evolve.
The ``Always Upgrade" user will update every day, and given users have the same upgrade criteria $U$, will generate the same CUDF* document.
These effects of these two users are further discussed in chapter \ref{experiments}.

\subsection{Limitations of \usermodel}
The limitations that \usermodel has are:
\begin{itemize}
  \item The initial set-up of the system by the user (as described in the user survey) is not included.
  \item Some types of requests are not included, notably remove requests.
  \item Cannot request to install many components on the same day, or at the same time.
  \item The list of packages that can be installed (from \texttt{app-install-data}) does not include many commonly installed packages.
\end{itemize}
Each of these points will be addressed in order.

The initial set-up of a system, which is performed by many respondents of the user survey, is not included in \usermodel.
This set-up includes upgrading their system, installing and removing components.
The initial upgrade is superfluous in the simulation as the initial system is already uptodate.
The initial installation and removal of components is not simulated as it will create different starting systems for different users,
making comparison between users more difficult.

In a real Ubuntu system, a user may request many different types of action, e.g. the \texttt{remove} component request.
Respondents to the user survey stated they request to remove a component infrequently, this is supported by the data in the user logs.
This is the justification for the exclusion from the \usermodel.

A user may want to install many components during a single day.
This can be accomplished with one request to install many components or many requests to install single components.
Both these situations commonly occur in the submitted user logs.
For example, \texttt{apt-get install ia32-libs ia32-libs-gtk libqt4-core libqt4-gui} is a single request to install multiple components,
and  \texttt{apt-get install autoconf} and \texttt{apt-get install checkinstall} are two requests performed on the same day.
Given the significant randomness in selecting a component to be installed, introducing either of these situations into \usermodel would only make results less valid.
For this reason, a user requesting to install many components on a single day is not simulated.

In reality, a user often requests to install packages that are not included in the list from the \texttt{app-install-data} package. 
An example of this extracted from the user logs is \texttt{apt-get install build-essential}\footnote{\texttt{build-essential} contains tools to build Debian packages},
this package cannot be installed during the simulation as it is no int the list from  \texttt{app-install-data}.
Much effort that went into creating \usermodel was directed at finding a list of the most commonly installed packages.
However, the core problem is that there exists very little information on what packages users commonly request to be installed.
This problem is seen as a possible future research that is described in chapter \ref{conclusion}. 

\subsection{Ubuntu Repository of \usermodel}
In a real Ubuntu system only a subset of the packages in the Ubuntu repository is visible to the package manager.
This ``view'' of the repository is the set of packages considered when resolving a request.
This is different to \modelname and \usermodel, 
where every request can consider all components that exist at the time of the request. 

To study the difference this view of the repository has to the simulation a virtual Ubuntu 11.10 system was created, 
and each day in the month of November 2011 the subset of components in its view of Ubuntu repository was stored.
This system used only the default ``view'' of the Ubuntu repository.
This ``view'' is compared to the changes in the repository of components over the month November 2009 in the \usermodel repository.

The \usermodel repository contains about 90,000 different package names, where the virtual ``view'' contained only about 40,000.
This difference is made up mostly of deprecated packages, i.e. packages that are no longer actively maintained.
It is also made up of packages that are have not yet been validated and put into the main ``view'', e.g. experimental packages. 
This causes the mean amount of packages that have a new version added to the repository to be much higher in the \usermodel repository,
126 packages per day compared to the virtual systems view of 37 packages per day.
The most dramatic difference though is the amount of removed packages from the ``view'' which is 8, where the \usermodel repository never removes a package.
This removal of packages reduces the complexity of the problem, 
though it also means that a real Ubuntu system may not have all information available when satisfying user requests. 

The reason a ``view'' of the repository was not used in the simulation is due to the fact that no data-source could be found 
that saved the set of packages that could be viewed in the Ubuntu repository at a particular time.
This lack of data lead to the decision that all components should be used.

\section{Summary}
This chapter presented the \usermodel model which is used to describe how a realistic user requests changes to their system.
This model was partially created from a survey, whose results where presented and discussed.
The variables that make up the \usermodel and the process used to create CUDF* documents were then described.
The data that was used to create the \usermodel was then described with the processes to collect and convert it.
This description of the validation of \usermodel was then described, and the differences to the reality of CSE discussed.
By exploring the differences to the reality of CSE the results from a simulation can be more accurately interpreted.

Although the \usermodel model can be used to create a description of the evolution of a component system, it may require significant calculation to find how exactly the component system evolves.
The constraints of each request must be satisfied, and the optimal system w.r.t. the criteria must be found.
The following chapter presents the mapping an instance of \modelname to a series of problems and the algorithms to solve such problems.
These alogirhtm are implemented in GJSolver, which is the final piece required to simulate CSE.



\chapter{Conclusion}
\label{conclusion}
%%%Answers to our initial questions
The main research goal of this thesis is:
\begin{quote}
Conceptualise component system evolution strategies and access them using a simulation that has been developed.
\end{quote}
This research has been directed towards this objective.
To attain this goal the semantics of component system evolution where defined, a CDR algorithm implemented and validated, user strategies for evolution explored,
and finally a simulation of the evolution of Ubuntu component systems conducted.
The assessment of the strategies took the form of answering various questions.

TODO DESCRIBE THE FOUND CONSEQUENCES I.E. How fast a system goes out of date when

How well has this research goal been attained?
Firstly; the strategies that have been accessed are component model (Ubuntu) specific. 
Secondly; not all strategies were enumerated and accessed.

Finding the consequences of component evolution on further component models (like the ones listed in chapter \ref{background}) to make generalisations is impractical and may be impossible.
This is because component models may not have enough information to create a simulation and the effort required to simulate them may exceed available resources of this project.

To simulate the evolution of a component model enough information to populate the variables of the conceptual model is required.
This information, like the likely hood a user will install a component, is often not accessible or does not exist.
The most difficult conceptual model attribute to assign is the repository function that returns a set of components that were available given a time. 
Many component models do not have a set of publicly available components, as their main users create proprietary components or do not see an advantage in sharing.
Component models that do make sets of component public often neglect to keep historical information, e.g. the OSGi repository OSGi Bundle Repository.
In some instances information is gathered but it may be incomplete or be stopped as it is seen to lack value,  
e.g. the Usage Data Collector\footnote{http://www.eclipse.org/epp/usagedata/faq.php} for the Eclipse project was shut down as it was deemed not to have a significant return on investment.  

If enough information was available for a simulation to be conducted, the effort of multiple simulations exceeds the resources of this project.
The computational power required to analyse just a single component model required many hours of high performance computers.
Each evolution problem is itself a hard problem to solve, to simulate generations of a single system was a massive undertaking.
The resources in both man hours and computational hours would need to be increased in order to analyse further component models.

Finding a complete set of all consequences for all user strategies is an impractical goal.
The amount of possible user strategies is massive, and only the common strategies where used in the simulation.
When these strategies where simulated only specific consequences where measured to answer the questions asked in chapter \ref{ubunutsimulation}.
Creating then simulating all possible user strategies would create a massive amount of data.
Then the extensive analysis of this data to find all consequences would likely be uninteresting and unusable.
Finding the consequences to a strategy that is not currently used provides little practical information.

What was presented in this thesis to answer the research question was 
\text{some} of the consequences when evolving an \textbf{Ubuntu} component system with CDR with \textbf{common} user strategies where described in this thesis.
This is believed to be significant enough to state that this research has been successful.  

To conclude this thesis, a list of contributions is briefly given as these are the ultimate outcomes of the research.
Then similar and used research from third parties are described to give a more general view of this domain.
Finally, the future of this research is outlined and the final thoughts are discussed.

\section{Contributions}
To answer the research questions various artifacts where created.
The artifacts that could be deemed useful to further study, that are also relatively unique are seen as contributions knowledge. 
These contributions include; the formal definition of component evolution, the implementation of CDR that evolves a component system,
the simulation that  

%%%Formalisation of evolution, the semantics

%%%GJSolver implementation 

%%%Simulation -- conceptual model

%%%List of consequences

%%%New stable components shown to reduce certain problems for certain user strategies

\section{Related Work}
Throughout this research the domains of software evolution, component based software engineering (CBSE) and component dependency resolution.
Research from these domains has shaped and informed this research, and in this section such research is discussed and contrasted with this thesis.

\subsection{Software Evolution}
%%%Lehman, other stuff on evolution
Software evolution is continual management of change to software. 
Software maintenance is often used to describe this,
however is considered an invalid description of evolution, as it conveys a
restoration of a system to a previous state, where evolution is a progression
of changes to a system \cite{lehman_software_2006}. Versioning software is the atomic mechanism of evolution, 
incrementing systems along the time axis.\\ 
Evolution is an intrinsic value of real world software
design and implementation \cite{lehman_programs_1980}. Laws of software
evolution identified by Lehman include continuing change, increasing complexity and
continuing growth \cite{lehman_rules_2001} to name a few. These empirically proven laws
display the importance of software evolution as a process within software
engineering.

This paper has many similarities to the empirical work into software evolution by Lehman \cite{lehman1980},
which looks for laws and relationships between software and its evolutionary process.
In this paper we aimed to look into the evolution of component systems through CDR,
and identify the consequences of different definitions of minimal change. 


\subsection{CBSE}
%%%All the CBSE stuff, Szyperski \ldots

Research often looks at the mechanisms of component evolution \citep{Wang2006}, how to add or remove a changed component in a system.  
However, this research looks at the architectural level into the properties of the component system as it evolves.
The mechanisms of exactly 

Research into how the finer-grained evolution of individual components integrates with the larger architectural view of component systems, which
usually focuses on component versioning \cite{Bauml2009,Stuckenholz2005}.
This work aims to add well-defined semantics to component versions, 
allowing them to describe if a component has changed its external requirements or if it has only changed internally.
Adding this to component models would allow CDR to make more informed decisions during evolution, 
and through experiments similar to the one described in this paper,
their semantics could be tested to find if their effect is positive to system evolution.

\subsection{Component Dependency Resolution}
%%%Le Berre and Parrain + Mancoosi 

%%%Mancoosi is by far the most related work, though its focus is slightly different than ours
As mentioned many times throughout this paper, 
the Mancoosi organisation is producing significant research in the area of Component Dependency Resolution.
They have defined the format CUDF \cite{treinen2009common}, to express and solve the CDR problems.
They also run constant comparisons against different CDR implementations through the Mancoosi International Solver Competition (MISC),
which we based our experiment on.
They also contribute significantly in the areas of modelling the problem \cite{Cosmo2010} 
and defining and researching solutions for it \cite{Trezentos2010}.
%TODO write more here

Le Berre and Parrain \cite{Berre2008} describe methods for solving dependency resolution with minimal optimisation.
Dependency resolution is used in such areas as software product lines,
for instance Savolainen et al. \cite{savolainen_analyzing_2007} uses dependency analysis to identify and remove unnecessary features.
The optimisation of such algorithms is usually hard coded maximising version while minimising components, 
as discussed in Section \ref{sec:motivation}.
The algorithms that do optimise for other properties, 
such as one proposed by Aleti et al. \cite{Aleti2009} that uses ant colony optimisation for quality optimisation,
usually are not based on a formal framework for multi-criteria optimisation but defines a specific model for each problem.

\section{Future Work}
%%%Future work may be looking at different component models, and the application of CDR to them (e.g. Web Services).

%%%Exploring more component models, and mapping them to CUDF to experiment with, and solve more problems.

%%%Making more criteria, based on testcases contracts user communities, as discuesed in chapter \ref{strategies}

\section{Final Thoughts}
The domain of software evolution is fundamental to software engineering, and this domain includes the evolution of component systems.
It is hoped that some aspects of component system evolution have been illuminated and that this research has contributed to the field. 

 


\chapter{appendix}

\section{Glossary}
Composer: The user who creates or alters component systems (compositions)

Versioning Model: A method to define and compare versions



\end{document}
