%%This is a very basic article template.
%%There is just one section and two subsections.
\documentclass{report}

\title{Component Dependency Resolution}

\begin{document}

\chapter{Introduction}
%%%The introduction gives examples leading to the idea of component development.

%TODO good quote about breaking a problem into smaller simpiler parts
When confronted with a large complex system, the first instinct, in an attempt to understand it, is to break it into smaller, simpler units, 
then examine each part individually while trying to understand the relationships between them.
Through doing this a complex system as a whole is able to be grasped through layers of abstraction.
Looking at the human anatomy, we can look at the nervous system, the cardiovascular system, the muscles, the skeleton \ldots,
each individual part with a distinct function and interactions with other parts.
Understanding a system as a set of units and interactions gives us tools for the study and maintenance of problems.  

In complex software systems, understanding the entire system as a set of parts with relationships has many of the same benefits analogous to the anatomy example.
%Specialization, Maintenance and problem detection, concurrent development, effect analysis
It allows specialization, 
by splitting the system into parts experts in those parts can focus on their own domain and only have to know how it interacts with other parts.

This same method is used when creating complex systems, first we divide the parts into individual functional units,
then define the relationships between the units.


Component Software consoldates code into units


\section{Overview of Thesis}
First Background,
Then define what a component is based on literature and examples

\chapter{Background}

\section{What is a Component?}
Discuss what a component is and who defines it as such

Give examples of OSGi Bundles, Eclipse Plugins, DS
Spring DM
Debian Packages
Maven POM
CUDF

Discuss differences and commonalities
Multiple package installations Maven Debain No, OSGi Eclipse DS and CUDF yes
Focus on interface not implementation dependence, Maven no, Debian Community defined virtual packages, 
OSGi kind of package name space related to installation, Eclipse Yes DS Yes.
Version Control, All Yes
Run time replacement Dynamic Dependence, Maven no (no runtime), Eclipse Kind of a bit buggy requires restarts,
OSGI DS Debian Yes, CUDF is abstract so has no real implementation
Conflicts, Maven?, Spring and DS no,  OSGi Singleton, Debian and CUDF are Expressive
Reccommendation, ??

\section{Component Evolution}
Versioning is the main mechanism in which components evolve.
There is also branching, and environment specific altering.

Versioning components has significant research behind it.

Branching is when a difference in opinion or idealogy on how a component should proceed into the future
causes the project to take two different routes given the same base code or idea.

Spring DM takes libraries and makes them OSGi compatable, 
Debian package can be compiled for different CPU architectures.


\section{Component System Evolution}
Version and Dynamic dependence, with focus on interface give extensive tools to the composer 
Add, remove, upgrade, downgrade components in the system while it is running.

There are some problems that must be solved within the component model like making sure that during the change,
the system is never invalid or unstable. 
This is a scheduling problem as described by %TODO

The composer must also make sure that all dependencies are meet 

\section{Criteria}
The two main criteria considered during component system evolution are to maximise the versions of individual components, 
while minimising the change caused by the evolution to the system. 
These two criteria are conflicting as increasing the version of a component is a change to the system.
They are also not strictly defined, as the version of the components must be aggregated to measure the entire system,
and the amount a system is changed can be measured in different ways.
The way in which the criteria relate to each other is different for different users.
For example, a server administrator will likely change as little as possible rather than risk a fault, 
where a developer/debugger will always wants the newest versions of components to stay up to date.

\subsection{Version vs. Minimal Change}
\quote{Why is having the newest version of a component important?}

A newer version exists because the component has changed to be better in some way.
This may be through a bug being fixed, a security hole being patched, or the functionality being extended.
Having a newer version of a component in the system, may also make all of the other components that depend on it better as well,
causing a propagation of system improvement with the upgrading of one component.

\quote{Why would you not want the newest version then?}

The change to the system that is caused when a newer version is installed may cause harm to your system.
In the same way that a version upgrade can propagate benefits through the system, 
friction or errors can propagate and cause a faulty or not functional system.

\subsection{Definitions}
\quote{How can we measure the version of a system from the versions of installed components?}

To answer this question over a set of components that are probably versioned differently, 
we need to explore functions that can aggregate the versions of different components together in a meaningful way.

Each component must be versioned to conform to a versioning model defined by the component model, 
however the way in which a developer uses this model to version their component is not usually strictly defined.
Each component is designed to be an independent unit, as such developers of components are separate from one another,
and the way in which they version their components can be different.
Therefore, version comparison is only useful between different versions of the same component;
e.g. a spell-checker component can be version 10 but a separate and superior spell-checker component may be only version 1.
This makes using component versions to decide between components impossible.


The Mancoosi organisation uses a metric that minimises the number of components in a system that have better versions, 
e.g. a system $a-1,b-1$ if $a-2$ exists is $1$ out of date. 
The Eclipse P2 implementation counts the amount of versions that are better for each component,
e.g. a system $a-1,b-1$ if $a-2$ and $a-3$ exist, is $2$ out of date.
Both of these take into account only the component and not the components that depend on it.

The measurement of change that a system goes through during evolution is also difficult to define.
The added, removed, updated, total changed, have all been considered as metrics before. 
As with the version metrics, none of these consider the dependencies when looking at the change that is performed on the system.


\subsection{The Gap}
Each time a user decides to evolve a component system, the decision must be made about the risks of the evolution.
In an environment which is mission critical, all risk is eliminated and an unecessary change to the system is too risky.
In a development environment where the user may be trying to fix potential problem, 
or test different packages, then the risk is accepted as the system is essentially disposable, and a complete re-installation is not out of the question.
These two strategies are represented in by the current strategies used, however very little middle ground is avaiable for the customisation of applications.
The user may want 

\chapter{Component Dependency Resolution}

Maven
Nuget -- C#

Nimble
OBR
Apt-get
Aptitude



Now we know the problems
\section{Formal Definition}

\section{Optimisation Notation}
%Lexicographic operator

%formal Definition with Optimisation constraints
\section{CUDF}

\section{CUDF Mapping}

\chapter{Implementation}

\section{SAT Solver}
\subsection{DPLL}
\subsection{CHAFF Watched Literals}
\subsection{Conflict Learning}
\subsection{Literal Order}

\section{Optimisation Representation}

\chapter{Component Models}
\section{Ubuntu Component Model}
\subsection{Mapping to CUDF}

\section{OSGi Component Model}
\subsection{Mapping to CUDF}

\section{Comparison}

\chapter{Investigations}

\section{Search Space Size}
%How big a search space are we looking at here

\section{SAT Optimisation}
%What is necessary in SAT
\subsection{Literal Order}
\subsection{Learning Choices}

\chapter{Simulation}
\label{simulation}
\epigraph{A model is a physical, mathematical, or logical representation of a system entity, phenomenon, or process. 
A simulation is the implementation of a model over time. 
A simulation brings a model to life and shows how a particular object or phenomenon will behave.}
{\textit{Systems Engineering Fundamentals. Defense Acquisition University Press, 2001}}

To study the CSE process is the main purpose of this thesis.
This study could be in the form of either looking at real systems and users, studying a system in a controlled environment or simulating the environment in an abstract manner.
Each of these options is discussed and weighed for their costs and benefits.
The simulation method is selected as it has the best attributes required for this research.

The methodology outlined by \cite{Law2005} is used to create a valid and credible simulation.
The basic artifact of this methodology is a ``conceptual model'', which is a simplified abstraction of reality.
A conceptual model can be used to create a CUDF* documents that can then be simulated to study the CSE process.
These relationships are described in figure \ref{sim.modeldiagram}.



\begin{figure}[htp]
\begin{center}
\digraph[scale=.5]{simmodeldiagram}{
rankdir=BT;
CMS[label=<<TABLE BORDER="0" CELLBORDER="1" CELLSPACING="5"><TR><TD WIDTH="150">CUDF*</TD></TR></TABLE>> shape=none];
subgraph {
	rank=same;
	CMI[label=<<TABLE BORDER="0" CELLBORDER="1" CELLSPACING="5"><TR><TD WIDTH="150">:CUDF*</TD></TR></TABLE>> shape=none];
	ConMI[label=<<TABLE BORDER="0" CELLBORDER="1" CELLSPACING="5"><TR><TD WIDTH="150">:Conceptual Model</TD></TR></TABLE>> shape=none];
}
ConM[label=<<TABLE BORDER="0" CELLBORDER="1" CELLSPACING="5"><TR><TD WIDTH="150">Conceptual Model</TD></TR></TABLE>> shape=none];
CMI -> CMS [ label="instantiates"];
ConMI -> ConM [ label="instantiates"];
ConMI -> CMI [ label="creates"];
}
  \caption[labelInTOC]{figureCaption}
  \label{sim.modeldiagram}
\end{center}
\end{figure}



\section{The study of Component System Evolution}
To study the CSE process, one must examine and investigate how a component system evolves in response to a request to change. 
This requires access to many component systems state before an after a change, and the requests that instigate the change.

To obtain this information three methods could be used:
\begin{enumerate}
  \item field data could be collected from real systems evolving ``in the wild''.
  \item a controlled environment could be created, in which systems are monitored as users are asked to request changes.
  \item a simulation of a systems evolution could be run, where variables of the simulation are altered to represent different environments and situations.
\end{enumerate} 

In this section the positive and negative aspects of each of these approaches are discussed,
and the reasons for the selection of the simulation method for this research are presented.

\subsection{Using Real Users and Systems}
Studying real systems that are evolved by real users is one method of studying CSE.
Users could participate by installing monitoring software in their component systems that collects data.
This data could then be used to study the process by which their systems evolve.

This approach  has significant drawbacks:
\begin{itemize}
  \item it will take long periods of time to find meaningful results.
  \item the monitoring of user's actions is an invasion of their privacy, which may limit the pool of willing participants.
  \item finding and convincing enough users to participate to generate useful results will require significant effort.
\end{itemize}

\subsection{Users in a Controlled Environment}
To increase the efficiency of collecting results and to eliminate privacy issues of users, a controlled environment may be created.
This environment is a component system where the parameters of its evolution could be controlled and monitored.
A user would then be asked to evolve the system to approximate their real actions, the resulting systems could then be studied.

Such a controlled environment could very quickly simulate  many interactions that would of occurred over a long period.  
It also removes the necessity user trust as the user is altering a system that is not theirs.
However, it will also produce less valid results, as it is still an approximation of a real system, and the users know this.
Any controlled environment would alter how the user interacts with their controlled system, making the results less reliable.

However, the largest problem with this method is the effort required to conduct the experiment on enough users to find significant results.
The core difficulty with this method is finding and convincing enough users to participate, and then monitoring and supporting them while they are in the controlled environment.

\subsection{Simulating the Problem}
CSE can be studied by simulating the users actions and the component system using reasonable approximations.
A simulation is a surrogate of the real system, 
such that it represents the core aspects of the reality while simplifying and abstracting away unnecessary detail.
The goal of a simulation is not to represent every aspect of the real world, 
but to make a ``close enough'' approximation so that the conclusions drawn from it are valid while minimising complexity.

When analysing the results and forming conclusions from a simulation, 
the assumptions and abstractions made, must be taken into account.
Therefore, the majority of the effort of creating a simulation is making sure the results from the simulation are valid.

\subsection{Why Simulate?}
The advantages of simulating are the speed at which ideas can be tested and evaluated;
the cost to test and get results; and the control over variables and configuration of the environment.

The core steps to generate new knowledge is the iterative process of creating a hypothesis, then testing its accuracy.
As the speed of this process increases, the greater amount of knowledge and meaningful information that can be generated. 
Using real users, or real systems, requires long periods of waiting and preparation where little or no progress is made.
However, if the actions and environment can be modeled and executed computationally, these times of no progress are eliminated.

%%%Cost
The cost of testing a hypothesis can be measured in the amount of time it takes to define an experiment and to collect the results.
Using real users requires significant amount of time to collect and convince users to participate.
However, with a simulation the main effort is ensuring that the returned results are valid.
Once the simulations validity is established, it takes very little time to test a hypothesis.

%%%Control
The control over the variables in a simulation allows for testing extreme and normal situations.
It also allows for the sensitivity analysis of different variables, finding how important they are to the simulated system.

%%%Final words on why we simulate
A simulation of CSE is selected in this research as it has a cost/benefit ratio that is desirable,
and provides an acceptable level of accuracy to draw meaningful conclusions.

\section{Methodology}
%%%We use the methodology from `` Build a Valid and Credible Simulation''
The core hurdle in creating a simulation is validating the returned are similar to reality.
To create a valid and credible simulation the methodology that \citep{Law2005} outlines is followed.
This methodology is a guideline for defining the study, collecting information, creating and validating models, and running the simulation.
In this section the methodology is described and aligned to this research's objectives. 

\subsection{Validation and Credibility}
%%%Why do we use this methodology and how is it relevant?
This methodology was created after the observation that validation was often ``attempted after the simulation models had already been developed'' \citep{Law2005}.
That is even if validation was attempted, it may only occurred if there was money and time left at the end of the project.
However, such simulations, that are not validated, can produce erroneous information that leads to bad, possibly costly decisions being made.
This reduces the credibility of the simulation to be used in future as a tool.

A simulation is an abstraction and simplification of reality, often created as using an actual system can be disruptive, not cost-effective, or simply impossible.
In this context,

\begin{quotation}
``\textit{Validation} is the process of determining whether a simulation is an accurate representation of the system, for the particular object of the study.'' \citep{Law2005}
\end{quotation}

The latter part is an important aspect of validation, as the accuracy of the simulation is directly dependent on the problem and questions the study addresses.
Therefore, the definition of the problem will directly lead to the modelling and scope of the simulation.

%%%What is credible
A simulation, and by extension its results, have \textit{credibility} if key stakeholders accept them as ``correct''.
A credible model is not necessarily valid, and vice-versa, as it involves the input of a person who decides if the goals of the simulation have been obtained.
Credibility of a simulation is then only attainable if the key personale from the project understand and are involved directly with the project.

The simulation produced by this study to identify the effect of different strategies on the evolution of component systems must be validated to produce meaningful results,
and must be credible for these results to be trusted.

\subsection{Seven Step Method}
The methodology presented in \citep{Law2005} has a seven step approach to creating a valid and credible simulation.
These steps are; formulating the problem; collecting information and data to construct a conceptual model; validating the conceptual model;
implementing (programming) the model; validating the programmed model; designing, conducting and analysing experiments; and documenting and presenting the simulation results.

\subsubsection{Step 1: Formulate the problem}
The first step is to formulate the problem as clearly as possible, this is usually done with core stakeholders in a ``kick-off meeting''.
The core artifacts from this step are the overall objectives of the study, specific questions wanting to be answered, scope of the study,
 and different configurations of the simulation with the measures used to evaluate their performance. 

\subsubsection{Step 2: Collect information and data to construct conceptual model}
The conceptual model is a description of how the simulation and system work relative to the problems earlier defined.
It is the most important artifact of the simulation.
It should be high level enough to be understood by the core stakeholders but detailed enough to be reused in future simulations.
It is created through interviews with subject matter experts and collecting relevant data like results from similar exiting systems.
Problems like the data not being representative of the model, not being in the appropriate format or type, and containing errors must be handled before use.

The conceptual model also contains all of the variables that can be configured including their documented assumptions. 
It is defined to the level of detail with respect to project objectives, performance measures, data availability, computer constraints, and resource constraints.

\subsubsection{Step 3: Conceptual model validation}
The conceptual model is the most important aspect of the simulation, thus its validation must be thorough.
The core method used to validate this model, is to discuss it with core-stakeholders and subject matter experts.
This provides feedback as to the direction of the conceptual model, ensuring that it will answer the questions posed in the study.

\subsubsection{Step 4: Implement the conceptual model}
The implementation of the conceptual model must also be executed and documented in a way that allows others to replicate and repeat the process.
The artifacts created during this process must be verified to work correctly, this can be accomplished through test-cases and debugging \citep{Pressman1992}.

\subsubsection{Step 5: Validate implementation}
There is no completely definitive approach to validating the simulation,
however, the most definitive test of a simulations validity is established by closely looking at the outputted results compared to that from an actual system \citep{Law2005}.

This is done through:
\begin{itemize}
  \item \textbf{Results validation: } a comparable system is used to create results and compared with the results from the simulation for validation.
  \item \textbf{Face Validation: } experts are given output of the simulation model and checked to see if it is consistent with how they perceive the system should operate.
\end{itemize}

Further validation of the implementation can be accomplished with sensitivity analysis,
which is performed on the simulation to find the factors with the greatest impact on the performance and results.

\subsubsection{Step 6: Design, conduct and analyse experiments}
For each of the experiments that are run, the time and number of independent runs must be defined.
If the results are inconclusive, or other aspects of interest have arisen, it may be necessary to run additional experiments.

\subsubsection{Step 7: Document and present results}
This step involves the presentation of the conceptual model, simulation, and results to the core-stakeholders.
This presentation is critical for the future re-use of the model, as it should promote credibility through describing the validation process.

\subsection{Differences in methodology}
A core goal of creating this simulation study is to produce valid and credible results.
However, this specific methodology has been created for large scale industrial projects with substantial resources.
As this study is smaller in scale and resources some of the procedures recommended in this methodology have been restricted and some removed.

The most significant difference between this study and the described methodology is the clear definition between decision-maker and simulation designer.
In larger projects a simulation designer is contracted by a decision-maker who has a problem to solve or question to answer.
In this project both these people are the same person, therefore meetings between them are not required.

Other people in the project including subject-matter experts, core-stakeholders, simulation analysts consist of survey participants and project supervisors.
This is because the limits of the projects resources excludes the employment of experts for validation.
This may reduce the validity of the end model, but these restrictions have been made only when necessary,
and done so in a manner that attempts to minimise negative effects.



\section{User Survey}
\label{strat.usersurvey}
The previous chapters have left the role a user plays in the evolution of their component system unexplored.
This chapter describes the motivations and strategies for users to evolve their component systems.

A strategy is the plan or pattern employed by a user to evolve a system.
A fundamental influence on the user's strategy is their perception of risk.
Each change to the system the user requests, and the criteria used to accomplish this change is effected by the risk of that change.
For example, upgrading a system may introduce new bugs and problems, yet not upgrading may let old vulnerabilities compromise the system.

To explore the users role in CSE a survey was given to users, the results are discussed in section \ref{strat.usersurvey}.
This survey was conducted in order to illuminate general user strategies and discuss aspects like interactions while evolving component systems.
The criteria a user judges solutions by are then discussed, the reasons and system objectives they fulfil are described.
Each criteria is mapped to the pseudo-Boolean function to be either minimised or maximised as discussed in chapter \ref{implementation}.
This chapter is finished with a brief discussion on how criteria can be composed to fulfil different user objectives. 

%%%The survey used to validate and refine the model
To describe and define the strategies that different users employ to evolve their component systems an online survey was conducted.
This survey targeted users of GNU/Linux distributions (specifically Ubuntu), and server administrators through the online forum reddit\footnote{http://reddit.com accessed 6/3/2012}.
In this context the way in which component systems are evolved is through package managers, as described in chapter \ref{background}.
It was completed by 59 users, who answered questions about themselves, the system they are evolving, and the way in which they evolve it.
It also involved the submission of package manager log files, which through analysis can provide greater detail of their behaviours.
These logs are not analysed here, but are used further discussed in chapter \ref{ubunutsimulation}.

In this section a brief description of the survey and an analysis of responses is given.
The full set of questions in the survey is presented in section \ref{apx.survey}. 

\subsection{Questions}
%%%The goals of the survey:
There are two types of questions in the survey; a set to identify the type of user, and a set to describe their interactions with package managers.

A set of questions used to identify the user are:
\begin{itemize}
  \item How experienced with package managers are you? 
  \item What system are you using?
  \item What package manager are you using?
\end{itemize}
These can be used to categorise the type of user and put into context the answers that they give.
They can also be used to weigh their answers for credibility and potentially exclude results from novice or inexperienced users. 

A set of questions asked about their use of package mangers are:
\begin{itemize}
  \item Describe your initial interaction with the package manager.
  \item Describe your day to day interactions with the package manager.
  \item What usage life cycle do you have with the package manager?
  \item Describe an unusual scenario in which the package manager was used.
\end{itemize}
The main focus of these questions is to identify the life-cycle of interactions that the user has when evolving a system.
The responses will provide reasons for these actions to occur, and illuminates the users objectives and strategies.

The survey finished with the question:
\begin{itemize}
  \item What other questions should I have asked?
\end{itemize}
This question tried to illicit comments and possible information that could of been gathered.
It also provided a small verification of the survey, as to see if anything was missing that should of been asked.

\subsection{Results}
%%%The results are summarised as:
The type of user that submitted the survey is important to be able to put in context their replies.
Most users (29/59) used the dpkg component model, either in Debian or Ubuntu systems.
The majority of these users used apt-get as their package manager.
The next largest component model used was ArchLinux\footnote{http://www.archlinux.org/ accessed 6/3/2012} (13/59),
all using the package manager pacman\footnote{http://www.archlinux.org/pacman/ accessed 6/3/2012}.
Other represented component models and package mangers include, Fedora with the rpm component model and the yum package manger,
Slackware with pkgtools, and Gentoo with portage.

The users mean experience was recorded as 3.9/5.
This is of course a subjective measure but describes the confidence in which the participants answered the questions. 

The life cycle questions have been summarised into the frequency of each action in table \ref{strat.tblaction}.

\begin{table}[htp]
\begin{tabular}{l | c | c | c | c |}
Request & Init. & Daily & Weekly & Monthly \\
Upgrade  & 45 & 27 & 16 & 0 \\
Install & 49 & 6 & 17 & 3 \\
Remove & 6 & 4 & 1 & 0\\
\end{tabular}
\caption{Summary of the frequency of user actions with package managers}
\label{strat.tblaction}
\end{table}

This table shows the amount of users from the survey who perform the actions of upgrading their system, installing or removing a package 
initially (when the system is first created), daily, weekly or monthly.
It shows that the vast majority of users when they create a system upgrade the system and install required packages.
This also shows that most users upgrade their system daily, and if they will likely install a package weekly.

The unusual scenarios which the user described using the package manager included 
11 of the 59 users stating that they had installed a set of packages to fulfil a specific task only to remove them the next day.
This is described by the users response 
\begin{quote}
``I sometimes install lots of packages very quickly as I need to accomplish a task, then remove them once that task has finished.''
\end{quote}

Also, 3 of the users stated that the package manager broke their system at some point, which then required to be fixed through the package manager.

In the final question asked if there is any more questions the general themes of responses include:
\begin{itemize}
  \item the mitigation of failed actions
  \item installing multiple versions of packages
  \item using a graphical user interface v.s. console based package manager
\end{itemize} 
The majority (37/59) users did not fill out this option in the survey, a potential indicator that the survey was complete.
One user even remarked to this question:
\begin{quote}
``Quite an all encompassing survey!''
\end{quote}

\subsection{Progressive vs. Conservative Users}
The main outcome of this survey was the identification of different user attitudes towards evolving a component system.
These attitudes have been used to define user stereotypes, which are described as either \textbf{conservative} or \textbf{progressive}.
These terms come from the domain of politics where conservatism is the philosophy that emphasises minimal and gradual change in society,
where progressivism promotes change and reform.
These ideologies can be caused because of a conflicting perception of risk; the risk of change introducing new problems, and the risk of not changing having old problems persist.

These ideologies are analogous to component system evolution where there is a force that wishes to reduce the risk to a system by not altering it,
and one to reduce risk by installing new versions of components.
The majority of users from the survey found some middle ground in which to mitigate both types of risk.

For example, a users response that express a slight conservative attitude:
\begin{quotation}
In production I rarely remove packages (easier to leave software as-is than risk breaking stuff).
\end{quotation}
That is they are more likely not to change the system, even though it may be beneficial as it may lead to a fault. 

Another example of a users response expressing a slight progressive attitude:
\begin{quotation}
I update my packages whenever I log in each day
\end{quotation}
Upgrading the system each day will ensure that no packages with fixed bugs will be installed, though may require significant change.

Both of these ideologies have extremes though, where users who try to eliminate all of a risk.

For example, a user's response in the survey with an extreme progressive attitude:
\begin{quotation}
I do run an unstable system all the time, I help mitigate this with some redundancy in my most frequently used components, 
using packages which perform the same function, but have different dependencies, since it's less likely to have multiple packages break at the same time. 
If something is rather buggy for me, I tend to update on a more frequent basis to check for the next stable point to jump into.
\end{quotation}
This user has components in their system that are so new as to not be fully tested, which are described as unstable packages.
To lower the risk that such packages have on the function of the system, this user has redundant functionality installed in their system to ensure that it can be used.
This users system will almost certainly be completely up to date, as any new component versions will be immediately integrated into their system.

An example of an extremely conservative user's response when asked about the frequency with which they interact with the package manager:
\begin{quotation}
As little as possible. I like build my box into whatever I'll need in the first couple of weeks after an install. 
Following the configuration and construction, only the occasional upgrade is necessary. 
Unless, of course, I receive a security notice about something.
\end{quotation}
So after the setting up of a system, this user will not evolve the system unless there is a direct security risk.
This lack of change will result in the system quickly becoming out of date.

%%%The reasons for keeping a system up-to-date; the fixing of bugs security, the adding of functionality, 
The reasons for these progressive attitudes of users is that maintaining an up-to-date system will eliminate known security exploits against the system,
and also introduce more efficient or new functionality into the system.
%%%The reasons for minimal change; through change unknown bugs could be introduced, the act of changing can create bugs, the effort of changing can consume resources  (e.g. network traffic).
Where the conservative attitudes come from the wither the perceived risk that unknown bugs could be introduced into the system, the act of changing a system can cause faults,
or the consumed resources (e.g. time, network traffic) are to expensive compared to the benefits.

Both concerns are valid assessments of the risk of component system evolution, the user must decide with respect to the system objectives what they require. 
For example, a server administrator that is running a mission critical system will likely change the system begrudgingly as any fault could cause massive damage.
However, a user running a desktop personal computer system may be more interested in new functionality than the up-time of the computer, so would change more readily.

\section{Conceptual Model}
In this section the ``conceptual model'' of the component system evolution process is presented.
The c the conceptual model is defined to represent \textbf{realistic} component system evolutions.

This relation to reality is the core difference between the conceptual model and the \modelname model.
For example, \modelname could describe a situation where a user may requests to upgrade their system every minute for a year.
This situation is clearly unrealistic, and not able to be represented by the conceptual model.

In this section, the conceptual models of the user and repository are described.
How an instance of the conceptual model can be used to create a realistic component system evolution, described by a CUDF* document, is also described.

\subsection{Conceptual User}
The important aspects to the conceptual user are the initial component system, the user requests to be made to the system, and the times at which these requests are made. 

Installing a component system is the first action a user takes during the CSE process.
This initial system is represented in the conceptual model as a CUDF* document.

In CUDF* three different actions are defined that a user can request to alter their system; installing, removing or upgrading a component.
However, as shown in the user survey presented in chapter \ref{strategies} the two main actions of users are the installation of a component, and the upgrading of the entire system.

The times at which the user requests these changes is also important.
For example, a user will typically not request to upgrade their system every minute of the day as this would be seen as superfluous.
Reasonably, if an action occurs its max

%%%We have limited it to two actions, install and update, as they are the core actions a user executes
The variables that make up the conceptual user are: 
\begin{itemize}
  \item the initial system as a CUDF* document
  \item the probability a user selects to update the system per day
  \item the probability a user selects to install a component per day
  \item the probability a component is selected to be installed
  \item the MOF criteria used to select an optimal system for an update request
  \item the MOF criteria used to select an optimal system for an install request
  \item The number of days they exist
\end{itemize}

The initial system the user installs is also modelled here, as this is the starting point of any evolution.

\subsubsection{Install}
A user typically requests to install a component to extend the functionality of their component system.
How often does a user requests to install a package, and what packages is the user likely to request to install are interesting questions. 
simulating, the information of what component a user will install and when will a user select to install it are needed. 

%%%What probability a package will be selected to be installed
Determining the probability a user will select any component to be installed is difficult given the enormous amount of factors this relies on.
The users job, location, current tasks, previously installed software, favourite colour and numerous other aspects can determine what the user will select to install.
Given that all this information is impractical to simulate, this problem is abstracted into the form of two questions;
\begin{itemize}
  \item What components may a user select to install?
  \item How likely would a user select to install any of these?
\end{itemize}
This is represented by a weighted list of components, where the weights represent popularity;
e.g. component ``A'' has a 10\% chance of being selected as the component to install, or component ``B'' has a 5\% chance.

%%%The core problem with this is correlation between packages and user
The core problem with this weighting of components is that it ignores the correlation between probability a component will be installed. 
Two or more components may complement their functionality, thus making their installation together more likely.
This is impossible to represent with a weighted list, however this type of information is very difficult to calculate.
It would require the analysis of a significant amount of systems, which would probably cost more than the benefits gained.

The next question is what is the probability that a user will install a component, and how many components?
This is represented by a probability for the amount of components that a user may install on a day,
e.g. they have a 80\% chance of installing nothing, a 15\% chance of installing one component.

\subsubsection{Initial System}
%%%The initial system to start from is important, typically there are many to choose from
The starting system will effect how the system is evolved.
Therefore, selecting an initial system to start evolving is an important issue.

%%%What initial systems are there to chose from
Many component frameworks will release various configurations of components that satisfy many use cases.
This is analogous to software product lines \citep{clements2001software}, which increases the re-use of components and satisfy many different users with minimal effort.
The Eclipse framework offers more than ten types of initial Eclipse installs for different users.

The Ubuntu distribution offers three different systems for use, server, desktop and alternative.
Each of these also includes the choice of either amd64 or i386 chipsets.

The selection of the initial system can depend on many aspects of the simulation, which should be considered when defining the questions to be asked.


\subsection{Repository Model}
The set of components that exist at a given time.

\subsubsection{Time Range}
%%%The range of time in which to look at
The range of time over which the simulation is run will determine the repositories that are required.
It is also important as it should be long enough to draw conclusions from,
but as the simulation can take considerable resources to execute, too long and it may make it impractical to execute all the iterations necessary.
Other external aspects such as policy changes in the way in which a repository is run, or release cycles of the component system, 
must be considered when defining the time range.
These may have an effect on the results therefore should be considered in the conclusions drawn.

\subsubsection{Variables}

%%%this model contains a record of packages in repository over time
This model then contains one set of information:
\begin{enumerate}
  \item A daily record of components stored in the repository
  \item A time frame, start and finish, over which the simulation is run
\end{enumerate}

\subsection{Configuration}
The simulation configuration is a set of variables in which the simulation can be altered through.
To treat these variables as a set of inputs, their format and constraints must be defined.

The format for the initial system and criteria have been defined in previously chapters.
The initial system and the repository information will be store in the CUDF format, as described in chapter \ref{background}. 
The criteria to update and install will be defined in our modified MANCOOSI format defined in chapter \ref{criteria}.

Some inputs are trivially defined;
the update probability is a number between 0 and 1;
the time frame is defined as the dates of the days (given in seconds since the epoch at Jan. 1st 1970) between a given start date and the number of days that the simulation should run for.

These final inputs, the probability distributions of whether a component will be selected to be installed and how many components a user will select to install on a given day,
are represented using a set of pairs, each pair containing the action and the probability the action will be taken.



\subsection{Processes}
These processes are attempts to abstract and simplify the reality of the component system evolution.
The process that simulates the evolution of a component system takes a configuration and first generates a set of ``user actions" a user may take.
These user actions are then iteratively applied to the initial system though generating a CUDF problem,
then passing it to the resolver with the appropriate criteria to find the resulting solution.

The results of this process is then a set of systems that are created through the user actions.
These will be different along the dimensions of the configuration.



\paragraph{Initialisation}
%%%Initialize the algorithm
The \verb+generateUserActions+ algorithm first initializes three variables, \verb+userActions+, \verb+keeps+ and \verb+days+.
The \verb+userActions+ list is returned at the end of the algorithm.

The variable \verb+keeps+ is an initially empty list, that is contains all the components the user has previously selected to install.
This list is later translated into a ``keep: package'' property of the installed component with defined semantics in the CUDF specification.
This ensures that after a component has been selected to be installed by a user a component is not later removed by an action, like an update.




\paragraph{Differences to Reality}
The core differences from this model of a user to a real user are life cycle actions and correlation between actions.

A user when first installing a system will likely have components that they require that are not in the default system, and therefore when starting will install a lot of components.
This type of reasoning leads to life cycle actions, actions that are caused by events in the life cycle of the system.
Another example of this may be that at the end of each month a user decides to look through what they have installed and remove unneeded packages.
These actions add an extra dimension to the user model and process and will clearly change some results.
However, it has been opted out of the simulation as a precise set of possible actions and associated parameters are difficult to define.
Also as stated in their name they occur infrequently, if at all, and therefore it has been decided that they may not be significant.  

One action may also impact the occurrence of another action, so that they are correlated to happen together.
For example, a user may select to always update before they install, or to only remove a component directly after it has been installed.
The complexity that can be introduced by these correlations could make this simulation significantly more complex.
As these values may be different for each type of user and is probably more related to the person than the strategy they employ these have been deemed 



\section{Simulation Validation}
%%%Validation of this simulation, what needs to be validated/why it should be validated
The conceptual model presented is a simplified abstraction of the reality in which users evolve component systems.
It describes the variables that effect component system evolution as a configuration,
and the processors used to execute the simulation given a configuration.

%%%What if it is wrong
If some significant aspect of the system was missed, or if some aspect was incorrectly defined, the simulation may produce results that are incomprehensible,
or worse, misunderstood.
Therefore, the validation of these artifacts is essential to move forward. 
This validation was accomplished though regular stakeholder meetings, and an online survey with subject matter experts (as described in chapter \ref{strategies}).

\subsection{Stakeholder Meetings}
%%%Weekly meetings with stakeholders (i.e. supervisors)
As described in the methodology; one effort to validate these artifacts is done through meetings and a structured walk-through with the core stakeholders.
In this simulation the core stakeholders are the project researcher and supervisors.
These are the people who are asking the questions and are also impacted by the outcome, therefore they are directly effected by the validity of the results.
Meetings where held at regular intervals to ensure the projects progress and direction where correct.

\subsection{Subject Matter Expert Survey}
%%%Results from what else should be asked, install stuff not from repository, installs break
The survey described in chapter \ref{strategies}, 
was conducted at a point in the project when the conceptual model was just being developed, so had considerable impact on these artifacts.
The questions asked in this survey helped gauge the necessity and frequency of user actions,
so that only the most important aspects of the problem can be selected to be simulated.
It also filled in gaps of what was missing from the survey and model, giving direction for exploration.

\subsubsection{Frequency of User Actions}
%%%How often do users do these actions
The more frequently a user selects an action to evolve their system, the more important it is to the evolution of their system.
The information gained from the survey provides confidence that not including the action to remove a component and abstract the requirement multiple repositories, 
would not damage the validity of the results.
It also helped us define and represent the update and install actions in the configuration.

The remove action can be ignored as it seemed many of the users do not use this frequently.
When they do select to remove a component it is usually directly after installing it, if the selection to install a component was seen as a mistake.
Although it is clearly an important function to be included when evolving a system, the assumption is made that it is unnecessary for this simulation.

This survey also clearly shows that the main actions of a user is to update their system,
with the installation of a component the second most used action.
This survey also showed that the update occurs at more regular intervals than the install components.
Therefore, only the update and install user actions were included in this simulation and it also defined their representation. 

\subsubsection{Component Fault Exploration}
Another aspect of this problem that was mentioned by the subject matter experts through the survey was that a system may break during a change.
This typically then requires a reversion of the system components to a previously stable state.
Simulating this effect was ultimately deemed outside the scope of the project as it is seen as a rare occurrence with modern systems.
However, before it was eliminated, it was explored for possible inclusion.

The core problem with including simulated faults in a component systems evolution, is that each component has a different likelihood of causing a fault.
Different properties of a component like development process or complexity can impact this value.
Therefore, the function to calculate the likelihood of a component failing could rely on many different properties.

Instead of creating a function that tried to calculate the probability of failure per component,
an effort to measure it was attempted through a small study of component bug reports was attempted.

The feedback generated when a component causes a fault should ideally be a bug report, for the Ubuntu distribution these are filed on the project hosting server LaunchPad.
By using the Launchpad API to extract bug information, the amount of bugs per component in the system was able to be measured.
It was initially assumed that the number of systems that a component was installed on would increase the number of bug reports generated,
as more users means more eyes and systems to find bugs.
The Ubuntu Debian popularity contest was used then to see if this relationship existed the graph in figure \ref{bugsvspop} was created.

\begin{figure}[htp]
\begin{center}
  \includegraphics[width=\textwidth]{simulationpics/bugsvspopularity}
  \caption[Bugs v.s. Popularity]{A plot of the bugs a package has compared to its popularity, with notable outliers labeled}
  \label{bugsvspop}
\end{center}
\end{figure}

The first thing to note is that there seems to be very little relationship between the two variables, other than the package with the most bug reports is also one of the most used components.
The second thing to note is that the packages with the largest amount of bugs are ``apt'' and ``aptitude'' the two most popular package managers.
This could be because those packages are very buggy, or it could be because problems caused by apt, e.g. trying and failing to install a faulty package, may be reported as a problem with apt.

The last thing to notice is that there are many less popular components that have many bug reports.
When identifying the purposes of these packages many are used by developers, e.g. emacs21 is a popular text editor to program in.
The reason for their increased amount of bug reports may be that the users have prior experience and appreciation for the bug reports and the maintenance process, so file more bugs.

Measuring a components likely hood of failure using bug reports is then hypothesised to be impractical if not impossible,
as a component purpose and a components users may affect the results of the best measurement method available.
Other methods of finding a components likelihood of failure have not been further explored since this variable was eliminated from the configuration.
Though, it is expected that this is an intractable problem that is likely impossible to simulate accurately.

\subsection{Further Validation}
The assignment of the configuration variables is a different stage in the validation of this simulation.
Clearly if you create a configuration that is completely unrealistic, the saying ``garbage in, garbage out'' applies to the results.
However, this is not a concern when validating the conceptual model, or the abstract processes.
Further discussion of the validation of the assignment of the configuration variables is in chapter \ref{ubunutsimulation}.

\section{Summary}
{}In this chapter possible options were discussed for studying various strategies employed when evolving component systems.
{}Simulation, through the methodology \citep{Law2005} describes, was selected, and the steps involved were described.
{}The central artifact of this methodology, the conceptual model, was broken down into models of the user, repository and solver, and the processes of simulation.
{}These models were validated through regular meetings with the core stakeholders, and a survey conducted on subject matter experts.
{}In the next chapter the configuration of the simulation is further defined, and the questions about component system evolution are attempted to be answered.


\section{User Survey}
\section{Log analysis}

\section{Criteria}

\section{Heuristics}
%Page Rank
%Predictive Use
%Trendy
%P2
%Custom Beast

\section{Experiment}
\section{Results}
\section{Analysis}
\section{Experiment Conclusions}

\chapter{Conclusion}
\section{Related Work}
\section{Future Work}

\section{Glossary}
Composer: The user who creates or alters component systems (compositions)

Versioning Model: A method to define and compare versions
\end{document}
