%What is component system evolution
There are an estimated 20 million Ubuntu GNU/Linux users and millions of users of the Eclipse IDE.
Ubuntu and Eclipse systems are constructed from components, called packages and bundles respectively, 
and can be changed by adding or removing components to and from their systems. 
Over time these systems will be continually changed to adapt to their software environment, accommodate new user requirements, 
fix errors and/or prevent errors from occurring in the future.
By repeatedly changing component systems they evolve; this is the Component System Evolution (CSE) process.

%How we study it
This thesis investigates CSE through measuring the change a system goes through and how out-of-date the system becomes over time.
To accomplish this, a developed simulation was used.
To create this simulation, CSE and the users that change component systems were modeled, 
and an efficient implementation that calculates the evolution of component systems was created.
Various experiments were conducted using this simulation to measure the change and out-of-dateness of Ubuntu systems evolving over a year. 
The results of these experiments enabled the development of two novel approaches that reduce change and out-of-dateness of systems.
Using the simulation these approaches were shown to be beneficial.

