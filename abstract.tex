%What is component system evolution
A component system can be constructed by composing together software components.
These systems can adapt to the changing software environment, accommodate new user requirements, 
fix errors and/or prevent errors from occurring in the future by changing their composition of components.
By repeatedly changing these systems, they evolve over time.

%why does it matter
There are an estimated 20 million Ubuntu GNU/Linux systems users and millions of users of the Eclipse IDE.
Ubuntu and Eclipse systems are constructed from components, called packages and bundles respectively, 
and can be changed by adding or removing components to and from their systems. 
Repeatedly applying such changes allows Ubuntu and Eclipse systems to evolve, which in the context of this thesis is the CSE process.
By studying the effects of CSE and providing recommendations, this research has the potential to impact millions of users and their systems.

%How we study it
This thesis investigates this evolution of such component systems, called component system evolution, using a developed simulation.
To create such a simulation models of CSE and the users that change component systems were developed.
An implementation that calculates the evolution of component systems was created.
In conjunction these were used to simulate the evolution of component systems.

%What the results are
Though conducting experiments using the developed simulation the change and out-of-dateness for various systems is measured during CSE. 
Two novel alterations are proposed to reduce change and out-of-dateness during CSE.
The effects of these are measured using the simulation and are shown to have benefits during CSE.
