\chapter{Simulation}
{}To evaluate different strategies to evolve a component system, we simulate the environment in which this evolution occurs.
{}This simulation models the user-interaction, the repository, and the system
{}to draw conclusions on the benefits and draw backs of a particular evolution strategy.
{}To ensure that the model is credible we use a methodology outlined by, %TODO cite How to Build a Valid and Credible Simulation
{}The data that is used was extracted from logs of real users, a user survey conducted on a popular Internet forum, 
{}component and temporal information collected from repositories, meta-information about those repositories.

{}In this simulation we take a component system and evolve it over time using real and approximate information and look at the resulting systems.

{}In this chapter we first describe our motivation for using a simulation,
{}then we present our methodology
{}first describing our formulation of the problem,
{}our reasons and methods for the collection of data,
{}discuss our simulation model validity when compared to the real world.
{}The design of the experiments are then described and the variables assigned values,
{}finally we present the results from this simulation.
 
\section{Why Simulate?}
%%%Why are we using a simulation instead of real systems, or a controlled environment
To evaluate an evolution strategy we could either look at a set of actual systems with real users and collect data,
we could create a controlled environment in which users are given tasks and monitored the outcomes,
or we could realistically simulate the necessary aspects, then study the results and implications.
As using real users with real systems will always return more valid results to simulating, as would a controlled study of users,
why would we opt to simulate?

%%%Because users dont trust the ``experimental" implementations
The main reason for simulating is that finding enough users
who would allow an experimental component resolution algorithm to alter their real system would be extremely difficult.
A user will likely not trust a newly released resolver, as their system is important to them and even the slightest error can cause
a system to become unstable. 
To gain the trust of possible users the experimental resolvers would have to be thoroughly tested through a repositories development cycle and be well maintained.
Moving a package through this cycle can be itself a massive undertaking lasting several months. 
For Eclipse Plug-ins it involves going a component going through four phases 
Proposal, Incubation, Mature, and Top-Level \footnote{http://www.eclipse.org/projects/dev\_process/development\_process\_2010.php};
similar to Debian's process of moving through the phases Unstable, Testing, Frozen, and finally stable \footnote{http://en.wikipedia.org/wiki/File:Debian-package-cycl.svg}.
After it has been through this cycle maintenance of the resolver is still required; 
for instance the resolver apt-get since its initial release has released more than 2 versions a month \footnote{http://changelogs.ubuntu.com/}.
This shows how much effort must be applied to earn a users trust, and this effort may outweigh the actual benefits when simulation is a cheaper alternative.

%%%Because getting enough users of different types would be difficult, and in a controlled environment the results may vary.
We could create a case study of users where users evolve systems in a monitored environment, where they select packages and we use these to create systems and analyse implications.
As with the real simulation, getting a broad enough user group that we can monitor is logistically difficult and impractical when compared to the benefits.
Any controlled environment would alter the state of how the user interacts with their "control" system, making the results similarly less reliable.
The one aspect that would be improved over simulation would be the possible more realistic user model, 
however as a user typically researches a package before having installing it because it would have real implications, in a controlled environment these implications are removed so the selection would still be of questionable validity.

%%%Instead we can simulate, which is an approximation of the real world.\\
%%%When drawing conclusions the accuracy of this approximation must be considered
Simulation is a surrogate of the real system, such that it represents the aspects core to the problem, 
the evolution of component systems.
As it is only an approximation of the real world, 
the accuracy to which it actually represents the real world is not 100\%.
The goal is then to make it a ``close enough'' approximation so that the conclusions drawn from it are valid in the real world.
So when analysing the results and forming conclusions, we must take into account the assumptions we have made creating the simulation to determine 
the conclusions overall accuracy.
Therefore, the majority of the effort when creating a simulation is gathering and using valid information when making assumptions.

\section{Methodology}
%%%We use the methodology from `` Build a Valid and Credible Simulation'', this requires the definition of our probelm. 
Creating a simulation that accurately represents the evolution of a component system so that we can 
analyse the effect different evolution strategies have on a systems properties is the core objective of this simulation.
To this end, we follow the a methodology outlined by, %TODO cite How to Build a Valid and Credible Simulation
This gives us a set of guidelines to follow when preparing a valid simulation including;
specifically defining the problem,
building a conceptual model,
collecting data,

\section{Conceptual Model}
%We first define the problem to be modeled
To create our conceptual model, we must first precisely describe the problem;
{}When evolving a component-based system, different strategies can be employed by the user and the resolver.
{}These strategies have unknown effects over a long period in which the user evolves the system.
{}Through simulating and analysing these effects we can choose a strategy more effectively.
{}There are two parts to these strategies, first how the user interacts with a resolver, secondly how the resolver handles these requests.

\subsection{Strategies}
%%%How do users use different strategies when evolving a system when installing and updating their systems
A user interacts with a resolver to evolve a system by installing, removing or updating packages.
The way in which these interactions occur is usually related to the purpose of the system, and what the type of user.
For instance, a server administrator is less likely to install a component into a system as the system has only one task and if it is working their is no need to change it, 
however a desktop user will more likely install different packages because they use their system for many different tasks.
How a user updates their system is also different along these lines, server admins. will likely only update when absolutely necessary as if it is not broken why fix it.
Yet desktop users will update more frequently as they want the best system they can have, and new features or better performance are usually wanted.

%%%What different strategies are deployed by resolvers
As described in previous chapters, dependency resolvers often try to minimise change and maximise the versions.
These two objectives have been implemented in many different ways, how these interact with each other is the strategy a resolver employs.
For instance, the trendy strategy from the Mancoosi organisation describes the strategy to minimise the removed components above all other criteria,
this is a common strategy because removal of a component is seen as very risky.

\subsection{Effects}
%%%What kind of effects are we looking at analysing?

%%%Static

%%%Dynamic

%%%Change Propogation

%%%Graph Analysis

%%%Coupling and Instability defined by JDepend

\subsection{Configuration}
%%%A configuration is the set of parameters of the simulation. What are these parameters which we must consider.
A configuration is the set of variables which define the parameters of the simulation.
How these variables are defined are derived from what goals of the simulation.
As the goal of this simulation is to look at different resolver criteria and user strategies,
these are the core changes that are made.
Given that we want to look at the effects of many different possible ways a system can evolve,
we also change the packages that are selected to be installed by the user.
The variables like, what repositories are used, and over what time period are seen as not being relevant to our eventual analysis,
therefore we do not change them and leave them static.

Given this simulation is looking at the effects of the strategies employed by users and resolvers on the component systems,
we must look at different resolvers, and different user strategies.
What packages the simulated user selects to install and 

%%%The criteria we are testing
We are testing resolver strategies P2, trendy, Hamming + max version, PageRank,\ldots 

%%%We are changing the distributions of user installaion, and the period between updates
The user installs different amount of packages per day, and updates at different intervals. 

%%%We are selecting different packages to install
We are selecting the packages to install that are different.

%%%The repository over which it runs changes over time
Time, We have decided not to 

\subsection{Sensitivity Analysis}
{}Through altering some of the configuration variables a small amounts then running the simulation, the variables that are most important will alter the results significantly.
{}This is an important step in our simulation as it shows us the variables that we must exert more effort to get correct. 

\section{Data-sets}
%%%Data collection is an important aspect of a simulations validity, where have we collected data from, and why have we collected it
The most complex and difficult part of this (or any) simulation is the human component, the user,
as such that is where much of our data collection focused.
Data has been collected from
a user survey which was performed on a popular Internet forum,
user logs from resolvers were collected with the survey.
The Ubuntu popularity contest is used to determine package popularity,
as well as a user forum thread where users posted their top ten packages.
A package that contains a list of applications in the Ubuntu repository was included.
The entire Ubuntu repository, with relevant date information about packages was used.

%%%User Survey, What they are, where they are from, what information we can get from them, problems with that information
A user survey was conducted online via a popular Internet forum http://reddit.com/r/ubuntu, this involved nearly 60 participants. %TODO attach survey to appendix. 
The survey focused on user interaction with a resolver and the lifecycle that is associated with their component system.
The results are summarised below, %Summarise results

%%%User Logs, What they are, where they are from, what information we can get from them, problems with that information
Resolvers often keep logs of their activites, these usually only include the changes to the systems that are made, and not what the usre requested.
Therefore, some of the information that can be obtained

%%%PopCon, What it is, where it is from, what information we can get from it, problems with that information
The Ubuntu Debian popularity contest\footnote{http://popcon.ubuntu.com/} is an excellent, accurate and broad data-set of information of the popularity of Ubuntu packages.
Each week this automated survey is submitted by nearly two million users including the currently installed packages they have on their system.

%%%User Forum top 10 posts,What it is, where it is from, what information we can get from it, problems with that information
People submit multiple times, some people submit more than 10, some less, package sanitisation, had to change names to fit package names

%%%app-install-data package
The package app-install-data contains a list of applications, that are available in the repository, which the user may wish to install.
This is useful for applications to use that help users search for and find an application that they may wish to install, like the Ubuntu Software Center.
%TODO how do they get this list
This comprehensive list currently\footnote{May 24th 2011} contains 2393 applications.

%%%Ubuntu Repositories
The Ubuntu Repository, as with most open and free software, is freely downloadable.
It contains all the packages that have ever been in the repository with the information of when the package was added.
We created a web scraper to download all the packages, and then we extracted their control files (the meta information file) and converted it to CUDF as precariously described in section. %TODO referecne
As the date a package was uploaded, we used the extensible CUDF syntax to include what the date when they were uploaded.

%%%Ubuntu installation
One of the aspects that is critical to a simulation is a time over which it is occuring, 
so the starting system is important aspect for this simulation.
Ubuntu has 6 monthly releases one, in April and one in October, the syntax of the version of each release is first the year,
then the month in which it was released, e.g. 10.04 is the release in April 2010.
Given that we are running this simulation over the course of a year, we are selecting that year to be 


%%%The user can install a single package, or update all, removing other more complex but rarely used actions
In our abstraction of reality the user has two actions with the resolver, either to install a new package or to try upgrade the entire system.  
In reality the user can do much more fine grained actions, where they can remove packages, 
upgrade individual packages or even create complex queries for the resolver to solve. 
These additional actions, along with being difficult to simulate, are rarely performed by a user, as found with our user survey. %TODO reference
How often a user installs a new package is found through looking at user logs, %TODO ref user logs
and how often a user updates a system is found through our user survey.
These pieces of information allow us to approximately represent real users.


\section{User Model}
{}The user model involves many aspects, including the lifecycle of specific actions, the probability of actions being taken, and also the parameters of an action.
{}Given the three actions update, install and remove, how often are these actions executed, and what are they executed on.
{}To answer these questions we first look at the analysis of our survey to find the lifecycle of these values, we further refine these values by looking at the submitted user logs.

\subsection{Survey analysis}
%%%The questions we want to answer are which actions are routinely performed and at what frequency?
%%%What life cycle specific actions are performed?

%%%Description of the Survey

%%%Results from the Survey

%%%Analysis of these results and how they apply to the user model

\subsection{Log Analysis}
%%%The question we want to answer is "How often do you install packages?"

%%%The distribution of actions, specifically installation, is difficult to ask a user, "How often do you install?", therefore we extract this information from logs, and bootstrap the distributions.

%%%How we analyised these logs, our main assumption that daily actions are independent, and that we group actions every few hours 
%TODO look at script to get specific values

%%%What are the drawbacks of this model? Mostly that the log information from non-apt-get resolvers had to be grouped.


\subsection{Package Popularity}
{}Determining the probability a user will select a package to be installed is difficult given the enormous amount of factors this relies on.
{}The users job, location, current tasks, previously installed software, favourite colour\ldots all may effect when and what package a user selects to install.
{}Given that all this information is impractical to simulate, we abstract this problem into the form of two questions;
{}what packages may a user select to install and how likely would a user select to install these.
{}We attempt to answer these questions through using the set of packages listed in the package app-install-package
{}weighted with their popularity from the Ubuntu popularity contest.
{}This method has some draw backs, not all packages a user may install are listed and there is no correlation between packages selected for install.
{}However, by limiting our approach we have answered these questions without sacrificing much simulation integrity.

%%%What packages may a user select to install? We can determine this by looking at applications that are listed in the app-install-data package
Many of the packages in a repository are not ones which a user would directly select to install.
Most packages provide libraries, background daemons, interfaces between services; packages that are only needed through dependencies.
A user would not likely install these as they do not directly allow the user to complete tasks in the system.
The list of applications from the package app-install-data is used to answer the first question, what packages may a user select to install.

%%%How likely would a user will select to install a package can be gathered from popcon, by looking at how many systems have that package installed
The second question is then answered through analysing the data-set and determining the probability a user will have a particular package installed.
The reason why we cannot use this information to also answer the first question is that many of the most packages installed are there because they are depended on by many different applications,
e.g. a media library that decodes a stream may be used by many different media playing applications therefore installed on many users systems, 
ranking it high in the popularity contest but it was never directly installed by the user.


%%%We validate this by comparing it against the list of users top 10's and stating that the pacakages that more than 5% of the users voted for is 90% accurate
This relies on the fact that a user would like to 

%%The core problem with this list is that more experienced users may install packages that are not applications, build-essential
Although a user will more likely install this list can be assumed to be a complete list of applications 
the main problem is that more experienced users may directly select to install packages that are not deemed applications.
For instance, the package build-essential includes tools in which a user can build their own Debian packages,
this is a task for many users, though not deemed an application therefore not included.

%%%The core problem with this probability, is that it doesnt measure corrolation between packages being installed
The core problem with this probability, is that it doesnt measure corrolation between packages being installed.
For instance, i


\subsubsection{Failed Attempts at ranking}
%%%We know popcon is an accurate and broad measure of users systems, we attempted to solve its problems through other measures with no sucess
In creating this set packages weighted to their popularity, we also  know the Ubuntu popularity contest is an excellent accurate and broad data-set of information with one main draw back of having superfluous packages,
we attempted to use other means to eliminate these packages and then  

%%%Method google completeion API
To estimate the popularity of a package, the Google API for automated search completion is used.
When given a query, this API returns an estimate of the amount this query has been searched for by other users,
it also returns a list of related searches that users have searched for.
What query is used is the most important aspect of this approach,
searching merely for the name of the package may return user queries from other domains, e.g. searching for the package ``wine'' may return oenophile sites.
Also using multiple different queries can allow for a more robust heuristic as it allows measuring their popularity from different perspectives,
as one query may not be used when users search for a particular package.

%%%How we validate the heurisitic? Via a list of popular packages from a popular forum
This is a very general approach, it involves many aspects that effect the results making them possibly inaccurate.
Therefore, the estimates are validated against a popular Ubuntu forums thread\footnote{http://ubuntuforums.org/showthread.php?t=35208} 
that asks the user to post the top packages they install in their ubunutu systems. 
The results of this thread are tallied to compare against the google approach.

%%%The queries we use and the way that we aggregate them
We have selected three queries to build our heuristic:
``apt-get install``, ``ubuntu'' and ``install'', 

%%%Corrolation, as google API returns similar values we can estimate the corrolation of packages

\section{Repository Model}
{}The model of the repository is complicated by the fact that our simulation takes into account the date, this means that the repository should only contain packages that existed at a certain date.
{}This is simplified by only using a single repository, the ubuntu repository, and only using a single date 10.2009-10.2010

%%%We create a repository for each different day and use that based on what day we are at

%%%The repository is selected because it has significant amount of users, so results will be applicable, and a significant amount of components.

%%%The dates are selected as they are between two major releases and over a year

%%%Analysis of the repository between the dates, number of changes, additions, different versions of the same components.

%%%This repository model differs from reality in the fact that we use all components, not just those available in the meta repository (ubuntu manacured)

\section{Criteria}
%%%What criteria are we using, Yet to be decided though likely contain P2, Hamming, PageRank, Trendy


\section{Simulation implementation}
%%%Here we describe how our simulation is implemented, specifics that are not previously mentioned, and the variables not yet defined

\section{Results}
%%%Here we describe results from our simulation

\section{Analysis}
%%%Analysis of the results, and the simulation

%%%Results analysis

\subsection{Simulation analysis}
%%%By comparing the results back to the user submitted logs we can partially validate the output.

\section{Conclusions}
%%%Discuss the overall conclusions. Bullet point the points most surprising, and useful for further study.

