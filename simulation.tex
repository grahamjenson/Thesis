\chapter{User Model}
\label{simulation}
\epigraph{All parts should go together without forcing. You must remember that the parts you are reassembling were disassembled by you. 
Therefore, if you can't get them together again, there must be a reason. By all means, do not use a hammer.}
{\textit{IBM maintenance manual, 1925}}

In the previous chapter, CUDF* documents were used to describe the evolution of component systems through a user's repeated requests to change.
Given that a goal of this research is to simulate the evolution of component systems,
it is necessary to model \textit{realistic} users. 
For this purpose the \usermodel (\textbf{Sim}ulated \textbf{User}) model was developed.
This model describes how a user would change systems,
and it can be used to create CUDF* documents.
These relationships are described in figure \ref{sim.modeldiagram}.

\begin{figure}[htp]
\begin{center}
\digraph[scale=.5]{simmodeldiagram}{
rankdir=BT;
CMS[label=<<TABLE BORDER="0" CELLBORDER="1" CELLSPACING="5"><TR><TD WIDTH="150">CUDF*</TD></TR></TABLE>> shape=none];
subgraph {
	rank=same;
	CMI[label=<<TABLE BORDER="0" CELLBORDER="1" CELLSPACING="5"><TR><TD WIDTH="150">:CUDF* Document</TD></TR></TABLE>> shape=none];
	ConMI[label=<<TABLE BORDER="0" CELLBORDER="1" CELLSPACING="5"><TR><TD WIDTH="150">:\usermodelwx</TD></TR></TABLE>> shape=none];
}
ConM[label=<<TABLE BORDER="0" CELLBORDER="1" CELLSPACING="5"><TR><TD WIDTH="150">\usermodelwx</TD></TR></TABLE>> shape=none];
CMI -> CMS [ label=" instantiates"];
ConMI -> ConM [ label=" instantiates"];
ConMI -> CMI [ label=" creates"];
}
  \caption{Relationships between the \usermodel model and CUDF*}
  \label{sim.modeldiagram}
\end{center}
\end{figure}

The \usermodel model was developed using the results of a survey conducted to determine how users change their component systems.
This survey is described in section \ref{strat.usersurvey}.

\usermodel contains variables to create a realistic scenario of CSE, e.g. a variable that describes the probability a user will upgrade their system on a given day.
In addition it contains assumptions (e.g. the probability a package will be selected to be installed) made in the simulation.
\usermodel is presented in section \ref{sim.usermodel} and the variables and assumptions for the simulation are discussed.

In order to define parts of \usermodel it was necessary to collect and convert data into usable formats.
The sources this data was collected from and the methods used to convert it are described in section \ref{sim.collection}.

This chapter concludes in section \ref{sim.modelvalidation} with a discussion over the validity of \usermodel, focusing on the differences from the reality of CSE.
These differences are important to explore, as they may reduce the validity of the results from the simulation.

\section{User Survey}
\label{strat.usersurvey}
To explore the user's role in CSE and to construct the \usermodel model, a survey was conducted on users of component systems.
This survey targeted users of GNU/Linux distributions (specifically Ubuntu) and server administrators through the online forum reddit\footnote{http://reddit.com accessed 6/3/2012}.
It was completed by 59 users, who answered questions about their background, the systems they use, and the ways they change their systems.
Typically the way in which GNU/Linux systems are changed is through interaction with package manager applications. 
Therefore, the questions were primarily about this interaction.

In this section, a description of the survey and an analysis of responses are given.
The results from this survey are used to describe and categorise the motivations and behaviours of users when changing their component systems.

\subsection{Questions}
%%%The goals of the survey:
The survey consisted of two groups of questions: questions that identified the type of user and questions that described their interactions with package managers.
These were asked in order to understand the user and how and why they changed their systems.
The set of questions used to identify the user were:
\begin{itemize}
  \item How experienced with package managers are you (based on scale of 1 to 5 where 1 is no experience and 5 is highly experienced)? 
  \item What system are you using?
  \item What package manager are you using?
\end{itemize}
These can be used to categorise the type of user and weigh their answers for credibility based on their self rated experience.

The set of questions that asked about the use of package managers were:
\begin{itemize}
  \item After you install a new system, what are your first interactions with the package manager?
  \item Describe your day to day interactions with the package manager?
  \item At what frequency are your typical interactions with the package manager?
\end{itemize}
These questions were focused on identifying the types and times the user's interacted with their package manager.
To answer these questions the user could input free text.

\subsection{Results}
%%%The results are summarised as:
As was already mentioned earlier, the survey was completed by 59 respondents.
The majority of these (29) were Debian-based operating systems users.
Among these users \texttt{apt-get} was the most popular package manager.
The other respondents used a variety of other GNU/Linux distributions and their package managers.
These package managers provide similar functionality to \texttt{apt-get}, a comparison can be seen at \texttt{http://distrowatch.com/dwres.php?resource=package-management}\footnote{accessed 7/8/2012}. 

The mean of the respondents self-ranked experience was 3.9/5.
While this can be considered a subjective measure, it is expected that the respondents were confident in their answers. 

The respondents answers to the questions about the types and the frequency of interactions with their package managers have been summarised in table \ref{strat.tblaction}.
\begin{table}[htp]
\centering
\begin{tabular}{| l | c | c | c | c |} \hline
\textbf{Request} & \textbf{Set-up} & \textbf{Daily} & \textbf{Weekly} & \textbf{Monthly} \\ \hline
Upgrade  & 45 & 27 & 16 & 0 \\
Install & 49 & 6 & 17 & 3 \\
Remove & 6 & 4 & 1 & 0\\ \hline
\end{tabular}
\caption{Summary of the survey respondents types and frequencies of interactions with their package managers.}
\label{strat.tblaction}
\end{table}

Based on this table, the vast majority of respondents upgrade and install new components when they first set-up their system.
This also shows that most users upgrade their system daily, and many install packages weekly.
Additionally, this shows that users do not often remove packages. 

\subsection{Progressive vs. Conservative Users}
One goal of this survey was to identify motivations of users to change their systems.
Based on the responses the users attitudes towards changing their systems are affected by the following risks:
\begin{itemize}
  \item the potential risk of changing the system and introducing new problems.
  \item the potential risk of becoming out-of-date and having less functionality and old problems persist.
\end{itemize}
The behaviours of these users can be described with two user stereotypes, \textbf{conservative} and \textbf{progressive}.
These terms come from the domain of politics where conservatism philosophy that emphasises minimal and gradual change in society,
which is contrasted by progressivism philosophy that promotes change and reform \citep{oed2010}.

Most users in the survey are partially motivated by both risks, e.g. a user's response that expresses a slight conservative attitude:
\begin{quotation}
``In production I rarely remove packages ([it is] easier to leave software as-is than risk breaking stuff).''
\end{quotation}
This quote shows that this respondent is less likely to change their system, even though it may be beneficial. 

Another example of a user's response that expresses a progressive attitude:
\begin{quotation}
``I update my packages whenever I log in each day''
\end{quotation}
The reason this user upgrades their system every day is likely to ensure that packages do not become out-of-date.
This behaviour may increase the functionality of their system and allow bugs in packages to be fixed, though it has the cost of additional change.

Both of these stereotypes have extremes.
One such extreme is when a user tries to eliminate all of the risks associated with being out-of-date.
For example, a user responded in the survey with an extreme progressive attitude:
\begin{quotation}
``I do run an unstable system all the time, I help mitigate this with some redundancy in my most frequently used components, 
using packages which perform the same function, but have different dependencies, since it's less likely to have multiple packages break at the same time. 
If something is rather buggy for me, I tend to update on a more frequent basis to check for the next stable point to jump into.''
\end{quotation}
This user has not fully tested components in their system (described as unstable packages).
To lower the risk that such packages have on the function of the system, this user has redundant functionality installed.
This way the user tries to ensure that potential problems will not be too severe.
The system of this user will never be out-of-date, but will be changed frequently.

An example of a very conservative user's response on the frequency with which they interact with the package manager:
\begin{quotation}
``As little as possible. I like build my box into whatever I'll need in the first couple of weeks after an install. 
Following the configuration and construction, only the occasional upgrade is necessary. 
Unless, of course, I receive a security notice about something.''
\end{quotation}
After the setting up of a system, this user will not change it unless there is a direct security risk.
This lack of change will result in the system quickly becoming out-of-date.

The ``progressive'' and ``conservative'' terminology is useful to describe different types of users and their motivations.
It is used further when describing the simulated users in chapter \ref{experiments}.

\section{\usermodel model}
\label{sim.usermodel}
The \usermodel model consists of: 
\begin{itemize}
  \item a set of variables that describe the user's behaviour when changing their component system.
  \item a set of assumptions used by the simulation.
\end{itemize}
This section presents the variables and assumptions of \usermodel, and how an instance of this model can be used to create a CUDF* document.

\subsection{Variables and Assumptions}
The \usermodel entity consists of the following variables: 
\begin{itemize}
  \item $u$ is the probability a user requests to upgrade the system per day.
  \item $i$ is the probability a user requests to install any component per day.
  \item $U$ is the MOF criteria used to select an optimal system for an upgrade request.
  \item $I$ is the MOF criteria used to select an optimal system for an install request.
\end{itemize}
An instance of \usermodel is an assignment to these variables.

The variable $u$ is the probability per day that a user will upgrade their system, and the variable $i$ is the probability a user will request to install a component.
The reason for selecting the ``per day'' resolution came from the user survey responses where typically the most frequent a user interacted with a package manager was daily.
The $I$ and $U$ criteria to install a component and upgrade a system are represented in MOF.
These criteria can be based on package manager criteria from currently applications like \texttt{apt-get} or on novel criteria.

The assumptions made in the \usermodel model are:
\begin{itemize}
  \item The starting time of the evolution $t_0$ is October 30th 2009.
  \item The initial component system $\alpha_{t_0}$ is Ubuntu 9.10 (i386) released October 29th 2009.
  \item The user interacts with the system over a year (365 days).
  \item The available components to the evolving system are located at the central Ubuntu repository.
  \item Each component name has a probability that a user will select to install it. 
\end{itemize}

The reason for selecting Ubuntu as the system to simulate is discussed in chapter \ref{introduction}.
The time frame, October 2009 to 2010, was selected to start at the release of Ubuntu 9.10 until the release of Ubuntu 10.10.
A year was selected as 30/59 respondents of the survey stated that their systems are less than or about one year old.
The Ubuntu system has 6 monthly releases, therefore the simulation will be over the release of Ubuntu 10.04 in April 2010.

The Ubuntu central repository is the core location for where packages are distributed to Ubuntu systems.
Using this as the set of packages because it is the default choice of newly installed Ubuntu systems.

Typically a user would not select a specific version of a package to be installed but select the package name and let the package manager install the most recent version.
Additionally, a user would select a package name to install a package for a purpose, to fulfil some requirement of their system.
As different users have different requirements, a user would more likely select some packages names over others.
However, this model assumes that all users will equally likely install the same packages, which is incorrect.
The impact on the validity of \usermodel caused by such assumptions is further discussed in section \ref{sim.modelvalidation}.

\subsection{CUDF* Document Creation}
The process used to create a CUDF* document from a \usermodel instance attempts to create a description of how such a user would change their system.
The first part of this process is to represent the initial system Ubuntu 9.10 and the Ubuntu repository as a CUDF* document template.
This template is a complete CUDF* document except that it contains no requests.
By adding requests to this template the evolution of a component system can be described.
How an Ubuntu 9.10 system and the Ubuntu repository are used to create the CUDF* template document is described in section \ref{sim.collection}.

Given this template and a \usermodel instance the process to create a CUDF* document is described in figure \ref{sim.CUDF*}.
\begin{figure}[h]
\begin{center}
\begin{alltt}
createCUDF*(template, \( u\), \( i\), \( U\), \( I\)):
 for \(day\) in 1 to 365:
  if random-probability() > \( u\):
    addUpgradeRequest(template,\(t\sb{0}+day\),\(U\))
  if random-probability() > \( i\):  
   cn = weighted selection without replacement from component names  
   addInstallRequest(template, cn, \(t\sb{0}+day+10 minutes\),\(I\))
\end{alltt}
  \caption{The process to create a CUDF* document from a \usermodel instance.}
  \label{sim.CUDF*}
\end{center}
\end{figure}

This process takes the CUDF* template and the variables from a \usermodel instance and adds requests to the template to  create a complete CUDF* document.
Such a document describes the realistic evolution of an Ubuntu system.

The main loop iterates over the 365 days in which the simulation occurs.
Each day the ``user'' randomly selects to upgrade their system and/or install a component.

The function \texttt{random-probability()} is defined to return a uniformly distributed random value between 0 and 1.
This function is used to randomly select what requests the user will make on a given day.

If a user requests to upgrade their system, 
the function \texttt{addUpgradeRequest} adds an upgrade request to the CUDF template at time initial time $t_0$ plus the number of elapsed days.
This request uses the MOF criteria $U$ to find an optimal system.
If the user requests to install a component, a component to be installed must first be selected.
This selection is: 
\begin{itemize}
  \item \textit{Weighted}: names with a higher weight are more likely to be requested to be installed.
  \item \textit{Without replacement}: a user cannot select to install the same component name more than once.
\end{itemize} 
The function \texttt{addInstallRequest} adds an install request to the CUDF template ten minutes after the initial time $t_0$ plus the number of elapsed days.
Adding ten minutes ensures that if a user selects to upgrade and install a component, then the installation request will occur after the upgrade request.
The install request will use the criteria $I$ from the user model.

The discussion over the validity of the resulting CUDF* documents is presented in section \ref{sim.modelvalidation}.

\section{\usermodel Data Collection and Conversion}
\label{sim.collection}
In order to create CUDF* documents from \usermodel instances, data about the Ubuntu system must be collected and converted into useful formats.
The collected data and their sources are:
\begin{enumerate}
  \item The set of Ubuntu packages and the times they became available, collected from the Ubuntu repository.
  \item The initial component system Ubuntu 9.10 collected from a virtual install.
  \item The probabilities different component names will be selected to be installed, from the \texttt{app-install-data} package and weighted using 
  the Ubuntu popularity contest\footnote{http://popcon.ubuntu.com/ accessed 6/3/2012}.
\end{enumerate}

This section describes the data collection and conversion to create \usermodel.

\subsection{Collecting the Components}
The Ubuntu repository located at \texttt{http://archive.ubuntu.com/} is the default location that Ubuntu packages are distributed from.
This repository contains a history of all packages that have ever been included with the precise minute that the package was uploaded.

To collect these packages and information from the repository first the repositories web site was parsed and all the packages were downloaded.
These Debian/Ubuntu packages are compressed with meta-information, code and binary files.

Second, each package was decompressed and the main meta-data file, the \textit{control} file, was extracted.
This control file was tagged with the upload time of the package to the Ubuntu Repository.

Third, all control files that did not have the key \verb+architecture+  include the value \verb+i386+ or \verb+all+ where removed.
This removes any package that would not work on the simulated Ubuntu system due to architecture incompatibility.

Fourth, the control files are converted into a single CUDF* document.
The conversion from Debian control files to a CUDF document is described by \cite{abatetrenen2010}.
This report describes the handling of virtual packages and the conversion of versions. 
This process is followed to create the CUDF* template with the exceptions that:
\begin{itemize}
  \item \texttt{apt-pinning} priorities that force the use of only particular versions of a package are ignored.
  \item If a package has the value \texttt{require} for key \texttt{priority}, the CUDF* package is given the property \texttt{keep: package} to ensure the package remains between requests.
  \item In the \texttt{preamble} the time $t_0$ is added.
  \item Each CUDF package has the key \texttt{time} added with the tagged time of their control file.
  \item A self conflict is not added to each package to simulate the restraint that only one version of each package is allowed to be installed.
\end{itemize} 
The final difference is significant as it differs from how real Ubuntu systems evolve and could invalidate the simulations results.
However, the restriction for these systems to only allow one version to be installed is an interesting point of study.
By allowing multiple packages to be installed means that this specific constraint can be studied.
For example, questions like ``how often does an install request require the installation of multiple packages?'' 
or ``how much does does this restriction effect the evolution of Ubuntu systems?'' can be answered by not enforcing this restriction.
It is still preferable for a user to not have multiple versions of a package.
Therefore, this restriction can be ``softly'' enforced through using a criterion to minimise the number of packages with multiple versions.
Such a criterion is defined in section \ref{impl.ipp} and it is shown in chapter \ref{experiments} that this alteration has almost no impact on the simulation.

\subsection{Probability a component will be selected}
%%%The probability a component will be selected
Different users will likely select different components to install.
For example, a user who is a graphic designer will more likely select graphics editor tools to be installed, and less likely select programming tools.
To simulate each user with their individual preferences of what they would install is impractical.
For this simulation each component name's probability to be installed will be the same for all users.

To define the probability a user will select a component to install, the problem is broken into two questions:
\begin{enumerate}
  \item What component would a user select to install?
  \item How often are these components selected to be installed?
\end{enumerate}
There are many packages in the Ubuntu repository that a user would not likely select for install.
Packages that provide libraries, background daemons, interfaces between services are usually installed because other packages depend on them, not because a user selected to install them.
What packages would a user select to install and how likely they will be selected are therefore important questions.
By looking at current systems and what packages they have installed, the probability that these packages are selected to be installed can be estimated.
These questions are answered using data from the package \texttt{app-install-package} and the Ubuntu popularity contest.

The package \texttt{app-install-package} contains a list of 2399 packages\footnote{as of May 24th 2011} with meta-data like icons and descriptions.
This data is used by other applications, like the Ubuntu Software Center, to provide a list of packages the user may wish to install.
Some of these packages are installed by default in Ubuntu systems, and some are not available in the Ubuntu repository.
After filtering out such packages from the provided list there remain 2087 packages that the user may select to install in their system. 

%%%PopCon
The probability a package from the \texttt{app-install-package} list is selected to be installed can be calculated using the Ubuntu popularity contest.
The Ubuntu popularity contest is a broad data-set of information of the popularity of Ubuntu packages.
Each week an automated survey is submitted by nearly two million users, that contains information on what packages a user has installed.
The results from this survey are processed and the number of systems that have a package installed is presented in the Ubuntu popularity contest.
By dividing the the number of systems each package from the \texttt{app-install-data} packages  by the total survey respondents, 
the probability a package is selected to be installed is estimated.

\section{\usermodel Validation}
\label{sim.modelvalidation}
The validation of \usermodel has been accomplished through:
\begin{itemize}
  \item Discussions with project supervisors and other stakeholders.
  \item Comparing \usermodel to the responses from the user survey.
  \item Compare generated CUDF* documents with \texttt{apt-get} logs collected from 19 
  respondents of the survey\footnote{Comparing these logs to the resulting simulated systems is described in section \ref{impl.validation}}.
  \item Creating a virtual Ubuntu 11.10 system to study its perspective of the Ubuntu repository changing over the month of November 2011.
\end{itemize}

Using these methods differences between the simulation and the reality of CSE can be identified and discussed.
These differences largely come from the randomness in \usermodel, e.g. randomly selecting packages to install,
and from the limitations of the model, e.g. the limited types of request.
Additional differences exist in the way in which \usermodel uses the Ubuntu repository.
This section discusses these differences and their impacts on the validity of \usermodel.

\subsection{Randomness of \usermodel}
\label{sim.randomness}
When generating a CUDF* document, each simulated day there are three points of randomness in \usermodel:
\begin{enumerate}
  \item The probability $u$ a user requests to upgrade their system.
  \item The probability $i$ a user requests to install a component.
  \item If the user requests to install, what package is selected to be installed.
\end{enumerate}
In reality, a user's request to upgrade the system or install a component are not random.
These requests are reasoned about using gathered information, preferences and external constraints.
Even the most insignificant aspect of a user may impact requests, e.g. the favourite colour of a user may impact their choices of components to install.
To model a user completely would be difficult, if not impossible, and would require significant effort.
\usermodel instead simulates the user with random behaviour.
The impact on validity of the randomness of these behaviours is discussed here.

The variables $u$ and $i$ introduce randomness into the simulation that does not exist in the evolution of real component systems.
For example, a user who upgrades their system each work day (Monday to Friday) would have a probability to upgrade their system as $5/7$.
Describing such a user with \usermodel and then generating a CUDF* document may create a situation where they will upgrade their system on a Saturday.
This is because users will not randomly select days to request changes, they have constraints (like the work week) and preferences that are not expressed in this model.

The most significant randomness in \usermodel comes from the selection of a components to be installed.
This randomness causes many differences to the reality of CSE:
\begin{itemize}
  \item A real user would typically not randomly select a component to be installed. a user would likely research a component before deciding to install.
  \item Each user would have different preferences of which components to install, e.g. a software developer will prefer components to aid in software development.
  \item Installing one package may change the preference of installing another, e.g. installing of browser \texttt{firefox} would decrease the probability to install another browser such as \texttt{chromium}.
\end{itemize}
To address these issues would require a more data to be collected (or created) and the modifications would make \usermodel significantly more complex.
The reduced effort and cost during data collection and the simplicity of the model was decided to take priority in this matter.

To mitigate the randomness in \usermodel many CUDF* documents will be created from a single \usermodel instance, then simulated.
The results from these simulations can then be aggregated and analysed.
Additionally, any conclusions derived from the results will be tempered by being aware of the randomness in \usermodel.

An important caveat to note is that there exist two assignments to the variables $u$ and $i$ that result in users that are not at all random:
\begin{enumerate}
  \item \textbf{Control} user: $u = 0$ and $i = 0$
  \item \textbf{Always Upgrade} user: $u = 1$ and $i = 0$
\end{enumerate}
The ``Control'' user never requests to change their system, therefore their system will always be the initial system and never evolve.
The ``Always Upgrade" user will update every day, and given two users that have the same upgrade criteria $U$, their systems will evolve identically.
These effects of these two users are further discussed in chapter \ref{experiments}.

\subsection{Limitations of \usermodel}
The limitations that \usermodel has are:
\begin{itemize}
  \item The initial set-up of the system by the user (as described in the user survey) is not included.
  \item Some types of requests are not included, notably remove requests.
  \item Cannot request to install many components on the same day, or at the same time.
  \item The list of packages that can be installed (from \texttt{app-install-data}) does not include many commonly installed packages.
\end{itemize}
Each of these points will be addressed in order.

The initial set-up of a system, which is performed by many respondents of the user survey, is not included in \usermodel.
This set-up includes upgrading their system, installing and removing components.
The initial upgrade is superfluous in the simulation as the initial system is already uptodate.
The initial installation and removal of components is not simulated as it will create different starting systems for different users,
making comparison between users more difficult.

In a real Ubuntu system, a user may request many different types of action, e.g. the \texttt{remove} component request.
Respondents to the user survey stated they request to remove a component infrequently, this is supported by the data in the user logs.
This is the justification for the exclusion from the \usermodel.

A user may want to install many components during a single day.
This can be accomplished with one request to install many components or many requests to install single components.
Both these situations commonly occur in the submitted user logs.
For example, \texttt{apt-get install ia32-libs ia32-libs-gtk libqt4-core libqt4-gui} is a single request to install multiple components,
and  \texttt{apt-get install autoconf} and \texttt{apt-get install checkinstall} are two requests performed on the same day.
Given the significant randomness in selecting a component to be installed, introducing either of these situations into \usermodel would only make results less valid.
For this reason, a user requesting to install many components on a single day is not simulated.

In reality, a user often requests to install packages that are not included in the list from the \texttt{app-install-data} package. 
An example of this extracted from the user logs is \texttt{apt-get install build-essential}\footnote{\texttt{build-essential} contains tools to build Debian packages},
this package cannot be installed during the simulation as it is no int the list from  \texttt{app-install-data}.
Much effort that went into creating \usermodel was directed at finding a list of the most commonly installed packages.
However, the core problem is that there exists very little information on what packages users commonly request to be installed.
This problem is seen as a possible future research that is described in chapter \ref{conclusion}. 

\subsection{Perspective of the Ubuntu Repository}
In a real Ubuntu system only a subset of the packages in the Ubuntu repository is visible to the package manager.
This ``view'' of the repository is the set of packages considered when resolving a request.
This is different to \modelname and \usermodel, 
where every request can consider all components that exist at the time of the request. 

To study the difference this view of the repository has to the simulation a virtual Ubuntu 11.10 system was created, 
and each day in the month of November 2011 the subset of components in its view of Ubuntu repository was stored.
This system used only the default ``view'' of the Ubuntu repository.
This ``view'' is compared to the changes in the repository of components over the month November 2009 in the \usermodel repository.

The \usermodel repository contains about 90,000 different package names, where the virtual ``view'' contained only about 40,000.
This difference is made up mostly of deprecated packages, i.e. packages that are no longer actively maintained.
It is also made up of packages that are have not yet been validated and put into the main ``view'', e.g. experimental packages. 
This causes the mean amount of packages that have a new version added to the repository to be much higher in the \usermodel repository,
126 packages per day compared to the virtual systems view of 37 packages per day.

The most dramatic difference though is the amount of removed packages from the ``view''.
The mean amount of packages removed per day from the view of the Ubuntu repository is eight, where the \usermodel repository never removes a package.
This removal of packages reduces the complexity of the problem, 
though it also means that a real Ubuntu system may not have all information available when satisfying user requests. 

The reason a ``view'' of the repository was not used in the simulation is due to the fact that no data-source could be found 
that saved the set of packages that could be viewed in the Ubuntu repository at a particular time.
This lack of data lead to the decision that all components should be used.

\section{Summary}
This chapter presented the \usermodel model which is used to describe how a realistic user requests changes to their system.
This model was partially created from a survey, whose results where presented and discussed.
The variables that make up the \usermodel and the process used to create CUDF* documents were then described.
The data that was used to create the \usermodel was then described with the processes to collect and convert it.
The validation of \usermodel was then described, and the differences to the reality of CSE discussed.
By exploring the differences to the reality of CSE the results from a simulation can be more accurately interpreted.

Although the \usermodel model can be used to create a description of the evolution of a component system, it may require significant calculation to find how exactly the component system evolves.
The constraints of each request must be satisfied, and the optimal system w.r.t. the criteria must be found.
The following chapter presents the mapping an instance of \modelname to a series of problems and the algorithms to solve such problems.
These algorithms are implemented in GJSolver, which is the final piece required to simulate CSE.

